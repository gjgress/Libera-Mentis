\documentclass{memoir}
\usepackage{notestemplate}

%\logo{~/School-Work/Auxiliary-Files/resources/png/logo.png}
%\institute{Rice University}
%\faculty{Faculty of Whatever Sciences}
%\department{Department of Mathematics}
%\title{Class Notes}
%\subtitle{Based on MATH xxx}
%\author{\textit{Author}\\Gabriel \textsc{Gress}}
%\supervisor{Linus \textsc{Torvalds}}
%\context{Well, I was bored...}
%\date{\today}

%\makeindex

\begin{document}

% \maketitle

% Notes taken on 


\begin{prop}
	The following conditions on an \(R\)-module \(V\) are equivalent:
	\begin{itemize}
		\item Every submodule of \(V\) is finitely generated
		\item There is no infinite strictly increasing chain \(W_1 < W_2 < \ldots\) of submodules of \(V\).
	\end{itemize}
\end{prop}

We formalize this notion via Noetherian rings and modules.

\begin{defn}[Noetherian Modules]
A left \(R\)-module \(\prescript{}{R}M\) is said to be a \textbf{Noetherian \(R\)-module} if it satisfies the ascending chain condition on submodules. That is, for any increasing chain of submodules of \(M\) 
\begin{align*}
	M_1\subset M_2 \subset M_3 \subset \ldots
\end{align*}
there is a positive integer \(m\) such that, for all \(k\geq m\), \(M_k = M_m\).\\

	A ring \(R\) is \textbf{Noetherian} if it is Noetherian as a left module over itself. That is, there are no infinite increasing chains of left ideals of \(R\).
\end{defn}
\begin{cor}
	If \(R\) is Noetherian then every ideal of \(R\) is finitely generated.
\end{cor}

\begin{thm}[Submodules of Noetherian]
	Let \(R\) be a ring and \(\prescript{}{R}M\) a left \(R\)-module. Then the following are equivalent:
	\begin{itemize}
		\item \(M\) is a Noetherian \(R\)-module
		\item Every nonempty set of submodules of \(M\) contains a maximal element under inclusion
		\item Every submodule of \(M\) is finitely generated
	\end{itemize}
\end{thm}
Notice that these conditions also imply that \(R\) is a Noetherian ring, and furthermore that every proper ideal of \(R\) is a maximal ideal.

\begin{cor}
	If \(R\) is a PID then every collection of ideals of \(R\) has a maximal element, and \(R\) is Noetherian.
\end{cor}

\begin{lemma}
	Let \(\varphi:V\to V'\) be a homomorphism of \(R\)-modules.
	\begin{itemize}
		\item If \(V\) is finitely generated and \(\varphi\) is surjective, then \(V'\) is finitely generated.
		\item If the kernal and image of \(\varphi\) are finitely generated, then \(V\) is finitely generated.
		\item Let \(W\) be a submodule of an \(R\)-module \(V\). If both \(W\) and \(\overline{V} = V / W\) are finitely generated, then \(V\) is finitely generated. If \(V\) is finitely generated, so is \(\overline{V}\).
	\end{itemize}
\end{lemma}

\begin{thm}[Hilbert Basis Theorem]
	Let \(R\) be a Noetherian ring. The polynomial ring \(R[x]\) is Noetherian.
\end{thm}

\begin{prop}[Quotients of Noetherian]
	Let \(R\) be a Noetherian ring, and let \(I\) be an ideal of \(R\). Any ring that is isomorphic to the quotient ring \(\overline{R} = R / I\) is Noetherian.
\end{prop}
\begin{cor}
	Let \(P\) be a polynomial ring in a finite number of variables over the integers/field. Any ring \(R\) that is isomorphic to the quotient ring \(P / I\) is Noetherian.
\end{cor}
\begin{lemma}
	Let \(R\) be a ring, let \(I\) be an ideal of the polynomial ring \(R[x]\). The set \(A\) whose elements are the leading coefficients of the nonzero polynomials in \(I\), together with the zero element of \(R\), is an ideal of \(R\), the \textbf{ideal of leading coefficients}.
\end{lemma}



% \printindex
\end{document}
