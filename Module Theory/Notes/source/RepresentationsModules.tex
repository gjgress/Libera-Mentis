\documentclass{memoir}
\usepackage{notestemplate}

%\logo{~/School-Work/Auxiliary-Files/resources/png/logo.png}
%\institute{Rice University}
%\faculty{Faculty of Whatever Sciences}
%\department{Department of Mathematics}
%\title{Class Notes}
%\subtitle{Based on MATH xxx}
%\author{\textit{Author}\\Gabriel \textsc{Gress}}
%\supervisor{Linus \textsc{Torvalds}}
%\context{Well, I was bored...}
%\date{\today}

%\makeindex

\begin{document}

% \maketitle

% Notes taken on 

\begin{defn}[Presentations]
	Let an \(m\times n\) \(R\)-matrix denoted by \(A\) be a homomorphism of \(R\)-modules
	\begin{align*}
		R^{n}\stackrel{A}{\to} R^{m}.
	\end{align*}
	We can denote its image by \(AR^{n}\). We say that the quotient module \(V = R^{m}/ AR^{n}\) is \textbf{presented} by the matrix \(A\). Any isomorphism \(\sigma:R^{m} / AR^{n} \to V\) is a \textbf{presentation} of a module \(V\), of which \(A\) is a \textbf{presentation matrix} for \(V\) if there is such an isomorphism.
\end{defn}

We use the canonical map \(\pi:R^{m}\to V = R^{m} / AR^{n}\) to interpret the quotient module as follows:
\begin{prop}
\(V\) is generated by a set of elements \(B = (v_1,\ldots,v_m)\), the images of the standard basis elements of \(R^{m}\). Furthermore, if \(Y\) is a column vector in \(R^{m}\), the element \(BY\) of \(V\) is zero if and only if \(Y\) is a linear combination of the columns of \(A\), with coefficients in \(R\), if and only if there exists a column vector \(X\) with entries in \(R\) such that \(Y=AX\).
\end{prop}
If a module \(V\) is generated by a set \(B = (v_1,\ldots,v_m)\), we call any element \(Y \in R^{m}\) such that \(BY = 0\) a \textbf{relation vector}, or simply a \textbf{relation} among the generators. A set \(S\) of relations is a \textbf{complete set} if every relation is a linear combination of \(S\) with coefficients in the ring.
\begin{prop}[Theoretical Method of Finding a Presentation]
	First, choose a set of generators \(B = (v_1,\ldots,v_m)\) for \(V\). These generators give a surjective homomorphism \(R^{m}\to V\) that sends a column vector \(Y\) to the linear combination \(BY = v_1y_1+\ldots+v_my_m\). Denote the kernel of the map by \(W\). It is the \textbf{module of relations}; its elements are the relation vectors.\\

	Repeat this procedure, choosing a set of generators \(C = (w_1,\ldots,w_m)\) for \(W\), and define a surjective map \(R^{n}\to W\) using them. Here the generators \(w_j\) are elements of \(R^{m}\), and thus column vectors. Assemble the coordinate vectors \(A_j\) of \(w_j\) into a matrix with \(A_i\) as column \(i\). Then multiplication by \(A\) defines
	\begin{align*}
		R^{n}\to^{A} R^{m}
	\end{align*}
	which sends \(e_j \mapsto A_j = w_j\), as it is the composition of \(R^{n}\to W\) with the inclusion \(W \subset R^{m}\). By construction \(W\) is its image and we denote it by \(AR^{n}\). Because the map \(R^{m}\to V\) is surjective, by the First Isomorphism Theorem, \(V\) is isomorphic to \(R^{m}/W = R^{m}/AR^{n}\). Hence \(V\) is presented by the matrix \(A\).\\

	In short the presentation matrix \(A\) for a module \(V\) is determined by the set of generators for \(V\), and the set of generators for the module of relations \(W\). Assuming the set of generators does not form a basis, the number of generators will be equal to the number of rows of \(A\).
\end{prop}
Note that this relies on the assumption that \(V\) has finite generators. We must also assume that \(W\) has a finite set of generators, which is slightly more problematic.

\begin{prop}[Rules for manipulating \(A\) without changing isomorphism class]
	Let \(A\) be an \(m \times n\) presentation matrix for a module \(V\). The following matrices \(A' \) present the same module \(V\):
	\begin{itemize}
		\item \(A' = Q^{-1}A\), \(Q \in GL_m(R)\)
		\item \(A' = AP\) with \(P \in GL_n(R)\)
		\item \(A'\) is obtained by deleting a column of zeroes
		\item if the \(j\)-th column of \(A\) is \(e_i\), then removing row \(i\) and column \(j\) preserves the presentation
	\end{itemize}
\end{prop}

% \printindex
\end{document}
