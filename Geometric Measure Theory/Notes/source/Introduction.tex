\documentclass{memoir}
\usepackage{notestemplate}

%\logo{~/School-Work/Auxiliary-Files/resources/png/logo.png}
%\institute{Rice University}
%\faculty{Faculty of Whatever Sciences}
%\department{Department of Mathematics}
%\title{Class Notes}
%\subtitle{Based on MATH xxx}
%\author{\textit{Author}\\Gabriel \textsc{Gress}}
%\supervisor{Linus \textsc{Torvalds}}
%\context{Well, I was bored...}
%\date{\today}

\begin{document}

% \maketitle

% Notes taken on 05/20/21

\chapter{Introduction}
\label{cha:introduction}

These notes are personal notes I have created in the process of studying geometric measure theory and contain a wide variety of definitions and techniques that appear often in the field. The notes were created primarily from a reading course taken with Dr. Gregory Chambers at Rice University in Spring 2021 that followed Dr. Kenneth Falconer's textbook \textit{Fractal Geometry}. One will notice that the proofs of major results are either lacking or not included in these notes. This is because this document is primarily intended as a reference-- if the reader is looking for a deeper insight as to why these results are true, I would highly recommend one read the details in \textit{Fractal Geometry}-- if the proof isn't in there, then I have listed the source separately with the statement.\\

The results in this book assume a basic understanding of measure theory, and so one should already know the definition of a measure, \(\sigma \)-algebra, Borel set, and so on. Introductory notes on this topic will be provided in the LibreMath repository (soon).

\end{document}
