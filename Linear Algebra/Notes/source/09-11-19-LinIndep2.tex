\documentclass{memoir}
\usepackage{notestemplate}

\begin{document}
If $\text{Span}(v_1,\ldots,v_m) = V$, we say that $v_1,\ldots,v_m$ "span" the space $V$. \\

We call the set of unit vectors of a space the \textbf{standard basis}.
\begin{defn}
	A vector space $V$ is \textbf{finite-dimensional} if $V = span(v_1,\ldots,v_m)$.
\end{defn}
If a vector space is not finite-dimensional, then we say it is \textbf{infinite-dimensional}. To maintain convention, if \(V = F[x]\), then \(\textrm{deg}(0) = -\infty\).
\begin{thm}
	\(F[x]\) is infinite-dimensional.
\end{thm}

\subsection{Linear Independence}
\begin{defn}[Linear independence]
	A list $v_1,\ldots,v_m$ of vectors is \textbf{linearly independent} if and only if
	\begin{align*}
	a_1v_1+\ldots+a_mv_m = \vec{0}\implies a_1,\ldots,a_m = 0
	\end{align*}
	
\end{defn}
	A list of vectors in $V$ is \textbf{linearly dependent} if it is not linearly independent.\\

	We can also say that a list in $V$ is linearly dependent if there exist $a_1,\ldots,a_m$ not all zero such that $a_1v_1+\ldots+a_mv_m = 0$.

\begin{lemma}
	If \(v_1,\ldots,v_m\) are linearly independent, then there is exactly one way to write $v \in \text{Span}(v_1,\ldots,v_m)$ as a linear combination of the vectors.
\end{lemma}
\begin{thm}
Let $V$ be a finite-dimensional vector space. If \(v_1,\ldots,v_m\) is an arbitrary set of linearly independent vectors in \(V\) and \(w_1,\ldots,w_k\) is an arbitrary set of vectors that span \(V\), then \(m\leq k\).

\end{thm}
\begin{lemma}[Linear Dependence Lemma]
	Let $v_1,\ldots,v_m \in V$ be a set of linearly dependent vectors. Then there exists an index $j$ such that \begin{itemize}
		\item $v_j \in \text{Span}(v_1,\ldots,v_{j-1})$ 
		\item $\text{Span}(v_1,\ldots,v_m) = \text{Span}(v_1,\ldots,v_{j-1},v_{j+1},\ldots,v_m)$ 
	\end{itemize}
\end{lemma}
\begin{lemma}[Finite-dimensional subspaces]
	Every subspace of a finite-dimensional vector space is finite-dimensional.
\end{lemma}

\end{document}
