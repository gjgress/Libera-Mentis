\documentclass{memoir}
\usepackage{linalg}

% \begin{figure}[ht]
%     \centering
%     \incfig{riemmans-theorem}
%     \caption{Riemmans theorem}
%     \label{fig:riemmans-theorem}
% \end{figure}

\begin{document}
\section{The Spectral Theorem}
\label{sec:the_spectral_theorem}

\subsection{The Complex Spectral Theorem}
\label{subsec:the_complex_spectral_theorem}

\begin{thm}[Complex Spectral Theorem]
	Suppose \(F = \C\) and \(T = \mathcal{L}(V)\). Then the following are equivalent:
	\begin{itemize}
		\item \(T\) is normal
		\item \(V\) has an orthonormal basis consisting of eigenvectors of \(T\).
		\item \(T\) has a diagonal matrix with respect to some orthonormal basis of \(V\).
	\end{itemize}
\end{thm}

\subsection{The Real Spectral Theorem}
\label{subsec:the_real_spectral_theorem}

\begin{lemma}
	Suppose \(T \in \mathcal{L}(V)\) is self-adjoint and \(b,c \in \R\) are such that \(b^2< 4c\). Then
	\begin{align*}
		T^2+bT+cI
	\end{align*}
is invertible.
\end{lemma}
\begin{lemma}
	Suppose \(V \neq \left\{ 0 \right\} \) and \(T \in \mathcal{L}(V)\) is a self-adjoint operator. Then \(T\) has an eigenvalue.
\end{lemma}
\begin{lemma}
	Suppose \(T \in \mathcal{L}(V)\) is self-adjoint and \(U\) is a subspace of \(V\) that is invariant under \(T\). Then
	\begin{itemize}
		\item \(U^{\perp}\) is invariant under \(T\) 
		\item \(T\mid_U \in \mathcal{L}(U)\) is self-adjoint
		\item \(T\mid_{U^{\perp}} \in \mathcal{L}(U^{\perp})\) is self-adjoint.
	\end{itemize}
\end{lemma}
\begin{thm}[Real Spectral Theorem]
	Suppose \(F = \R\) and \(T \in \mathcal{L}(V)\). Then the following are equivalent:
	\begin{itemize}
		\item \(T\) is self-adjoint
		\item \(V\) has an orthonormal basis consisting of eigenvectors of \(T\).
		\item \(T \) has a diagonal matrix with respect to some orthonormal basis of \(V\).
	\end{itemize}
\end{thm}

\end{document}
