\documentclass{memoir}
\usepackage{linalg}

% \begin{figure}[ht]
%     \centering
%     \incfig{riemmans-theorem}
%     \caption{Riemmans theorem}
%     \label{fig:riemmans-theorem}
% \end{figure}

\begin{document}
\section{Matrices}
\label{cha:matrices}
\subsection{Representing a Linear Map by Matrices}
\begin{defn}[Matrix]
	Let $m$ and $n$ denote positive integers. An $m$-by-$n$ \textbf{matrix} $A$ is a rectangular array of elements of $F$ with $m$ rows and $n$ columns:
\begin{align*}
	A = \begin{bmatrix} A_{1,1} & \dots & A_{1,n} \\
		\vdots & \ddots & \vdots \\
		A_{m,1} & \dots & A_{m,n}
	\end{bmatrix} 
\end{align*}
where $j$ refers to the row number, and $k$ refers to the column number.
\end{defn}
\begin{defn}[Matrix of a linear map]

Suppose $T \in \mathcal{L}(V,W)$, $v_1,\ldots,v_n$ a basis of $V$, and $w_1,\ldots,w_m$ a basis of $W$. Then, the \textbf{matrix of $T$} with respect to these bases is the $m$-by-$n$ matrix $\mathcal{M}(T)$ whose entries $A_{j,k}$ are defined by
\begin{align*}
	Tv_k = A_{1,k}w_1 + \ldots + A_{m,k}w_m .
\end{align*}
If the bases are not clear from context, we use the notation $\mathcal{M}(T,(v_1,\ldots,v_n),(w_1,\ldots,w_m))$ is used.

\end{defn}
\color{gray}
\begin{exmp}[Examples of matrices of linear maps]
	Consider $T:\R^2\to \R^3$ defined by $(x,y) \mapsto (x+3y,2x+5y,7x+9y)$. By applying this to the standard basis for $\R^2$ and $\R^3$, we can determine the matrix:
	\begin{align*}
		T(v_1) = T(1,0) = (1,2,7) \\
		T(v_2) = T(0,1) = (3,5,9) \\
		\text{ so then } \mathcal{M}(T) = \begin{bmatrix} 
			1 & 3\\
			2 & 5\\
			7 & 9
		\end{bmatrix} 
	\end{align*}

	Another important example is the differentiation map $D: P_3(\R) \to P_2(\R)$ defined by $p \mapsto p'$.\\

	Choose bases $P_3(\R) = \left\{1,x,x^2,x^3 \right\}$ and $P_2(\R) = \left\{1,x,x^2 \right\} $. Then
	\begin{align*}
		D(v_1) = D(1) = 0 = 0\cdot w_1 + 0\cdot w_2 + 0\cdot w_3 \\
		D(v_2) = D(x) = 1 = 1\cdot w_1 + 0 \cdot w_2 + 0 \cdot w_3 \\
		D(v_3) = D(x^2) = 2x = 0\cdot w_1 + 2 \cdot w_2 + 0 \cdot w_3 \\
		D(v_4) = D(x^3) = 3x^2 = 0 \cdot w_1 + 0\cdot w_2 + 3\cdot w_3 \\
		\text{ So therefore } \mathcal{M}(D) = \begin{bmatrix} 
			0 & 0 & 0\\
			1 & 0 & 0\\
			0 & 2 & 0 \\
			0 & 0 & 3 
		\end{bmatrix} 
	\end{align*}
\end{exmp}
\color{black}

\section{Addition and Scalar Multiplication of Matrices of Linear Maps}
\label{sec:addition_and_scalar_multiplication_of_matrices_of_linear_maps}
\begin{cor}[Linearity of Matrices]
	Suppose $S,T \in \mathcal{L}(V,W)$. Then $\mathcal{M}(S+T) = \mathcal{M}(S) + \mathcal{M}(T)$.\\

	Suppose $\lambda \in F$ and $T \in \mathcal{L}(V,W)$. Then $\mathcal{M}(\lambda T) = \lambda \mathcal{M}(T)$.
\end{cor}
Notation: The set of all $m$-by-$n$ matrices with entries in $F$ is denoted by $F^{m,n}$.
\begin{cor}
	If $m,n$ are positive integers, then $F^{m,n}$ is a vector space with dimension $mn$.
\end{cor}
\end{document}
