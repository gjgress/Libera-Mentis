\documentclass{memoir}
\usepackage{notestemplate}

%\logo{~/School-Work/Auxiliary-Files/resources/png/logo.png}
%\institute{Rice University}
%\faculty{Faculty of Whatever Sciences}
%\department{Department of Mathematics}
%\title{Class Notes}
%\subtitle{Based on MATH xxx}
%\author{\textit{Author}\\Gabriel \textsc{Gress}}
%\supervisor{Linus \textsc{Torvalds}}
%\context{Well, I was bored...}
%\date{\today}

%\makeindex

\begin{document}

% \maketitle

% Notes taken on 

%Proof of the simplicity of A_n. Requires Sylow's theorem
\begin{thm}[Simplicity of A_n revisited]
	\(A_n\) is simple for \(n\geq 5\).
\end{thm}

One can show this by showing that \(A_n\) is generated by 3-cycles. Then a normal subgroup must contain one 3-cycle, and hence all 3-cycles, and hence cannot be a proper subgroup. However, this approach is computationally heavy-- we will show a nicer approach.

\begin{proof}
	We will show this by induction on \(n\). We have already proven separately that the \(A_5\) case holds. Now assume \(n\geq 6\), and assume there exists \(H \triangleleft A_n\) with \(H\neq 1,G\).\\

	Consider the natural action of \(A_n\) on \(\left\{ 1,2,\ldots,n \right\} \), and let \(G_i\) be the stabilizer of \(i \in \left\{ 1,2,\ldots,n \right\} \). Then it holds that \(G_i \leq A_n\), and \(G_i \cong A_{n-1}\). By the induction hypothesis, \(G_i\) is simple for all \(i\).\\

	Suppose there is a nontrivial element \(\tau  \in H\) which stabilizes some \(i\)-- that is, \(\tau (i) = i\). Then it holds that \(\tau  \in H\cap G_i\). Because \(H\) is normal, \(H\cap G_i \triangleleft G_i\). But our induction hypothesis tells us that \(G_i\) is simple, and so because \(H\) is nontrivial, it must hold that \(H\cap G_i = G_i\), and hence \(G_i \leq H\). One can check that \(\sigma G_i \sigma ^{-1} = G_{\sigma (i)}\), and hence for all \(i\), \(\sigma G_i \sigma ^{-1} \leq \sigma H \sigma ^{-1} = H\), and hence
	\begin{align*}
		G_i \leq  H
	\end{align*}
	for ALL \(i \in \left\{ 1,2,\ldots,n \right\} \). Any \(\sigma  \in A_n\) can be written as a product of an even number of transpositions. Of course, because \(n>4\), a transposition fixes SOME element in \(G\), and hence each even transposition in \(\sigma \) is an element of some \(G_i\). Thus
	\begin{align*}
		G = \langle G_1,G_2,\ldots,G_n \rangle \leq H
	\end{align*}
	which is a contradiction.\\

	This tells us that if \(\tau \in H \)  is nontrivial, then \(\tau (i) \neq i\) for all \(i \left\{ 1,2,\ldots,n \right\} \). Furthermore, notice that if \(\tau_1(i) = \tau_2(i)\), then \(\tau_1=\tau 2\)-- otherwise \(\tau_2^{-1}\tau_1(i) = i\).\\

	Suppose there is a permutation \(\tau  \in H\) so that the cycle decomposition has an \(n\)-cycle for \(n\geq 3\). Then there exists another permutation \(\sigma \) that fixes the first two elements of a cycle (call them \(a_1,a_2\)), but not the third (call it \(a_3\)). Then \(\sigma \tau \sigma^{-1}\) is a distinct permutation such that \(\tau(a_1) = \sigma \tau \sigma^{-1}(a_1) = a_2\), contradicting our previous statement. So, no \(\tau  \in H\) can have a cycle of length 3 or more.\\

	This leaves us with only one possibility-- \( \tau  \in H\) is a product of 2-cycles:
	\begin{align*}
		\tau = (a_1a_2)(a_3a_4)(a_5a_6)\ldots
	\end{align*}
	Let \(\sigma = (a_1a_2)(a_3a_5) \in G\). Then \(\sigma \tau \sigma^{-1}(a_1) = \tau (a_1) = a_2\), but once again is distinct from \(\tau\), giving us a contradiction. Hence we have that \(H\) cannot be non-trivial.
\end{proof}

% \printindex
\end{document}
