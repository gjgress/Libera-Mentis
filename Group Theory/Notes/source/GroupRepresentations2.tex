\documentclass{memoir}
\usepackage{notestemplate}
%\logo{~/School-Work/Auxiliary-Files/resources/png/logo.png}
%\institute{Rice University}
%\faculty{Faculty of Whatever Sciences}
%\department{Department of Mathematics}
%\title{Class Notes}
%\subtitle{Based on MATH xxx}
%\author{\textit{Author}\\Gabriel \textsc{Gress}}
%\supervisor{Linus \textsc{Torvalds}}
%\context{Well, I was bored...}
%\date{\today}

%\makeindex

\begin{document}

% \maketitle

% Notes taken on 06/03/21

\section{Group Representations and Free Groups}
\label{sec:group_representations_and_free_groups}
We revisit group representations by introducing some new concepts to incorporate, and see how that allows us to expand our theory.

\begin{defn}[Commutator]
	Let \(x,y \in G\) be elements of a group, and let \(A,B \subset G\) be nonempty subsetf of \(G\). The \textbf{commutator of \(x\) and \(y\)} is denoted by
	\begin{align*}
		[x,y] = x ^{-1}y^{-1}xy
	\end{align*}
	and the group generated by commutators of elements from \(A,B\) is denoted by
	\begin{align*}
		[A,B] = \langle [a,b] \mid a \in A, b \in B \rangle .
	\end{align*}
	We can also define a subgroup of \(G\) by the group generated by commutators of elements of \(G\):
	\begin{align*}
		G' = \langle [x,y] \mid x,y \in G \rangle 
	\end{align*}
	We call this the \textbf{commutator subgroup} of \(G\).
\end{defn}
This terminology arises because the commutator of \(x,y\) is 1 if and only if \(x\) and \(y \) commute.

\begin{prop}[Properties of commutators]
	Let \(x,y \in G\) be elements of a group and let \(H\leq G\). Then
	\begin{itemize}
		\item \(xy = yx[x,y]\)
		\item \(H \triangleleft G\) if and only if \([H,G] \leq H\) 
		\item \(\sigma [x,y] = [\sigma (x),\sigma (y)]\) for any automorphism \(\sigma \) of \(G\). Hence, \(G' \textrm{char}G\), and \(G / G'\) is abelian.
		\item If \(H \triangleleft G\) and \(G/H\) is abelian, then \(G' \leq H\). Conversely, if \(G' \leq H\), then \(H \triangleleft G\) and \(G / H\) is abelian.
		\item If \(\varphi :G\to H\) is a homomorphism of \(G\) into \(H\) and \(H\) is abelian, then \(G' \leq \textrm{Ker}\varphi \) and the following diagram commutes:
\begin{center}
			\begin{tikzpicture}
  \matrix (m)
    [
      matrix of math nodes,
      row sep    = 3em,
      column sep = 4em
    ]
    {
	    G & G / G' \\
	     & H            \\
    };
  \path
    (m-1-2) edge [->] node {} (m-2-2)
    (m-1-1.east |- m-1-2)
      edge [->] node {} (m-1-2)
      (m-1-1) edge [->] node [below] {$\varphi$} (m-2-2);
\end{tikzpicture}
\end{center}
	\end{itemize}
\end{prop}
The way to think about this is that by passing to the quotient by the commutator subgroup of \(G\), we collapse all commutators to identity. Hence, all elements in the quotient group commute. This is why we have such a strong property in that, if \(G' \leq H\), then \(G / H\) must be abelian.\\

One word of caution-- there can be elements of the commutator subgroup that \textit{cannot} be written as a single commutator \([x,y]\) for any \(x,y\). In other words, \(G'\) is not just the set of single commutators, but is the group generated by elements of that form.

\begin{prop}
	Let \(H,K \leq G\) be subgroups. The number of distinct ways of writing each element of the set \(HK\) in the form \(hk\), for some \(h \in H\), \(k \in K\), is \(\left| H \cap K \right| \).\\

	If \(H\cap K = 1\), then each element of \(HK\) can be written uniquely as a product \(hk\) for some \(h \in H\), \(k \in K\).
\end{prop}

\begin{thm}
	Let \(H,K\leq G\) be subgroups of \(G\) such that \(H,K \triangleleft G\) and \(H\cap K = 1\). Then
	\begin{align*}
		HK \cong H\times K
	\end{align*}
\end{thm}

\subsection{Free Groups}
\label{sub:free_groups}

The idea of the free group is to define a group \(F(S)\) to be generated by some set \(S\) with no relations on any of the elements of \(S\). For example, if \(S = \left\{ a,b \right\} \), then some elements of \(F(S)\) would be of the form \(a,aa,ab,abab,bab\), as well as the inverses of these elements. We call elements of a free group \textbf{words}. Then we can multiply elements in the free group simply by concatenation. Our goal will be to define this formally and show it indeed satisfies the necessary properties.

\begin{general}[Construction of Free Groups]
	Let \(S\) be a set, and let \(S^{-1}\) be a set disjoint from \(S\) such that there is a bijection from \(S\) to \(S^{-1}\). We denote the corresponding element for \(s \in S\) to be \(s\mapsto s^{-1}\in S^{-1}\), and furthermore we denote \((s^{-1})^{-1} = s\). Finally, we add a third singleton set disjoint from \(S,S^{-1}\) and call it \(\left\{ 1 \right\} \), and define it so \(1^{-1} = 1\). We also define that for any \(x \in S \cup S^{-1}\cup \left\{ 1 \right\} \), \(x^{1} = x\).\\

	A \textbf{word} on \(S\) is a sequence \((s_1,s_2,s_3,\ldots)\) where \(s_i \in S\cup S^{-1} \cup \left\{ 1 \right\} \), and \(s_i = 1\) for all \(i\geq N\) for some arbitrarily large \(N\) (so that words are "infinite", but not in practice). In order to get uniqueness of words, we say a word is \textbf{reduced} if
	\begin{align*}
		s_{i+1} \neq s_{i}^{-1} \quad \forall i, s_i \neq 1\\
		s_k = 1 \implies s_i = 1 \; \forall i \geq k
	\end{align*}
	We refer to the special word given by
	\begin{align*}
		(1,1,1,\ldots)
	\end{align*}
	to be the \textbf{empty word} and denote it by \(1\). Let \(F(S)\) be the set of reduced words on \(S\), and embed mKS into \(F(S)\) by
	\begin{align*}
		s \mapsto (s,1,1,1,\ldots)
	\end{align*}
	Hence we identify \(S\) with its image and consider \(S\subset F(S)\). Notie that if \(S = \emptyset\), \(F(S) = \left\{ 1 \right\} \).\\

	Now we simply introduce a binary operation on \(F(S)\), so that two words in \(F(S)\) are concatenated, then reduced to their reduced word form. We leave the details of defining this binary operation to the reader, but one can check that this operation is well-defined and satisfies all the properties of a group operation.
\end{general}

\begin{thm}
	\(F(S)\) is a group by the binary operation of word concatenation with reduction.
\end{thm}
Furthermore, free groups satisfy a special kind of universal property.
\begin{thm}
	Let \(G\) be a group, \(S\) a set, and \(\varphi :S\to G\) a set map. There is a unique group homomorphism \(\Phi :F(S) \to G\) such that the following diagram commutes:
\begin{center}
			\begin{tikzpicture}
  \matrix (m)
    [
      matrix of math nodes,
      row sep    = 3em,
      column sep = 4em
    ]
    {
	    S & F(S) \\
	     & G            \\
    };
  \path
	  (m-1-2) edge [->] node [right] {\(\Phi \)} (m-2-2)
    (m-1-1.east |- m-1-2)
    edge [->] node [above] {inclusion} (m-1-2)
      (m-1-1) edge [->] node [below] {$\varphi$} (m-2-2);
\end{tikzpicture}
\end{center}

\end{thm}
This further shows that \(F(S)\) is unique up to a unique isomorphism, which is the identity map on the set \(S\).

\begin{defn}[Free Group]
	The group \(F(S)\) is called the \textbf{free group} on the set \(S\). A group \(F\) is a \textbf{free group} if there is some set \(S\) such that \(F = F(S)\), in which case we call \(S\) a set of \textbf{free generators} of \(F\). The cardinality of \(S\) is called the \textbf{rank} of the free group.
\end{defn}

\begin{thm}
	Subgroups of a free group are free.
\end{thm}
Furthermore, if \(G\leq F\) are free and \([F:G] = m\), then
\begin{align*}
	\textrm{rank}(G) = 1 + m(\textrm{rank}(F)-1)
\end{align*}
Proving this requires a lot of other tools, such as covering spaces.

\subsection{Presentations}
\label{sub:presentations}

Notice that if we take \(S = G\), then we can view \(G\) as a homomorphic image of the free group \(F(G)\) onto \(G\). Moreover, if \(G= \langle S \rangle \), there is a unique surjective homomorphism from  \(F(S)\) onto \(G\) which is the identity on \(S\). This allows us to construct a more powerful construction of presentations, generators, and relations.

\begin{defn}
	A subset \(S\subset G\) \textbf{generates \(G\)} by \(G = \langle S \rangle \) if and only if the map \(\pi :F(S) \to G\) which extends the identity map of \(S\) to \(G\) is surjective.
\end{defn}
This is distinct but equivalent to our earlier notion for subsets generating a group. However, it is more flexible, so we will use this from here on out.

\begin{defn}[Presentations, Generators, and Relations]
	Let \(S\subset G\) be a subset of \(G\) such that \(G = \langle S \rangle \). A \textbf{presentation} for \(G\) is a pair \((S,R)\), where \(R\) is a set of words in \(F(S)\) such that
	\begin{align*}
		\textrm{ncl}_{F(S)}(\langle R \rangle ) = \textrm{Ker}(\pi )
	\end{align*}
	where \(\textrm{ncl}\) denotes the normal closure (the smallest normal subgroup containing \(\langle R \rangle \)). The elements of \(S\) are called \textbf{generators}, and the elements of \(R\) are called \textbf{relations} of \(G\).\\

	We say \(G\) is \textbf{finitely generated} if there is a presentation \((S,R)\) such that \(S\) is finite. Furthermore, \(G\) is \textbf{finitely presented} if \(R\) is also finite.
\end{defn}
A word of caution-- the kernel of the map \(F(S) \to G\) is \textit{not} \(\langle R \rangle \), but instead the union of all subsets conjugate to \(\langle R \rangle \) (including \(\langle R \rangle \) itself). Furthermore, even if \(S\) is fixed, a group will have many different presentations.\\

Finally, often when writing relations, if we have \(w_1w_2^{-1} = 1\), we might instead write \(w_1=w_2\), or vice versa.

\subsection{Applying presentations to find homomorphisms and automorphisms}
\label{sub:applying_presentations_to_find_homomorphisms_and_automorphisms}

Suppose \(G\) is presented by \((\langle a,b \rangle , \langle r_1,\ldots,r_k \rangle )\). Then if \(a',b' \in H\) are elements that satisfy \(r_1,\ldots,r_k\), then there is a homomorphism from \(G\) into \(H\). If \(\pi :F(\left\{ a,b \right\} )\to G\) is the presentation homomorphism, we can define
\begin{align*}
	\pi ':F(\left\{ a,b \right\} ) \to H\\
	\pi'(a) = a', \; \pi'(b) = b'.
\end{align*}
This works because \(\textrm{Ker}\pi \leq \textrm{Ker}\pi'\), and so \(\pi '\) factors through \(\textrm{Ker}\pi \) and we get
\begin{align*}
	G \cong F(\left\{ a,b \right\} ) / \textrm{Ker}\pi \to H
\end{align*}
Moreover, if \(\langle a',b' \rangle = H = G\), then this homomorphism is an automorphism of \(G\)(!!). In the other direction, any automorphism on a presentation must send a set of generators to another set of generators satisfying the same relations.

\begin{exmp}[Dihedral presentation]
	Consider \(D_8 = \langle a,b \mid a^2 = b^{4} =1, aba = b^{-1} \rangle \). Any pair of elements \(a',b'\) that are of order 2 and 4 (and \(a'\) is noncentral) must satisfy the same relations. There are four noncentral elements of order 2, and two elements of order 4, so \(D_8\) has 8 automorphisms.
\end{exmp}
Similarly, any distinct pair of elements of order \(4\) in \(Q_8\) that are not inverses of each other necessarily generate \(Q_8\) and satisfy its relations. There are \(24\) such pairs, so \(\left| \textrm{Aut}(Q_8) \right| =24\). As one can see, free groups are an incredibly useful tool to classify these maps.

% \printindex
\end{document}
