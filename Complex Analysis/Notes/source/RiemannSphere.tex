\documentclass{memoir}
\usepackage{notestemplate}

%\logo{~/School-Work/Auxiliary-Files/resources/png/logo.png}
%\institute{Rice University}
%\faculty{Faculty of Whatever Sciences}
%\department{Department of Mathematics}
%\title{Class Notes}
%\subtitle{Based on MATH xxx}
%\author{\textit{Author}\\Gabriel \textsc{Gress}}
%\supervisor{Linus \textsc{Torvalds}}
%\context{Well, I was bored...}
%\date{\today}

%\makeindex

\begin{document}

% \maketitle

% Notes taken on 

\subsection{Riemann Sphere}
\label{sub:riemann_sphere}


%% ADD RIEMANN SPHERE


\subsection{Linear Transformations}
\label{sub:linear_transformations}

\begin{defn}[Linear Fractional Transformation]
	A \textbf{linear fractional transformation} or \textbf{Möbius transformation} is a linear transformation of the complex plane given by
	\begin{align*}
		S(z) := \frac{az+b}{cz+d}
	\end{align*}
	for complex numbers \(a,b,c,d \in \C\) with one of \(c,d\) non-zero.\\

	We extend \(S\) to \(\overline{\C}\) by defining the following values:
\begin{align*}
	S(\infty) = \begin{cases}
		\sfrac{a}{c} & c\neq 0\\
		\infty & c=0
	\end{cases}\\
	S(-\sfrac{d}{c}) = \infty \text{ if }c\neq 0.
\end{align*}
	This extends the mapping so that \(S\) is a topological mapping of the extended plane onto itself, with the topology of distances on the Riemannian sphere.
\end{defn}
This is the generalized form of all linear transformations on \(\overline{\C}\)-- translations, reflections, rotations, and more can be represented as linear fractional transformations.

\begin{anki}
TARGET DECK
Complex Qual::Complex Analysis
START
MathJaxCloze
Text: A **linear fractional transformation** or **Möbius transformation** is a linear transformation of the complex plane given by
{{c1::\(\begin{align*}
        	S(z) := \frac{az+b}{cz+d}
        \end{align*}\)}}
for complex numbers \(a,b,c,d \in \C\) with one of \(c,d\) non-zero.

If we write \(z = \frac{z_1}{z_2}\), we can express \(S(z)\) as a matrix via
{{c1::\(\begin{align*}
		\begin{pmatrix} a & b \\ c & d \end{pmatrix} \begin{pmatrix} z_1 \\ z_2 \end{pmatrix} = \begin{pmatrix} w_1 \\ w_2 \end{pmatrix} 
	\end{align*}\)}} 
where \(w := \frac{w_1}{w_2} = S(z)\). If {{c1::\(ad-bc = 1\)}}, we say the linear fractional transformation is **normalized**.

Extra: We extend \(S\) to \(\overline{\C}\) by defining the following values:
\(\begin{align*}
  	S(\infty) = \begin{cases}
  		\sfrac{a}{c} & c\neq 0\\
  		\infty & c=0
  	\end{cases}\\
  	S(-\sfrac{d}{c}) = \infty \text{ if }c\neq 0.
  \end{align*}\)
This extends the mapping so that \(S\) is a topological mapping of the extended plane onto itself, with the topology of distances on the Riemannian sphere.
Tags: analysis complex_analysis complex_geometry defn
<!--ID: 1624827134946-->
END
\end{anki}

\begin{prop}
	Let \(S\) be a linear fractional transformation
	\begin{align*}
		S(z) = \frac{az+b}{c z + d}
	\end{align*}
	with \(ad-bc\neq 0\). Then
	\begin{align*}
		S:\overline{\C}\to \overline{\C}
	\end{align*}
	is an isomorphism. Furthermore, the inverse of \(S\) is given by
	\begin{align*}
		S^{-1}(z) = \frac{dz -b}{a-cz}.
	\end{align*}
	If \(ad-bc = 1\), then we say the linear fractional transformation is \textbf{normalized}.
\end{prop}

Every linear fractional transformation admits a normalized form. In fact, there are exactly two, which are obtained from each other by changing the sign of the coefficients.

\begin{anki}
START
MathJaxCloze
Text: Let \(S\) be a linear fractional transformation
\(\begin{align*}
  	S(z) = \frac{az+b}{c z + d}
\end{align*}\)
	with \(ad-bc\neq 0\). Then
{{c1::\(\begin{align*}
	S:\overline{\C}\to \overline{\C}
      \end{align*}\)}} 
	is {{c1::an isomorphism}}. Furthermore, the inverse of \(S\) is given by
{{c1::\(\begin{align*}
	S^{-1}(z) = \frac{dz -b}{a-cz}.
      \end{align*}\)}}
	If {{c1::\(ad-bc = 1\)}}, then we say the linear fractional transformation is \textbf{normalized}.
Extra: Every linear fractional transformation admits a normalized form. In fact, there are exactly two, which are obtained from each other by changing the sign of the coefficients.
Tags: analysis complex_analysis complex_geometry
<!--ID: 1626152754001-->
END
\end{anki}


\begin{general}[Matrix Formulation of Linear Fractional Transformations]
	Let \(z = \frac{z_1}{z_2}\) be an arbitrary complex number and
	\begin{align*}
		S(z) = \frac{az+b}{cz+d}.
	\end{align*}
	Then we can express this linear fractional translation by
	\begin{align*}
		\begin{pmatrix} a & b \\ c & d \end{pmatrix} \begin{pmatrix} z_1 \\ z_2 \end{pmatrix} = \begin{pmatrix} w_1 \\ w_2 \end{pmatrix} 
	\end{align*}
	where \(w := \frac{w_1}{w_2} = S(z)\). This representation as matrices is useful as it satisfies the standard matrix operations when it comes to addition, composition, inversion, and so on.\\

	Hence, the set of linear fractional transformations forms a (matrix) group. The identity transformation is given by the identity matrix. The rest of the group properties we leave to be checked by the reader.\\

	In fact, if we restrict ourselves to normalized representations, this matrix group is isomorphic to \(SL(2,\C)\). If furthermore we form an equivalence class on normalized representations, then the group is isomorphic to the \textbf{projective special linear group} \(PSL_2(\C)\).
\end{general}

If
\begin{align*}
	A = \begin{pmatrix} a & b \\ c & d \end{pmatrix} 
\end{align*}
then we may denote the linear fractional transformation
\begin{align*}
	S(z) = \frac{az+b}{cz+d}
\end{align*}
simply by \(S_A\).

\begin{prop}[Equivalence of Fractional Linear Transformations]
	Let \(A,A' \in \C^{2\times 2}\) with \(\textrm{det}(A)\neq 0\) and \(\textrm{det}(A') \neq 0\). Then
	\begin{align*}
		S_A = S_{A'} \iff A' = \lambda A
	\end{align*} 
	for some \(\lambda \in \C\) non-zero.
\end{prop}

\begin{anki}
START
MathJaxCloze
Text: **Equivalence of Fractional Linear Transformations**
Let \(A,A' \in \mathbb{C}^{2\times 2}\) with \(\textrm{det}(A)\neq 0\) and \(\textrm{det}(A') \neq 0\). Then
{{c1::\(\begin{align*}
        	S_A = S_{A'} \iff A' = \lambda A
        \end{align*} \)}}
	for some \(\lambda \in \C\) non-zero.
Tags: analysis complex_analysis complex_geometry
<!--ID: 1626152754018-->
END
\end{anki}


\begin{defn}[Important Classes of Linear Transformations]
	The linear fractional transformations of the form
	\begin{align*}
		T_\alpha (z) := \begin{pmatrix} 1 & \alpha \\ 0 & 1 \end{pmatrix} 
	\end{align*}
	for \(\alpha  \in \C\) are called \textbf{parallel translations}, as they correspond to the transformation \(S(z) = z + \alpha \).\\

	The linear fractional transformations of the form
	\begin{align*}
		R_\alpha (z) := \begin{pmatrix} \alpha  & 0 \\ 0 & 1 \end{pmatrix} 
	\end{align*}
	are referred to as \textbf{rotations} if \(\left| \alpha  \right| = 1\), or a \textbf{homothetic transformation} if \(\alpha \) is real with \(\alpha >0\). For arbitrary complex \(\alpha \neq 0\), we can write
	\begin{align*}
		\alpha  = \left| \alpha  \right| \frac{\alpha }{\left| \alpha  \right| }
	\end{align*}
	and hence the general form can be viewed as a composition of a homothetic transformation with a rotation.\\

	The linear fractional transformation
	\begin{align*}
	(z)^{-1} := \begin{pmatrix} 0 & 1 \\ 1 & 0 \end{pmatrix} 
	\end{align*}
	is referred to as an \textbf{inversion}, as it represents \(S(z) = \frac{1}{z}\).
\end{defn}
In some sense, we can view these classes of translations as the fundamental linear transformations. If \(S(z)\) is given by
\begin{align*}
	\frac{az+b}{cz+d}
\end{align*}
with \(c\neq 0\), then we can write
\begin{align*}
	S(z) &= \frac{az+b}{c z+d} = \frac{bc-ad}{c^2(z+\frac{d}{c})}+ \frac{a}{c}\\
	&=  T_{\sfrac{a}{c}}\circ R_{\frac{bc-ad}{c^2}}\circ (\cdot)^{-1} \circ T_{\sfrac{d}{c}}
\end{align*}
In the simpler case where \(c=0\), we have
\begin{align*}
	S(z) &= \frac{az+b}{d}\\
	&= T_{\sfrac{b}{d}}\circ R_{\sfrac{a}{d}}
\end{align*}

\begin{anki}
START
MathJaxCloze
Text: The linear fractional transformations of the form
 {{c1::\(\begin{align*}
         	T_\alpha (z) := \begin{pmatrix} 1 & \alpha \\ 0 & 1 \end{pmatrix} 
         \end{align*}\)}} 
for \(\alpha  \in \C\) are called **parallel translations**, as they correspond to the transformation {{c1::\(S(z) = z + \alpha \)}}.

The linear fractional transformations of the form
 {{c2::\(\begin{align*}
         	R_\alpha (z) := \begin{pmatrix} \alpha  & 0 \\ 0 & 1 \end{pmatrix} 
         \end{align*}\)}}
are referred to as **rotations** if {{c2::\(\left| \alpha  \right| = 1\)}}, or a **homothetic transformation** if {{c2::\(\alpha \) is real with \(\alpha >0\)}}. For arbitrary complex \(\alpha \neq 0\), we can write
 {{c2::\(\begin{align*}
        	\alpha  = \left| \alpha  \right| \frac{\alpha }{\left| \alpha  \right| }
        \end{align*}\)}} 
and hence the general form can be viewed as {{c2::a composition of a homothetic transformation with a rotation}}.

The linear fractional transformation
 {{c3::\(\begin{align*}
        (z)^{-1} := \begin{pmatrix} 0 & 1 \\ 1 & 0 \end{pmatrix} 
        \end{align*}\)}} 
is referred to as an **inversion**, as it represents {{c3::\(S(z) = \frac{1}{z}\)}} .
Extra: If \(S(z)\) is given by
\(\begin{align*}
  	\frac{az+b}{cz+d}
  \end{align*}\)
with \(c\neq 0\), then we can write
\(\begin{align*}
  	S(z) &= \frac{az+b}{c z+d} = \frac{bc-ad}{c^2(z+\frac{d}{c})}+ \frac{a}{c}\\
  	&=  T_{\sfrac{a}{c}}\circ R_{\frac{bc-ad}{c^2}}\circ (\cdot)^{-1} \circ T_{\sfrac{d}{c}}
  \end{align*}\)
In the simpler case where \(c=0\), we have
\(\begin{align*}
  	S(z) &= \frac{az+b}{d}\\
  	&= T_{\sfrac{b}{d}}\circ R_{\sfrac{a}{d}}
  \end{align*}\)
Tags: analysis complex_analysis complex_geometry defn
<!--ID: 1624827135038-->
END
\end{anki}

\begin{thm}
	A fractional linear transformation maps straight lines and circles onto straight lines and circles.
\end{thm}
\begin{proof}
	
\end{proof}

\begin{anki}
START
MathJaxCloze
Text: A fractional linear transformation \(S(z) = \frac{az+b}{c z + d}\), \(ac-bd\neq 0\) maps straight lines and circles onto {{c1::straight lines and circles}}.
Tags: analysis complex_analysis complex_geometry
<!--ID: 1626152754035-->
END
\end{anki}

\begin{lemma}
	Let \(S\) be a fractional linear map. If \(S(\infty) = \infty\), then
	\begin{align*}
		S(z) = az+b
	\end{align*}
	for \(a,b \in \C\).
\end{lemma}

\begin{anki}
START
MathJaxCloze
Text: Let \(S\) be a fractional linear map. If \(S(\infty) = \infty\), then
{{c1::\(\begin{align*}
        	S(z) = az+b
        \end{align*}\)
      	for \(a,b \in \C\)}}.
Tags: analysis complex_analysis complex_geometry
<!--ID: 1626152754052-->
END
\end{anki}


\begin{thm}
	Given any three distinct points \(\left\{ z_1,z_2,z_3 \right\} \) with \(z_i \in \overline{\C}\) and three distinct points \(\left\{ w_1,w_2,w_3 \right\} \) with \(w_i \in \overline{\C}\), there exists a unique fractional linear map \(S\) with
	\begin{align*}
		S(z_i) = w_i.
	\end{align*}
	If \(\left\{ w_1,w_2,w_3 \right\} = \left\{ 0,\infty,1 \right\} \) then the map is given by
	\begin{align*}
		S(z) = \frac{z-z_1}{z-z_2}\cdot \frac{z_3-z_2}{z_3-z_1}.
	\end{align*}
	We can apply this to the general case to get the relation
	\begin{align*}
		\frac{w-w_1}{w-w_2}\cdot \frac{w_3-w_2}{w_3-w_1} = \frac{z-z_1}{z-z_2}\cdot \frac{z_3-z_2}{z_3-z_1}.
	\end{align*}
	We can use this relation to derive an explicit formula for specific values.
\end{thm}

\begin{lemma}
	If \(S\) is a fractional linear map with three fixed points, then \(S\) is the identity map.
\end{lemma}

\begin{proof}
	
\end{proof}

\begin{anki}
START
MathJaxCloze
Text: Given any three distinct points \(\left\{ z_1,z_2,z_3 \right\} \) with \(z_i \in \overline{\C}\) and three distinct points \(\left\{ w_1,w_2,w_3 \right\} \) with \(w_i \in \overline{\C}\), there exists a {{c1::unique fractional linear map \(S\)::map}} with
{{c1::\(\begin{align*}
        	S(z_i) = w_i.
        \end{align*}\)}}
If \(\left\{ w_1,w_2,w_3 \right\} = \left\{ 0,\infty,1 \right\} \) then the map is given by
 {{c1::\(\begin{align*}
        	S(z) = \frac{z-z_1}{z-z_2}\cdot \frac{z_3-z_2}{z_3-z_1}.
        \end{align*}\)}}
We can apply this to the general case to get the relation
 {{c1::\(\begin{align*}
        	\frac{w-w_1}{w-w_2}\cdot \frac{w_3-w_2}{w_3-w_1} = \frac{z-z_1}{z-z_2}\cdot \frac{z_3-z_2}{z_3-z_1}.
        \end{align*}\)}} 
We can use this relation to derive an explicit formula for specific values.
Extra: If \(S\) is a fractional linear map with three fixed points, then \(S\) is the identity map.
Tags: analysis complex_analysis complex_geometry
<!--ID: 1626152754068-->
END
\end{anki}


%% Lots more details can be filled here, TBD based on later content


% \printindex
\end{document}
