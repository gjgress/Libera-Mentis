\documentclass{memoir}
\usepackage{notestemplate}

%\logo{~/School-Work/Auxiliary-Files/resources/png/logo.png}
%\institute{Rice University}
%\faculty{Faculty of Whatever Sciences}
%\department{Department of Mathematics}
%\title{Class Notes}
%\subtitle{Based on MATH xxx}
%\author{\textit{Author}\\Gabriel \textsc{Gress}}
%\supervisor{Linus \textsc{Torvalds}}
%\context{Well, I was bored...}
%\date{\today}

%\makeindex

\begin{document}

% \maketitle

% Notes taken on 07/03/21

Before we explicitly state Cauchy's integration formulas, it will be beneficial to look at a particular class of curves. This will give us important intuition when working more generally with Cauchy's integration formulas later.

\subsection{Winding Number}
\label{sub:winding_number}

Recall that if \(\gamma \) is the boundary of the unit disc, then
\begin{align*}
	\int_{\gamma } \frac{1}{z} \,d t = 2\pi i. 
\end{align*}
When we initially discussed this notion, we implicitly took that \(\gamma \) was simple. Instead, consider the situation where \(\gamma(t) = e^{2it}\) on the interval \([0,2\pi ]\). Then one can check that
\begin{align*}
	\int_\gamma  \frac{1}{z}\,d t = 4\pi i.
\end{align*}
In fact, it happens that if \(\gamma(t) = e^{k(it)}\) for \(k \in \Z\), then
\begin{align*}
	\int_\gamma \frac{1}{z}\,d t = (2\pi i)k.
\end{align*}

We refer to \(k\) as the \textbf{winding number of \(\gamma \) around zero}. Now we construct this notion more generally.

\begin{lemma}
	Let \(\gamma \) be a piecewise smooth closed curve, and assume that \(\gamma (t)\neq z_0\) for all \(t \in [t_0,t_1]\). Then
	\begin{align*}
		\int_{\gamma } \frac{1}{z-z_0}\,d t = k(2\pi i) 
	\end{align*}
	for some \(k \in \Z\).
\end{lemma}
\begin{proof}
	
\end{proof}

\begin{defn}[Winding Number]
	The \textbf{winding number of \(z_0\) with respect to \(\gamma \)}, also called the \textbf{index of \(z_0\) with respect to \(\gamma \)} is defined by
	\begin{align*}
		n(\gamma ,z_0) = \frac{1}{2\pi i} \int_\gamma \frac{1}{z-z_0}\,d t.
	\end{align*}
\end{defn}

\begin{anki}
TARGET DECK
Complex Qual::Complex Analysis
START
MathJaxCloze
Text: The **winding number of \(z_0\) with respect to \(\gamma \)**, also called the **index of \(z_0\) with respect to \(\gamma \)** is defined by
 {{c1::\(\begin{align*}
         	n(\gamma ,z_0) = \frac{1}{2\pi i} \int_\gamma \frac{1}{z-z_0}\,d t.
         \end{align*}\)}}
Extra: This follows from the fact that if \(\gamma(t) = e^{k(it)}\) for \(k \in \Z\), then
\(\begin{align*}
  	\int_\gamma \frac{1}{z}\,d t = (2\pi i)k.
  \end{align*}\)
Tags: analysis complex_analysis complex_integration defn
<!--ID: 1625372340718-->
END
\end{anki}


\begin{prop}[Properties of Winding Numbers]
	\begin{itemize}
		\item \(n(-\gamma ,z_0) = -n(\gamma,z_0)\) 
		\item If \(z_0\) lies on the exterior of \(\gamma \), then \(n(\gamma ,z_0) = 0\)
		\item \(n(\gamma ,z_0)=n(\gamma ,z_1)\) if there exists a path between \(z_0\) and \(z_1\) that does not intersect \(\gamma \) 
	\end{itemize}
\end{prop}

In fact, we can define the notion of winding number and obtain the first two properties even if \(\gamma \) is only continuous. This goes a long way to classifying curves as a whole, and in fact is useful in proving the Jordan curve theorem.

\begin{anki}
START
MathJaxCloze
Text: Let \(\gamma \) be a piecewise smooth curve and \(z_0\) an arbitrary point in \(\C\).

* {{c1::\(n(-\gamma ,z_0) = -n(\gamma,z_0)\)::reverse of curve}}  
* If \(z_0\) {{c2::lies on the exterior of \(\gamma \)}}, then \(n(\gamma ,z_0) = 0\)
* \(n(\gamma ,z_0)=n(\gamma ,z_1)\) if there {{c3::exists a path between \(z_0\) and \(z_1\) that does not intersect \(\gamma \)}} 
Extra: The first two properties hold even if \(\gamma \) is only continuous (instead of piecewise-smooth)
Tags: analysis complex_analysis complex_integration
<!--ID: 1625372340727-->
END
\end{anki}


\begin{exmp}[Jordan Curve Theorem]
	The Jordan curve theorem states that every Jordan curve in the plane determines exactly two regions. We will prove via winding numbers that the complement of a Jordan curve \(\gamma \) has at least two components. Observe that this holds if there exists a point \(z_0\) such that \(n(\gamma ,z_0)\neq 0\).\\

	Without loss of generality, assume that \(\textrm{Re}(\gamma )>0\) and furthermore that there exist points \(z_1,z_2 \in \gamma (t)\) so that \(\textrm{Im}(z_1)<0\) and \(\textrm{Im}(z_2)>0\). In fact, we can choose \(z_1,z_2\) so that the line segments \(\ell(0,z_1)\) and \(\ell(0,z_2)\) do not intersect \(\gamma \).\\

	Let \(\gamma_1\) be the subcurve between \(z_1,z_2\), and \(\gamma_2\) the subcurve that takes \(z_2\) to \(z_1\) along \(\gamma \) (so that \(\gamma = \gamma_1+\gamma_2\)). Let \(\sigma_1\) be the curve formed by following \(\ell(0,z_1) + \gamma_1 + \ell(0,z_2)^{^{-}}\), and let \(\sigma_2\) be the curve formed by following \(\ell(0,z_1) + \gamma_2^{-} + \ell(0,z_2)^{-}\) (so that \(\sigma_1-\sigma_2 = \gamma \)).\\

	Observe that the positive real axis intersects both \(\gamma_1,\gamma_2\) at points which we will refer to by \(x_1,x_2\) respectively. Then we show that:
	\begin{itemize}
		\item \(n(\sigma_1,x_2) = 0\) and hence \(n(\sigma_1,\gamma_2) = 0\) 
		\item \(n(\sigma_1,z) = n(\sigma_2,z) = 1\) for \(0<z<x_1\)
		\item \(n(\sigma_2,z_1) = 1\) and hence \(n(\sigma_2,\gamma_1) = 1\) 
	\end{itemize}
\end{exmp}

\subsection{Cauchy's Integral Formulas}
\label{sub:cauchy_s_integral_formulas}

\begin{thm}[Cauchy's Integral Formula]
	Let \(\Omega \subset \C\) be an open region and \(f\) a holomorphic function on \(\Omega \). Let \(\gamma \) be a closed curve in \(\Omega \) homologous to a point. Then for all \(z \in \Omega \) with \(n(\gamma ,z)\neq 0\),
	\begin{align*}
		f(z) = \frac{1}{n(\gamma,z)2\pi i} \int_\gamma \frac{f(\xi )}{\xi -z}\,d t.
	\end{align*}
	If we restrict our \(z\) further into a connected region with \(n(\gamma ,z)=1\), then we obtain Cauchy's integral formula:
	\begin{align*}
		f(z) = \frac{1}{2\pi i} \int_\gamma \frac{f(\xi )}{\xi -z}\,d t
	\end{align*}
	and hence defines a holomorphic function equal to \(f\) on this region.
\end{thm}
Note that we are implicitly requiring that \(z\) not be on the curve for \(\gamma \), as the winding number is not well-defined on the curve.

\begin{proof}% Ahlfors p.118
	
\end{proof}

\begin{anki}
START
MathJaxCloze
Text: Let \(\Omega \subset \C\) be an open region and \(f\) a holomorphic function on \(\Omega \). Let \(\gamma \) be a closed curve in \(\Omega \) homologous to a point. Then for all \(z \in \Omega \) with \(n(\gamma ,z)\neq 0\),
{{c1::\(\begin{align*}
f(z) = \frac{1}{n(\gamma,z)2\pi i} \int_\gamma \frac{f(\xi )}{\xi -z}\,d t.
\end{align*}\)}} 
If we restrict our \(z\) further into a connected region with \(n(\gamma ,z)=1\), then we obtain **Cauchy's integral formula**:
{{c1::\(\begin{align*}
	f(z) = \frac{1}{2\pi i} \int_\gamma \frac{f(\xi )}{\xi -z}\,d t
\end{align*}\)}}
and hence defines a holomorphic function equal to \(f\) on this region.
Extra: Note that we are implicitly requiring that \(z\) not be on the curve for \(\gamma \), as the winding number is not well-defined on the curve.
Tags: analysis complex_analysis complex_integration
<!--ID: 1625525402615-->
END
\end{anki}

This allows us to get infinite differentiability of holomorphic functions.

\begin{cor}
	Let \(f\) be a holomorphic function on an open set \(\Omega \) containing an open disc \(D_r(z_0)\subset \Omega \) for some \(z_0 \in \Omega \), \(r>0\), and let \(\gamma \) be a closed curve in \(D_r(z_0)\). Then \(f\) has infinitely many complex derivatives in \(\Omega \). Furthermore, for all \(z \in D_r(z_0)\), we have
	\begin{align*}
		f^{(n)}(z) = \frac{n!}{n(\gamma,z)2\pi i} \int_\gamma \frac{f(\xi )}{(\xi  -z)^{n+1}}\,d t
	\end{align*}
\end{cor}

Notice that because analytic functions are holomorphic, this also shows infinite complex differentiability of analytic functions. All that remains is to show that holomorphic functions are analytic.

\begin{anki}
START
MathJaxCloze
Text: Let \(f\) be a holomorphic function on an open set \(\Omega \) containing an open disc \(D_r(z_0)\subset \Omega \) for some \(z_0 \in \Omega \), \(r>0\), and let \(\gamma \) be a closed curve in \(D_r(z_0)\). Then \(f\) has infinitely many complex derivatives in \(\Omega \). Furthermore, for all \(z \in D_r(z_0)\), we have
 {{c1::\(\begin{align*}
        	f^{(n)}(z) = \frac{n!}{n(\gamma,z)2\pi i} \int_\gamma \frac{f(\xi )}{(\xi  -z)^{n+1}}\,d t
        \end{align*}\)::value of higher derivatives}} 
Extra: Notice that because analytic functions are holomorphic, this also shows infinite complex differentiability of analytic functions.
Observe that this also gives us uniform convergence on the higher order derivatives by simply taking the modulus of both sides:
\(\begin{align*}
  	\left| f^{(n)}(z_0) \right| \leq \frac{\left| f(z) \right|_{z \in \partial D_r(z_0)} n!}{r^{n}}.
  \end{align*}\)
Tags: analysis complex_analysis complex_integration
<!--ID: 1625525402632-->
END
\end{anki}


\begin{proof}
	
\end{proof}

Observe that this also gives us uniform convergence on the higher order derivatives by simply taking the modulus of both sides:
\begin{align*}
	\left| f^{(n)}(z_0) \right| \leq \frac{\left| f(z) \right|_{z \in \partial D_r(z_0)} n!}{r^{n}}.
\end{align*}

%\begin{cor}[Cauchy inequalities]
%       If \(f\) is holomorphic in an open set that contains the closure of a disc \(D\) centered at \(z_0\) of radius \(R\), then
%       \begin{align*}
%       	\left| f^{(n)}(z_0) \right| \leq \frac{n! \|f\|_C}{R^{n}}
%       \end{align*}
%       where \(\|f\|_C = \sup_{z \in C} \left| f(z) \right| \) denotes the supremum of \(\left| f \right| \) on the boundary of \(C\).
%\end{cor}

\begin{thm}[Taylor's Theorem]
	Suppose \(f\) is holomorphic in an open set \(\Omega \subset \C\). Let \(D_r(z_0)\) be a disc with radius \(r>0\) with \(\overline{D_r(z_0)}\subset \Omega \). Then \(f\) has a power series expansion at \(z_0\) :
	\begin{align*}
		f(z) = \sum_{n=0}^{\infty} a_n(z-z_0)^{n}\\
		a_n := \frac{f^{(n)}(z_0)}{n!} \text{ for all } n\geq 0
	\end{align*}
	which converges absolutely and uniformly for all \(z \in D_r(z_0)\).
\end{thm}
This finally gives us the equivalence between holomorphic and analytic functions.\\

The power series expansion of holomorphic functions converges uniformly, and hence this theorem is a stronger version of Taylor's theorem for complex functions. That is, a holomorphic function can be approximated by partial sums of the power series expansion with a remainder that limits to zero at \(z_0\).

\begin{proof}
	
\end{proof}

\begin{anki}
START
MathJaxCloze
Text: **Taylor's Theorem**
Suppose \(f\) is holomorphic in an open set \(\Omega \subset \C\). Let \(D_r(z_0)\) be a disc with radius \(r>0\) with \(\overline{D_r(z_0)}\subset \Omega \). Then \(f\) has a power series expansion at \(z_0\) :
 {{c1::\(\begin{align*}
         	f(z) = \sum_{n=0}^{\infty} a_n(z-z_0)^{n}\\
         	a_n := \frac{f^{(n)}(z_0)}{n!} \text{ for all } n\geq 0
         \end{align*}\)}}
which {{c1::converges absolutely and uniformly}}  for all \(z \in D_r(z_0)\).
Extra: The power series expansion of holomorphic functions converges uniformly, and hence this theorem is a stronger version of Taylor's theorem for complex functions. That is, a holomorphic function can be approximated by partial sums of the power series expansion with a remainder that limits to zero at \(z_0\).
Tags: analysis complex_analysis complex_integration
<!--ID: 1625525402654-->
END
\end{anki}

In fact, this isn't the strongest form we can get of Taylor's theorem. Utilizing our earlier construction of Laurent series allows us to obtain an equivalence within an annulus.
\begin{thm}
	Let \(\Omega \subset \C\) be an annulus centered at \(z_0\) with inner radius \(r\) and outer radius \(R\) satisfying \(0\leq r<R\). Suppose \(f\) is holomorphic on \(\Omega \)-- then for all \(s,S\) satisfying \(r<s<S<R\), \(f\) has a Laurent expansion
	\begin{align*}
		f(z) = \sum_{n=-\infty}^{\infty} a_n (z-z_0)^{n}\\
		a_n = \begin{cases}
			\frac{1}{2\pi i} \int_{\partial D_R(z_0)} \frac{f(\xi )}{(\xi -z_0)^{n+1}}\,d t & n\geq 0\\
			\frac{1}{2\pi i} \int_{\partial D_r(z_0)} \frac{f(\xi )}{(\xi -z_0)^{n+1}}\,d t & n< 0\\
		\end{cases}
	\end{align*}
	which converges absolutely and uniformly on \(s\leq \left| z \right| \leq S\).
\end{thm}
Thus, if \(f\) is holomorphic on \(D_r(z_0)\setminus\left\{ z_0 \right\} \) for some \(r>0\), then the theorem implies that \(f\) has a unique Laurent expansion in \(D_r(z_0)\). This Laurent expansion is not holomorphic on \(D_r(z_0)\), but is holomorphic on \(D_r(z_0)\setminus\left\{ z_0 \right\} \). This will be useful later when dealing with meromorphic functions.

\begin{anki}
START
MathJaxCloze
Text: Let \(\Omega \subset \C\) be an annulus centered at \(z_0\) with inner radius \(r\) and outer radius \(R\) satisfying \(0\leq r<R\). Suppose \(f\) is holomorphic on \(\Omega \)-- then for all \(s,S\) satisfying \(r<s<S<R\), \(f\) has a Laurent expansion
 {{c1::\(\begin{align*}
         	f(z) = \sum_{n=-\infty}^{\infty} a_n (z-z_0)^{n}\\
         	a_n = \begin{cases}
         		\frac{1}{2\pi i} \int_{\partial D_R(z_0)} \frac{f(\xi )}{(\xi -z_0)^{n+1}}\,d t & n\geq 0\\
         		\frac{1}{2\pi i} \int_{\partial D_r(z_0)} \frac{f(\xi )}{(\xi -z_0)^{n+1}}\,d t & n< 0\\
         	\end{cases}
         \end{align*}\)}} 
	which {{c1::converges absolutely and uniformly}} on \(s\leq \left| z \right| \leq S\).
Extra: Thus, if \(f\) is holomorphic on \(D_r(z_0)\setminus\left\{ z_0 \right\} \) for some \(r>0\), then the theorem implies that \(f\) has a unique Laurent expansion in \(D_r(z_0)\). This Laurent expansion is not holomorphic on \(D_r(z_0)\), but is holomorphic on \(D_r(z_0)\setminus\left\{ z_0 \right\} \).
Tags: analysis complex_analysis complex_integration
<!--ID: 1625525402672-->
END
\end{anki}


% \printindex
\end{document}
