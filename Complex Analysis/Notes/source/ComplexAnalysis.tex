\documentclass[oneside]{memoir}

\usepackage{notestemplate}

\logo{~/LibreMath/Auxiliary Resources/resources/png/logo.png}
\institute{Rice University}
%\faculty{Faculty of Whatever Sciences}
\department{Department of Mathematics}
\title{Introduction to Complex Analysis}
%\subtitle{Based on MATH xxx}
\author{\textit{Author}\\Gabriel \textsc{Gress}}
%\supervisor{Linus \textsc{Torvalds}}
%\context{Well, I was bored...}
\date{\today}


\begin{document}

\maketitle

\tableofcontents

\setcounter{chapter}{-1}

\chapter{Preamble}
\label{cha:preamble}

\documentclass{memoir}
\usepackage{notestemplate}

%\logo{./resources/pdf/logo.pdf}
%\institute{Rice University}
%\faculty{Faculty of Whatever Sciences}
%\department{Department of Mathematics}
%\title{Class Notes}
%\subtitle{Based on MATH xxx}
%\author{\textit{Author}\\Gabriel \textsc{Gress}}
%\supervisor{Linus \textsc{Torvalds}}
%\context{Well, I was bored...}
%\date{\today}

\begin{document}

% \maketitle

% Notes take on 01/25/21

\chapter{Introduction}
\label{cha:introduction}

These notes covers modules, unital and commutative rings, and fields.\\

The lecture notes are based off two main sources. The overall outline and the major statements of theorems and definitions are based off lecture notes from Dr.\@ Chelsea Walton during the Spring 2021 teaching of Rice's course MATH 357 -- \textit{Abstract Algebra II}. These notes are supplemented by exercises from Dummitt and Foote's \textit{Abstract Algebra}, and some of the basic ring material is based on Goodman's \textit{Algebra: Abstract and Concrete}. Finally, the applications of Galois theory are based on a section from Hungerford's \textit{Algebra}.

\end{document}


\chapter{The Complex Field}
\label{cha:the_complex_field}

\section{Arithmetic of \(\C\)}
\label{sec:arithmetic_and_geometry_of_c}

\documentclass{memoir}
\usepackage{notestemplate}

% \begin{figure}[ht]
%     \centering
%     \incfig{riemmans-theorem}
%     \caption{Riemmans theorem}
%     \label{fig:riemmans-theorem}
% \end{figure}

\begin{document}
%\section{Prototypical Examples}
%\label{sec:prototypical_examples}
%Let us consider two analytic functions on \(\R\) :
%\begin{itemize}
%	\item \(f(x) = \frac{1}{1+x^2}\)
%	\item \(g(x) = e^{-x^2}\)
%\end{itemize}
%Their graphs look similar, but they are very different. Consider their Taylor series:
%\begin{align*}
%	f(x) = 1-x^2+x^3-x^{6}+\ldots
%\end{align*}
%and
%\begin{align*}
%	g(x) = 1-x^2+\frac{x^{4}}{2!}-\frac{x^{6}}{3!}+\ldots
%\end{align*}
%Observe that the Taylor Series of \(f(x)\) only converges for \(|x|<1\), due to the denominators. However, the Taylor Seris of \(g(x)\) converges for all \(x \in \R\).\\
%The reason \(f(x)\) has issues is the complex plane. \(g(x)\) is well-defined on the complex plane, but \(f(x)\) has holes at \(i,-i\).\\
%
%Now consider \(h(x) = \begin{cases}
%	e^{\frac{-1}{x^2}}, & x\neq 0\\
%	0, & x=0
%\end{cases}\)\\
%Once again, on the real line it looks fine, and seems perfectly differentiable. But this cannot be extended to the complex plane because as you approach zero from the imaginary axis, the function blows up.
%

\subsection{Formal Definition}
\label{sub:formal_definition}

First, we construct the field \(\C\) so that the addition and multiplication operations correspond to natural geometric transformations.

\begin{defn}[Complex Field]
	We define the \textbf{complex field} \(\C\) to be the field of ordered pairs
	\begin{align*}
		(\alpha ,\beta )
	\end{align*}
	where \(\alpha ,\beta  \in \R\). The addition and multiplication operations on \(\C\) are defined by
	\begin{align*}
		(\alpha_1,\beta_1) + (\alpha_2,\beta_2) &= (\alpha_1+\alpha_2,\beta_1+\beta_2)\\
		(\alpha_1,\beta_1)\cdot (\alpha_2,\beta_2) &= (\alpha_1\beta_1 - \alpha_2\beta_2, \alpha_1\beta_2+\alpha_2\beta_1)
	\end{align*}
	Elements of \(\C\) are called \textbf{complex numbers}. We refer to the first element of the ordered pair as the \textbf{real part} (denoted \(\textrm{Re}(\alpha ,\beta )\)), and the second element as the \textbf{imaginary part} (denoted \(\textrm{Im}(\alpha ,\beta )\) of the complex number.\\

	The identities of \(\C\) are hence
	\begin{align*}
		0_{\C} &= (0_\R,0_\R)\\
		1_\C &= (1_\R,0_\R)
	\end{align*}
	and the inverses of an element are given by
	\begin{align*}
		-(\alpha,\beta ) &= (-\alpha ,-\beta )\\
		(\alpha ,\beta)^{-1} &= \left( \frac{\alpha }{\alpha^2+\beta^2}, \frac{-\beta }{\alpha^2+\beta^2} \right) 
	\end{align*}
\end{defn}

\begin{anki}
TARGET DECK
Complex Qual::Complex Analysis
START
MathJaxCloze
Text: We define the **complex field** \(\C\) to be the field of ordered pairs
\(\begin{align*}
  	(\alpha ,\beta )
  \end{align*}\)
where \(\alpha ,\beta  \in \R\). The addition and multiplication operations on \(\C\) are defined by
{{c1::\(\begin{align*}
        	(\alpha_1,\beta_1) + (\alpha_2,\beta_2) &= (\alpha_1+\alpha_2,\beta_1+\beta_2)\\
        	(\alpha_1,\beta_1)\cdot (\alpha_2,\beta_2) &= (\alpha_1\beta_1 - \alpha_2\beta_2, \alpha_1\beta_2+\alpha_2\beta_1)
        \end{align*}\)}}
Elements of \(\C\) are called **complex numbers**. We refer to the first element of the ordered pair as the **real part**, and the second element as the **imaginary part** of the complex number.

The identities of \(\C\) are hence
\(\begin{align*}
  	0_{\C} &= (0_\R,0_\R)\\
  	1_\C &= (1_\R,0_\R)
  \end{align*}\)
and the inverses of an element are given by
 {{c2::\(\begin{align*}
        	-(\alpha,\beta ) &= (-\alpha ,-\beta )\\
        	(\alpha ,\beta)^{-1} &= \left( \frac{\alpha }{\alpha^2+\beta^2}, \frac{-\beta }{\alpha^2+\beta^2} \right) 
        \end{align*}\)}} 
Extra: Sometimes, we might write a complex number \((\alpha ,\beta )\) as the sum
\(\begin{align*}
	\alpha + i\beta .
\end{align*}\)
Tags: analysis complex_analysis complex_numbers defn
<!--ID: 1624142599825-->
END
\end{anki}

\begin{hw}
	Check that the identities and inverses above indeed satisfy the necessary conditions to be identities and inverses of a field. Furthermore, verify that the addition and multiplication operations are associative, commutative, and satisfy the distributive property.
\end{hw}

Sometimes, we might write a complex number \((\alpha ,\beta )\) as the sum
\begin{align*}
	\alpha + i\beta .
\end{align*}
In other words, we choose to define
\begin{align*}
	i:= (0,1)
\end{align*}
and will often use \(i\) for shorthand in formulas.\\

If \(\beta=0\), we say a number is \textbf{real}, and if \(\alpha=0\), we say a number is \textbf{imaginary}. There is a natural isomorphism from the set of real complex numbers to the real numbers, as well as a natural isomorphism from the set of imaginary complex numbers to the real numbers.\\

Of course, this construction does not give an intuition for why the operations above are defined so. One should first observe that the definition gives a field structure to the set, unlike the standard operations of \(\R^2\). Later, we will also show that the field is algebraically closed, and is in fact the algebraic extension of \(\R\).\\

If we graph the real part of a complex number onto an \(x\)-axis of an \(xy\)-grid, and the imaginary part to the \(y\)-axis, then the addition and multiplication operations correspond to geometric translations:
\begin{itemize}
	\item Adding a fixed complex number to a complex variable is geometrically a translation:
		\begin{align*}
			(\alpha ,\beta ) + (x,y) = (\alpha +x,\beta +y)
		\end{align*}
		% Figure
	\item Multiplying a fixed positive number to a complex variable is geometrically a dilation:
	\begin{align*}
		a(x+yi) = ax + ayi
	\end{align*}
	% Figure
	\item Multiplying by a fixed complex number of length 1 and an argument or angle \(t\) is geomerically a rotation by \(t\):
		\begin{align*}
			e^{it} = \cos(t) + i \sin(t)
		\end{align*}
	% Figure
	This definition allows us to add angles when we multiply by \(e^{it'}\)
\end{itemize}
This geometric interpretation immediately gives us new intuition when working with complex numbers. For example, a natural question arises-- is there a polar form for complex numbers in a similar sense as \(\R^2\)?\\

In fact, every non-zero complex number has a unique decomposition into a product of a positive number and a number of unit length, given by \(z = re^{it}\). Then, in a formula, complex multiplcation is defined by
\begin{align*}
	re^{it} \cdot r'e^{it'} = (r\cdot r')e^{i(t+t')}
\end{align*}
% Figure

\begin{hw}
	Show that the square root of a complex number is given by
	\begin{align*}
		(\alpha ,\beta )^{\sfrac{1}{2}} = \pm \left( \sqrt{ \frac{\alpha + \sqrt{\alpha^2+\beta^2} }{2}} , \frac{\beta }{\left| \beta  \right| } \sqrt{ \frac{-\alpha + \sqrt{\alpha^2+\beta^2} }{2}} \right) .
	\end{align*}
	where the square roots of positive numbers are taken to be the positive root. Furthermore, use this to show that the square root of any complex number always exists and has two opposite values, coinciding only if \((\alpha ,\beta )=0 \)
\end{hw}

\begin{anki}
START
MathJaxCloze
Text: The square root of a complex number is given by
 {{c1::\(\begin{align*}
         	(\alpha ,\beta )^{\sfrac{1}{2}} = \pm \left( \sqrt{ \frac{\alpha + \sqrt{\alpha^2+\beta^2} }{2}} , \frac{\beta }{\left| \beta  \right| } \sqrt{ \frac{-\alpha + \sqrt{\alpha^2+\beta^2} }{2}} \right) .
         \end{align*}\)}}
where the square roots of positive numbers are taken to be the positive root. 
Extra: The square root of any complex number always exists and has two opposite values, coinciding only if \((\alpha ,\beta )=0 \)
Tags: analysis complex_analysis complex_numbers
<!--ID: 1624142599862-->
END
\end{anki}

This exercise has some subtle deep implications. Recall that in \(\R\), a polynomial equation may not have all its roots in \(\R\). However, in \(\C\) this holds-- hence if we apply the natural isomorphism on polynomials in \(\R\) to polynomials in \(\C\), the polynomials will have all its solutions in \(\C\). In fact, \(\C\) is the smallest field for which this isomorphism is possible, and hence \(\C\) is the algebraic completion of \(\R\).\\

There is much to be shown in order to demonstrate that the algebraic completion of \(\R\) corresponds to the construction of \(\C\) above. For brevity, this will be skipped, but for first-time readers, the author highly encourages one to read this construction in detail elsewhere.

\subsection{Conjugation in \(\C\)}
\label{sub:conjugation_in_c}

While there are infinitely many automorphisms of \(\C\) (of which we will discuss in detail later), there is a special automorphism that is worth discussing first.\\

\begin{defn}[Complex Conjugation]
	Let \((\alpha ,\beta ) \in \C\) be a complex number. Then we define the \textbf{complex conjugate of \((\alpha ,\beta )\)} to be the complex number given by the transformation:
	\begin{align*}
		\overline{(\alpha ,\beta )} = (\alpha ,-\beta )
	\end{align*}
	Furthermore, complex conjugation satisfies the following:
	\begin{align*}
		\overline{a+b} = \overline{a} + \overline{b}\\
		\overline{ab} = \overline{a}\cdot \overline{b}.
	\end{align*}
	Finally, complex conjugation is an \textbf{involuntary transformation}. That is, it satisfies
	\begin{align*}
		\overline{ \left( \overline{(\alpha ,\beta )} \right) } = (\alpha ,\beta )
	\end{align*}
\end{defn}
\begin{anki}
% Up to 5 consequences
START
Definition
Name: Complex Conjugate
Premise 1: Let \(a = (\alpha ,\beta ) \in \C\)
Consequence 1: The complex conjugate is \(\overline{a} := (\alpha ,-\beta )\)
Tags: analysis complex_analysis complex_numbers defn 
<!--ID: 1624237478645-->
END
\end{anki}

One can check that we can obtain formulas
\begin{align*}
	\textrm{Re}(a) = \frac{a + \overline{a}}{2}, \quad \textrm{Im}(a) = \frac{a - \overline{a}}{2}
\end{align*}

\begin{defn}[Modulus]
	Let \(a = (\alpha ,\beta ) \in \C\). The \textbf{absolute value} or \textbf{modulus} of \(a\) is defined as
	\begin{align*}
		\left| a \right| := (a \overline{a})^{\sfrac{1}{2}} = (\alpha^2+\beta^2)^{\sfrac{1}{2}}.
	\end{align*}
\end{defn}
	Notice that \(\left| a \right| \geq 0\) for all \(a \in \C\), and furthermore, \(\left| a \right| = 0\) if and only if \(a=0\). It also holds that \(\left| \overline{a} \right| = \left| a \right| \) and 
	\begin{align*}
		\left| ab \right| = \left| a \right| \cdot \left| b \right| 
	\end{align*}
	This has the properties of a norm, and hence for sums we have
	\begin{align*}
		\left| a + b \right|^2 &= \left| a \right|^2 + \left| b \right|^2 + 2 \textrm{Re}a \overline{b}\\
		\left| a-b \right|^2 &= \left| a \right|^2 + \left| b \right|^2 - 2 \textrm{Re} a \overline{b}\\
				     &\implies \left| a+b \right|^2 + \left| a-b \right|^2 = 2\left( \left| a \right|^2 + \left| b \right|^2 \right) 
	\end{align*}

\begin{anki}
START
MathJaxCloze
Text: Let \(a = (\alpha ,\beta ) \in \C\). The **absolute value** or **modulus** of \(a\) is defined as
{{c1::\(\begin{align*}
        	\left| a \right| := (a \overline{a})^{\sfrac{1}{2}} = (\alpha^2+\beta^2)^{\sfrac{1}{2}}.
        \end{align*}\)}} 
Extra: Notice that \(\left| a \right| \geq 0\) for all \(a \in \C\), and furthermore, \(\left| a \right| = 0\) if and only if \(a=0\). It also holds that \(\left| \overline{a} \right| = \left| a \right| \) and 
\(\begin{align*}
  	\left| ab \right| = \left| a \right| \cdot \left| b \right| 
  \end{align*}\)
	This has the properties of a norm, and hence for sums we have
	\(\begin{align*}
	  	\left| a + b \right|^2 &= \left| a \right|^2 + \left| b \right|^2 + 2 \textrm{Re}a \overline{b}\\
	  	\left| a-b \right|^2 &= \left| a \right|^2 + \left| b \right|^2 - 2 \textrm{Re} a \overline{b}\\
	  			     &\implies \left| a+b \right|^2 + \left| a-b \right|^2 = 2\left( \left| a \right|^2 + \left| b \right|^2 \right) 
	  \end{align*}\)
Tags: analysis complex_analysis complex_numbers defn
<!--ID: 1624237478679-->
END
\end{anki}


\begin{thm}[Triangle Inequality]
Let \(a,b \in \C\). Then
\begin{align*}
	\left| a + b \right| \leq \left| a \right| + \left| b \right|.
\end{align*}
The equation is an equality if and only if \(a \overline{b}\) is real and non-negative. In fact, this holds for extended finite sums if the ratio of any two nonzero terms is positive.
\end{thm}

\begin{anki}
% Up to 4 premises
% Up to 4 equivalences
START
Theorem
Name: Complex Triangle Inequality
Premise 1: Let \(a,b \in \C\)
Consequence 1: \(\left| a+b \right| \leq \left| a \right| + \left| b \right| \) 
Consequence 2: \(\left| a+b \right| = \left| a \right| + \left| b \right| \) if and only if \(a \overline{b}\) is real and non-negative
Tags: analyis complex_analysis complex_numbers
<!--ID: 1624237478709-->
END
\end{anki}

\begin{thm}[Cauchy's Inequality]
	Let \(a_i,b_i \in \C\) for \(i \in \left\{ 1,\ldots,n \right\} \). Then
	\begin{align*}
		\left| \sum_{i=1}^{n} a_ib_i \right|^2 \leq \sum_{i=1}^{n} \left| a_i \right|^2 \sum_{i=1}^{n} \left| b_i \right|^2.
	\end{align*}
\end{thm}

\begin{anki}
% Up to 4 premises
% Up to 4 equivalences
START
Theorem
Name: Complex Cauchy's Inequality
Premise 1: Let \(a_i, b_i \in \C\) for \(i \in \left\{ 1,\ldots,n \right\} \)
Consequence 1: \(\left| \sum_{i=1}^{n} a_ib_i \right|^2 \leq \sum_{i=1}^{n} \left| a_i \right|^2 \sum_{i=1}^{n} \left| b_i \right|^2\)
Tags: analysis complex_analysis complex_numbers
<!--ID: 1624237478740-->
END
\end{anki}


\end{document}


\section{Geometry in \(\C\)}
\label{sec:geometry_in_c}

\documentclass{memoir}
\usepackage{notestemplate}

% \begin{figure}[ht]
%     \centering
%     \incfig{riemmans-theorem}
%     \caption{Riemmans theorem}
%     \label{fig:riemmans-theorem}
% \end{figure}

\begin{document}

A lot of geometric theorems can be visualized in the complex plane, in which the complex operations are immensely useful for proving theorems.\\

It is important to establish that the complex plane indeed inherits the metric properties of \(\R^2\).

\begin{defn}[Scalar Product and Angles]
	Let \(z,w \in \C\). The \textbf{scalar product} of \(z\) and \(w\) is defined by
	\begin{align*}
		\langle z,w \rangle := \textrm{Re}(z \overline{w}).
	\end{align*}
	We define the \textbf{angle between \(z\) and \(w\)} to be
	\begin{align*}
		\theta (z,w):= \cos ^{-1} \left( \frac{\langle z, w \rangle }{\left| z \right| \left| w \right| } \right) .
	\end{align*}
\end{defn}
The angle formula is chosen so that
\begin{align*}
	\cos \theta (z,w) = \frac{\langle z,w \rangle }{\left| z \right| \left| w \right| }\\
	\sin\theta (z,w) = \frac{\langle z, -iw \rangle }{\left| z \right| \left| w \right| }.
\end{align*}

\begin{anki}
TARGET DECK
Complex Qual::Complex Analysis
START
MathJaxCloze
Text: Let \(z,w \in \C\). The **scalar product** of \(z\) and \(w\) is defined by
 {{c1::\(\begin{align*}
         	\langle z,w \rangle := \textrm{Re}(z \overline{w}).
         \end{align*}\)}} 
	We define the **angle between \(z\) and \(w\)** to be
	 {{c1::\(\begin{align*}
	         	\theta (z,w):= \cos ^{-1} \left( \frac{\langle z, w \rangle }{\left| z \right| \left| w \right| } \right) .
	         \end{align*}\)}} 
Extra: The angle formula is chosen so that
\(\begin{align*}
  	\cos \theta (z,w) = \frac{\langle z,w \rangle }{\left| z \right| \left| w \right| }\\
  	\sin\theta (z,w) = \frac{\langle z, -iw \rangle }{\left| z \right| \left| w \right| }.
  \end{align*}\)
Tags: analysis complex_analysis defn complex_geometry
<!--ID: 1624668023893-->
END
\end{anki}

\begin{hw}
	Show that the scalar product defined above indeed satisfies the necessary properties for a general scalar product.
\end{hw}

%%% Might need to relabel things

%\begin{thm}[Pythagorem Theorem]
%	In a right triangle, the square of the hypotenuse is equal to the sum of squares of the legs.By plotting the triangle in the complex plane, we can restate it as \(\left| x+yi \right|^2 = x^2+y^2\). Writing \(x+yi\) as the polar form, and writing \(x,y\) as \(r^2\) which is the product of complex conjugates, everything checks out.
%\end{thm}
%\begin{thm}[]
%	The blue triangle in \ref{fig:triangles} is an equilateral triangle.	
%\end{thm}
%
%\begin{figure}[ht]
%    \centering
%     \def\svgwidth{1\linewidth}
%     \input{./figures/triangles.pdf_tex}
%    \caption{Complex Pythagorem Theorem}
%    \label{fig:triangles}
%\end{figure}
%
%Let \(a,b,c\) of the original triangle, and let \(a',b',c'\) be the vertices of the blue triangle. Then our goal is to do the mult. and get the side lengths we expect.
%\begin{thm}
%	Consider a regular \(n\)-gon in the unit circle. From a vertex, draw a line segment to the other vertices. The product of the lengths of these \(n-1\) line segments is \(n\).
%\end{thm}
%\begin{proof}
%WLOG, the vertices are the \(n\)-th roots of unity: \(w^{k}\) (\(k = 0,\ldots,n-1\) ) where \(w = e^{2\pi \frac{i}{n}}\). Also WLOG, the vertex we start with is \(1\). Then we want to show that \(\prod_{k=1}^{n-1} |1-w^{k}| = n \). Equivalently, \(\left| \prod_{k=1}^{n-1} (1-w^{k})  \right| = n\). Now we evaluate the product inside. To do this, we substitute in \(z\) for \(1\), so we calculate \(\prod_{n=1}^{n-1} (z-w^{k}) \). We then think of the inside as a polynomial. It has roots for \(w^{k}\), except for \(1\), so we multiply by \((z-1)\), and now it has roots for \(w^{k}\) including 0. So this is really \(z^{n}-1\). So our original product is simply \(\frac{z^{n}-1}{z-1} = 1 + z + z^2 + \ldots + z^{n-1}\). Plugging in \(1\), you get \(n\). We are done.
%\end{proof}
%
%\begin{thm}
%	Let \(A_1,\ldots,A_n\) be a regular \(n\)-gon in the unit circle. Let \(p\) be an arbitrary point on the unit circle. Then the maximum of \(pA_1 \cdot \ldots \cdot pA_n\) equals \(2\) (where p varies).
%\end{thm}
%
%\begin{proof}
%Think of \(A_1,\ldots,A_n\) as complex numbers \(a_1,\ldots,a_n\). WLOG, these \(n\) numbers are the \(n \)-th roots of \(-1\), or \(w^{\frac{2k-1}{n}\pi i}\). Relabeling \(p\) as \(z\), we want to show
%\begin{align*}
%	\textrm{max}_{|z|=1}\prod_{k=1}^{n} |z-a_k| = 2 
%\end{align*}
%Equivalently,
%\begin{align*}
%	\textrm{max}_{|z|=1}\left| \prod_{k=1}^{n} (z-a_k)  \right| = 2
%\end{align*}because the \(a_k's\) are roots of the RHS.
%So the statement reduces to
%\begin{align*}
%	\textrm{max}_{|z|=1} |z^{n}+1| = 2
%\end{align*}
%By the triangle inequality, it is clear that this statement holds. Indeed, for any \(z\) with \(|z| = 1\), \(|z^{n}+1| \leq |z^{n}| + 1 = |z|^{n}+1 = 2\). 
%\end{proof}
%
%
%\begin{thm}
%	Let \(A_1,\ldots,A_n\) be a non-regular \(n\)-gon in the unit circle. Then
%	\begin{align*}
%		\textrm{max}_{p \text{ on the unit circle}} pA_1 \cdot \ldots\cdot pA_n > 2
%	\end{align*}
%\end{thm}
%
%\begin{proof}
%We shall think of the vertices as complex numbers \(a_1,\ldots,a_n\). and \(p\) as a complex number. We want to show that
%\begin{align*}
%	\textrm{max}_{|z|=1} \left| \prod_{k=1}^{n} (z-a_k)  \right| > 2
%\end{align*}
%Look at the constant term of the polynomial inside: \((-a_1)(-a_2)\ldots(-a_n) = (-1)^{n}a_1\ldots a_n\). Rotating all the \(a_k's\) by the same angle does not change the maximum. In other words, for a given \(u \in \C\) on the unit circle, we can replace each \(a_k\) by \(ua_k\), without changing the maximum in question. This changes the constant term to \(u^{n}(-1)^{n}a_1\ldots a_n\). Now we will pick \(u\) so the constant term is \(1\). WLOG, this results in
% \begin{align*}
%	 \prod_{k=1}^{n} (z-a_k) = z^{n} + c_{n-1}z^{n-1} + \ldots + c_1 z + 1 
%\end{align*}
%We will call the terms in the middle (not \(z^n\) and not 1) \(p(z)\). Then we want to show that
%\begin{align*}
%	\textrm{max}_{|z| = 1} |z^{n}+1+p(z)| > 2
%\end{align*}
%Plugging in \(n\)-th roots of unity, we simply need to show that \(p(w^{k})\) can be positive. One of the roots must be non-zero, as it is of smaller degree and cannot be identically zero (would be a regular \(n\)-gon). Moreover, the sum of all the \(p(w^{k})\) must be zero. We will show this, which will then show that there is a \(k\) such that \(p(w^{k})\) has a positive real part. Now let's prove that statement. Recall that
%\begin{align*}
%	p(z) = \sum_{m=1}^{n-1} c_m z^{m}
%\end{align*}
%So
%\begin{align*}
%	\sum_{k=0}^{n-1} p(w^{k}) = \sum_{k=0}^{n-1} \sum_{m=1}^{n-1} c_m w^{km} = \sum_{m=1}^{n-1} c)_m \sum_{k=0}^{n-1} w^{km} = \sum_{m=1}^{n-1} \frac{w^{nm}-1}{w^{m}-1} = \sum_{m=1}^{n-1} c_m * 0 = 0
%\end{align*}
%\end{proof}

\end{document}


\section{Riemannian Sphere and Its Linear Transformations}
\label{sec:riemannian_sphere_and_its_linear_transformations}

\documentclass{memoir}
\usepackage{notestemplate}

%\logo{~/School-Work/Auxiliary-Files/resources/png/logo.png}
%\institute{Rice University}
%\faculty{Faculty of Whatever Sciences}
%\department{Department of Mathematics}
%\title{Class Notes}
%\subtitle{Based on MATH xxx}
%\author{\textit{Author}\\Gabriel \textsc{Gress}}
%\supervisor{Linus \textsc{Torvalds}}
%\context{Well, I was bored...}
%\date{\today}

%\makeindex

\begin{document}

% \maketitle

% Notes taken on 

\subsection{Riemann Sphere}
\label{sub:riemann_sphere}


%% ADD RIEMANN SPHERE

Under construction

\subsection{Linear Transformations}
\label{sub:linear_transformations}

\begin{defn}[Linear Fractional Transformation]
	A \textbf{linear fractional transformation} or \textbf{Möbius transformation} is a linear transformation of the complex plane given by
	\begin{align*}
		S(z) := \frac{az+b}{cz+d}
	\end{align*}
	for complex numbers \(a,b,c,d \in \C\) with one of \(c,d\) non-zero.\\

	We extend \(S\) to \(\overline{\C}\) by defining the following values:
\begin{align*}
	S(\infty) = \begin{cases}
		\sfrac{a}{c} & c\neq 0\\
		\infty & c=0
	\end{cases}\\
	S(-\sfrac{d}{c}) = \infty \text{ if }c\neq 0.
\end{align*}
	This extends the mapping so that \(S\) is a topological mapping of the extended plane onto itself, with the topology of distances on the Riemannian sphere.
\end{defn}
This is the generalized form of all linear transformations on \(\overline{\C}\)-- translations, reflections, rotations, and more can be represented as linear fractional transformations.

\begin{anki}
TARGET DECK
Complex Qual::Complex Analysis
START
MathJaxCloze
Text: A **linear fractional transformation** or **Möbius transformation** is a linear transformation of the complex plane given by
{{c1::\(\begin{align*}
        	S(z) := \frac{az+b}{cz+d}
        \end{align*}\)}}
for complex numbers \(a,b,c,d \in \C\) with one of \(c,d\) non-zero.

If we write \(z = \frac{z_1}{z_2}\), we can express \(S(z)\) as a matrix via
{{c1::\(\begin{align*}
		\begin{pmatrix} a & b \\ c & d \end{pmatrix} \begin{pmatrix} z_1 \\ z_2 \end{pmatrix} = \begin{pmatrix} w_1 \\ w_2 \end{pmatrix} 
	\end{align*}\)}} 
where \(w := \frac{w_1}{w_2} = S(z)\). If {{c1::\(ad-bc = 1\)}}, we say the linear fractional transformation is **normalized**.

Extra: We extend \(S\) to \(\overline{\C}\) by defining the following values:
\(\begin{align*}
  	S(\infty) = \begin{cases}
  		\sfrac{a}{c} & c\neq 0\\
  		\infty & c=0
  	\end{cases}\\
  	S(-\sfrac{d}{c}) = \infty \text{ if }c\neq 0.
  \end{align*}\)
This extends the mapping so that \(S\) is a topological mapping of the extended plane onto itself, with the topology of distances on the Riemannian sphere.
Tags: analysis complex_analysis complex_geometry defn
<!--ID: 1624827134946-->
END
\end{anki}

\begin{prop}
	Let \(S\) be a linear fractional transformation
	\begin{align*}
		S(z) = \frac{az+b}{c z + d}
	\end{align*}
	with \(ad-bc\neq 0\). Then
	\begin{align*}
		S:\overline{\C}\to \overline{\C}
	\end{align*}
	is an isomorphism. Furthermore, the inverse of \(S\) is given by
	\begin{align*}
		S^{-1}(z) = \frac{dz -b}{a-cz}.
	\end{align*}
	If \(ad-bc = 1\), then we say the linear fractional transformation is \textbf{normalized}.
\end{prop}

Every linear fractional transformation admits a normalized form. In fact, there are exactly two, which are obtained from each other by changing the sign of the coefficients.

\begin{anki}
START
MathJaxCloze
Text: Let \(S\) be a linear fractional transformation
\(\begin{align*}
  	S(z) = \frac{az+b}{c z + d}
\end{align*}\)
	with \(ad-bc\neq 0\). Then
{{c1::\(\begin{align*}
	S:\overline{\C}\to \overline{\C}
      \end{align*}\)}} 
	is {{c1::an isomorphism}}. Furthermore, the inverse of \(S\) is given by
{{c1::\(\begin{align*}
	S^{-1}(z) = \frac{dz -b}{a-cz}.
      \end{align*}\)}}
	If {{c1::\(ad-bc = 1\)}}, then we say the linear fractional transformation is \textbf{normalized}.
Extra: Every linear fractional transformation admits a normalized form. In fact, there are exactly two, which are obtained from each other by changing the sign of the coefficients.
Tags: analysis complex_analysis complex_geometry
<!--ID: 1626152754001-->
END
\end{anki}


\begin{general}[Matrix Formulation of Linear Fractional Transformations]
	Let \(z = \frac{z_1}{z_2}\) be an arbitrary complex number and
	\begin{align*}
		S(z) = \frac{az+b}{cz+d}.
	\end{align*}
	Then we can express this linear fractional translation by
	\begin{align*}
		\begin{pmatrix} a & b \\ c & d \end{pmatrix} \begin{pmatrix} z_1 \\ z_2 \end{pmatrix} = \begin{pmatrix} w_1 \\ w_2 \end{pmatrix} 
	\end{align*}
	where \(w := \frac{w_1}{w_2} = S(z)\). This representation as matrices is useful as it satisfies the standard matrix operations when it comes to addition, composition, inversion, and so on.\\

	Hence, the set of linear fractional transformations forms a (matrix) group. The identity transformation is given by the identity matrix. The rest of the group properties we leave to be checked by the reader.\\

	In fact, if we restrict ourselves to normalized representations, this matrix group is isomorphic to \(SL(2,\C)\). If furthermore we form an equivalence class on normalized representations, then the group is isomorphic to the \textbf{projective special linear group} \(PSL_2(\C)\).
\end{general}

If
\begin{align*}
	A = \begin{pmatrix} a & b \\ c & d \end{pmatrix} 
\end{align*}
then we may denote the linear fractional transformation
\begin{align*}
	S(z) = \frac{az+b}{cz+d}
\end{align*}
simply by \(S_A\).

\begin{prop}[Equivalence of Fractional Linear Transformations]
	Let \(A,A' \in \C^{2\times 2}\) with \(\textrm{det}(A)\neq 0\) and \(\textrm{det}(A') \neq 0\). Then
	\begin{align*}
		S_A = S_{A'} \iff A' = \lambda A
	\end{align*} 
	for some \(\lambda \in \C\) non-zero.
\end{prop}

\begin{anki}
START
MathJaxCloze
Text: **Equivalence of Fractional Linear Transformations**
Let \(A,A' \in \mathbb{C}^{2\times 2}\) with \(\textrm{det}(A)\neq 0\) and \(\textrm{det}(A') \neq 0\). Then
{{c1::\(\begin{align*}
        	S_A = S_{A'} \iff A' = \lambda A
        \end{align*} \)}}
	for some \(\lambda \in \C\) non-zero.
Tags: analysis complex_analysis complex_geometry
<!--ID: 1626152754018-->
END
\end{anki}


\begin{defn}[Important Classes of Linear Transformations]
	The linear fractional transformations of the form
	\begin{align*}
		T_\alpha (z) := \begin{pmatrix} 1 & \alpha \\ 0 & 1 \end{pmatrix} 
	\end{align*}
	for \(\alpha  \in \C\) are called \textbf{parallel translations}, as they correspond to the transformation \(S(z) = z + \alpha \).\\

	The linear fractional transformations of the form
	\begin{align*}
		R_\alpha (z) := \begin{pmatrix} \alpha  & 0 \\ 0 & 1 \end{pmatrix} 
	\end{align*}
	are referred to as \textbf{rotations} if \(\left| \alpha  \right| = 1\), or a \textbf{homothetic transformation} if \(\alpha \) is real with \(\alpha >0\). For arbitrary complex \(\alpha \neq 0\), we can write
	\begin{align*}
		\alpha  = \left| \alpha  \right| \frac{\alpha }{\left| \alpha  \right| }
	\end{align*}
	and hence the general form can be viewed as a composition of a homothetic transformation with a rotation.\\

	The linear fractional transformation
	\begin{align*}
	(z)^{-1} := \begin{pmatrix} 0 & 1 \\ 1 & 0 \end{pmatrix} 
	\end{align*}
	is referred to as an \textbf{inversion}, as it represents \(S(z) = \frac{1}{z}\).
\end{defn}
In some sense, we can view these classes of translations as the fundamental linear transformations. If \(S(z)\) is given by
\begin{align*}
	\frac{az+b}{cz+d}
\end{align*}
with \(c\neq 0\), then we can write
\begin{align*}
	S(z) &= \frac{az+b}{c z+d} = \frac{bc-ad}{c^2(z+\frac{d}{c})}+ \frac{a}{c}\\
	&=  T_{\sfrac{a}{c}}\circ R_{\frac{bc-ad}{c^2}}\circ (\cdot)^{-1} \circ T_{\sfrac{d}{c}}
\end{align*}
In the simpler case where \(c=0\), we have
\begin{align*}
	S(z) &= \frac{az+b}{d}\\
	&= T_{\sfrac{b}{d}}\circ R_{\sfrac{a}{d}}
\end{align*}

\begin{anki}
START
MathJaxCloze
Text: The linear fractional transformations of the form
 {{c1::\(\begin{align*}
         	T_\alpha (z) := \begin{pmatrix} 1 & \alpha \\ 0 & 1 \end{pmatrix} 
         \end{align*}\)}} 
for \(\alpha  \in \C\) are called **parallel translations**, as they correspond to the transformation {{c1::\(S(z) = z + \alpha \)}}.

The linear fractional transformations of the form
 {{c2::\(\begin{align*}
         	R_\alpha (z) := \begin{pmatrix} \alpha  & 0 \\ 0 & 1 \end{pmatrix} 
         \end{align*}\)}}
are referred to as **rotations** if {{c2::\(\left| \alpha  \right| = 1\)}}, or a **homothetic transformation** if {{c2::\(\alpha \) is real with \(\alpha >0\)}}. For arbitrary complex \(\alpha \neq 0\), we can write
 {{c2::\(\begin{align*}
        	\alpha  = \left| \alpha  \right| \frac{\alpha }{\left| \alpha  \right| }
        \end{align*}\)}} 
and hence the general form can be viewed as {{c2::a composition of a homothetic transformation with a rotation}}.

The linear fractional transformation
 {{c3::\(\begin{align*}
        (z)^{-1} := \begin{pmatrix} 0 & 1 \\ 1 & 0 \end{pmatrix} 
        \end{align*}\)}} 
is referred to as an **inversion**, as it represents {{c3::\(S(z) = \frac{1}{z}\)}} .
Extra: If \(S(z)\) is given by
\(\begin{align*}
  	\frac{az+b}{cz+d}
  \end{align*}\)
with \(c\neq 0\), then we can write
\(\begin{align*}
  	S(z) &= \frac{az+b}{c z+d} = \frac{bc-ad}{c^2(z+\frac{d}{c})}+ \frac{a}{c}\\
  	&=  T_{\sfrac{a}{c}}\circ R_{\frac{bc-ad}{c^2}}\circ (\cdot)^{-1} \circ T_{\sfrac{d}{c}}
  \end{align*}\)
In the simpler case where \(c=0\), we have
\(\begin{align*}
  	S(z) &= \frac{az+b}{d}\\
  	&= T_{\sfrac{b}{d}}\circ R_{\sfrac{a}{d}}
  \end{align*}\)
Tags: analysis complex_analysis complex_geometry defn
<!--ID: 1624827135038-->
END
\end{anki}

\begin{thm}
	A fractional linear transformation maps straight lines and circles onto straight lines and circles.
\end{thm}
\begin{proof}
	
\end{proof}

\begin{anki}
START
MathJaxCloze
Text: A fractional linear transformation \(S(z) = \frac{az+b}{c z + d}\), \(ac-bd\neq 0\) maps straight lines and circles onto {{c1::straight lines and circles}}.
Tags: analysis complex_analysis complex_geometry
<!--ID: 1626152754035-->
END
\end{anki}

\begin{lemma}
	Let \(S\) be a fractional linear map. If \(S(\infty) = \infty\), then
	\begin{align*}
		S(z) = az+b
	\end{align*}
	for \(a,b \in \C\).
\end{lemma}

\begin{anki}
START
MathJaxCloze
Text: Let \(S\) be a fractional linear map. If \(S(\infty) = \infty\), then
{{c1::\(\begin{align*}
        	S(z) = az+b
        \end{align*}\)
      	for \(a,b \in \C\)}}.
Tags: analysis complex_analysis complex_geometry
<!--ID: 1626152754052-->
END
\end{anki}


\begin{thm}
	Given any three distinct points \(\left\{ z_1,z_2,z_3 \right\} \) with \(z_i \in \overline{\C}\) and three distinct points \(\left\{ w_1,w_2,w_3 \right\} \) with \(w_i \in \overline{\C}\), there exists a unique fractional linear map \(S\) with
	\begin{align*}
		S(z_i) = w_i.
	\end{align*}
	If \(\left\{ w_1,w_2,w_3 \right\} = \left\{ 0,\infty,1 \right\} \) then the map is given by
	\begin{align*}
		S(z) = \frac{z-z_1}{z-z_2}\cdot \frac{z_3-z_2}{z_3-z_1}.
	\end{align*}
	We can apply this to the general case to get the relation
	\begin{align*}
		\frac{w-w_1}{w-w_2}\cdot \frac{w_3-w_2}{w_3-w_1} = \frac{z-z_1}{z-z_2}\cdot \frac{z_3-z_2}{z_3-z_1}.
	\end{align*}
	We can use this relation to derive an explicit formula for specific values.
\end{thm}

\begin{lemma}
	If \(S\) is a fractional linear map with three fixed points, then \(S\) is the identity map.
\end{lemma}

\begin{proof}
	
\end{proof}

\begin{anki}
START
MathJaxCloze
Text: Given any three distinct points \(\left\{ z_1,z_2,z_3 \right\} \) with \(z_i \in \overline{\C}\) and three distinct points \(\left\{ w_1,w_2,w_3 \right\} \) with \(w_i \in \overline{\C}\), there exists a {{c1::unique fractional linear map \(S\)::map}} with
{{c1::\(\begin{align*}
        	S(z_i) = w_i.
        \end{align*}\)}}
If \(\left\{ w_1,w_2,w_3 \right\} = \left\{ 0,\infty,1 \right\} \) then the map is given by
 {{c1::\(\begin{align*}
        	S(z) = \frac{z-z_1}{z-z_2}\cdot \frac{z_3-z_2}{z_3-z_1}.
        \end{align*}\)}}
We can apply this to the general case to get the relation
 {{c1::\(\begin{align*}
        	\frac{w-w_1}{w-w_2}\cdot \frac{w_3-w_2}{w_3-w_1} = \frac{z-z_1}{z-z_2}\cdot \frac{z_3-z_2}{z_3-z_1}.
        \end{align*}\)}} 
We can use this relation to derive an explicit formula for specific values.
Extra: If \(S\) is a fractional linear map with three fixed points, then \(S\) is the identity map.
Tags: analysis complex_analysis complex_geometry
<!--ID: 1626152754068-->
END
\end{anki}


%% Lots more details can be filled here, TBD based on later content


% \printindex
\end{document}


\section{Topology of \(\C\)}
\label{sec:topology_of_c}

\documentclass{memoir}
\usepackage{notestemplate}

% \begin{figure}[ht]
%     \centering
%     \incfig{riemmans-theorem}
%     \caption{Riemmans theorem}
%     \label{fig:riemmans-theorem}
% \end{figure}

\begin{document}
The topology of \(\C\) is based on the metric \((\C,d)\), where \(d(z,z') = \left| z-z' \right| \).  Equivalently, the topology of \(\C\) is based on the notion of \textit{open discs}.
\begin{defn}[Open Disc]
	Let \(z_0 \in \C\) and \(r>0\). The \textbf{open disc \(D_r(z_0)\) of radius \(r\) centered at \(z_0\)} is the set
	\begin{align*}
		D_r(z_0) = \left\{z \in \C \mid \left| z-z_0 \right| <r \right\} .
	\end{align*}
The \textbf{closed disc \(\overline{D_r}(z_0)\) of radius \(r\) centered at \(z_0\)} is defined by
	\begin{align*}
	\overline{D_r}(z_0) = \left\{z \in \C \mid |z-z_0|\leq r \right\} .
	\end{align*}
\end{defn}
We may write \(D_r\) to denote the disc of radius \(r\) centered at \(0\).

\begin{anki}
TARGET DECK
Complex Qual::Complex Analysis
START
MathJaxCloze
Text: Let \(z_0 \in \C\) and \(r>0\). The **open disc \(D_r(z_0)\) of radius \(r\) centered at \(z_0\)** is the set
 {{c1::\(\begin{align*}
         	D_r(z_0) = \left\{z \in \C \mid \left| z-z_0 \right| <r \right\} .
         \end{align*}\)}}
The **closed disc \(\overline{D_r}(z_0)\) of radius \(r\) centered at \(z_0\)** is defined by
{{c1::\(\begin{align*}
        \overline{D_r}(z_0) = \left\{z \in \C \mid |z-z_0|\leq r \right\} .
        \end{align*}\)}}
Tags: analysis complex_analysis complex_topology defn
<!--ID: 1624844998803-->
END
\end{anki}

\begin{defn}[Isolated Points and Discrete Sets]
	Let \(\Omega \subset \C\) and let \(z_0 \in \Omega \). If there exists a disc \(D_{r}(z_0)\) for some \(r>0\) so that
	 \begin{align*}
		 D_r(z_0) \cap \Omega = \left\{ z_0 \right\}
	\end{align*}
then we say that \(z_0\) is \textbf{isolated}.\\

If every \(z_0 \in \Omega \) is isolated, then we say that \(\Omega \) is \textbf{discrete}.
\end{defn}
Later, we will see that the set of zeroes of a non-constant holomorphic function is discrete.

\begin{anki}
START
MathJaxCloze
Text: Let \(\Omega \subset \C\) and let \(z_0 \in \Omega \). If there exists a disc \(D_{r}(z_0)\) for some \(r>0\) so that
 {{c1::\( \begin{align*}
        	 D_r(z_0) \cap \Omega = \left\{ z_0 \right\}
        \end{align*}\)}}
then we say that \(z_0\) is **isolated**.

If every \(z_0 \in \Omega \) is isolated, then we say that \(\Omega \) is **discrete**.
Extra: If \(f\) is non-constant holomorphic on \(\Omega \), then its zeroes form a discrete set in \(\C\). Furthermore, if \(f=g\) on a non-discrete subset \(U\subset \Omega \) for \(f,g\) holomorphic functions, then \(f=g\) on \(\Omega \).
Tags: analysis complex_analysis complex_topology defn
<!--ID: 1625188420444-->
END
\end{anki}


\begin{defn}[Interior, Exterior, and Boundary Points]
	Let \(\Omega \subset \C\), and let \(z \in \C\). 
	\begin{itemize}
		\item \(z\) is an \textbf{interior point of \(\Omega \)} if there exists an \(r>0\) such that \(D_r(z) \subset \Omega \). The set of interior points of \(\Omega\) is called the \textbf{interior of \(\Omega\)}, denoted \textrm{int}\(\Omega\).
	\item \(z\) is an \textbf{exterior point of \(\Omega\)} if there exists an \(r>0\) such that \(D_r(z) \cap \Omega = \emptyset\). The set of exterior points of \(\Omega\) is called the \textbf{exterior of \(\Omega\)}, denoted \textrm{ext}\(\Omega\).
	\item \(z\) is a \textbf{boundary point of \(\Omega\)} if it is neither an interior point or exterior point. The set of boundary points of \(\Omega\) is called the \textbf{boundary of \(\Omega\)}, denoted \(\partial \Omega\).
	\end{itemize}
\end{defn}

\begin{anki}
START
MathJaxCloze
Text: Let \(\Omega\subset \C\), and let \(z \in \C\). 

* \(z\) is an **interior point of \(\Omega\)** if {{c1::there exists an \(r>0\) s.t. \(D_r(z) \subset \Omega\)}}. The {{c1::set of interior points of \(\Omega\)}} is called the **interior of \(\Omega\)**, denoted \textrm{int}\(\Omega\).
* \(z\) is an **exterior point of \(\Omega\)** if there exists an \(r>0\) s.t. \(D_r(z) \cap \Omega = \emptyset\). The set of exterior points of \(\Omega\) is called the **exterior of \(\Omega\)**, denoted \textrm{ext}\(\Omega\).
* \(z\) is a **boundary point of \(\Omega\)** if it is neither an interior point or exterior point. The set of boundary points of \(\Omega\) is called the **boundary of \(\Omega\)**, denoted \(\partial \Omega\).
Tags: analysis complex_analysis complex_topology defn
<!--ID: 1624844998844-->
END
\end{anki}

\begin{defn}[Closure]
	Let \(\Omega \subset \C\). The \textbf{closure} of \(\Omega\) is the union of the interior and the boundary of \(\Omega\). Alternatively, it is the complement of the exterior of \(\Omega\). We denote the closure of \(\Omega\) by \(\overline{\Omega}\).
\end{defn}

\begin{anki}
% Up to 5 consequences
START
Definition
Name: Closure
Premise 1: Let \(\Omega\subset \C\)
Consequence 1: The **closure** of \(\Omega\) is defined as \(\overline{\Omega} := \textrm{int} (\Omega) \cup \partial \Omega\).
Tags: analysis complex_analysis complex_topology defn
<!--ID: 1624844998884-->
END
\end{anki}

\begin{prop}
	Let \(\Omega\subset \C\), and let \(z \in \C\). Then \(z \in \overline{\Omega}\) if and only if for every \(r>0\), \(D(z,r) \cap \Omega \neq \emptyset\).
\end{prop}

\begin{anki}
START
MathJaxCloze
Text: Let \(\Omega\subset \C\), and let \(z \in \C\). Then \(z \in \overline{\Omega}\) if and only if {{c1::for every \(r>0\), \(D(z,r) \cap \Omega \neq \emptyset\)}}.
Extra: There is a partition of \(\C\) relative to any arbitrary \(\Omega\) given by
\begin{align*}
	\C = \textrm{int}\Omega \cup \textrm{ext}\Omega \cup \partial \Omega
\end{align*}
which in turn tells us that the interior of the complement is the exterior, and vice versa.
Tags: 
<!--ID: 1624844998921-->
END
\end{anki}


We have introduced a partition relative to an arbitrary \(\Omega \subset \C\), namely:
\begin{align*}
	\C = \textrm{int} \Omega \cup \partial \Omega \cup \textrm{ext} \Omega
\end{align*}
where each set is disjoint. Another partition:
\begin{align*}
	\C = \textrm{ext}\Omega^{c} \cup \partial \Omega^{c} \cup \textrm{int}\Omega^{c} 
\end{align*}
Therefore the interior of a set is the exterior of the complement, and the boundaries are equal.
\begin{defn}[Open and Closed Sets]
	Let \(\Omega\subset \C\). Then
	\begin{itemize}
		\item \(\Omega\) is \textbf{open} if \(\Omega\) contains none of its boundary points; \(\Omega \cap \partial \Omega = \emptyset\)
		\item \(\Omega\) is \textbf{closed} if \(\Omega\) contains all of its boundary points; \(\partial \Omega \subset \Omega\)
	\end{itemize}
\end{defn}

\begin{anki}
START
MathJaxCloze
Text: Let \(\Omega\subset \C\). Then
* \(\Omega\) is **open** if \(\Omega\) contains none of its boundary points; \(\Omega \cap \partial \Omega = \emptyset\)
* \(\Omega\) is **closed** if \(\Omega\) contains all of its boundary points; \(\partial \Omega \subset \Omega\)
Tags: analysis complex_analysis complex_topology defn
<!--ID: 1624844998958-->
END
\end{anki}

\begin{exmp}[Properties of Open/Closed Sets]
\begin{itemize}
	\item \(\Omega\) is open if and only if \( \textrm{int}\Omega = \Omega\) 
	\item \(\Omega\) is closed if and only if \(\overline{\Omega} = \Omega\) 
	\item The topology of \(\C\) is the set of all open sets
	\item Most sets in \(\C\) are neither open nor closed; there are only continuum many open/closed sets
	\item \(\emptyset,\C\) are both open and closed (the only two sets with this property). This implies that every other set in \(\C\) has a non-empty boundary
\end{itemize}
\end{exmp}

\begin{prop}
	Let \(\Omega\subset \C\). Then \(\Omega\) is open if and only if \(\Omega^{c}\) is closed.
\end{prop}

\begin{prop}[More Properties of Open/Closed Sets]
	\begin{itemize}
		\item The union of an arbitrary family of open sets is open
		\item The finite intersection of an arbitrary family of open sets is open
		\item The intersection of an arbitrary family of closed sets is closed
		\item The finite union of an arbitrary family of closed sets is closed
	\end{itemize}
\end{prop}

\begin{prop}
	Let \(A\subset B\subset \C\). Then 
	\begin{itemize}
		\item \( \textrm{int}A \subset \textrm{ int}B\) 
		\item \(\overline{A}\subset \overline{B}\)
	\end{itemize}
\end{prop}

\begin{anki}
START
MathJaxCloze
Text: Let \(A\subset B\subset \C\). Then 
* \( \textrm{int}A \subset \textrm{ int}B\
* \(\overline{A}\subset \overline{B}\)
Tags: analysis complex_analysis complex_topology
<!--ID: 1624844998996-->
END
\end{anki}

\begin{thm}
	Let \(\Omega\subset \C\). Then
	 \begin{itemize}
		\item \( \textrm{int}\Omega\) is open; it is the largest open subset of \(\Omega\)
		\item \( \overline{\Omega}\) is closed; it is the smallest closed superset of \(\Omega\).
	\end{itemize}
\end{thm}

\begin{cor}
	Let \(\Omega\subset \C\). Then
	\begin{itemize}
		\item \( \textrm{ext}\Omega\) is open
		\item \(\partial \Omega\) is closed
	\end{itemize}
\end{cor}

\begin{anki}
START
MathJaxCloze
Text: Let \(\Omega\subset \C\). Then
* \( \textrm{int}\Omega\) is open; it is the largest open subset of \(\Omega\)
* \( \overline{\Omega}\) is closed; it is the smallest closed superset of \(\Omega\).
* \( \textrm{ext}\Omega\) is open
* \(\partial \Omega\) is closed
Tags: analysis complex_analysis complex_topology
<!--ID: 1624844999032-->
END
\end{anki}


\end{document}


\subsection{Sequences in \(\C\)}
\label{sub:sequences_in_c}

\documentclass{memoir}
\usepackage{notestemplate}

% \begin{figure}[ht]
%     \centering
%     \incfig{riemmans-theorem}
%     \caption{Riemmans theorem}
%     \label{fig:riemmans-theorem}
% \end{figure}

\begin{document}
\begin{defn}[Converging to a point]
	Let \(\left\{ z_n \right\}  \subset \C\) be a sequence of complex numbers. Let \(z \in \C\) be a complex number. We say that \((z_n)\) \textbf{converges to \(z\)} if \(\lim_{n \to \infty} \left| z_n-z \right| =0\).
\end{defn}
Of course this is identical to the coordinate-wise convergence in \(\R^2\). In other words, a complex sequence converges if and only if the real part converges to the real part of \(z\) and the imaginary part converges to the imaginary part of \(z\).\\

Note that \(\C\) is Hausdorff and thus if a sequence converges it only converges to one point in the complex plane.
\begin{defn}[Convergent and Cauchy Sequences]
	Let \(\left\{ z_n \right\} \subset \C\). Then
	\begin{itemize}
		\item \((z_n)\) is \textbf{convergent} if it converges to some \(z \in \C\).
		\item \((z_n)\) is a \textbf{Cauchy sequence} if for all \(\varepsilon>0, \exists n\geq N\) such that \(\left| z_n-z_m \right|\leq \varepsilon\) for all \(n,m \geq N\).
	\end{itemize}
\end{defn}

\begin{anki}
TARGET DECK
Complex Qual::Complex Analysis
START
MathJaxCloze
Text: Let \(\left\{ z_n \right\} \subset \C\). Then

* \((z_n)\) is **convergent** if
{{c1::\(\begin{align*}
        	\lim_{n \to \infty} \left| z_n - z \right| = 0
        \end{align*}\)}} 
for some \(z \in \C\).
* \((z_n)\) is a **Cauchy sequence** if for all \(\varepsilon>0\) there exists \( n\geq N\) such that
 {{c2::\(\begin{align*}
        	\left| z_n-z_m \right|\leq \varepsilon
        \end{align*}\)}} 
for all \(n,m \geq N\).
Tags: analysis complex_analysis complex_convergence
<!--ID: 1624245565291-->
END
\end{anki}


\begin{thm}[Completeness of \(\C\) ]
	\(\C\) is complete. That is, a sequence is convergent if and only if it is Cauchy.
\end{thm}

\begin{anki}
START
MathJaxCloze
Text: \(\C\) is {{c1::complete}}. That is, a sequence is convergent if and only if {{c1::it is Cauchy}}.
Tags: analysis complex_analysis complex_convergence
<!--ID: 1624245565361-->
END
\end{anki}


\begin{prop}
	Let \(z_n\to z,w_n\to w\) be two convergent sequences of complex numbers. Then:
	\begin{itemize}
		\item \(z_n+w_n \to z+w\) 
		\item \(z_n - w_n = z-w\)
		\item \(z_nw_n \to zw\) 
		\item \(\frac{z_n}{w_n} \to \frac{z}{w}\) when \(w_n\neq 0\)
	\end{itemize}
	In other words, \(\C\) is a topological field.
\end{prop}

\begin{prop}
	Let \(\Omega\subset \C\). Let \(z \in \C\). Then \(z \in \overline{\Omega}\) if and only if there is a convergent sequence in \(\Omega\) that converges to \(z\).
\end{prop}

\begin{cor}
	Let \(\Omega\subset \C\). Then \(\Omega\) is closed if and only if, for every convergent sequence in \(\Omega\), the limit also lies in \(\Omega\).
\end{cor}

\begin{anki}
START
MathJaxCloze
Text: Let \(\Omega\subset \C\). Let \(z \in \C\). Then \(z \in \overline{\Omega}\) if and only if there is {{c1::a convergent sequence in \(\Omega\) that converges to \(z\)::sequence}} .
Extra: \(\Omega\) is closed if and only if, for every convergent sequence in \(\Omega\), the limit also lies in \(\Omega\).
Tags: analysis complex_analysis complex_convergence
<!--ID: 1624245565405-->
END
\end{anki}


\begin{defn}[Diameter and Boundedness]
	Let \(\Omega\subset \C\). The \textbf{diameter} of \(\Omega\) is
	\begin{align*}
		\sup \left\{\left| w-z \right|  \mid w,z \in \Omega \right\} .
	\end{align*}
	We say that \(\Omega\) is \textbf{bounded} if the diameter is finite.
\end{defn}

\begin{anki}
START
MathJaxCloze
Text: Let \(\Omega\subset \C\). The **diameter** of \(\Omega\) is
 {{c1::\(\begin{align*}
         	\sup \left\{\left| w-z \right|  \mid w,z \in \Omega \right\} .
         \end{align*}\)}} 
We say that \(\Omega\) is **bounded** if {{c1::the diameter is finite}}.
Tags: analysis complex_analysis complex_convergence
<!--ID: 1624245565441-->
END
\end{anki}

\begin{thm}
	Let \(F_1 \supset F_2 \supset F_3 \supset \ldots\) be a decreasing chain of non-empty closed subsets of \(\C\). Assume that the diameter of \(F_n\to 0\). Then \(\bigcap_{n=1}^{\infty}F_n\) is a singular set \(\left\{ z \right\} \).
\end{thm}
This sequence converges if even one \(F_n\) is bounded, but the convergent set may not be singular.

\begin{anki}
START
MathJaxCloze
Text: Let \(F_1 \supset F_2 \supset F_3 \supset \ldots\) be a decreasing chain of non-empty closed subsets of \(\C\). Assume that the diameter of \(F_n\to 0\). Then {{c1::\(\bigcap_{n=1}^{\infty}F_n\) is a singular set \(\left\{ z \right\} \)}}.
Extra: The diameter converges if even one \(F_n\) is bounded, but the convergent set may not be singular.
Tags: analysis complex_analysis complex_convergence
<!--ID: 1624245565476-->
END
\end{anki}

\begin{defn}[Compactness]
	A set \(\Omega\subset \C\) is \textbf{compact} if every open cover of \(\Omega\) has a finite subcover.
\end{defn}

This is inherited from the topological compactness via the basis of open discs of \(\C\).

\begin{thm}
	Let \(\Omega \subset \C\). The following are equivalent:
	\begin{itemize}
		\item \(\Omega\) is compact
		\item Every sequence in \(\Omega\) has a subsequence converging to a point in \(\Omega\)
		\item \(\Omega\) is closed and bounded
	\end{itemize}
\end{thm}

\begin{anki}
START
MathJaxCloze
Text: 
Let \(\Omega \subset \C\). The following are equivalent:
* {{c1::\(\Omega\) is compact}}
* {{c2::Every sequence in \(\Omega\) has a subsequence converging to a point in \(\Omega\)}}
* {{c3::\(\Omega\) is closed and bounded}} 
Tags: analysis complex_analysis complex_convergence
<!--ID: 1624245565513-->
END
\end{anki}

\begin{defn}[Connected Set]
	Let \(\Omega\subset \C\) be an open set. \(\Omega\) is \textbf{connected} if there does not exist two disjoint non-empty open sets \(\Omega_1,\Omega_2\subset \C\) such that
	\begin{align*}
		\Omega = \Omega_1\cup \Omega_2.
	\end{align*}
	A connected open set in \(\C\) is often referred to as a \textbf{region}.
\end{defn}

\begin{thm}
	Let \(K\subset G\subset \C\), where \(K\) is compact and \(G\) is open. Then there is an \(r>0\) such that, for any \(z \in K\),
	\begin{align*}
		D(z,r) \subset G.
	\end{align*}
\end{thm}

\begin{anki}
START
MathJaxCloze
Text: Let \(K\subset G\subset \C\), where \(K\) is compact and \(G\) is open. Then there is an \(r>0\) such that, for any \(z \in K\),
 {{c1::\(\begin{align*}
        	D(z,r) \subset G.
        \end{align*}\)}}
Tags: analysis complex_analysis complex_topology
<!--ID: 1624845437350-->
END
\end{anki}


\end{document}


\chapter{Complex Functions}
\label{cha:complex_functions}

\documentclass{memoir}
\usepackage{notestemplate}

% \begin{figure}[ht]
%     \centering
%     \incfig{riemmans-theorem}
%     \caption{Riemmans theorem}
%     \label{fig:riemmans-theorem}
% \end{figure}

\begin{document}
% \section{}	

Now we consider functions on the complex field. Namely, we are characterizing functions \(f:\C\to \C\). It is immediate that any function \(f\) can be parametrized by
\begin{align*}
	f:\C\to \C\\
	(x,y) \mapsto (u(x,y),v(x,y))
\end{align*}
for \(u,v \) real-valued functions. Sometimes we might write \(f = u + iv\) for shorthand of the above parametrization. This notation arises because
\begin{align*}
	f(x,y) = (u(x,y),v(x,y)) &\implies\\
	f &= (u(x,y),0) + (0,v(x,y)) \implies\\
	f &= u + iv.
\end{align*}
Now we construct the equivalent analytic structures on complex functions.
\begin{defn}[Complex Continuity]
	Let \(\Omega\subset \C\). A function \(f:\Omega\to \C\) is \textbf{continuous} if for all \(\varepsilon >0\), for all \(z \in \Omega\), there exists a \(\delta>0\) such that, for all \(w \in \Omega\),
	\begin{align*}
		|w-z| < \delta \implies \left| f(w)-f(z) \right| < \varepsilon
	\end{align*}
Equivalently, if \((z_n) \subset \Omega\) is an arbitrary sequence in \(\Omega\) converging to \(z \in \C\), then \(f(z_n) \to f(z)\).\\

We denote the space of complex continuous functions on \(\Omega\) by \(C(\Omega)\).
\end{defn}

\begin{anki}
TARGET DECK
Complex Qual::Complex Analysis

START
MathJaxCloze
Text: Let \(\Omega\subset \C\). A function \(f:\Omega\to \C\) is **continuous** if {{c2::for all \(\varepsilon >0\)}}, {{c2::for all \(z \in \Omega\)}}, {{c2::there exists a \(\delta>0\)}} such that, {{c2::for all \(w \in \Omega\)}},
{{c1::\(\begin{align*}
        	|w-z| < \delta \implies \left| f(w)-f(z) \right| < \varepsilon
        \end{align*}\)}} 
Equivalently, if \((z_n) \subset \Omega\) is an arbitrary sequence in \(\Omega\) converging to \(z \in \C\), then {{c1::\(f(z_n) \to f(z)\)}}.
Extra: Of course, we can alternatively define continuity of a function at a point, then refer to a continuous function as one that's continuous at all points.
Tags: analysis complex_analysis complex_analyticity defn
<!--ID: 1624399037000-->
END
\end{anki}

Of course, we can alternatively define continuity of a function at a point, then refer to a continuous function as one that's continuous at all points.\\

The sum, difference, product, ratio, and composition of continuous functions is continuous. We leave the proof of this as an exercise to the reader.
\begin{defn}[Uniform Continuity]
	Let \(\Omega \subset \C\). A function \(f:\Omega\to \C\) is \textbf{uniformly continuous} if for all \(\varepsilon\), there exists a \(\delta>0\) such that for all \(z,w \in \Omega\),
	\begin{align*}
		|w-z| < \delta \implies \left| f(w)-f(z) \right| < \varepsilon
	\end{align*}
\end{defn}

\begin{anki}
START
MathJaxCloze
Text: Let \(\Omega \subset \C\). A function \(f:\Omega\to \C\) is **uniformly continuous** if {{c1::for all \(\varepsilon\)}}, {{c1::there exists a \(\delta>0\)}} such that {{c1::for all \(z,w \in \Omega\)}},
{{c2::\(\begin{align*}
        	|w-z| < \delta \implies \left| f(w)-f(z) \right| < \varepsilon
        \end{align*}\)}}
Extra: In other words, we choose \(\delta \) first before being given the point \(z \in \Omega\). On unbounded domains, this is a much stronger property than continuity. 
Tags: analysis complex_analysis complex_analyticity defn
<!--ID: 1624399037047-->
END
\end{anki}

In other words, we choose \(\delta \) first before being given the point \(z \in \Omega\). On unbounded domains, this is a much stronger property than continuity. However, under many circumstances the notions are equivalent:
\begin{thm}
	On a compact set in \(\C\), every continuous function is uniformly continuous.
\end{thm}
\begin{anki}
START
MathJaxCloze
Text: On a compact set in \(\C\), every continuous function is {{c1::uniformly continuous}}.
Tags: analysis complex_analysis complex_analyticity
<!--ID: 1624399037095-->
END
\end{anki}

Many of the theorems here are equivalent to the real-valued version, and hence unless the proof differs, proofs will be omitted as they can be found in a standard real analysis textbook.\\

\begin{thm}
	The continuous image of a compact set in \(\C\) is compact.
\end{thm}
This is actually a corollary via point-set topology.

\begin{cor}
	A continuous real-valued function on a compact set has a maximum and a minimum.
\end{cor}
This doesn't seem very relevant immediately, as the range of a complex function is not well-ordered. However, by considering the modulus of complex functions, we can obtain a similar result.

\begin{defn}[Maximum and Minimum of Complex Functions]
	Let \(f:\C\to \C\) be a complex-valued function. We say that \(f\) attains a \textbf{maximum} at \(z_0 \in \Omega \) if
	\begin{align*}
		\left| f(z) \right| \leq \left| f(z_0) \right| 
	\end{align*}for all \(z \in \Omega \).\\

	If instead
	\begin{align*}
		\left| f(z) \right| \geq \left| f(z_0) \right| 
	\end{align*} for all \(z\in \Omega \) then we say \(z_0\) is a \textbf{minimum} for \(f\).
\end{defn}
Notice that if \(f\) is continuous, then \(z\mapsto \left| f(z) \right| \) is continuous (by the triangle inequality).

\begin{anki}
START
MathJaxCloze
Text: Let \(f:\C\to \C\) be a complex-valued function. We say that \(f\) attains a **maximum** at \(z_0 \in \Omega \) if
{{c1::\(\begin{align*}
        	\left| f(z) \right| \leq \left| f(z_0) \right| 
        \end{align*}\)}}
for all \(z \in \Omega \).

If instead
{{c1::\(\begin{align*}
      \left| f(z) \right| \geq \left| f(z_0) \right| 
      \end{align*}\)}}
for all \(z\in \Omega \) then we say \(z_0\) is a **minimum** for \(f\).
Extra: Notice that if \(f\) is continuous, then \(z\mapsto \left| f(z) \right| \) is continuous (by the triangle inequality).
Tags: analysis complex_analysis complex_analyticity defn
<!--ID: 1624399037133-->
END
\end{anki}

\begin{cor}
	A continuous (real or complex-valued) function on a compact set \(\Omega\) is bounded and attains a maximum and minimum on \(\Omega\).
\end{cor}

\begin{anki}
START
MathJaxCloze
Text: A continuous (real or complex-valued) function on a compact set \(\Omega \) is {{c1::bounded}} and {{c1::attains a maximum and minimum on \(\Omega\)}}.
Tags: analysis complex_analysis complex_analyticity
<!--ID: 1624399037173-->
END
\end{anki}

\subsection{Curves in \(\C\)}
\label{sub:curves_in_c}

\begin{defn}[Continuous Curve]
	Let \(\Omega\subset \C\). A \textbf{continuous curve} in \(\Omega\) is a continuous function \(\gamma :[t_0,t_1]\to \Omega\) where \([t_0,t_1]\subset \R\).
\end{defn}

\begin{anki}
% Up to 5 consequences
START
Definition
Name: Continuous Curve in \(\C\)
Premise 1: \(\Omega\subset \C\), \([t_0,t_1] \subset \R\)
Consequence 1: \(\gamma \) is a continuous curve if \(\gamma:[t_0,t_1]\to \Omega\) is a continous function
Tags: analysis complex_analysis complex_analyticity defn
<!--ID: 1624399037216-->
END
\end{anki}

Because every complex function can be parametrized by \(F(x,y) = (u(x,y),v(x,y))\), it follows that we can write a curve \(\gamma \) by
\begin{align*}
	\gamma (t) = (x(t),y(t))
\end{align*}
for real-valued curves \(x,y\). This will prove useful later when we introduce complex differentiation.

\begin{defn}[Path-Connected]
	Let \(\Omega\subset \C\). Then \(\Omega\) is path-connected if every two points in  \(\Omega\) can be joined by a continuous curve in \(\Omega\).
\end{defn}

\begin{anki}
% Up to 5 consequences
START
Definition
Name: Path-Connected Set in \(\C\)
Premise 1: \(\Omega\subset \C\)
Consequence 1: \(\Omega\) is path-connected if \(\forall z_0,z_1 \in \Omega\), \(\exists \gamma:[t_0,t_1]\to \Omega\) with \(\gamma(t_i)=z_i\)
Tags: analysis complex_analysis complex_topology defn
<!--ID: 1624399037262-->
END
\end{anki}


\begin{thm}
	Let \(\Omega\subset \C\) be open, Then the following are equivalent:
	\begin{itemize}
		\item \(\Omega\) is connected
		\item Every two points in \(\Omega\) can be joined by a broken line consisting of (finitely many) horizontal and vertical line segments within \(\Omega\)
		\item \(\Omega\) is path-connected
	\end{itemize}
\end{thm}

\begin{anki}
% Up to 4 premises
% Up to 5 equivalences
START
Equivalence
Name: Connectedness of open \(\Omega\subset \C\)
Premise 1:  \(\Omega\subset \C\) open set
Equivalence 1: \(\Omega\) is connected
Equivalence 2: Every \(z_0,z_1 \in \Omega\) is joined by line consisting of (finitely many) horizontal and vertical line segments in \(\Omega\)
Equivalence 3: \(\Omega\) is path-connected
Tags: analysis complex_analysis complex_topology
<!--ID: 1624399037300-->
END
\end{anki}


\end{document}


\section{Differentiability in \(\C\)}
\label{sec:differentiability_in_c}

\documentclass{memoir}
\usepackage{notestemplate}

% \begin{figure}[ht]
%     \centering
%     \incfig{riemmans-theorem}
%     \caption{Riemmans theorem}
%     \label{fig:riemmans-theorem}
% \end{figure}

\begin{document}

\begin{defn}[Holomorphic] % Testing
\label{defn:Holomorphic}
	Let \(\Omega\subset \C\) be open. A complex-valued function \(f:\Omega\to \C\) is \textbf{holomorphic} (or \textbf{regular}, \textbf{complex differentiable}) if for every \(z\in \Omega\), the limit exists:
	\begin{align*}
		\lim_{h \to 0} \frac{f(z+h)-f(z)}{h} := f'(z)
	\end{align*}
	\(f'\) is called the \textbf{complex derivative} of \(f\).
\end{defn}
Note that \(h\) is a non-zero complex number in \(\Omega\). Hence the derivative has to exist and be equal for any possible direction that \(h\) may approach zero.\\

\begin{anki}
TARGET DECK
Complex Qual::Complex Analysis
START
MathJaxCloze
Text: Let \(\Omega\subset \C\) be open. A complex-valued function \(f:\Omega\to \C\) is **holomorphic** (or **regular**, **complex differentiable**) if for every \(z\in \Omega\), the limit exists:
 {{c1::\(\begin{align*}
         	\lim_{h \to 0} \frac{f(z+h)-f(z)}{h} := f'(z)
         \end{align*}\)}} 
	\(f'\) is called the **complex derivative** of \(f\).
Extra: Note that \(h\) is a non-zero complex number in \(\Omega\). Hence the derivative has to exist and be equal for any possible direction that \(h\) may approach zero.
Tags: analysis complex_analysis complex_analyticity defn
<!--ID: 1624408998120-->
END
\end{anki}

\begin{exmp}[Holomorphic and non-holomorphic functions]
	Any polynomial \(p \in \C[z]\) defined by
	\begin{align*}
		p(z) = a_0 + a_1z + \ldots + a_n z^{n}
	\end{align*}
	is holomorphic in the entire complex plane with
	\begin{align*}
		p'(z) = a_1 + \ldots + na_n z^{n-1}.
	\end{align*}
	However, \(f(z) = 1 / z\) is only holomorphic on open sets that do not contain the origin.\\

	For an example of a function that is never holomorphic, consider the involuntary transformation
	\begin{align*}
		f(z) = \overline{z}.
	\end{align*}
	One can see that
	\begin{align*}
		\frac{f(z_0+h) - f(z_0)}{h} = \frac{\overline{h}}{h}
	\end{align*}
	which has no limit as \(h\to 0\), as if \(h\) approaches zero on the real axis, then \(\frac{\overline{h}}{h} = 1\), and if it approaches zero on the imaginary axis, then \(\frac{\overline{h}}{h}= -1\).
\end{exmp}

It is useful to review the definition of real differentiation in \(\R^2\). At first, it seems there should be no reason to view complex and real differentiation differently, but we will start to see some subtle and important differences soon.
\begin{defn}[Real Differentiation]
	Let \(\Omega\subset \R^2\) be open. Let \(F:\Omega\to \R^2\). Then \(F\) is real-differentiable if there exists a linear transformation \(J:\R^2\to \R^2\) such that, for every \(z \in \Omega\)
	\begin{align*}
		\lim_{\left| h \right|  \to 0} \frac{\left| F(z+h)-F(z)-J_F(h) \right| }{\left| h \right| } = 0
	\end{align*}
	where \(h \in \R^2\). \(J_F\) is called the Jacobian and is exactly the \(2\times 2\) real matrix of partial derivatives
\begin{align*}
	J = J_F(x,y) = \begin{pmatrix} \sfrac{\partial u}{\partial x} & \sfrac{\partial u}{\partial y} \\ \sfrac{\partial v}{\partial x} & \sfrac{\partial v}{\partial y}  \end{pmatrix} 
\end{align*}
\end{defn}
When \(F\) is a complex function, there is a special relation between the entries of the Jacobian, if it is holomorphic.

\begin{prop}[Cauchy-Riemann Equations]
	Let \(f\) be a complex function with \(f = u + iv\) for \(u,v\) real-valued functions. If \(f\) is holomorphic, then \(f\) satisfies the \textbf{Cauchy-Riemann equations}:
\begin{align*}
	\frac{\partial f}{\partial x} = \frac{1}{i}\frac{\partial f}{\partial y} &\implies \\
	\frac{\partial u}{\partial x} &= \frac{\partial v}{\partial y} \quad \frac{\partial u}{\partial y} = -\frac{\partial v}{\partial x} 
\end{align*}
where \((x,y)\) is a complex variable.
\end{prop}

\begin{anki}
START
MathJaxCloze
Text: Let \(f\) be a complex function with \(f = u + iv\) for \(u,v\) real-valued functions. If \(f\) is holomorphic, then \(f\) satisfies the **Cauchy-Riemann equations**:
 {{c1::\(\begin{align*}
         	\frac{\partial f}{\partial x} = \frac{1}{i}\frac{\partial f}{\partial y} \implies \\
         	\frac{\partial u}{\partial x} = \frac{\partial v}{\partial y} \quad \frac{\partial u}{\partial y} = -\frac{\partial v}{\partial x} 
         \end{align*}\)}} 
where \((x,y)\) is a complex variable.
Extra: The reverse direction holds: \(f\) is complex-differentiable iff \(f\) is real-differentiable and satisfies the Cauchy-Riemann equation.
Tags: analysis complex_analysis complex_analyticity
<!--ID: 1624408998160-->
END
\end{anki}


\begin{proof}[Construction of Cauchy-Riemann Equations]
	Recall that any complex-valued function \(f\) can be parametrized by some mapping
	\begin{align*}
		f = F(x,y) = (u(x,y), v(x,y))
	\end{align*}
	where \(x,y\) represent the real and imaginary coordinate respectively, and \(u,v\) are real-valued functions. If \(F\) is real-differentiable, then the partial derivatives of \(u,v\) exist and hence
	\begin{align*}
		J_F(x,y) = \begin{pmatrix} \sfrac{\partial u}{\partial x} & \sfrac{\partial u}{\partial y} \\ \sfrac{\partial v}{\partial x} & \sfrac{\partial v}{\partial y}  \end{pmatrix}
	\end{align*}
	satisfies the necessary properties as \(h\to 0\). However, there is an implicit relation imposed, as we utilize the parametrization \(h = (h_1,h_2)\). Observe that we can treat \(x\) or \(y\) as fixed when approaching from the imaginary or real axes respectively, and get
	\begin{align*}
		f'(z) &= \lim_{h_1 \to 0} \frac{f(x+h_1,y) - f(x,y)}{h_1} = \frac{\partial f}{\partial x} (z)\\
		f'(z) &= \lim_{h_2 \to 0} \frac{f(x,y+h) - f(x,y)}{i h_2} = \frac{1}{i} \frac{\partial f}{\partial y} (z)
	\end{align*}
	Hence, if \(f\) is holomorphic, these limits must be equal and thus
	\begin{align*}
		\frac{\partial f}{\partial x} = \frac{1}{i} \frac{\partial f}{\partial y} 
	\end{align*}
	Substituting \(f = u + iv\), we get the relations
	\begin{align*}
		\frac{\partial u}{\partial x} = \frac{\partial v}{\partial y} \quad \frac{\partial u}{\partial y} = - \frac{\partial v}{\partial x} 
	\end{align*}
\end{proof}

\begin{thm}
	Let \(\Omega\subset \C\) be open. Let \(f:\Omega\to \C\). Express \(z = x+yi\) and \(f = u+vi\) in the usual way. Then \(f\) is complex-differentiable if and only if \(f\) is real-differentiable and the Cauchy-Riemann equations are satisfied:
	\begin{align*}
		\frac{\partial u}{\partial x} = \frac{\partial v}{\partial y} \quad \frac{\partial u}{\partial y} = -\frac{\partial v}{\partial x} 
	\end{align*}
\end{thm}
This follows directly from the work above.

\begin{prop}[Properties of Complex Differentiation]\label{prop:properties_of_complex_differentiation}
	Let \(f,g:\Omega \to \C\) be holomorphic complex-valued functions. Then
	\begin{align*}
		(f+g)' &= f'+g'\\
		(fg)' &= f'g + fg'\\
		\left( \frac{f}{g} \right)' &= \frac{f'g-fg'}{g^2} \text{ at all \(g(z)\neq 0\)}\\
		(f \circ g)' &= (f'\circ g)\cdot g' \text{ at all \(g(z) \in \Omega\)}
	\end{align*}
	and hence all of these compositions are holomorphic functions.
\end{prop}

\begin{hw}
	Prove \ref{prop:properties_of_complex_differentiation}.
\end{hw}

\subsection{Complex Differential Forms}
\label{sub:complex_differential_forms}

When first encountering the complex plane and complex functions, it is tempting to "view" these structures as a variant on \(\R^2\). While this interpretation is not wrong, it can be restrictive at times. While it is true that a complex function is a parametrization \(f = (u,v)\) where each component represents the real and imaginary coordinate, the underlying relation between the real and imaginary coordinate can be hidden.\\

Thus, we encourage the reader to view complex functions \(f:\C\to \C\) as mappings \(f:z\mapsto f(z)\). This interpretation comes in handy when working with differential forms on \(\C\). Let \(f(z) = f((x,y))\) be given. Then the differential \(df\) can be given by
\begin{align*}
	df = \frac{\partial f}{\partial x} \,d x + \frac{\partial f}{\partial y} \,d y.
\end{align*}
This form is perfectly valid, but can be clunky depending on how \(f\) is defined. Instead, let \(z = (x,y)\). Then we have a relation given by
\begin{align*}
	z &= (x,y) &\quad x &= (\sfrac{1}{2},0)(z+\overline{z})\\
	\overline{z}&=(x,-y) &\quad y &= (0,\sfrac{1}{2})(z-\overline{z})
\end{align*}

These identities carry over to the differentials \(dx, \,d y\) so that
\begin{align*}
	dz &= (dx,dy) & \quad dx &= (\sfrac{1}{2},0)(dz + d\overline{z}) \\
	d \overline{z} &= (dx,-dy) &\quad \,d y &= (0,\sfrac{1}{2})(dz - d \overline{z})
\end{align*}
This change of variable offers a few distinct advantages, but first we point out its limitations. One convenience offered by \(dx,\,d y\) is that they are independent of one another, unlike \(z\) and \(\overline{z}\). We cannot treat \(dz,\,d \overline{z}\) as independent variables-- however, we do have a form of independence:
\begin{prop}
	Let \(g,h\) be complex-valued functions. Then
	\begin{align*}
		g \,d z + h \,d \overline{z} = 0 \iff g= h = 0.
	\end{align*}
\end{prop}

This differential form is not particularly useful if we cannot define the differential \(df\) in the form
\begin{align*}
	df = \frac{\partial f}{\partial z} \,d z + \frac{\partial f}{\partial \overline{z}} \,d \overline{z}.
\end{align*}
We can already obtain
\begin{align*}
	df &= \frac{\partial f}{\partial x} \,d x + \frac{\partial f}{\partial y} \,d y\\
	   &= \frac{\partial f}{\partial x} \left( (\sfrac{1}{2},0)(dz + d \overline{z} ) \right) + \frac{\partial f}{\partial y} \left( (0,\sfrac{1}{2})(dz - d \overline{z} \right) \\
	   &= \frac{1}{2}\left( \frac{\partial f}{\partial x} , - \frac{\partial f}{\partial y}  \right) \,d z + \frac{1}{2} \left( \frac{\partial f}{\partial x} , \frac{\partial f}{\partial y}  \right) \,d \overline{z}.
\end{align*}
This makes it clear what a natural definition for \(\frac{\partial }{\partial z} \) and \(\frac{\partial }{\partial \overline{z}} \) should look like. With this, we formally define what we have just discussed:

\begin{defn}[Complex Differential 1-form]
	Let \((x,y)\) be complex variables with corresponding differentials \(dx , \,d y\) and partial derivatives \(\frac{\partial }{\partial x} , \frac{\partial }{\partial y} \). Then taking \(z = (x,y)\), we define the \textbf{complex differential 1-form} by
	\begin{align*}
		dz = (dx, dy)\\
		d \overline{z} = (dx,-dy)
	\end{align*}
	so that
	\begin{align*}
		dx = (\sfrac{1}{2},0) (dz + d \overline{z})\\
		dy = (0, \sfrac{1}{2}) (dz - d \overline{z})
	\end{align*}
	is satisfied. Furthermore, we define the **complex differential operators** by
	\begin{align*}
		\frac{\partial }{\partial z} &= \frac{1}{2}\left( \frac{\partial }{\partial x} , \frac{\partial }{\partial y}  \right)\\
		\frac{\partial }{\partial \overline{z}} &= \frac{1}{2} \left( \frac{\partial }{\partial x} , - \frac{\partial }{\partial y}  \right) 
	\end{align*}
	so that
	\begin{align*}
		df = \frac{\partial f}{\partial z} \,d z + \frac{\partial f}{\partial \overline{z}} \,d \overline{z}
	\end{align*}
	is satisfied.
\end{defn}

\begin{anki}
START
MathJaxCloze
Text: Let \((x,y)\) be complex variables with corresponding differentials \(dx , \,d y\) and partial derivatives \(\frac{\partial }{\partial x} , \frac{\partial }{\partial y} \). Then taking \(z = (x,y)\), we define the **complex differential 1-form** by
\(\begin{align*}
  	dz = (dx, dy)\\
  	d \overline{z} = (dx,-dy)
  \end{align*}\)
so that
{{c1::\(\begin{align*}
      	dx = (\sfrac{1}{2},0) (dz + d \overline{z})\\
      	dy = (0, \sfrac{1}{2}) (dz - d \overline{z})
        \end{align*}\)}}
is satisfied. Furthermore, we define the \textbf{complex differential operators} by
{{c2::\(\begin{align*}
        	\frac{\partial }{\partial z} &= \frac{1}{2}\left( \frac{\partial }{\partial x} , \frac{\partial }{\partial y}  \right)\\
        	\frac{\partial }{\partial \overline{z}} &= \frac{1}{2} \left( \frac{\partial }{\partial x} , - \frac{\partial }{\partial y}  \right) 
        \end{align*}\)}} 
so that
\(\begin{align*}
  	df = \frac{\partial f}{\partial z} \,d z + \frac{\partial f}{\partial \overline{z}} \,d \overline{z}
  \end{align*}\)
is satisfied.
Extra: One convenience offered by \(dx,\,d y\) is that they are independent of one another, unlike \(z\) and \(\overline{z}\). We cannot treat \(dz,\,d \overline{z}\) as independent variables-- however, we do have a form of independence. If \(g,h\) are complex functions, then:
\(\begin{align*}
  	g \,d z + h \,d \overline{z} = 0 \iff g= h = 0.
  \end{align*}\)
Tags: analysis complex_analysis complex_analyticity defn
<!--ID: 1626483183627-->
END
\end{anki}


\begin{exmp}
	What is the derivative of \(f(z) = z^2 \)?
\end{exmp}

\begin{prop}
	A complex-valued function \(f\) is holomorphic if and only if
	\begin{align*}
		\frac{\partial f}{\partial \overline{z}} = 0.
	\end{align*}
\end{prop}
This follows directly from applying \(\frac{\partial }{\partial \overline{z}} \) to \(f = (u,v)\). We encourage the reader to compute this by hand as a short exercise if it is not immediately apparent.\\

\begin{anki}
START
MathJaxCloze
Text: A complex-valued function \(f\) is holomorphic if and only if
 {{c1::\(\begin{align*}
        	\frac{\partial f}{\partial \overline{z}} = 0.
        \end{align*}\)::complex differential operator}} 
Tags: analysis complex_analysis complex_analyticity defn
<!--ID: 1626483183638-->
END
\end{anki}


One should also verify that \(\frac{\partial }{\partial z} \) and \(\frac{\partial }{\partial \overline{z}} \) commute as one expects partial derivatives to commute from real analysis. Furthermore, they satisfy the product rule and chain rule.\\

We conclude the discussion on the complex differential with one last identity of note. We will use this tool throughout, although we will not see the full scope of the complex differential operator until we discuss harmonic functions.
\begin{prop}
If \(f = (u,v)\), then we can express the complex derivative by
\begin{align*}
	f'(z) = 2 \frac{\partial u}{\partial z} = \frac{\partial u}{\partial x} - i \frac{\partial u}{\partial y} .
\end{align*}
\end{prop}

% Info on potential functions??

\subsection{Analytic Continuity}
\label{sub:analytic_continuity}

We briefly mentioned that the set of zeroes of a non-constant holomorphic function is discrete. This property is a remarkably strong statement on holomorphic functions. First, we state it a bit more formally.

\begin{thm}[Discreteness of Holomorphic Zeroes]
	Let \(\Omega \subset \C\) be a connected open set. If \(f:\Omega \to \C\) is holomorphic on \(\Omega \) and non-constant, then \(U = f^{-1}(\left\{ 0 \right\}\subset \Omega  )\) is discrete.\\

	In particular, if \(f,g\) are holomorphic on \(\Omega \), and there exists \(U\subset \Omega \) non-discrete so that \(f(z)=g(z)\) for \(z \in U\), then \(f(z)=g(z)\) for \(z \in \Omega \).
\end{thm}

\begin{proof}
	The second statement follows from the first by observing that \(f-g \) is holomorphic and hence must have a discrete set of zeroes. Hence, if \(f-g=0\) on a non-discrete set, \(f-g\) must be constant.\\

	We will prove the first part once we introduce the notion of locally constant.
\end{proof}

\begin{anki}
START
MathJaxCloze
Text: Let \(\Omega \subset \C\) be a connected open set. If \(f\) is holomorphic on \(\Omega \) and non-constant, then the set of zeroes of \(f\) on \(\Omega \) is {{c1::discrete}}.

	In particular, if \(f,g\) are holomorphic on \(\Omega \), and there exists \(U\subset \Omega \) {{c1::non-discrete}} so that \(f(z)=g(z)\) for \(z \in U\), then {{c1::\(f(z)=g(z)\) for \(z \in \Omega \)}}.
Tags: analysis complex_analysis complex_analyticity
<!--ID: 1625191420102-->
END
\end{anki}


The theorem above is so powerful because it gives us a uniqueness of holomorphic functions.

\begin{defn}[Analytic Continuation]
	Let \(f\) be a holomorphic function on an open set \(U\) with \(U\subset \Omega \subset \C \) for some \(\Omega \) open. Suppose \(g\) is a holomorphic function on \(\Omega \) with \(f(z)=g(z)\) for \(z \in U\). Then \(g\) is the \textbf{analytic continuation} of \(f\) to \(\Omega \).\\

	Instead, suppose there exists an open set \(V \subset \C\), \(U,V\) connected, with \(U\cap V \neq \emptyset\). Then if \(g\) is holomorphic on \(V\) and equal to \(f\) on \(U\cap V\), then we also say that \(g\) is the analytic continuation of \(f\) to \(V\).
\end{defn}
In general, there are many ways to extend \(f\) depending on the structure of our sets by ensuring it agrees on non-isolated sets. We may refer to any of these extensions as analytic continuations. The theorem above guarantees that these extensions are unique.

\begin{anki}
START
MathJaxCloze
Text: Let \(f\) be a holomorphic function on an open set \(U\) with \(U\subset \Omega \subset \C \) for some \(\Omega \) open. Suppose \(g\) is {{c1::a holomorphic function}} on \(\Omega \) with {{c1::\(f(z)=g(z)\)}} for \(z \in U\). Then \(g\) is the **analytic continuation** of \(f\) to \(\Omega \).
Extra: Another form of analytic continuity: suppose there exists an open set \(V \subset \C\), \(U,V\) connected, with \(U\cap V \neq \emptyset\). Then if \(g\) is holomorphic on \(V\) and equal to \(f\) on \(U\cap V\), then we also say that \(g\) is the analytic continuation of \(f\) to \(V\).
Tags: analysis complex_analysis complex_analyticity defn
<!--ID: 1625191420112-->
END
\end{anki}


\subsection{Conformality of Holomorphic Functions}
\label{sub:conformality_of_holomorphic_functions}


Holomorphic functions play an important role with regard to curves.\\

We will ask that curves in \(\C\) are differentiable (and not, say, holomorphic). This is because we cannot construct a concept of holomorphicity that makes sense with complex curves at the moment (why?).\\

Hence, we merely say that a curve \(\gamma \) is differentiable provided its components are differentiable. That is, if \(\gamma = (x,y)\) for real-valued curves \(x,y\), then \(\gamma \) is holomorphic if and only if \(x,y\) are differentiable. Differentiable curves get a special name.

\begin{defn}[Smooth]
	Let \(\gamma =(x,y)\) be a curve in \(\C\), where \(x,y\) are real-valued curves. Then \(\gamma \) is \textbf{smooth} if it is component-wise \(C^{1}\)-- that is, \(x,y \in C^{1}\), and define
	\begin{align*}
		\gamma '(t) = (x'(t),y'(t)).
	\end{align*}
	Furthermore, we require that \(\gamma'(t) \neq 0\) for any time \(t\).
\end{defn}
A smooth curve is \textbf{closed} if \(\gamma(t_0) = \gamma (t_1)\), and \textbf{simple} if \(\gamma(t) \neq \gamma(s)\) for \(t\neq s\), where \(t,s\) are not both endpoints.

\begin{defn}[Tangent Vectors]
	Let \(\gamma ,\eta \) be smooth curves passing through a point \(z_0 \in \C\), that is:
	\begin{align*}
		\gamma (\tau_1) = \eta (\tau_2) = z_0
	\end{align*}
	for some \(\tau_1,\tau_2 \in \R\). Then \(\gamma '(\tau_1)\) is the \textbf{tangent vector of \(\gamma \) at \(z_0\)}.\\

	The \textbf{angle between \(\gamma ,\eta \)} is defined to be the angle between \(\gamma'(\tau_1)\) and \(\eta'(\tau_2)\), that is
	\begin{align*}
		\theta_{z_0} (\gamma ,\eta ) = \theta (\gamma'(\tau_1), \eta'(\tau_2))
	\end{align*}
\end{defn}

\begin{prop}[Preservation of Angles]
	Let \(\gamma ,\eta \) be smooth curves passing through a point \(z_0 \in \Omega\subset \C\), and let \(f:\Omega \to \C\) be holomorphic. Then
	\begin{align*}
		\frac{\partial }{\partial t} f(\gamma (t)) = f'(\gamma (t)) \gamma '(t). 
	\end{align*}
	and
	\begin{align*}
		\theta_{z_0}(\gamma,\eta ) = \theta_{f(z_0)}(f\circ \gamma , f\circ \eta ).
	\end{align*}
\end{prop}
In other words, holomorphic functions preserve the angles between curves.\\

Any isomorphism between metric spaces with curves that satisfies
\begin{align*}
	\theta_{z_0}(\gamma ,\eta ) = \theta_{f(z_0)}(f\circ \gamma , f\circ \eta )
\end{align*}
is called \textbf{conformal}. Hence, the proposition above states that all holomorphic isomorphisms are conformal.

\begin{anki}
START
MathJaxCloze
Text: Any isomorphism between metric spaces with curves that satisfies for every \(z_0\) in the metric space and \(\gamma ,\eta \) curves within the space:
 {{c1::\(\begin{align*}
        	\theta_{z_0}(\gamma ,\eta ) = \theta_{f(z_0)}(f\circ \gamma , f\circ \eta )
        \end{align*}\)}}
is called \textbf{conformal}.
Tags: analysis complex_analysis defn
<!--ID: 1624675761922-->
END
\end{anki}

\begin{anki}
START
MathJaxCloze
Text: Holomorphic isomorphisms in \(\C\) are {{c1::conformal}} maps.
Tags: 
<!--ID: 1624675761966-->
END
\end{anki}


\begin{proof}
	The derivative of the composition follows from the rule of complex differentiation on compositions. For the second part, let \(f'(z_0) = \alpha \). Then
	\begin{align*}
		\langle \alpha z, \alpha w \rangle = \textrm{Re}(\alpha z \overline{\alpha } \overline{w}) = \alpha \overline{\alpha } \textrm{Re}(z \overline{w}) = \left| \alpha  \right|^2 \langle z,w \rangle .
	\end{align*}
	Hence
	\begin{align*}
		\langle (f\circ \gamma)' , (f\circ \eta)'  \rangle = \langle \alpha \gamma' , \alpha \eta'  \rangle \\
		\left| (f\circ \gamma)'  \right| = \left| \alpha  \right|\left| \gamma ' \right| 
	\end{align*}
	which implies that
	\begin{align*}
		\theta ( (f\circ \gamma)' , (f \circ \eta )') = \theta (\alpha \gamma , \alpha \eta )
	\end{align*}
	One can check that scalar multiplication of complex vectors does not change the angle between complex vectors, and hence it holds that \(f\) is conformal.
\end{proof}
The reverse is false in general.\\

This gives an interesting new light to holomorphic functions. We can instead treat functions that are holomorphic on \(\C\) as isomorphisms of \(\C\). Hence we can view functions as holomorphic transformations of the complex plane. This equivalence allows us to infer a lot of information of geometric transformations of the complex plane.\\

Furthermore, one can show that \(\overline{f}\) is \textbf{indirectly conformal}-- that is, angles are preserved but reversed in direction. In general, holomorphic transformations scale continuously the area of sets. That is, let \(E\subset \C\) be a measurable set. Then
\begin{align*}
	A(E) &= \int_{E} 1 \,d \mu \\
	     &\implies A(f(E)) = \int_E \left| f' \right| \,d \mu .
\end{align*}
This holds because in general for any real-differentiable isomorphism \(f = u+iv\)
\begin{align*}
	A(f(E)) = \int_E \left| u_x v_y - u_y v_x \right| \,d x \,d y
\end{align*}
When \(f\) is conformal, we have \(u_xv_y - u_yv_x = \left| f'(z) \right|^2\) and hence the statement holds.

\begin{anki}
START
MathJaxCloze
Text: One can show that \(\overline{f}\) is **indirectly conformal**-- that is, {{c1::angles are preserved but reversed in direction}}. 

Holomorphic transformations {{c2::scale continuously the area of sets}} . That is, let \(E\subset \C\) be a measurable set. Then
 {{c2::\(\begin{align*}
        	A(E) &= \int_{E} 1 \,d \mu \\
        	     &\implies A(f(E)) = \int_E \left| f' \right| \,d \mu .
        \end{align*}\)}} 

Extra: In general, for any real-differentiable isomorphism \(f = u+iv\)
\(\begin{align*}
  	A(f(E)) = \int_E \left| u_x v_y - u_y v_x \right| \,d x \,d y
  \end{align*}\)
When \(f\) is conformal, we have \(u_xv_y - u_yv_x = \left| f'(z) \right|^2\) and hence the statement holds.
Tags: analysis complex_analysis defn complex_analyticity
<!--ID: 1624675762007-->
END
\end{anki}


\end{document}
	
\documentclass{memoir}
\usepackage{notestemplate}

%\logo{~/School-Work/Auxiliary-Files/resources/png/logo.png}
%\institute{Rice University}
%\faculty{Faculty of Whatever Sciences}
%\department{Department of Mathematics}
%\title{Class Notes}
%\subtitle{Based on MATH xxx}
%\author{\textit{Author}\\Gabriel \textsc{Gress}}
%\supervisor{Linus \textsc{Torvalds}}
%\context{Well, I was bored...}
%\date{\today}

%\makeindex

\begin{document}

% \maketitle

% Notes taken on 06/27/21

\subsection{Inverse and Open Mapping Theorems}
\label{sub:inverse_and_open_mapping_theorems}

\begin{defn}[Analytic Isomorphism]
	Let \(f\) be a holomorphic function on an open set \(X\subset \C\), and let \(f(X )=Y\). If \(Y\) is open and there exists a holomorphic function \(g:Y\to X\) with
	\begin{align*}
		f\circ g = \textrm{Id}_Y\\
		g \circ f = \textrm{Id}_X
	\end{align*}
	then \(f\) and \(g\) are \textbf{analytic isomorphisms}.\\

	If instead there is a point \(z_0 \in X\) such that \(f\) is an analytic isomorphism for some open neighborhood \(U_{z_0}\subset X\), then we say \(f\) is a \textbf{local analytic isomorphism} or \textbf{locally invertible} at \(z_0\).
\end{defn}
Note that the word "inverse" here always refers to the composition inverse. When the multiplicative and composition inverse differ, if we wish to refer to the multiplicative inverse, we will explicitly say "multiplicative inverse".

\begin{anki}
TARGET DECK
Complex Qual::Complex Analysis
START
MathJaxCloze
Text: Let \(f\) be a holomorphic function on an open set \(X\subset \C\), and let \(f(X )=Y\). If \(Y\) is {{c1::open}} and there exists a {{c1::holomorphic function \(g:Y\to X\)}} with
{{c1::\(\begin{align*}
        	f\circ g = \textrm{Id}_Y\\
        	g \circ f = \textrm{Id}_X
        \end{align*}\)}} 
	then \(f\) and \(g\) are **analytic isomorphisms**.

	If instead there is a point \(z_0 \in X\) such that {{c2::\(f\) is an analytic isomorphism for some open neighborhood \(U_{z_0}\subset X\)}}, then we say \(f\) is a **local analytic isomorphism** or **locally invertible** at \(z_0\).
Extra: The inverse function \(g\) of an analytic isomorphism is unique.
Tags: analysis complex_analysis complex_analyticity defn
<!--ID: 1624940600638-->
END
\end{anki}


\begin{prop}
	The inverse function \(g\) of an analytic isomorphism is unique.
\end{prop}

\begin{thm}[Complex Inverse Function Theorem]
	Let \(f\) be a holomorphic function on an open set \(\Omega \subset \C\). Suppose that  \(f'(z_0)\neq 0\) for some \(z_0 \in \Omega \). Then \(f\) is a local analytic isomorphism at \(z_0\).
\end{thm}

This can be proven via formal power series. In fact, this can also be proven the standard route fron real analysis. If one assumes the real analysis version of the theorem, then this can be proven by simply decomposing \(f = u + iv\), in which case it applies to \(u,v\).\\

% Proof via VI Theorem 1.7 in Lang?

\begin{anki}
START
MathJaxCloze
Text: **Complex Inverse Function Theorem**
Let \(f\) be a holomorphic function on an open set \(\Omega \subset \C\). Suppose that  {{c1::\(f'(z_0)\neq 0\)}} for some \(z_0 \in \Omega \). Then {{c1::\(f\) is a local analytic isomorphism at \(z_0\)}}.
Extra: This can be proven via formal power series. In fact, this can also be proven the standard route fron real analysis. If one assumes the real analysis version of the theorem, then this can be proven by simply decomposing \(f = u + iv\), in which case it applies to \(u,v\).
Tags: analysis complex_analysis complex_analyticity
<!--ID: 1624940600654-->
END
\end{anki}

Recall from topology that \(f\) is an \textbf{open mapping} if for every open subset \(U\subset X\), \(f(U)\) is open.

\begin{thm}[Open Mapping Theorem]
	Let \(f\) be holomorphic on an open set \(X\subset C\). If for every \(z_0 \in X\), \(f\) is non-constant on every neighborhood \(U_{z_0}\subset X\), then \(f\) is an open mapping.
\end{thm}
\begin{proof}
	
\end{proof}

\begin{anki}
START
MathJaxCloze
Text: **Open Mapping Theorem**
Let \(f\) be holomorphic on an open set \(X\subset C\). If for every \(z_0 \in X\), \(f\) is {{c1::non-constant on every neighborhood \(U_{z_0}\subset X\)}}, then {{c1::\(f\) is an open mapping}}.
Extra: Recall from topology that \(f\) is an **open mapping** if for every open subset \(U\subset X\), \(f(U)\) is open.
Tags: analysis complex_analysis complex_analyticity
<!--ID: 1624940600665-->
END
\end{anki}


\begin{thm}[Change of Coordinates]
	Let \(f\) be holomorphic at a point \(z_0\), with \(f(z_0)\neq 0\). Then there exists a local analytic isomorphism \(\varphi \) at 0 such that
	\begin{align*}
		f(z) = f(z_0) + (\varphi (z-z_0))^{m}
	\end{align*}
	where \(m\) is the smallest value (greater than \(0\)) for which \(f^{(m)}(z_0)\neq 0\).
\end{thm}

This theorem allows us to view functions holomorphic at a point \(z_0\) equivalently with functions holomorphic at \(0\). We can always find some \(\varphi \) that will "shift" our function so it is holomorphic at \(0\).

\begin{anki}
START
MathJaxCloze
Text: Let \(f\) be holomorphic at a point \(z_0\), with \(f(z_0)\neq 0\). Then there exists a {{c1::local analytic isomorphism}} \(\varphi \) at 0 such that
{{c1::\(\begin{align*}
         	f(z) = f(z_0) + (\varphi (z-z_0))^{m}
         \end{align*}\)}}
where \(m\) is the smallest value (greater than \(0\)) for which {{c1::\(f^{(m)}(z_0)\neq 0\)}}.
Extra: This theorem allows us to view functions holomorphic at a point \(z_0\) equivalently with functions holomorphic at \(0\). We can always find some \(\varphi \) that will "shift" our function so it is holomorphic at \(0\).
Tags: analysis complex_analysis complex_analyticity
<!--ID: 1624940600673-->
END
\end{anki}


\begin{thm}[The Natural Holomorphic Isomorphism]
	Let \(f\) be holomorphic on an open set \(\Omega \subset \C\). If \(f\) is injective, then
	\begin{align*}
		f:\Omega \to f(\Omega )
	\end{align*}
	is a holomorphic isomorphism, and hence \(f'(z)\neq 0\) for all \(z \in \Omega \).
\end{thm}
The first part of the theorem should not be surprising to the reader. The injectivity of \(f\) gives us the non-degeneracy of the derivative-- in particular, \(f'(z)\) cannot be constant, and hence the complex inverse function theorem tells us its inverse is holomorphic at \(f(z_0)\), giving the second statement.

\begin{anki}
START
MathJaxCloze
Text: Let \(f\) be holomorphic on an open set \(\Omega \subset \C\). If \(f\) is injective, then
\(\begin{align*}
  	f:\Omega \to f(\Omega )
  \end{align*}\)
is a {{c1::holomorphic isomorphism}}, and hence {{c1::\(f'(z)\neq 0\)}} for all \(z \in \Omega \).
Extra: The first part of the theorem should not be surprising to the reader. The injectivity of \(f\) gives us the non-degeneracy of the derivative-- in particular, \(f'(z)\) cannot be constant, and hence the complex inverse function theorem tells us its inverse is holomorphic at \(f(z_0)\), giving the second statement.
Tags: analysis complex_analysis complex_analyticity
<!--ID: 1624940600682-->
END
\end{anki}


\subsection{Local Maximum Modulus Principle}
\label{sub:local_maximum_modulus_principle}

\begin{defn}[Locally Constant]
	A function \(f\) is \textbf{locally constant} at a point \(z_0\) if there exists an open neighborhood \(U_{z_0}\) such that \(f\) is constant on \(U_{z_0}\).
\end{defn}

\begin{anki}
START
MathJaxCloze
Text: A function \(f\) is **locally constant** at a point \(z_0\) if there exists {{c1::an open neighborhood \(U_{z_0}\)}} such that {{c1::\(f\) is constant on \(U_{z_0}\)}}.
Tags: analysis complex_analysis complex_analyticity defn
<!--ID: 1624940600690-->
END
\end{anki}


\begin{thm}[Local Maximum Modulus Principle]
	Let \(f\) be holomorphic on an open set \(\Omega \). Suppose that \(z_0 \in \Omega \) is a maximum for \(\left| f \right| \). Then \(f\) is locally constant at \(z_0\).
\end{thm}

\begin{proof}
	Assume for the sake of contradiction that there does not exist an open neighborhood \(U_{z_0}\) for which \(f(z)=f(z_0)\) for \(z \in U_{z_0}\). By the open mapping theorem, \(f\) is an open mapping on \(U_{z_0}\), and hence the image \(f(U_{z_0})\) contains a disc \(D_{f(z_0)}(s)\subset f(U_{z_0})\). But then there must exist an element \(z_1 \in U_{z_0}\) for which
	 \begin{align*}
		 \left| f(z_1) \right| \geq \left| f(z_0) \right| 
	\end{align*}
	and hence \(z_0\) cannot be a maximum for \(\left| f \right| \), yielding a contradiction.
\end{proof}

\begin{cor}[Global Maximum Modulus Principle]
Let \(f\) be a holomorphic function on an open connected set \(\Omega \). If \(z_0\in \Omega \) is a maximum for \(\left| f \right| \) for all \(z \in \Omega \), then \(f\) is constant on \(\Omega \).
\end{cor}
In other words, a non-trivial holomorphic function on an open connected set must attain its maximum on the boundary. In fact, if \(\overline{\Omega }\) is closed and bounded, then this maximum always exists.

\begin{anki}
% Up to 4 premises
% Up to 4 equivalences
START
Theorem
Name: Local Maximum Modulus Principle
Premise 1: \(f\) holomorphic on \(\Omega \) open
Premise 2: \(z_0 \in \Omega \) is maximum for \(\left| f \right| \)
Consequence 1: \(f\) is locally constant at \(z_0\)
Tags: analysis complex_analysis complex_analyticity
<!--ID: 1624940600699-->
END
\end{anki}

\begin{anki}
% Up to 4 premises
% Up to 4 equivalences
START
Theorem
Name: Global Maximum Modulus Principle
Premise 1: \(f\) holomorphic on \(\Omega \) open and connected
Premise 2: \(z_0 \in \Omega \) is maximum for \(\left| f \right| \)
Consequence 1: \(f\) is constant on \(\Omega \)
Tags: analysis complex_analysis complex_analyticity
<!--ID: 1624941073717-->
END
\end{anki}


\begin{cor}
	Let \(f\) be holomorphic on an open set \(\Omega \), and suppose that \(z_0 \in \Omega \) is a maximum for \(\textrm{Re}f\), that is,
	\begin{align*}
		\textrm{Re}f(z_0) \geq \textrm{Re}f(z)
	\end{align*}
	for all \(z \in \Omega \). Then \(f\) is locally constant at \(z_0\).
\end{cor}

This gives us a lot of powerful applications.
\begin{thm}
	Let
	\begin{align*}
		f(z) = \sum_{n=0}^{k} a_nz^{n} 
	\end{align*}
	be a non-constant complex polynomial with \(a_k\neq 0\). Then there exists \(z_0 \in \C\) so that \(f(z_0) = 0\).
\end{thm}

We will prove this later via power series.

% \printindex
\end{document}


\section{Power Series}
\label{sec:power_series}

\documentclass{memoir}
\usepackage{notestemplate}

% \begin{figure}[ht]
%     \centering
%     \incfig{riemmans-theorem}
%     \caption{Riemmans theorem}
%     \label{fig:riemmans-theorem}
% \end{figure}

\begin{document}

\begin{defn}[Power Series]
A \textbf{power series} is an infinite sum of monomials
\begin{align*}
	\sum_{n=0}^{\infty} a_n (z-z_0)^{n}
\end{align*}
where \(\left\{ a_n \right\},z_0  \in \C\) and \(z\) is a complex variable. We call \(z_0\) the \textbf{center} of the power series.
\end{defn}
Notice that we make no statements thus far in terms of convergence. Furthermore, one can take \(a_n = 0\) for \(n\geq N\) in order to express a finite power series.\\

\begin{anki}
TARGET DECK
Complex Qual::Complex Analysis
START
MathJaxCloze
Text: A **power series** is an infinite sum of monomials
{{c1::\(\begin{align*}
        	\sum_{n=0}^{\infty} a_n (z-z_0)^{n}
        \end{align*}\)}}
where \(\left\{ a_n \right\},z_0  \in \C\) and \(z\) is a complex variable. We call \(z_0\) the \textbf{center} of the power series.
Extra: One can take \(a_n = 0\) for \(n\geq N\) in order to express a finite power series.
Tags: analysis complex_analysis power_series defn
<!--ID: 1624504053797-->
END
\end{anki}


We briefly describe a stronger form of convergence before looking closer at the convergence of power series:
\begin{defn}[Uniform Convergence]
	Let \(\left\{ f_n \right\} \) be a sequence of complex-valued functions on a set \(\Omega\subset \C\). The sequence \(\left\{ f_n \right\} \) is \textbf{uniformly convergent} on \(E\) with
	\begin{align*}
		\lim_{n \to \infty} f_n = f
	\end{align*}
	if for every \(\varepsilon>0\), there exists an \(N \in \N\) such that, for all \(n\geq N\) and \(z \in \Omega\)
	\begin{align*}
		\left| f_n(z) - f(z) \right| \leq \varepsilon.
	\end{align*}
\end{defn}

\begin{anki}
START
MathJaxCloze
Text: Let \(\left\{ f_n \right\} \) be a sequence of complex-valued functions on a set \(\Omega\subset \C\). The sequence \(\left\{ f_n \right\} \) is **uniformly convergent** on \(E\) with
\(\begin{align*}
  	\lim_{n \to \infty} f_n = f
  \end{align*}\)
	if {{c1::for every \(\varepsilon>0\)}}, {{c1::there exists an \(N \in \N\)}} such that, {{c1::for all \(n\geq N\)}} and {{c1::\(z \in \Omega\)}}:
	{{c1::\(\begin{align*}
	        	\left| f_n(z) - f(z) \right| \leq \varepsilon.
	        \end{align*}\)}} 
Tags: analysis complex_analysis defn complex_convergence
<!--ID: 1624504053886-->
END
\end{anki}

Of course, there are less strong and more general notions of convergence:

\begin{defn}[Absolute Convergence]
	A sequence of complex numbers \(\left\{ z_n \right\}\subset \C \) \textbf{absolutely converges} if
	\begin{align*}
	\sum_{n=0}^{\infty} \left| z_n \right| 
	\end{align*}
	converges.
\end{defn}

\begin{anki}
START
MathJaxCloze
Text: A sequence of complex numbers \(\left\{ z_n \right\} \subset \C \) **absolutely converges** if
 {{c1::\(\begin{align*}
        \sum_{n=0}^{\infty} \left| z_n \right| 
        \end{align*}\)}} 
	converges.
Extra: A power series is absolutely convergent at a point \(z_1\) if the power series evaluated at \(z_1\) is absolutely convergent.
Tags: 
<!--ID: 1624504053923-->
END
\end{anki}

We urge the reader to be cautious. The sum above is not a power series, and so it does not make semantic sense for a power series to absolutely converge. However, if we fix \(z = z_1\), then we can ask if the power series (absolutely) converges for \(z = z_1\).\\

\begin{prop}
	Absolute convergence implies convergence.
\end{prop}

Observe that if \(z = z_1\) is fixed and the power series is absolutely convergent for that \(z\):
\begin{align*}
	\sum_{n=0}^{\infty} \left| a_n (z-z_0)^{n} \right| 
\end{align*}
then
\begin{align*}
	\sum_{n=0}^{\infty} \left| a_n (z-z_0)^{n} \right| = \sum_{n=0}^{\infty} \left| a_n \right| \left| z-z_0 \right|^{n}.
\end{align*}
We will use this form when discussing the absolute convergence of a power series at a point. One can verify that the first sum converges if and only if the second sum converges.

\begin{prop}
	Given a power series
	\begin{align*}
	\sum_{n=0}^{\infty} a_n (z-z_0)^{n},
	\end{align*}
	assume it converges absolutely for some \(z=z_1\). Then the power series converges for all \(z\) such that \(\left| z-z_0 \right| \leq \left| z_1-z_0 \right| \).
\end{prop}
This is a slight simplification. Absolute convergence is uniform in every closed subdisc-- that is, every compact subset within the disc is absolutely convergent. In other words, if a power series absolutely converges for \(z = z_1\), it convergences absolutely and locally uniformly within the disc determined by \(z_1-z_0 \).

\begin{thm}
	Given a power series \(\sum_{n=0}^{\infty} a_n (z-z_0)^{n}\) there exists \(0\leq R\leq \infty\) such that:
	\begin{itemize}
		\item If \(\left| z-z_0 \right| <R\) the series converges absolutely
		\item If \(\left| z -z_0\right| >R\) the series diverges
	\end{itemize}
	Using the convention that \(\frac{1}{0}= \infty\) and \(\frac{1}{\infty}=0\), then \(R\) is given by Hadamard's formula:
	\begin{align*}
		\frac{1}{R}= \limsup \left| a_n \right|^{1 / n} .
	\end{align*}
	The number \(R\) is called the \textbf{radius of convergence} of the power series, and the region \(\left| z-z_0 \right| <R\) is the \textbf{disc of convergence}.
\end{thm}

Notice that the theorem above makes no statement as to convergence on the boundary. On the boundary, it is unclear whether we have convergence or divergence.

\begin{proof}
	
\end{proof}

\begin{anki}
START
MathJaxCloze
Text: Given a power series \(\sum_{n=0}^{\infty} a_n (z-z_0)^{n}\) there exists \(0\leq R\leq \infty\) such that:

* {{c1::If \(\left| z-z_0 \right| <R\) the series converges absolutely}}
* {{c1::If \(\left| z -z_0\right| >R\) the series diverges}}

Using the convention that \(\frac{1}{0}= \infty\) and \(\frac{1}{\infty}=0\), then \(R\) is given by Hadamard's formula:
 {{c2::\(\begin{align*}
        	\frac{1}{R}= \limsup \left| a_n \right|^{1 / n} .
        \end{align*}\)}} 
	The number \(R\) is called the \textbf{radius of convergence} of the power series, and the region {{c1::\(\left| z-z_0 \right| <R\)}}  is the \textbf{disc of convergence}.
Tags: analysis complex_analysis power_series defn
<!--ID: 1624504053960-->
END
\end{anki}


\begin{exmp}[Trigonometric Functions]
	Consider the power series given by
	\begin{align*}
		e^{z} &:= \sum_{n=0}^{\infty} \frac{1}{n!}z^{n}\\
		\cos (z) &:= \sum_{n=0}^{\infty} a_n z^{n}\\
		\sin (z) &:= \sum_{n=0}^{\infty} b_n z^{n}
	\end{align*}
	where
	\begin{align*}
		a_n &= \begin{cases}
			\frac{(-1)^{n}}{(2n)!} & n \equiv 0 \pmod{2} \\
			0 & n \equiv 1 \pmod{2}
		\end{cases}\\
		b_n &= \begin{cases}
			0 & n \equiv 0 \pmod{2}\\
			\frac{(-1)^{n}}{(2n+1)!} & n \equiv 1 \pmod{2}
		\end{cases}.
	\end{align*}
	These power series are absolutely convergent in the whole complex plane. One can check that they agree with the usual exponential, cosine, and sine function of the real plane when \(z\) is real. Within the context of complex analysis, we instead choose to define the functions by the above series.
\end{exmp}

\begin{anki}
START
MathJaxCloze
Text: Consider the power series given by
	\begin{align*}
		e^{z} &:= {{c1::\sum_{n=0}^{\infty} \frac{1}{n!}z^{n}}} \\
		\cos (z) &:= \sum_{n=0}^{\infty} a_n z^{n}\\
		\sin (z) &:= \sum_{n=0}^{\infty} b_n z^{n}
	\end{align*}
	where
	\begin{align*}
		a_n &= {{c2::\begin{cases}
			\frac{(-1)^{n}}{(2n)!} & n \equiv 0 \pmod{2} \\
			0 & n \equiv 1 \pmod{2}
		\end{cases} }} \\
		b_n &= {{c3::\begin{cases}
			0 & n \equiv 0 \pmod{2}\\
			\frac{(-1)^{n}}{(2n+1)!} & n \equiv 1 \pmod{2}
		\end{cases} }} .
	\end{align*}
Extra: These power series are absolutely convergent in the whole complex plane. One can check that they agree with the usual exponential, cosine, and sine function of the real plane when \(z\) is real. Within the context of complex analysis, we instead choose to define the functions by the above series.
Tags: analysis complex_analysis power_series
<!--ID: 1624504053995-->
END
\end{anki}

\begin{hw}
	Show that \(e^{z}, \cos(z), \sin(z)\) converge for all \(z \in \C\). Then show that
	\begin{align*}
		\cos(z) = \frac{1}{2}\left( e^{iz} + e^{-iz} \right) \\
		\sin(z) = \frac{1}{2i} \left( e^{iz} - e^{-iz} \right) .
	\end{align*}
	We call the above formulas the \textbf{Euler formulas} for the cosine and sine functions.
\end{hw}

\begin{hw}
	Show that \(e^{z}\) is the only solution to
	\begin{align*}
		f'(z) = f(z)
	\end{align*}
	with \(f(0) = 1\). Use this formulation of \(e^{z}\) to show that
	\begin{align*}
		e^{a+b} = e^{a}e^{b}
	\end{align*}
	Many references choose to define \(e^{z}\) as the unique solution to the differential equation above. Proving this bridges the gap between the two definitions, and now either formulation can be used interchangably.
\end{hw}

\begin{prop}
	\begin{align*}
		e^{2 \pi i} = 1.
	\end{align*}
\end{prop}
\begin{proof}
	Observe that
	\begin{align*}
		e^{z} = (\cos(z),\sin(z))
	\end{align*}
	and hence
	\begin{align*}
		e^{2\pi i} = \left( \cos(2\pi i), \sin(2\pi i) \right) = \left( 1,0 \right) = 1.
	\end{align*}
\end{proof}

\begin{exmp}[Logarithm]
We define the \textbf{logarithm} to be the inverse function of the exponential:
\begin{align*}
	\log: \C\setminus\left\{ 0 \right\} \to \C
	\log(e^{z}) := z
\end{align*}
Hence by definition, the domain of \(\log\) is the range of \(e^{z}\), that is, \(\C\setminus \left\{ 0 \right\} \).\\

First, notice that the logarithm is not injective. To see this, write \(z = (x,y)\) and notice that
\begin{align*}
	e^{z} = e^{(x,y)} = e^{(x,0)}e^{(0,y)} = e^{(x,0)}e^{(0,2\pi k+ y}
\end{align*}
for \(k \in \Z\). Hence the real part is unique, but the imaginary part is only unique up to multiples of \(2\pi \). We refer to the real part of the logarithm as the \textbf{real logarithm}, and note that it is given by
\begin{align*}
	\log(e^{(x,0)}) = \left| z \right| .
\end{align*}
The imaginary part is referred to as the \textbf{argument of \(z\)} and is given by
\begin{align*}
	\log(e^{(0,y)}) = \sfrac{z}{\left| z \right| }.
\end{align*}
When taking the argument of a complex number, we first must choose a \textbf{branch}-- an interval of length \(2\pi \) in which the argument is to lie. If unstated, then we implicitly are choosing the canonical branch \(0\leq \textrm{arg}(z)<2\pi \).\\

Geometrically, we can view the argument of \(z\) as the angle. Hence, for all \(z \in \C\setminus\left\{ 0 \right\} \), we have
\begin{align*}
	\log(z) = (\log\left| z \right|, \textrm{arg}(z)).
\end{align*}
Let the canonical branch be chosen. This function is not holomorphic on \(\C\setminus \left\{ 0 \right\}\), but is holomorphic on \(\C\setminus \left\{ z\in \R \mid z \leq 0 \right\} \). This is because there is a discontinuity on the negative real line.
\end{exmp}

\begin{thm}
	The power series \(f(z) = \sum_{n=0}^{\infty} a_n (z-z_0)^n\) is a holomorphic function in its disc of convergence. The derivative of \(f\) is also a power series obtained by differentiating term by term the series for \(f\), that is,
	\begin{align*}
		f'(z) = \sum_{n=0}^{\infty} na_n(z-z_0)^{n-1}
	\end{align*}
	Moreover, \(f'\) has the same radius of convergence as \(f\).
\end{thm}
\begin{proof}
	
\end{proof}

\begin{anki}
START
MathJaxCloze
Text: The power series \(f(z) = \sum_{n=0}^{\infty} a_n (z-z_0)^n\) is a {{c1::holomorphic function}} in its disc of convergence. The derivative of \(f\) is also a {{c1::power series}} obtained by {{c1::differentiating term by term the series for \(f\)}}, that is,
 {{c1::\(\begin{align*}
        	f'(z) = \sum_{n=0}^{\infty} na_n(z-z_0)^{n-1}
        \end{align*}\)}} 
	Moreover, \(f'\) has the {{c1::same radius of convergence}} as \(f\).
Extra: A power series is infinitely complex differentiable in its disc of convergence, and the higher derivatives are also power series obtained by termwise differentiation.
Tags: analysis complex_analysis power_series complex_analyticity
<!--ID: 1624504054031-->
END
\end{anki}


\begin{cor}
	A power series is infinitely complex differentiable in its disc of convergence, and the higher derivatives are also power series obtained by termwise differentiation.
\end{cor}
This is an incredibly powerful statement. Compare this to real analysis-- in real analysis, we cannot infer a function has higher derivatives from the existence of a first derivative. The strength of this tool is that now if we want to show a complex equation is holomorphic in a region, we simply show that it is equal to a power series within the region, then show the region is within the power series' region of convergence. Now we formalize this idea:

\begin{defn}[Analytic]
	A function \(f\) defined on an open set is said to be \textbf{analytic} at a point \(z_0\) if there exists a power series centered at \(z_0\) with positive radius of convergence such that
	\begin{align*}
		f(z) = \sum_{n=0}^{\infty} a_n(z-z_0)^{n}
	\end{align*}
	for all \(z\) in a neighborhood of \(z_0\).\\

	If \(f\) has a power series expansion at every point in the open set, it is \textbf{analytic} on the open set.
\end{defn}
It follows immediately that an analytic function on \(\Omega \) is holomorphic on \(\Omega \). We will later show the converse.

\begin{anki}
START
MathJaxCloze
Text: A function \(f\) defined on an open set is said to be **analytic** at a point \(z_0\) if {{c1::there exists a power series centered at \(z_0\)}} with {{c1::positive radius of convergence}} such that
{{c1::\(\begin{align*}
        	f(z) = \sum_{n=0}^{\infty} a_n(z-z_0)^{n} 
        \end{align*}\)}}
for all \(z\) in a neighborhood of \(z_0\).

If \(f\) has {{c1::a power series expansion}} at every point in the open set, it is **analytic** on the open set.
Extra: An analytic function on \(\Omega\) is holomorphic on \(\Omega\) (the converse also holds)
Tags: analysis complex_analysis defn power_series complex_analyticity
<!--ID: 1624504054064-->
END
\end{anki}

\begin{prop}
	If \(f,g\) are power series which converge absolutely on \(D(z_0,R)\), then \(f+g\) and \(fg\) converge absolutely on \(D(z_0,R)\). Furthermore, if \(\alpha  \in \C\), then \(\alpha f\) converges absolutely on \(D(z_0,R)\). In fact, we have:
	\begin{align*}
		(f+g)(z-z_0) = f(z-z_0) + g(z-z_0)\\
		(fg)(z-z_0) = f(z-z_0)g(z-z_0)\\
		(\alpha f)(z-z_0) = \alpha f(z-z_0)
	\end{align*}
	for all \(z \in D(z_0,R)\).
\end{prop}

\begin{anki}
START
MathJaxCloze
Text: If \(f,g\) are power series which converge absolutely on \(D(z_0,R)\), then \(f+g\) and \(fg\) {{c1::converge absolutely on \(D(z_0,R)\)}}. Furthermore, if \(\alpha  \in \C\), then {{c1::\(\alpha f\)}} converges absolutely on \(D(z_0,R)\). In fact, we have:
 {{c1::\(\begin{align*}
         	(f+g)(z-z_0) = f(z-z_0) + g(z-z_0)\\
         	(fg)(z-z_0) = f(z-z_0)g(z-z_0)\\
         	(\alpha f)(z-z_0) = \alpha f(z-z_0)
         \end{align*}\)}} 
for all \(z \in D(z_0,R)\).
Tags: analysis complex_analysis power_series
<!--ID: 1624845302985-->
END
\end{anki}


This leads to the following theorem:
\begin{thm}
	\begin{enumerate}[(a).]
		\item Let \(f(z) = \sum_{n=0}^{\infty} a_n z^{n}\) be a non-constant power series with non-zero radius of convergence. If \(f(0) = 0\), then there exists a disc of radius \(s>0\) such that
			\begin{align*}
				f(z)\neq 0
			\end{align*}
			for all \(z \in D(0,s)\setminus\left\{ 0 \right\} \).
		\item Suppose that \(f,g\) are convergent power series with
			\begin{align*}
				f(z) = \sum_{n=0}^{\infty} a_n z^n\\
				g(z) = \sum_{n=0}^{\infty} b_n z^{n}.
			\end{align*}
			If \(f(z)=g(z)\) in any infinite set \(A\) with \(0 \in \overline{A}\), then \(f(z)=g(z)\) everywhere-- i.e. \(a_n = b_n\) for all \(n\).
	\end{enumerate}
\end{thm}
\begin{proof}
	
\end{proof}
This theorem is extremely useful for proving the uniqueness of holomorphic functions, as well as distinguishing holomorphic functions.

\begin{anki}
START
MathJaxCloze
Text: 
* Let \(f(z) = \sum_{n=0}^{\infty} a_n z^{n}\) be a non-constant power series with non-zero radius of convergence. If \(f(0) = 0\), then there exists a {{c1::disc of radius \(s>0\)}} such that
{{c1::\(\begin{align*}
        f(z)\neq 0
        \end{align*}\)}} 
for all {{c1::\(z \in D(0,s)\setminus\left\{ 0 \right\} \)}}.
* Suppose that \(f,g\) are convergent power series with
\(\begin{align*}
  f(z) = \sum_{n=0}^{\infty} a_n z^n\\
  g(z) = \sum_{n=0}^{\infty} b_n z^{n}.
  \end{align*}\)
If \(f(z)=g(z)\) in {{c2::any infinite set \(A\) with \(0 \in \overline{A}\)}}, then {{c2::\(f(z)=g(z)\) everywhere}}-- i.e. {{c2::\(a_n = b_n\) for all \(n\)::coefficients}}.
Tags: analysis complex_analysis power_series
<!--ID: 1624845303027-->
END
\end{anki}


\begin{prop}[Composition of Power Series]
	Let
	\begin{align*}
		f(z) = \sum_{n=0}^{\infty} a_n z^{n}\\
		g(z) = \sum_{n=0}^{\infty} b_n z^{n}
	\end{align*}
	be convergent power series, and assume that \(b_0 = 0\). If \(f(z)\) is absolutely convergent for \(z \in D(0,R)\), \(R>0\), and there exists an integer \(s>0\) so that
	\begin{align*}
		\sum_{n=0}^{\infty} \left| b_n \right| s^{n} \leq R
	\end{align*}
	then
	\begin{align*}
		h(z) = \sum_{n=0}^{\infty} a_n \left( \sum_{m=0}^{\infty} b_m z^{m} \right)^{n}
	\end{align*}
	converges absolutely for \(z \in D(0,s)\), and within this disc satisfies
	\begin{align*}
		h = f\circ g.
	\end{align*}
\end{prop}

\begin{anki}
START
MathJaxCloze
Text: Let
\(\begin{align*}
  	f(z) = \sum_{n=0}^{\infty} a_n z^{n}\\
  	g(z) = \sum_{n=0}^{\infty} b_n z^{n}
  \end{align*}\)
be convergent power series, and assume that \(b_0 = 0\). If \(f(z)\) is {{c1::absolutely convergent}} for {{c1::\(z \in D(0,R)\)}}, \(R>0\), and there exists {{c1::an integer \(s>0\)}} so that
{{c1::\(\begin{align*}
        	\sum_{n=0}^{\infty} \left| b_n \right| s^{n} \leq R
        \end{align*}\)}} 
then
\(\begin{align*}
  	h(z) = \sum_{n=0}^{\infty} a_n \left( \sum_{m=0}^{\infty} b_m z^{m} \right)^{n}
  \end{align*}\)
{{c1::converges absolutely}} for {{c1::\(z \in D(0,s)\)}}, and within this disc satisfies
{{c1::\(\begin{align*}
        	h = f\circ g.
        \end{align*}\)}} 
Tags: analysis complex_analysis power_series
<!--ID: 1624845303068-->
END
\end{anki}

There is a slightly more general form of power series that will be useful when discussing holomorphic functions.

\begin{defn}[Laurent Series]
A \textbf{Laurent series} is an infinite sum of monomials
\begin{align*}
	f(z) = \sum_{n-\infty}^{\infty} a_n (z-z_0)^{n}
\end{align*}
where \(\left\{ a_n \right\},z_0  \in \C\) and \(z\) is a complex variable. We say that the Laurent series \textbf{converges absolutely} on \(\Omega \subset \C\) if
\begin{align*}
	f^{+}(z) = \sum_{n=0}^{\infty} a_n (z-z_0)^{n}\\
	f^{-}(z) = \sum_{n=-\infty}^{-1} a_n (z-z_0)^{n}
\end{align*}
converges absolutely on \(\Omega \). Notice that if this holds, then
\begin{align*}
	f = f^{+}+ f^{-}.
\end{align*}
\end{defn}

\begin{anki}
START
MathJaxCloze
Text: A **Laurent series** is an infinite sum of monomials
 {{c1::\(\begin{align*}
         	f(z) = \sum_{n-\infty}^{\infty} a_n (z-z_0)^{n}
         \end{align*}\)}} 
where \(\left\{ a_n \right\},z_0  \in \C\) and \(z\) is a complex variable. We say that the Laurent series **converges absolutely** on \(\Omega \subset \C\) if
{{c1::\(\begin{align*}
         	f^{+}(z) = \sum_{n=0}^{\infty} a_n (z-z_0)^{n}\\
         	f^{-}(z) = \sum_{n=-\infty}^{-1} a_n (z-z_0)^{n}
         \end{align*}\)}} 
converges absolutely on \(\Omega \). Notice that if this holds, then
 {{c1::\(\begin{align*}
        	f = f^{+}+ f^{-}.
        \end{align*}\)}} 
Tags: analysis complex_analysis power_series defn
<!--ID: 1625522318703-->
END
\end{anki}


\subsection{Obtaining Convergence and Absolute Convergence}
\label{sub:obtaining_absolute_convergence}
We will briefly develop a few tools that will help us show a series is absolutely convergent. As discussed earlier, this is a vital step in obtaining holomorphicity of a function.

\begin{thm}[Weierstrass M test]
	Let \(\left\{ f_n \right\} \) be a sequence of real or complex-valued functions, and \(\left\{ A_n \right\} \) a sequence of non-negative real numbers so that
	\begin{align*}
		\left| f_n(z) \right| \leq A_n
	\end{align*}
	for all \(z\) in some region \(\Omega\). If the sum
	\begin{align*}
		\sum_{n=0}^{\infty} A_n
	\end{align*}
	converges, then
	\begin{align*}
		\sum_{n=0}^{\infty} f_n(z)
	\end{align*}
	converges absolutely and uniformly on \(\Omega\).\\

	In this case, we call \(A_n\) a \textbf{majorant} of the \textbf{minorant} \(\left\{ f_n \right\} \). 
\end{thm}

\begin{anki}
START
MathJaxCloze
Text: **Weierstrass M-test**
	Let \(\left\{ f_n \right\} \) be a sequence of real or complex-valued functions, and \(\left\{ A_n \right\} \) a sequence of non-negative real numbers so that
	{{c1::\(\begin{align*}
	        	\left| f_n(z) \right| \leq A_n
	        \end{align*}\)}} 
	for all \(z\) in some region \(\Omega\). If
	{{c1::\(\begin{align*}
	        	\sum_{n=0}^{\infty} A_n
	        \end{align*}\)}} 
	converges, then
	{{c1::\(\begin{align*}
	        	\sum_{n=0}^{\infty} f_n(z)
	        \end{align*}\)}} 
	converges absolutely and uniformly on \(\Omega\).

In this case, we call \(A_n\) a **majorant** of the **minorant** \(\left\{ f_n \right\} \). 
Tags: analysis complex_analysis power_series
<!--ID: 1624504054101-->
END
\end{anki}


We have another useful tool:

\begin{thm}[Abel's Limit Theorem]
	Let \(G(z)\) be a power series given by
	\begin{align*}
		G(z) = \sum_{n=0}^{\infty} a_n z^{n}
	\end{align*}
	Suppose that the sum below converges:
	\begin{align*}
		\sum_{n=0}^{\infty} a_n = a
	\end{align*}
	for some \(a \in \C\). Then
	\begin{align*}
		\lim_{z \to 1} G(z) = \sum_{n=0}^{\infty} a_n
	\end{align*}
	provided that \(z\) remains within a \textbf{Stolz sector}, that is, satisfies
	\begin{align*}
		\frac{\left| 1-z \right| }{1-\left| z \right| } \leq M
	\end{align*}
	for some \(M \in \R\).
\end{thm}
This is most useful when the radius of convergence of a power series is \(1\), as it can be used to find the limit of the power series from within the disc of convergence. Hence, even if the power series does not have a limit on the radius of convergence, we might be able to obtain an "inner limit" that we can use for calculations.

\begin{anki}
START
MathJaxCloze
Text: **Abel's Limit Theorem**
	Let \(G(z)\) be a power series given by
	\(\begin{align*}
	  	G(z) = \sum_{n=0}^{\infty} a_n z^{n}
	  \end{align*}\)
	Suppose that the sum below converges:
	\(\begin{align*}
	  	\sum_{n=0}^{\infty} a_n = a
	  \end{align*}\)
	for some \(a \in \C\). Then
	{{c1::\(\begin{align*}
	        	\lim_{z \to 1} G(z) = \sum_{n=0}^{\infty} a_n
	        \end{align*}\)}} 
	provided that \(z\) remains within a \textbf{Stolz sector}, that is, satisfies
	{{c1::\(\begin{align*}
	        	\frac{\left| 1-z \right| }{1-\left| z \right| } \leq M	
	        \end{align*}\)}} 
	for some \(M \in \R\).
Tags:  analysis complex_analysis power_series
<!--ID: 1624504054138-->
END
\end{anki}

One last tool that will be helpful is Stirling's formula.

\begin{prop}[Stirling's Formula]
	\begin{align*}
		\sqrt{2\pi n} (n^{n}e^{-n}) \leq n! \leq e\sqrt{n} (n^{n}e^{-n})
	\end{align*}
	for all \(n \in \N\). Furthermore, \(n!\) limits towards the lower bound-- that is,
	\begin{align*}
		\lim_{n \to \infty} \frac{n!}{\sqrt{2\pi n} (n^{n}e^{-n})} =1.
	\end{align*}
	Sometimes, the approximation is written by
	\begin{align*}
		\ln n! \approx n\ln n - n.
	\end{align*}
\end{prop}
This can be useful when comparing power series to show radius of convergence.

\begin{exmp}[Radius of Convergence of Various Series]
	Consider the power series given below by
	\begin{align}
		\sum_{n=0}^{\infty} n! z^{n}\\
		\sum_{n=0}^{\infty} \frac{1}{n!}z^{n}\\
		\sum_{n=0}^{\infty} \frac{n!}{n^{n}}z^{n}\\
	\end{align}
	The radius of convergence of the first power series is 0 because \(\frac{n^{n}}{e^{n}z^{n}}\) is unbounded as \(n\to \infty\). Likewise, \(\frac{z^{n}}{n^{n}e^{n}}\) approaches zero as \(n\to \infty\) and hence the second series has infinite radius of convergence. A similar trick can be used to show that the third series' radius of convergence is \(e\).\\

	In general the ratio test states that if
	\begin{align*}
		\lim_{n \to \infty} \frac{a_{n+1}}{a_n} = A\geq 0
	\end{align*}
	for positive numbers \(a_n\), then
	\begin{align*}
		\lim_{n \to \infty} a_n^{\sfrac{1}{n}}=A.
	\end{align*}
\end{exmp}

\begin{exmp}[Binomial Series]
	Let \(\alpha  \in \C\) be a non-zero complex number. The \textbf{binomial coefficients} are given by
	\begin{align*}
	{\alpha\choose{n}} := \frac{\alpha (\alpha-1)\ldots(\alpha -n+1)}{n!}\\
	{\alpha \choose{0}}=1
	\end{align*}
	and the \textbf{binomial series} by
	\begin{align*}
		B_\alpha (T) := \sum_{n=0}^{\infty} {\alpha \choose{n}}z^{n} = (1+z)^{\alpha }.
	\end{align*}
	One can check that the second equality indeed holds, and in fact the radius of convergence of the binomial series is \(1\) provided that \(\alpha \) is not an integer \(\geq 0\).
\end{exmp}
\begin{proof}
	Observe that
	\begin{align*}
		\left| \frac{ {\alpha \choose{n+1}}}{ {\alpha \choose{n}}} \right| = \left| \frac{\alpha -n}{n+1} \right| 
	\end{align*}
	and hence limits to \(1\). By the ratio test, the binomial sum has the radius of convergence desired.
\end{proof}

\end{document}



\chapter{Complex Integration}
\label{cha:complex_integration}

\section{Integration on Curves}
\label{sec:integration_on_curves}

\documentclass{memoir}
\usepackage{notestemplate}


\begin{document}

Before we define the complex line integral, we will construct a few definitions that will allow us to work in more generality.
\begin{defn}[Piecewise-smooth curve]
	A curve \(\gamma :[t_0,t_1]\to \C\) is piecewise-smooth if \(\gamma \) is continuous on \([t_0,t_1]\) and if there exist a set of points
	\begin{align*}
		t_0 = a_0<a_1<\ldots<a_n = t_1
	\end{align*}
	where \(\gamma \) is smooth on the intervals \([a_k,a_{k+1}]\).
\end{defn}
Because line integrals tend to have decomposability as a property, we only need piecewise-smooth curves to define integration, as we can choose our intervals so each integral is over smooth segments.

\begin{anki}
TARGET DECK
Complex Qual::Complex Analysis
START
MathJaxCloze
Text: A curve \(\gamma :[t_0,t_1]\to \C\) is piecewise-smooth if \(\gamma \) is continuous on \([t_0,t_1]\) and if there exist a set of points
 {{c1::\(\begin{align*}
         	t_0 = a_0<a_1<\ldots<a_n = t_1
         \end{align*}\)}}
where \(\gamma \) is {{c1::smooth on the intervals \([a_k,a_{k+1}]\)}}.
Tags: analysis complex_analysis complex_analyticity defn
<!--ID: 1625192000546-->
END
\end{anki}

\begin{defn}[Length]
	The length of a smooth curve \(\gamma:[t_0,t_1]\to \C\) is defined by
	\begin{align*}
		\textrm{length}(\gamma) = \int_{t_0}^{t_1} \left| \gamma '(t) \right| \,d t 
	\end{align*}
\end{defn}
Of course, this extends to piecewise-smooth curves if we integrate along the partition. That is, if \(\gamma \) is piecewise-smooth via a partition of the interval \([t_0,t_1]\) by \(\left\{ a_i \right\}_{i=0}^{n}\), then
\begin{align*}
	\textrm{length}(\gamma ) = \int_{t_0}^{t_1} \left| \gamma'(t) \right| \,d t = \sum_{i=0}^{n-1} \int_{a_i}^{a_{i+1}} \left| \gamma'(t) \right| \,d t.
\end{align*}
This construction is perfectly valid as we are using the real line integral to define these notions, and we urge the reader to review real line integrals if these notions are troubling.\\

We may choose to extend the notion of length to all curves \(\gamma \), and say that a curve is \textbf{unrectifiable} if \(\textrm{length}(\gamma ) = \infty\). Otherwise, if the curve has finite length, it is \textbf{rectifiable}. For a more formal construction, we refer the reader to the real analysis line integral construction once again.\\

\begin{anki}
START
MathJaxCloze
Text: The length of a smooth curve \(\gamma:[t_0,t_1]\to \C\) is defined by
 {{c1::\(\begin{align*}
        	\textrm{length}(\gamma) = \int_{t_0}^{t_1} \left| \gamma '(t) \right| \,d t 
        \end{align*}\)}}
Extra: Of course, this extends to piecewise-smooth curves if we integrate along the partition. That is, if \(\gamma \) is piecewise-smooth via a partition of the interval \([t_0,t_1]\) by \(\left\{ a_i \right\}_{i=0}^{n}\), then
\(\begin{align*}
  	\textrm{length}(\gamma ) = \int_{t_0}^{t_1} \left| \gamma'(t) \right| \,d t = \sum_{i=0}^{n-1} \int_{a_i}^{a_{i+1}} \left| \gamma'(t) \right| \,d t.
  \end{align*}\)
Tags: analysis complex_analysis complex_integration defn
<!--ID: 1625192000556-->
END
\end{anki}

One may be concerned that there may be curves for which this notion is not well-defined, but it turns out that line integrals are equivalent over reparametrizations, and with some other properties we assure the reader that the line integral is well-defined.

\begin{defn}[Equivalence]
	Two curves \(\gamma_1:[t_0,t_1]\to \C\) and \(\gamma_2:[\tau_1,\tau_2]\to \C\) are \textbf{equivalent} if there exists a strictly increasing, continuously differentiable bijection \(\varphi \) from \([t_0,t_1]\) onto \([\tau_1,\tau_2]\) so that
	\begin{align*}
		\gamma_1(t) = \gamma_2(\varphi (t)).
	\end{align*}
\end{defn}
This forms an equivalence class, with elements called \textbf{reparametrizations}.
\begin{lemma}
	Equivalent curves have the same length.
\end{lemma}
\begin{defn}[Reverse]
	The \textbf{reverse} of a curve \(\gamma:[t_0,t_1]\to \C\) is \(\gamma^{-}:[t_0,t_1]\to \C\) defined by 
\begin{align*}
	\gamma^{-}(t) := \gamma (t_0+t_1-t).
\end{align*}
\end{defn}

Now that we have established a standard for our curves, we finally introduce the complex line integral.

\begin{defn}[Complex Line Integral]
	Let \(\gamma :[t_0,t_1]\to \C\) be a piecewise-smooth curve and \(f\) a complex-valued function continuous on \(\gamma \). We define the \textbf{line integral of \(f\) on \(\gamma \)} by
	\begin{align*}
		\int_\gamma f \,d t := \int_{t_0}^{t_1} f(\gamma (t)) \gamma'(t)\,d t .
	\end{align*}
\end{defn}
One might object that the above definition is ill-defined, as we have not explicitly stated how to evaluate the right-hand side. We omit this in the definition above simply because we evaluate it piecewise. That is, if \(F:[a,b]\to \C\) is continuous with \(F = (u,v)\), then
\begin{align*}
	\int_{a}^{b} F \,d t = \left( \int_{a}^{b} u \,d t, \int_{a}^{b} v \,d t \right)  .
\end{align*}
Hence, for the above definition, \(f(\gamma (t))\gamma'(t)\) is a continuous function on the interval and decomposes into real and imaginary-valued parts, and so is well-defined.\\

Of course, actually computing this by hand is difficult at best and impossible at worst-- even if both \(f\) and \(\gamma \) are nice, there is no guarantee that \((f\circ \gamma )\gamma'\) will be nice.\\

\begin{anki}
START
MathJaxCloze
Text: Let \(\gamma :[t_0,t_1]\to \C\) be a piecewise-smooth curve and \(f\) a complex-valued function continuous on \(\gamma \). We define the **line integral of \(f\) on \(\gamma \)** by
 {{c1::\(\begin{align*}
         	\int_\gamma f \,d t := \int_{t_0}^{t_1} f(\gamma (t)) \gamma'(t)\,d t .
         \end{align*}\)}}
Extra: Computing this directly requires splitting each function into real and imaginary parts.
Tags: analysis complex_analysis complex_integration defn
<!--ID: 1625192000566-->
END
\end{anki}

The component-wise integral definition is convenient because it inherits the properties of the real line integral by construction. For posteriety, we include them here, but omit the proof.

\begin{prop}[Properties of Complex Line Integrals]
	Let \(\gamma\) be a piecewise-smooth curve and \(f,g\) complex-valued functions continuous on \(\gamma \). Then
	\begin{itemize}
		\item If \(\alpha ,\beta  \in \C\), then
			\begin{align*}
				\int _{\gamma }(\alpha f + \beta g) = \alpha \int_{\gamma }f + \beta \int_{\gamma }g.
			\end{align*}
		\item If \(\gamma^{-}\) is the reverse of \(\gamma \), then
			\begin{align*}
				\int_\gamma f = - \int_{\gamma^{-}}f.
			\end{align*}
		\item 
			\begin{align*}
				\left| \int_\gamma f(z) \,d z \right| \leq \sup_{z \in \gamma } \left| f(z) \right| \cdot \textrm{length}(\gamma ).
			\end{align*}
	\end{itemize}
\end{prop}

As we have seen previously, a curve has many different reparametrizations. In order to make sure our line integral definition behaves properly, we must check that the integral is independent of reparametrization.

\begin{hw}
	Let \(\gamma:[t_0,t_1] \to \C\) be a smooth curve and \(f\) a complex-valued function continuous on \(\gamma \). Let \(\overline{\gamma }\) be a reparametrization of \(\gamma \)-- prove that
	\begin{align*}
		\int_{\gamma }f \,d t = \int_{\overline{\gamma }}f \,d t.
	\end{align*}
\end{hw}

\end{document}


\section{Integration on Closed Curves}
\label{sec:integration_on_closed_curves}

\documentclass{memoir}
\usepackage{notestemplate}

% \begin{figure}[ht]
%     \centering
%     \incfig{riemmans-theorem}
%     \caption{Riemmans theorem}
%     \label{fig:riemmans-theorem}
% \end{figure}

\begin{document}

\begin{defn}[Primitive]
	Let \(f\) be a function on the open set \(\Omega\). A \textbf{primitive} for \(f\) on \(\Omega\) is a function \(F\) that is holomorphic on \(\Omega\) such that \(F'(z) = f(z)\) for all \(z \in \Omega\).
\end{defn}
In other words, a primitive is an antiderivative. We use this nomenclature to prevent confusion with the real counterpart.

\begin{thm}[Newton-Leibniz Formula]
	If a continuous function \(f\) has a primitive \(F\) in \(\Omega\), and \(\gamma\) is a curve in \(\Omega\) that begins at \(z_0\) and ends at \(z_1\), then
	\begin{align*}
		\int_{\gamma} f(z) \,d z = F(z_1)-F(z_0)
	\end{align*}
\end{thm}
Note that while \(F\) and hence \(f\) is necessarily holomorphic along \(\gamma \), holomorphicity of \(f\) is not sufficient to get the statement-- not all holomorphic functions have primitives.

\begin{cor}
	If \(\gamma\) is a closed curve in an open set \(\Omega\), and \(f\) is continuous and has a primitive in \(\Omega\), then
	\begin{align*}
		\int_{\gamma} f(z) \,d z = 0 
	\end{align*}
\end{cor}
We can actually use this to show that a function does not have a primitive by integrating closed curves and getting non-zero values. In fact, if a function does integrate to \(0\) for every closed curve, then it must have an antiderivative.

\begin{exmp}
	Consider the function
	\begin{align*}
		f(z) = \frac{1}{z}
	\end{align*}
	on the domain \(\C\setminus \left\{ 0 \right\} \). If we take \(\gamma \) to be the boundary of the unit disc, that is, \(\gamma  = e^{it}\) on the interval \([0,2\pi ]\) then we observe that
	\begin{align*}
		\int_{\gamma } f \,d t = 2\pi i
	\end{align*}
	and hence \(f\) does not have a primitive in \(\C\setminus\left\{ 0 \right\} \). This exemplifies why we cannot define \(\log(z)\) on \(\C\setminus\left\{ 0 \right\} \)-- we have to specify a branch cut. If we integrate \(f\) over \(\gamma \) in the branch cut, we will indeed get zero once again, and hence \(f\) has a primitive within the branch cut.
\end{exmp}


\end{document}


\section{Cauchy's Theorem}
\label{sec:cauchy_s_theorem}

\documentclass{memoir}
\usepackage{notestemplate}

%\logo{~/School-Work/Auxiliary-Files/resources/png/logo.png}
%\institute{Rice University}
%\faculty{Faculty of Whatever Sciences}
%\department{Department of Mathematics}
%\title{Class Notes}
%\subtitle{Based on MATH xxx}
%\author{\textit{Author}\\Gabriel \textsc{Gress}}
%\supervisor{Linus \textsc{Torvalds}}
%\context{Well, I was bored...}
%\date{\today}

%\makeindex

\begin{document}

% \maketitle

% Notes taken on 

\subsection{Free Homotopy of Curves}
\label{sub:free_homotopy_of_curves}

In order to state Cauchy's theorem in complete generality, we need to construct a notion of homotopy for curves.

\begin{defn}[Free Homotopy]
	A \textbf{free homotopy} of closed curves in \(\Omega \) is a continuous map \(\gamma(\tau,t)\) from \([0,1]\times [t_0,t_1]\) to \(\Omega \) such that
\begin{align*}
	\gamma(\tau,t_0) = \gamma(\tau,t_1)
\end{align*}
for every \(\tau\in [0,1]\). We can denote \(\gamma_\tau(t) := \gamma(\tau,t)\).\\

We say \(\gamma_0,\gamma_1\) are \textbf{homotopic} if there exists a free homotopy with \(\gamma(0,t) = \gamma_0\) and \(\gamma(1,t) = \gamma_1\).
\end{defn}
Of course, if \(\Omega \) is convex, then any two curves are automatically freely homotopic.


% \printindex
\end{document}


\documentclass{memoir}
\usepackage{notestemplate}

% \begin{figure}[ht]
%     \centering
%     \incfig{riemmans-theorem}
%     \caption{Riemmans theorem}
%     \label{fig:riemmans-theorem}
% \end{figure}

\begin{document}

In short, Cauchy's theorem tells us that the integral of a holomorphic function on the boundary of an open region in \(\C\) is zero. Proving this statement is non-trivial and requires a lot of intermediate theorems which will later be superceded by Cauchy's Theorem. Understanding the development of these intermediate theorems is important, but not necessary to commit to memory. We state and prove the intermediate steps for posteriety, but note that many of the theorems will go unused after this section.

\subsection{Goursat's Theorem}
\label{sec:goursat_s_theorem}
\begin{thm}
	Let \(\Omega \subset \C\) be an open set, and \(T\subset \Omega \) a triangle with interior in \(\Omega \), and let \(f\) be a holomorphic function on \(\Omega \). Then
	\begin{align*}
		\int_T f(z) \,d z = 0.
	\end{align*}
\end{thm}

This will be a useful start in the direction of proving Cauchy's Theorem. We can infer from Goursat's Theorem that holomorphic functions integrate to zero on convex polygons, as we can partition convex polygons into triangles. We will use this style of construction to build torwards Cauchy's Theorem.

\begin{proof}% Stein Shakarchi pg 34
	
\end{proof}

\subsection{Local existence of primitives and Cauchy's theorem in a disc}
\label{sub:local_existence_of_primitives_and_cauchy_s_theorem_in_a_disc}

We mentioned that holomorphicity of a function does not give existence of a primitive alone. However, we do have local existence-- this will prove to be a crucial step to obtaining Cauchy's Theorem.

\begin{thm}[Existence of Local Primitives]
	Let \(f\) be a complex-valued function. If \(f\) is holomorphic on an open disc \(D_r(z_0)\), then \(f\) has a primitive in \(D_r(z_0)\).
\end{thm}

\begin{cor}[Cauchy's theorem for a disc]
	If \(f\) is holomorphic in an open disc, then
	\begin{align*}
		\int_\gamma f(z) \,d z = 0
	\end{align*}
	for any closed curve \(\gamma\) in that disc.
\end{cor}

\begin{proof}[Proof on Open Discs]% Stein Shakarchi p.39
	
\end{proof}

As the proof indicates, we can extend this to simply connected regions. This brings us to the full generality of Cauchy's Theorem.

\begin{thm}[Cauchy's Theorem]
	Let \(\Omega \subset \C\) be an open set, and let \(f\) be a holomorphic function on \(\Omega \). Suppose that \(\gamma \) is a smooth closed curve in \(\Omega \). If \(\gamma \) is homotopic to a constant curve, then
	\begin{align*}
		\int_\gamma f \,d t = 0.
	\end{align*}
\end{thm}

In fact, we can state Cauchy's Theorem in slightly more generality that will allow us to construct and fully utilize Cauchy's Integral Formulas.

\begin{cor}
	Let \(\Omega \subset \C\) be an open set, and let \(f\) be a holomorphic function on \(\Omega \). Suppose that \(\gamma_0,\gamma_1\) are two closed curves in \(\Omega \) that are freely homotopic to each other. Then
	\begin{align*}
		\int_{\gamma_0} = \int_{\gamma_1}f.
	\end{align*}
\end{cor}

Before we see what results Cauchy's theorem gives us, we want to clarify to what degree we are able to extend holomorphic functions.

\begin{prop}[Removable Singularities]
	Let \(\Omega \subset \C\) be an open bounded set in \(\C\). Suppose that \(f\) is holomorphic in a subset \(\Omega \setminus \left\{ z_i \right\}_{i=1}^{\infty}\) where \(\left\{ z_i \right\}_{i=1}^{\infty}\) is isolated within \(\Omega \). Then provided that
	\begin{align*}
		\lim_{z \to z_i} (z-z_i)f(z) = 0
	\end{align*}
	then \(f\) can be analytically continued onto \(\Omega \). Furthermore, Cauchy's theorem applies for \(f\) on \(\Omega \).
\end{prop}
We call these points \textbf{removable singularities} because \(f\) can be effortlessly extended onto these singularities, and hence we can mostly ignore them.

\begin{anki}
TARGET DECK
Complex Qual::Complex Analysis
START
MathJaxCloze
Text: **Cauchy's Theorem**
{{c1::Let \(\Omega \subset \C\) be an open set, and let \(f\) be a holomorphic function on \(\Omega \). Suppose that \(\gamma \) is a smooth closed curve in \(\Omega \). If \(\gamma \) is homotopic to a constant curve, then
      \(\begin{align*}
        	\int_\gamma f \,d t = 0.
        \end{align*}\)}} 
Extra: In fact, we can state Cauchy's Theorem in slightly more generality.
Let \(\Omega \subset \C\) be an open set, and let \(f\) be a holomorphic function on \(\Omega \). Suppose that \(\gamma_0,\gamma_1\) are two closed curves in \(\Omega \) that are freely homotopic to each other. Then
\(\begin{align*}
  	\int_{\gamma_0} = \int_{\gamma_1}f.
  \end{align*}\)
Tags: analysis complex_analysis complex_integration
<!--ID: 1625364556702-->
END
\end{anki}

\begin{anki}
START
MathJaxCloze
Text: Let \(\Omega \subset \C\) be an open bounded set in \(\C\). Suppose that \(f\) is holomorphic in a subset \(\Omega \setminus \left\{ z_i \right\}_{i=1}^{\infty}\) where \(\left\{ z_i \right\}_{i=1}^{\infty}\) is isolated within \(\Omega \). Then provided that
 {{c1::\(\begin{align*}
         	\lim_{z \to z_i} (z-z_i)f(z) = 0
         \end{align*}\)}} 
	then \(f\) can be analytically continued onto \(\Omega \). Furthermore, Cauchy's theorem applies for \(f\) on \(\Omega \). We call the points \(\left\{ z_i \right\}_{i=1}^{\infty}\) **removable singularities**.
Tags: analysis complex_analysis singularities_residues defn
END
\end{anki}


\end{document}


\section{Cauchy's Integral Formulas}
\label{sec:cauchy_s_integral_formulas}

\documentclass{memoir}
\usepackage{notestemplate}

%\logo{~/School-Work/Auxiliary-Files/resources/png/logo.png}
%\institute{Rice University}
%\faculty{Faculty of Whatever Sciences}
%\department{Department of Mathematics}
%\title{Class Notes}
%\subtitle{Based on MATH xxx}
%\author{\textit{Author}\\Gabriel \textsc{Gress}}
%\supervisor{Linus \textsc{Torvalds}}
%\context{Well, I was bored...}
%\date{\today}

%\makeindex

\begin{document}

% \maketitle

% Notes taken on 07/03/21

Before we explicitly state Cauchy's integration formulas, it will be beneficial to look at a particular class of curves. This will give us important intuition when working more generally with Cauchy's integration formulas later.

\subsection{Winding Number}
\label{sub:winding_number}

Recall that if \(\gamma \) is the boundary of the unit disc, then
\begin{align*}
	\int_{\gamma } \frac{1}{z} \,d t = 2\pi i. 
\end{align*}
When we initially discussed this notion, we implicitly took that \(\gamma \) was simple. Instead, consider the situation where \(\gamma(t) = e^{2it}\) on the interval \([0,2\pi ]\). Then one can check that
\begin{align*}
	\int_\gamma  \frac{1}{z}\,d t = 4\pi i.
\end{align*}
In fact, it happens that if \(\gamma(t) = e^{k(it)}\) for \(k \in \Z\), then
\begin{align*}
	\int_\gamma \frac{1}{z}\,d t = (2\pi i)k.
\end{align*}

We refer to \(k\) as the \textbf{winding number of \(\gamma \) around zero}. Now we construct this notion more generally.

\begin{lemma}
	Let \(\gamma \) be a piecewise smooth closed curve, and assume that \(\gamma (t)\neq z_0\) for all \(t \in [t_0,t_1]\). Then
	\begin{align*}
		\int_{\gamma } \frac{1}{z-z_0}\,d t = k(2\pi i) 
	\end{align*}
	for some \(k \in \Z\).
\end{lemma}
\begin{proof}
	
\end{proof}

\begin{defn}[Winding Number]
	The \textbf{winding number of \(z_0\) with respect to \(\gamma \)}, also called the \textbf{index of \(z_0\) with respect to \(\gamma \)} is defined by
	\begin{align*}
		n(\gamma ,z_0) = \frac{1}{2\pi i} \int_\gamma \frac{1}{z-z_0}\,d t.
	\end{align*}
\end{defn}

\begin{anki}
TARGET DECK
Complex Qual::Complex Analysis
START
MathJaxCloze
Text: The **winding number of \(z_0\) with respect to \(\gamma \)**, also called the **index of \(z_0\) with respect to \(\gamma \)** is defined by
 {{c1::\(\begin{align*}
         	n(\gamma ,z_0) = \frac{1}{2\pi i} \int_\gamma \frac{1}{z-z_0}\,d t.
         \end{align*}\)}}
Extra: This follows from the fact that if \(\gamma(t) = e^{k(it)}\) for \(k \in \Z\), then
\(\begin{align*}
  	\int_\gamma \frac{1}{z}\,d t = (2\pi i)k.
  \end{align*}\)
Tags: analysis complex_analysis complex_integration defn
<!--ID: 1625372340718-->
END
\end{anki}


\begin{prop}[Properties of Winding Numbers]
	\begin{itemize}
		\item \(n(-\gamma ,z_0) = -n(\gamma,z_0)\) 
		\item If \(z_0\) lies on the exterior of \(\gamma \), then \(n(\gamma ,z_0) = 0\)
		\item \(n(\gamma ,z_0)=n(\gamma ,z_1)\) if there exists a path between \(z_0\) and \(z_1\) that does not intersect \(\gamma \) 
	\end{itemize}
\end{prop}

In fact, we can define the notion of winding number and obtain the first two properties even if \(\gamma \) is only continuous. This goes a long way to classifying curves as a whole, and in fact is useful in proving the Jordan curve theorem.

\begin{anki}
START
MathJaxCloze
Text: Let \(\gamma \) be a piecewise smooth curve and \(z_0\) an arbitrary point in \(\C\).

* {{c1::\(n(-\gamma ,z_0) = -n(\gamma,z_0)\)::reverse of curve}}  
* If \(z_0\) {{c2::lies on the exterior of \(\gamma \)}}, then \(n(\gamma ,z_0) = 0\)
* \(n(\gamma ,z_0)=n(\gamma ,z_1)\) if there {{c3::exists a path between \(z_0\) and \(z_1\) that does not intersect \(\gamma \)}} 
Extra: The first two properties hold even if \(\gamma \) is only continuous (instead of piecewise-smooth)
Tags: analysis complex_analysis complex_integration
<!--ID: 1625372340727-->
END
\end{anki}


\begin{exmp}[Jordan Curve Theorem]
	The Jordan curve theorem states that every Jordan curve in the plane determines exactly two regions. We will prove via winding numbers that the complement of a Jordan curve \(\gamma \) has at least two components. Observe that this holds if there exists a point \(z_0\) such that \(n(\gamma ,z_0)\neq 0\).\\

	Without loss of generality, assume that \(\textrm{Re}(\gamma )>0\) and furthermore that there exist points \(z_1,z_2 \in \gamma (t)\) so that \(\textrm{Im}(z_1)<0\) and \(\textrm{Im}(z_2)>0\). In fact, we can choose \(z_1,z_2\) so that the line segments \(\ell(0,z_1)\) and \(\ell(0,z_2)\) do not intersect \(\gamma \).\\

	Let \(\gamma_1\) be the subcurve between \(z_1,z_2\), and \(\gamma_2\) the subcurve that takes \(z_2\) to \(z_1\) along \(\gamma \) (so that \(\gamma = \gamma_1+\gamma_2\)). Let \(\sigma_1\) be the curve formed by following \(\ell(0,z_1) + \gamma_1 + \ell(0,z_2)^{^{-}}\), and let \(\sigma_2\) be the curve formed by following \(\ell(0,z_1) + \gamma_2^{-} + \ell(0,z_2)^{-}\) (so that \(\sigma_1-\sigma_2 = \gamma \)).\\

	Observe that the positive real axis intersects both \(\gamma_1,\gamma_2\) at points which we will refer to by \(x_1,x_2\) respectively. Then we show that:
	\begin{itemize}
		\item \(n(\sigma_1,x_2) = 0\) and hence \(n(\sigma_1,\gamma_2) = 0\) 
		\item \(n(\sigma_1,z) = n(\sigma_2,z) = 1\) for \(0<z<x_1\)
		\item \(n(\sigma_2,z_1) = 1\) and hence \(n(\sigma_2,\gamma_1) = 1\) 
	\end{itemize}
\end{exmp}

\subsection{Cauchy's Integral Formulas}
\label{sub:cauchy_s_integral_formulas}

\begin{thm}[Cauchy's Integral Formula]
	Let \(\Omega \subset \C\) be an open region and \(f\) a holomorphic function on \(\Omega \). Let \(\gamma \) be a closed curve in \(\Omega \) homologous to a point. Then for all \(z \in \Omega \) with \(n(\gamma ,z)\neq 0\),
	\begin{align*}
		f(z) = \frac{1}{n(\gamma,z)2\pi i} \int_\gamma \frac{f(\xi )}{\xi -z}\,d t.
	\end{align*}
	If we restrict our \(z\) further into a connected region with \(n(\gamma ,z)=1\), then we obtain Cauchy's integral formula:
	\begin{align*}
		f(z) = \frac{1}{2\pi i} \int_\gamma \frac{f(\xi )}{\xi -z}\,d t
	\end{align*}
	and hence defines a holomorphic function equal to \(f\) on this region.
\end{thm}
Note that we are implicitly requiring that \(z\) not be on the curve for \(\gamma \), as the winding number is not well-defined on the curve.

\begin{proof}% Ahlfors p.118
	
\end{proof}

\begin{anki}
START
MathJaxCloze
Text: Let \(\Omega \subset \C\) be an open region and \(f\) a holomorphic function on \(\Omega \). Let \(\gamma \) be a closed curve in \(\Omega \) homologous to a point. Then for all \(z \in \Omega \) with \(n(\gamma ,z)\neq 0\),
{{c1::\(\begin{align*}
f(z) = \frac{1}{n(\gamma,z)2\pi i} \int_\gamma \frac{f(\xi )}{\xi -z}\,d t.
\end{align*}\)}} 
If we restrict our \(z\) further into a connected region with \(n(\gamma ,z)=1\), then we obtain **Cauchy's integral formula**:
{{c1::\(\begin{align*}
	f(z) = \frac{1}{2\pi i} \int_\gamma \frac{f(\xi )}{\xi -z}\,d t
\end{align*}\)}}
and hence defines a holomorphic function equal to \(f\) on this region.
Extra: Note that we are implicitly requiring that \(z\) not be on the curve for \(\gamma \), as the winding number is not well-defined on the curve.
Tags: analysis complex_analysis complex_integration
<!--ID: 1625525402615-->
END
\end{anki}

This allows us to get infinite differentiability of holomorphic functions.

\begin{cor}
	Let \(f\) be a holomorphic function on an open set \(\Omega \) containing an open disc \(D_r(z_0)\subset \Omega \) for some \(z_0 \in \Omega \), \(r>0\), and let \(\gamma \) be a closed curve in \(D_r(z_0)\). Then \(f\) has infinitely many complex derivatives in \(\Omega \). Furthermore, for all \(z \in D_r(z_0)\), we have
	\begin{align*}
		f^{(n)}(z) = \frac{n!}{n(\gamma,z)2\pi i} \int_\gamma \frac{f(\xi )}{(\xi  -z)^{n+1}}\,d t
	\end{align*}
\end{cor}

Notice that because analytic functions are holomorphic, this also shows infinite complex differentiability of analytic functions. All that remains is to show that holomorphic functions are analytic.

\begin{anki}
START
MathJaxCloze
Text: Let \(f\) be a holomorphic function on an open set \(\Omega \) containing an open disc \(D_r(z_0)\subset \Omega \) for some \(z_0 \in \Omega \), \(r>0\), and let \(\gamma \) be a closed curve in \(D_r(z_0)\). Then \(f\) has infinitely many complex derivatives in \(\Omega \). Furthermore, for all \(z \in D_r(z_0)\), we have
 {{c1::\(\begin{align*}
        	f^{(n)}(z) = \frac{n!}{n(\gamma,z)2\pi i} \int_\gamma \frac{f(\xi )}{(\xi  -z)^{n+1}}\,d t
        \end{align*}\)::value of higher derivatives}} 
Extra: Notice that because analytic functions are holomorphic, this also shows infinite complex differentiability of analytic functions.
Observe that this also gives us uniform convergence on the higher order derivatives by simply taking the modulus of both sides:
\(\begin{align*}
  	\left| f^{(n)}(z_0) \right| \leq \frac{\left| f(z) \right|_{z \in \partial D_r(z_0)} n!}{r^{n}}.
  \end{align*}\)
Tags: analysis complex_analysis complex_integration
<!--ID: 1625525402632-->
END
\end{anki}


\begin{proof}
	
\end{proof}

Observe that this also gives us uniform convergence on the higher order derivatives by simply taking the modulus of both sides:
\begin{align*}
	\left| f^{(n)}(z_0) \right| \leq \frac{\left| f(z) \right|_{z \in \partial D_r(z_0)} n!}{r^{n}}.
\end{align*}

%\begin{cor}[Cauchy inequalities]
%       If \(f\) is holomorphic in an open set that contains the closure of a disc \(D\) centered at \(z_0\) of radius \(R\), then
%       \begin{align*}
%       	\left| f^{(n)}(z_0) \right| \leq \frac{n! \|f\|_C}{R^{n}}
%       \end{align*}
%       where \(\|f\|_C = \sup_{z \in C} \left| f(z) \right| \) denotes the supremum of \(\left| f \right| \) on the boundary of \(C\).
%\end{cor}

\begin{thm}[Taylor's Theorem]
	Suppose \(f\) is holomorphic in an open set \(\Omega \subset \C\). Let \(D_r(z_0)\) be a disc with radius \(r>0\) with \(\overline{D_r(z_0)}\subset \Omega \). Then \(f\) has a power series expansion at \(z_0\) :
	\begin{align*}
		f(z) = \sum_{n=0}^{\infty} a_n(z-z_0)^{n}\\
		a_n := \frac{f^{(n)}(z_0)}{n!} \text{ for all } n\geq 0
	\end{align*}
	which converges absolutely and uniformly for all \(z \in D_r(z_0)\).
\end{thm}
This finally gives us the equivalence between holomorphic and analytic functions.\\

The power series expansion of holomorphic functions converges uniformly, and hence this theorem is a stronger version of Taylor's theorem for complex functions. That is, a holomorphic function can be approximated by partial sums of the power series expansion with a remainder that limits to zero at \(z_0\).

\begin{proof}
	
\end{proof}

\begin{anki}
START
MathJaxCloze
Text: **Taylor's Theorem**
Suppose \(f\) is holomorphic in an open set \(\Omega \subset \C\). Let \(D_r(z_0)\) be a disc with radius \(r>0\) with \(\overline{D_r(z_0)}\subset \Omega \). Then \(f\) has a power series expansion at \(z_0\) :
 {{c1::\(\begin{align*}
         	f(z) = \sum_{n=0}^{\infty} a_n(z-z_0)^{n}\\
         	a_n := \frac{f^{(n)}(z_0)}{n!} \text{ for all } n\geq 0
         \end{align*}\)}}
which {{c1::converges absolutely and uniformly}}  for all \(z \in D_r(z_0)\).
Extra: The power series expansion of holomorphic functions converges uniformly, and hence this theorem is a stronger version of Taylor's theorem for complex functions. That is, a holomorphic function can be approximated by partial sums of the power series expansion with a remainder that limits to zero at \(z_0\).
Tags: analysis complex_analysis complex_integration
<!--ID: 1625525402654-->
END
\end{anki}

In fact, this isn't the strongest form we can get of Taylor's theorem. Utilizing our earlier construction of Laurent series allows us to obtain an equivalence within an annulus.
\begin{thm}
	Let \(\Omega \subset \C\) be an annulus centered at \(z_0\) with inner radius \(r\) and outer radius \(R\) satisfying \(0\leq r<R\). Suppose \(f\) is holomorphic on \(\Omega \)-- then for all \(s,S\) satisfying \(r<s<S<R\), \(f\) has a Laurent expansion
	\begin{align*}
		f(z) = \sum_{n=-\infty}^{\infty} a_n (z-z_0)^{n}\\
		a_n = \begin{cases}
			\frac{1}{2\pi i} \int_{\partial D_R(z_0)} \frac{f(\xi )}{(\xi -z_0)^{n+1}}\,d t & n\geq 0\\
			\frac{1}{2\pi i} \int_{\partial D_r(z_0)} \frac{f(\xi )}{(\xi -z_0)^{n+1}}\,d t & n< 0\\
		\end{cases}
	\end{align*}
	which converges absolutely and uniformly on \(s\leq \left| z \right| \leq S\).
\end{thm}
Thus, if \(f\) is holomorphic on \(D_r(z_0)\setminus\left\{ z_0 \right\} \) for some \(r>0\), then the theorem implies that \(f\) has a unique Laurent expansion in \(D_r(z_0)\). This Laurent expansion is not holomorphic on \(D_r(z_0)\), but is holomorphic on \(D_r(z_0)\setminus\left\{ z_0 \right\} \). This will be useful later when dealing with meromorphic functions.

\begin{anki}
START
MathJaxCloze
Text: Let \(\Omega \subset \C\) be an annulus centered at \(z_0\) with inner radius \(r\) and outer radius \(R\) satisfying \(0\leq r<R\). Suppose \(f\) is holomorphic on \(\Omega \)-- then for all \(s,S\) satisfying \(r<s<S<R\), \(f\) has a Laurent expansion
 {{c1::\(\begin{align*}
         	f(z) = \sum_{n=-\infty}^{\infty} a_n (z-z_0)^{n}\\
         	a_n = \begin{cases}
         		\frac{1}{2\pi i} \int_{\partial D_R(z_0)} \frac{f(\xi )}{(\xi -z_0)^{n+1}}\,d t & n\geq 0\\
         		\frac{1}{2\pi i} \int_{\partial D_r(z_0)} \frac{f(\xi )}{(\xi -z_0)^{n+1}}\,d t & n< 0\\
         	\end{cases}
         \end{align*}\)}} 
	which {{c1::converges absolutely and uniformly}} on \(s\leq \left| z \right| \leq S\).
Extra: Thus, if \(f\) is holomorphic on \(D_r(z_0)\setminus\left\{ z_0 \right\} \) for some \(r>0\), then the theorem implies that \(f\) has a unique Laurent expansion in \(D_r(z_0)\). This Laurent expansion is not holomorphic on \(D_r(z_0)\), but is holomorphic on \(D_r(z_0)\setminus\left\{ z_0 \right\} \).
Tags: analysis complex_analysis complex_integration
<!--ID: 1625525402672-->
END
\end{anki}


% \printindex
\end{document}


\documentclass{memoir}
\usepackage{notestemplate}

%\logo{~/School-Work/Auxiliary-Files/resources/png/logo.png}
%\institute{Rice University}
%\faculty{Faculty of Whatever Sciences}
%\department{Department of Mathematics}
%\title{Class Notes}
%\subtitle{Based on MATH xxx}
%\author{\textit{Author}\\Gabriel \textsc{Gress}}
%\supervisor{Linus \textsc{Torvalds}}
%\context{Well, I was bored...}
%\date{\today}

%\makeindex

\begin{document}

% \maketitle

% Notes taken on 07/04/21

\subsection{Applications of Cauchy's Integral Formulas}
\label{sub:applications_of_cauchy_s_integral_formulas}

\begin{defn}[Entire]
	A complex function \(f\) is \textbf{entire} if it is holomorphic on \(\C\).
\end{defn}

\begin{cor}[Liouville's Theorem]
	If \(f\) is entire and bounded, then \(f\) is constant.
\end{cor}
\begin{proof}
	
\end{proof}

\begin{anki}
TARGET DECK
Complex Qual::Complex Analysis
START
MathJaxCloze
Text: **Liouville's Theorem**
{{c1::If \(f\) is entire and bounded, then \(f\) is constant.}} 
Extra: Recall that an entire function is a function that is holomorphic on \(\C\).

We can use Liouville's theorem to prove the fundamental theorem of algebra.
Tags: analysis complex_analysis complex_analyticity
<!--ID: 1625527512323-->
END
\end{anki}


We can use Liouville's theorem to prove the fundamental theorem of algebra.

\begin{cor}
	Every non-constant polynomial \(p \in \C[x]\) has a root in \(\C\).
\end{cor}
\begin{proof}
	
\end{proof}

\begin{cor}
	Every polynomial \(p \in \C[x]\) of degree \(n\geq 1\) has exactly \(n\) roots in \(\C\). If \(\left\{ w_i \right\}_{i=1}^{n}\) are the roots of \(p\) so that \(p(w_i) = 0\), then it holds that
	\begin{align*}
		p(z) = a_n \prod_{i=1}^{n} (z-w_i).
	\end{align*}
\end{cor}
This follows by factoring \(p\) by the first root and then reapplying the fundamental theorem of algebra to the remaining polynomial.

\begin{anki}
START
MathJaxCloze
Text: **Fundamental Theorem of Algebra**
{{c1::Every non-constant polynomial \(p \in \C[x]\) has a root in \(\C\).}} 

Every polynomial \(p \in \C[x]\) of degree \(n\geq 1\) has {{c1::exactly \(n\)}}  roots in \(\C\). If \(\left\{ w_i \right\}_{i=1}^{n}\) are the roots of \(p\) so that \(p(w_i) = 0\), then it holds that
 {{c1::\(\begin{align*}
         	p(z) = a_n \prod_{i=1}^{n} (z-w_i).
         \end{align*}\)}}
Extra: The second part follows by factoring \(p\) by the first root and then reapplying the fundamental theorem of algebra to the remaining polynomial.
Tags: analysis complex_analysis complex_analyticity
<!--ID: 1625527512341-->
END
\end{anki}


\begin{cor}[Morera's Theorem]
	If \(f\) is continuous in an open bounded set \(\Omega \subset \C\), and if \(\int_\gamma f \,d t = 0\) for all closed curves \(\gamma \) in \(\Omega \), then \(f\) is holomorphic.
\end{cor}
\begin{proof}
	
\end{proof}

\begin{anki}
START
MathJaxCloze
Text: **Morera's Theorem**
If \(f\) is {{c1::continuous}} in an open bounded set \(\Omega \subset \C\), and if {{c1::\(\int_\gamma f \,d t = 0\)}} for all closed curves \(\gamma \) in \(\Omega \), then \(f\) is holomorphic.
Tags: analysis complex_analysis complex_integration
<!--ID: 1625527512362-->
END
\end{anki}


\begin{thm}[Weierstrass' Theorem]
	Let \(\left\{ f_n \right\}_{n=1}^{\infty}\) be a sequence of holomorphic functions on \(\Omega \subset \C\) that converges uniformly to a function \(f\) in every compact subset \(\overline{U}\subset \Omega \):
	\begin{align*}
		\lim_{n \to \infty} \left\{ f_n \right\}_{n=1}^{\infty} = f \text{ uniformly}.
	\end{align*}
	Then \(f\) is holomorphic on \(\Omega \). Furthermore, the sequence \(\left\{ f^{(k)}_n \right\}_{n=1}^{\infty}\) satisfies
	\begin{align*}
		\lim_{n \to \infty} \left\{ f^{(k)}_n\right\}_{n=1}^{\infty} = f^{(n)} \text{ uniformly} 
	\end{align*}
	on every compact subset of \(\Omega \).
\end{thm}

Obviously this theorem does not apply in the real case-- a sequence of continuously differentiable functions may not be differentiable.\\

\begin{anki}
START
MathJaxCloze
Text: **Weierstrass' Theorem**
Let \(\left\{ f_n \right\}_{n=1}^{\infty}\) be a sequence of holomorphic functions on \(\Omega \subset \C\) that converges uniformly to a function \(f\) in every compact subset \(\overline{U}\subset \Omega \):
\(\begin{align*}
  	\lim_{n \to \infty} \left\{ f_n \right\}_{n=1}^{\infty} = f \text{ uniformly}.
  \end{align*}\)
Then \(f\) is {{c1::holomorphic on \(\Omega \)}}. Furthermore, the sequence \(\left\{ f^{(k)}_n \right\}_{n=1}^{\infty}\) satisfies
 {{c1::\(\begin{align*}
        	\lim_{n \to \infty} \left\{ f^{(k)}_n\right\}_{n=1}^{\infty} = f^{(n)} \text{ uniformly} 
        \end{align*}\)}}
on every compact subset of \(\Omega \).
Extra: Obviously this theorem does not apply in the real case-- a sequence of continuously differentiable functions may not be differentiable.
Tags: analysis complex_analysis complex_analyticity
<!--ID: 1625527512378-->
END
\end{anki}

\begin{cor}
	Let \(\left\{ f_n \right\}_{n=1}^{\infty}\) be a sequence of non-vanishing holomorphic functions on \(\Omega \subset \C\) that converges uniformly to a holomorphic function \(f\) in every compact subset \(\overline{U}\subset \Omega \):
	\begin{align*}
		\lim_{n \to \infty} \left\{ f_n \right\}_{n=1}^{\infty} = f \text{ uniformly}.
	\end{align*}
	Then \(f(z)(\) is either identically zero or non-vanishing.
\end{cor}

\begin{thm}[Integration of Holomorphic Functions]
	Let \(F(z,t)\) be defined for \((z,t) \in \Omega  \times [0,1]\), where \(\Omega \subset \C\) is an open set. Suppose \(F\) satisfies the following properties:
	\begin{itemize}
		\item \(F(z,t)\) is holomorphic in \(z\) for each \(t\), and
		\item \(F\) is continuous on \(\Omega  \times [0,1]\)
	\end{itemize}
	Then \(f\) defined on \(\Omega \) by
	\begin{align*}
		f(z) = \int_{0}^{1} F(z,t)\,d t 
	\end{align*}
	is holomorphic.
\end{thm}
Some functions can only be defined by this form of integral, hence the theorem's usefulness.

\begin{anki}
START
MathJaxCloze
Text: Let \(F(z,t)\) be defined for \((z,t) \in \Omega  \times [0,1]\), where \(\Omega \subset \C\) is an open set. Suppose \(F\) satisfies the following properties:

* \(F(z,t)\) is holomorphic in \(z\) for each \(t\), and
* \(F\) is continuous on \(\Omega  \times [0,1]\)

Then \(f\) defined on \(\Omega \) by
 {{c1::\(\begin{align*}
         	f(z) = \int_{0}^{1} F(z,t)\,d t 
         \end{align*}\)}} 
is holomorphic.
Extra: Some functions can only be defined by this form of integral, hence the theorem's usefulness.
Tags: analysis complex_analysis complex_integration
<!--ID: 1625527512394-->
END
\end{anki}

Cauchy's theorem has far more applications, but we will have to develop other tools first before we can explore this further. Nevertheless, the tools above give the reader more than enough to approach the integration of many complex functions, and so we end this section with a few problems that illustrate the power of these tools.

% \printindex
\end{document}


\chapter{Singularities and Residues}
\label{cha:singularities_and_residues}

\documentclass{memoir}
\usepackage{notestemplate}

%\logo{~/School-Work/Auxiliary-Files/resources/png/logo.png}
%\institute{Rice University}
%\faculty{Faculty of Whatever Sciences}
%\department{Department of Mathematics}
%\title{Class Notes}
%\subtitle{Based on MATH xxx}
%\author{\textit{Author}\\Gabriel \textsc{Gress}}
%\supervisor{Linus \textsc{Torvalds}}
%\context{Well, I was bored...}
%\date{\today}

%\makeindex

\begin{document}

% \maketitle

% Notes taken on 

Cauchy's theorem gives us a rather detailed picture of holomorphic functions-- perhaps too detailed. While the properties we have shown are remarkable, they can be difficult to satisfy, and so holomorphic functions are too restrictive of a class. The problem is that many complex functions have isolated points-- singularities-- where the function is not well-defined. We can classify these singularities and deal with a larger class of complex functions-- meromorphic functions-- which behave nicely outside of these points, and are tameable near the singularities.

\subsection{Motivation}
\label{sub:motivation}

Before we discuss more generally singularities, it will be helpful to develop an intuition on a simpler class of complex functions.

\begin{defn}[Rational Functions]
	Let \(p,q \in \C[z]\) be two polynomials with no common factors. Then
	\begin{align*}
		r(z) = \frac{p(z)}{q(z)}
	\end{align*}
	is a \textbf{rational function} in the extended plane-- if \(q(z) = 0\), then we assign \(r(z) = \infty\). This function is continuous. We call the zeros of \(q\) the \textbf{poles} of \(r\), and the \textbf{order of a pole} is the order of the corresponding zero of \(q\).
\end{defn}
Note that the derivative
\begin{align*}
	r'(z) = \frac{p'(z)q(z) - q'(z)p(z)}{q(z)^2}
\end{align*}
exists only when \(q(z) \neq 0\), but as a rational function, it has the same poles as \(r(z)\), but with one additional order on the poles. Furthermore, note that if \(\lambda \in \C\) is a constant, then \(\lambda r\) has the same poles as \(r\) and hence the same order. Therefore, a rational function \(r\) of order \(p\) has \(p\) zeros and \(p\) poles, and every equation \(r(z) = \lambda \) has exactly \(p\) roots.\\

Finally, by partial fraction decomposition, there is a decomposition \(r(z) = g(z) + h(z)\), where \(g\) does not have a constant term, and \(h(z)\) is finite at infinity. The degree of \(g(z)\) is the order of the pole at \(\infty\) and so the polynomial is referred to the \textbf{singular part} of \(r(z)\) at \(\infty\). Another form that is useful is
\begin{align*}
	r(z) = g(z) + \sum_{j=1}^{q} g_j \left( \frac{1}{z-\beta_j} \right) 
\end{align*}
where \(\beta_j\) are the distinct finite poles of \(r(z)\).\\

With this definition alone, we can already introduce a powerful statement that follows from a theorem of Runge's that will give us uniform approximation of holomorphic functions:

\begin{thm}
	Let \(f\) be a holomorphic function on a neighborhood of a compact set \(K\). Then \(f\) can be approximated uniformly on \(K\) by rational functions with poles in \(K^{c}\).\\

	Furthermore, if \(K^{c}\) is connected, then \(f\) can be uniformly approximated by polynomials.
\end{thm}

\begin{anki}
TARGET DECK
Complex Qual::Complex Analysis
START
MathJaxCloze
Text: Let \(f\) be a holomorphic function on a neighborhood of a compact set \(K\). Then \(f\) can be {{c1::approximated uniformly}} on \(K\) by {{c1::rational functions with poles in \(K^{c}\)}}.

Furthermore, if \(K^{c}\) is connected, then \(f\) can be {{c1::uniformly approximated by polynomials}}.
Tags: analysis complex_analysis complex_analyticity
<!--ID: 1625608497996-->
END
\end{anki}


% \printindex
\end{document}


\section{Zeroes and Singularities}
\label{sec:zeroes_and_singularities}

\documentclass{memoir}
\usepackage{notestemplate}

%\logo{~/School-Work/Auxiliary-Files/resources/png/logo.png}
%\institute{Rice University}
%\faculty{Faculty of Whatever Sciences}
%\department{Department of Mathematics}
%\title{Class Notes}
%\subtitle{Based on MATH xxx}
%\author{\textit{Author}\\Gabriel \textsc{Gress}}
%\supervisor{Linus \textsc{Torvalds}}
%\context{Well, I was bored...}
%\date{\today}

%\makeindex

\begin{document}

% \maketitle

% Notes taken on 

\begin{defn}[Zero]
	Let \(\Omega \subset \C\) and \(z_0 \in \Omega \) be given. Suppose \(f\) is a complex-valued function holomorphic on \(D_r(z_0)\).Then \(z_0\) is a \textbf{zero of order \(k\)} for a unique \(k \in \Z_+\) if
	\begin{align*}
		0 < \lim_{z \to z_0} \frac{f(z)}{(z-z_0)^{k}} < \infty.
	\end{align*}
	It follows that \((z-z_0)^{-k}f\) is holomorphic and non-vanishing. We say that \(z_0\) is \textbf{simple} if \(k=1\).
\end{defn}
Recall that by analytic continuation, all zeroes of non-constant holomorphic functions are isolated. Of course, it is equivalent to say that \(z_0\) is a zero if \(f(z_0) = 0\)-- but we state it as above to capture the notion of multiplicity.\\

In general, we may write the order \(k\) of a point \(z_0\) by
\begin{align*}
	k_{z_0} := \textrm{ord}(z_0) := \textrm{order}(z_0)
\end{align*}
when the expression is unambiguous.

\begin{anki}
TARGET DECK
Complex Qual::Complex Analysis
START
MathJaxCloze
Text: Let \(\Omega \subset \C\) and \(z_0 \in \Omega \) be given. Suppose \(f\) is a complex-valued function holomorphic on \(D_r(z_0)\).Then \(z_0\) is a **zero of order \(k\)** for a unique \(k \in \Z_+\) if
 {{c1::\(\begin{align*}
         	0 < \lim_{z \to z_0} \frac{f(z)}{(z-z_0)^{k}} < \infty.
         \end{align*}\)}}
It follows that {{c1::\((z-z_0)^{-k}f\)}} is {{c1::holomorphic}} and {{c1::non-vanishing}}. We say that \(z_0\) is **simple** if {{c1::\(k=1\)}}.
Extra: Recall that by analytic continuation, all zeroes of non-constant holomorphic functions are isolated. Of course, it is equivalent to say that \(z_0\) is a zero if \(f(z_0) = 0\)-- but we state it as above to capture the notion of multiplicity.
Tags: analysis complex_analysis singularities_residues defn
<!--ID: 1625609844275-->
END
\end{anki}

\begin{thm}
	Let \(\Omega \subset \C\) be a simply-connected open set. Suppose \(f\)  is a complex-valued function holomorphic on \(\Omega \) with isolated zeros \(\left\{ z_i \right\}_{i=1}^{\infty} \subset \C\). For every closed curve \(\gamma\) in \(\Omega \) with \(\gamma(t) \neq \left\{ z_i \right\}_{i=1}^{\infty}\) for all \(t \in [t_0,t_1]\):
	\begin{align*}
		\sum_{i=1}^{\infty} n(\gamma,z_i) k_{z_i} = \frac{1}{2\pi i} \int_{\gamma} \frac{f'(z )}{f(z )}\,d t, 
	\end{align*}
	when the sum consists of finitely many non-zero terms.
\end{thm}
If we know that \(n(\gamma,z_i)\) must be either 0 or 1, then this formula yields the total number of zeros enclosed by \(\gamma\). We can use this formula to solve equations as well-- applying the theorem to \(f(z)-a\) allows one to count the number of solutions within the curve.

\begin{thm}[Local Correspondence]
	Let \(z_0 \in \C\) and let \(f\) be a complex-valued function holomorphic at \(z_0\). Suppose \(f(z_0)=w_0\), so that \(f(z)-w_0\) has a zero of order \(n\) at \(z_0\). For all small \(\varepsilon>0\), there exists a \(\delta >0\) such that, for all \(w\) satisfying
	\begin{align*}
		\left| w-w_0 \right| <\delta 
	\end{align*}
	the equation \(f(z)=w\) has exactly \(k\) roots in the disk \(\left| z-z_0 \right| <\varepsilon\).
\end{thm}
This gives us another way to rederive the open mapping theorem, the global maximum principle, and more.

There are a variety of circumstances under which a function may not be defined at a point. We first classify these singularities.\\

For posterity, we restate the definition of a removable singularity:
\begin{defn}[Removable Singularity]
	Let \(\Omega \subset \C\) and \(z_0\in \Omega \) be given. Suppose \(f\) is a complex-valued function holomorphic on \(D_r(z_0)\setminus \left\{ z_0 \right\} \) for some \(r>0\). Then \(z_0\) is a \textbf{removable singularity} if
	\begin{align*}
		\lim_{z \to z_0} (z-z_0)f(z) = 0.
	\end{align*}
	Stated less formally, \(z_0\) is a removable singularity if \(f\) can be analytically continued onto \(\Omega \).
\end{defn}

It naturally follows that the other types of singularities will not limit to zero.

\begin{defn}[Poles]
	Let \(\Omega \subset \C\) and \(z_0 \in \Omega \) be given. Suppose \(f\) is a complex-valued function holomorphic on \(D_r(z_0)\setminus\left\{ z_0 \right\} \). Then \(z_0\) is a \textbf{pole} or \textbf{infinite singularity} if
	\begin{align*}
		\lim_{z \to z_0} (z-z_0)f(z) = \infty.
	\end{align*}
\end{defn}
Note that this is equivalent to saying that
\begin{align*}
	\lim_{z \to z_0} f(z) = \infty.
\end{align*}
Furthermore, observe that by construction, poles are necessarily isolated.

\begin{anki}
START
MathJaxCloze
Text: Let \(\Omega \subset \C\) and \(z_0 \in \Omega \) be given. Suppose \(f\) is a complex-valued function holomorphic on \(D_r(z_0)\setminus\left\{ z_0 \right\} \). Then \(z_0\) is a **pole** or **non-essential singularity** if
{{c1::\(\begin{align*}
        	\lim_{z \to z_0} (z-z_0)f(z) = \infty.
        \end{align*}\)}}
Extra: Note that this is equivalent to saying that
\(\begin{align*}
  	\lim_{z \to z_0} f(z) = \infty.
  \end{align*}\)
Furthermore, observe that by construction, poles are necessarily isolated.
Tags: analysis complex_analysis singularities_residues defn
<!--ID: 1625609844293-->
END
\end{anki}

\begin{prop}[Pole Equivalence and Order of Pole]
	Let \(f\) be a complex-valued function defined on \(D_r(z_0)\setminus \left\{ z_0 \right\} \subset \C\) for some \(r>0\). Then \(f\) has a pole at \(z_0\) if and only if
	\begin{align*}
		g(z) := \begin{cases}
			\frac{1}{f(z)} & z\neq z_0\\
			0 & z=z_0
		\end{cases}
	\end{align*}
	is holomorphic on \(D_r(z_0)\). Furthermore, there exists a unique positive integer \(k\) so that
	\begin{align*}
		0 < \lim_{z \to z_0} \frac{(z-z_0)^{k}}{f(z)}< \infty.
	\end{align*}
	We call \(k\) the \textbf{order of the pole}. If \(k=1\), then the pole is \textbf{simple}.
\end{prop}
Recall from our discussion on Cauchy's integral formula that \(f\) has a unique Laurent expansion locally at a pole \(z_0\). We point out that the above definition tells us that the Laurent expansion at \(z_0\) has finitely many negative terms-- in fact, it has negative terms up to \(-k\). That is, \(f\) can be expressed locally at \(z_0\) by
\begin{align*}
	f(z) = \sum_{i=-k}^{\infty} a_n (z-z_0)^{n}.
\end{align*}

Notice that our definition of a pole for rational functions falls under this definition now.

\begin{anki}
START
MathJaxCloze
Text: Let \(f\) be a complex-valued function defined on \(D_r(z_0)\setminus \left\{ z_0 \right\} \subset \C\) for some \(r>0\). Then \(f\) has a pole at \(z_0\) if and only if
 {{c1::\(\begin{align*}
        	g(z) := \begin{cases}
        		\frac{1}{f(z)} & z\neq z_0\\
        		0 & z=z_0
        	\end{cases}
        \end{align*}\)}} 
	is holomorphic on \(D_r(z_0)\). Furthermore, there exists a unique positive integer \(k\) so that
	{{c1::\(\begin{align*}
	        	0 < \lim_{z \to z_0} \frac{(z-z_0)^{k}}{f(z)}< \infty.
	        \end{align*}\)}} 
We call \(k\) the **order of the pole**. If {{c1::\(k=1\)}}, then the pole is **simple**.
Extra: Recall from our discussion on Cauchy's integral formula that \(f\) has a unique Laurent expansion locally at a pole \(z_0\). We point out that the above definition tells us that the Laurent expansion at \(z_0\) has finitely many negative terms-- in fact, it has negative terms up to \(-k\). That is, \(f\) can be expressed locally at \(z_0\) by
\(\begin{align*}
  	f(z) = \sum_{i=-k}^{\infty} a_n (z-z_0)^{n}.
  \end{align*}\)
Tags: analysis complex_analysis singularities_residues defn
<!--ID: 1625609844310-->
END
\end{anki}

\begin{defn}[Essential Singularity]
	Let \(\Omega \subset \C\) and \(z_0 \in \Omega \) be given. Suppose \(f\) is a complex-valued function holomorphic on \(D_r(z_0) \setminus\left\{ z_0 \right\} \). Then \(z_0\) is an \textbf{essential singularity} of \(f\) if it is neither a pole nor a removable singularity.\\

	More formally, \(z_0\) is an essential singularity if
	\begin{align*}
		\lim_{z \to z_0} (z-z_0)f(z)
	\end{align*}
	does not exist.
\end{defn}
The definition is equivalent to the Laurent series of \(f\) at \(z_0\) having infinitely many negative terms.

\begin{anki}
START
MathJaxCloze
Text: Let \(\Omega \subset \C\) and \(z_0 \in \Omega \) be given. Suppose \(f\) is a complex-valued function holomorphic on \(D_r(z_0) \setminus\left\{ z_0 \right\} \). Then \(z_0\) is an **essential singularity** of \(f\) if {{c1::it is neither a pole nor a removable singularity}}.

More formally, \(z_0\) is an essential singularity if
{{c1::\(\begin{align*}
        	\lim_{z \to z_0} (z-z_0)f(z)
        \end{align*}\)
does not exist.}} 
Extra: The definition is equivalent to the Laurent series of \(f\) at \(z_0\) having infinitely many negative terms.
Tags: analysis complex_analysis singularities_residues defn
<!--ID: 1625609844327-->
END
\end{anki}


Thus, we have officially classified all isolated singularities as either:
\begin{itemize}
	\item a removable singularity
	\item an infinite singularity
	\item an essential singularity
\end{itemize}
We can characterize all but the essential singularity very succinctly:
\begin{defn}[Algebraic Order]
	Let \(\Omega \subset \C\) and \(z_0 \in \Omega \) be given. Suppose \(f\) is a complex-valued function holomorphic on \(D_r(z_0)\setminus\left\{ z_0 \right\} \). Suppose that for some \(k \in \Z\) and \(M \in \C\)
	\begin{align*}
	\lim_{z \to z_0} (z-z_0)^{k}f(z) =M
	\end{align*}
	Then \(k\) is the \textbf{algebraic order} of \(f\) at \(z_0\). If \(k>0\) and \(M\neq 0\), then \(z_0\) is a zero. If \(k=1\) and \(M=0\), then \(z_0\) is a removable singularity (and may or may not be a zero). If \(k=0\), then \(f\) is analytic and non-vanishing at \(z_0\). Finally, if \(k<0\) then \(z_0\) is a pole. 
\end{defn}
Notice that we make no comment on situations where \(k \in \Q\)-- this is because, remarkably, this situation cannot occur! In order for the limit to converge, \(k\) must be an integer.\\

\begin{anki}
START
MathJaxCloze
Text: Let \(\Omega \subset \C\) and \(z_0 \in \Omega \) be given. Suppose \(f\) is a complex-valued function holomorphic on \(D_r(z_0)\setminus\left\{ z_0 \right\} \). Suppose that for some \(k \in \Z\) and \(M \in \C\)
 {{c1::\(\begin{align*}
         \lim_{z \to z_0} (z-z_0)^{k}f(z) =M
         \end{align*}\)}}
	Then \(k\) is the **algebraic order** of \(f\) at \(z_0\). If {{c1::\(k>0\)}} and {{c1::\(M\neq 0\)}}, then \(z_0\) is a zero. If {{c1::\(k=1\)}} and {{c1::\(M=0\)}}, then \(z_0\) is a removable singularity (and may or may not be a zero). If {{c1::\(k=0\)}}, then \(f\) is analytic and non-vanishing at \(z_0\). Finally, if {{c1::\(k<0\)}} then \(z_0\) is a pole. 
Extra: Notice that we make no comment on situations where \(k \in \Q\)-- this is because, remarkably, this situation cannot occur! In order for the limit to converge, \(k\) must be an integer.
Tags: analysis complex_analysis singularities_residues defn
<!--ID: 1625609844344-->
END
\end{anki}

\begin{exmp}
	\begin{itemize}
		\item \(\frac{1}{z}\) 
		\item \(\frac{1}{\sin(z)}\)
	\end{itemize}
\end{exmp}

Essential singularities are naturally very difficult to characterize. To see this, we state a classical theorem of Weierstrass:
\begin{thm}[Casorati-Weierstrass]
	Let \(\Omega \subset \C\) and \(z_0 \in \Omega \) be given. Suppose \(f\) is a complex-valued function holomorphic on \(D_r(z_0)\setminus \left\{ z_0 \right\} \) with an essential singularity at \(z_0\). Then
	\begin{align*}
		\overline{f(D_r(z_0)\setminus\left\{ z_0 \right\} )} = \C.
	\end{align*}
	That is, the image of \(f\) in any neighborhood of \(z_0\) is dense in \(\C\)-- \(f\) comes arbitrarily close to every complex number.
\end{thm}
\begin{proof}
	
\end{proof}

\begin{anki}
START
MathJaxCloze
Text: **Casorati-Weierstrass**
Let \(\Omega \subset \C\) and \(z_0 \in \Omega \) be given. Suppose \(f\) is a complex-valued function holomorphic on \(D_r(z_0)\setminus \left\{ z_0 \right\} \) with an essential singularity at \(z_0\). Then
 {{c1::\(\begin{align*}
         	\overline{f(D_r(z_0)\setminus\left\{ z_0 \right\} )} = \C.
         \end{align*}\)}}
That is, the image of \(f\) in any neighborhood of \(z_0\) is {{c1::dense in \(\C\)}}-- \(f\) {{c1::comes arbitrarily close to every complex number}}.
Tags: analysis complex_analysis singularities_residues
<!--ID: 1625609844360-->
END
\end{anki}


We state but do not prove a stronger result:
\begin{thm}[Picard's Great Theorem]
	Let \(\Omega \subset \C\) and \(z_0 \in \Omega \) be given. Suppose \(f\) is a complex-valued function holomorphic on \(D_r(z_0)\setminus \left\{ z_0 \right\} \) with an essential singularity at \(z_0\). Then for all \(z \in \C\) with at most one exception,
	\begin{align*}
		f^{-1}(z) \cap D_r(z_0) \neq \emptyset.
	\end{align*}
	Moreover, the intersection has infinitely many elements. In other words, \(f(z)\) takes on all possible complex values (with at most one exception) infinitely often in a neighborhood of \(z_0\).
\end{thm}

\begin{cor}
	The only holomorphic automorphisms of \(\C\) are functions of the form \(f(z) = az+b\) for \(a,b \in \C\) with \(a\neq 0\).
\end{cor}

\begin{anki}
START
MathJaxCloze
Text: **Picard's Great Theorem**
Let \(\Omega \subset \C\) and \(z_0 \in \Omega \) be given. Suppose \(f\) is a complex-valued function holomorphic on \(D_r(z_0)\setminus \left\{ z_0 \right\} \) with an essential singularity at \(z_0\). Then for all \(z \in \C\) with at most one exception,
{{c1::\(\begin{align*}
        	f^{-1}(z) \cap D_r(z_0) \neq \emptyset.
        \end{align*}\)}}
Moreover, the intersection has {{c1::infinitely many elements}}. In other words, \(f(z)\) {{c1::takes on all possible complex values (with at most one exception) infinitely often}} in a neighborhood of \(z_0\).
Tags: analysis complex_analysis singularities_residues
<!--ID: 1625609844377-->
END
\end{anki}


Picard also gave a stronger version of Liouville's theorem, which we state now:
\begin{thm}[Picard's Little Theorem]
	If \(f\) is entire and non-constant, then \(f(\C) = \C\) or \(f(\C) = \C\setminus\left\{ z_0 \right\} \) for some \(z_0 \in \C\).
\end{thm}
Hence, if \(f\) omits even two points from its range, it must be constant. We do not have the tools to prove this yet, but the intuition behind this is that if \(f(z)\) doesn't hit a value \(z_0\), then \(f(z)-z_0 = e^{p(z)}\) for some polynomial \(p \in \C[z]\).\\

\begin{anki}
START
MathJaxCloze
Text: **Picard's Little Theorem**
If \(f\) is entire and non-constant, then {{c1::\(f(\C) = \C\)}} or {{c1::\(f(\C) = \C\setminus\left\{ z_0 \right\} \)}} for some \(z_0 \in \C\).
Extra: Hence, if \(f\) omits even two points from its range, it must be constant. We do not have the tools to prove this yet, but the intuition behind this is that if \(f(z)\) doesn't hit a value \(z_0\), then \(f(z)-z_0 = e^{p(z)}\) for some polynomial \(p \in \C[z]\).
Tags: analysis complex_analysis singularities_residues
<!--ID: 1625609844393-->
END
\end{anki}

Before we look closer at functions with these singularities, we note that we can technically include \(z_0=\infty\) during the discussion of these singularities. This is the idea behind the Riemann sphere-- and hence all the discussion above is perfectly valid provided we clarify what some of our constructions mean at \(z_0 = \infty\).

\begin{defn}[Singularity at Infinity]
	Let \(f\) be a function holomorphic on \(\C\setminus D_R(0)\) for some \(R>0\). Then we say \(f\) has a \textbf{singularity at infinity} if
	\begin{align*}
		F(z) := f(\sfrac{1}{z})
	\end{align*}
	has a singularity at zero. Furthermore, the type of singularity \(f\) has at infinity is defined to be the type of singularity \(F\) has at zero.\\

	If \(f\) has a non-essential singularity at infinity, then \(f\) is said to be \textbf{meromorphic in the extended complex plane}.
\end{defn}

\begin{prop}
	If \(f\) is a function meromorphic in the extended complex plane, then \(f\) is a rational function of the form \(\frac{p(z)}{q(z)}\) for \(p,q \in \C[z]\).
\end{prop}

\begin{anki}
START
MathJaxCloze
Text: Let \(f\) be a function holomorphic on \(\C\setminus D_R(0)\) for some \(R>0\). Then we say \(f\) has a **singularity at infinity** if
 {{c1::\(\begin{align*}
         	F(z) := f(\sfrac{1}{z})
         \end{align*}\)}}
has {{c1::a singularity at zero}}. Furthermore, the type of singularity \(f\) has at infinity is defined to be the type of singularity \(F\) has {{c1::at zero}}.

If \(f\) has a non-essential singularity at infinity, then \(f\) is said to be **meromorphic in the extended complex plane**.
Extra: If \(f\) is a function meromorphic in the extended complex plane, then \(f\) is a rational function of the form \(\frac{p(z)}{q(z)}\) for \(p,q \in \C[z]\).
Tags: analysis complex_analysis singularities_residues defn
<!--ID: 1625688164963-->
END
\end{anki}


% \printindex
\end{document}


\section{Calculus of Residues}
\label{sec:calculus_of_residues}

\documentclass{memoir}
\usepackage{notestemplate}

%\logo{~/School-Work/Auxiliary-Files/resources/png/logo.png}
%\institute{Rice University}
%\faculty{Faculty of Whatever Sciences}
%\department{Department of Mathematics}
%\title{Class Notes}
%\subtitle{Based on MATH xxx}
%\author{\textit{Author}\\Gabriel \textsc{Gress}}
%\supervisor{Linus \textsc{Torvalds}}
%\context{Well, I was bored...}
%\date{\today}

%\makeindex

\begin{document}

% \maketitle

% Notes taken on 

\subsection{Meromorphic Functions and Residues}
\label{sub:meromorphic_functions_and_residues}



\begin{defn}[Meromorphic Function]
	Let \(\Omega \subset \C\) be an open subset. Suppose that a function \(f\) has poles at \(\left\{ z_i \right\}_{i=1}^{\infty}\). Then \(f\) is \textbf{meromorphic} if \(f\) is holomorphic on \(\Omega \setminus\left\{ z_i \right\}_{i=1}^{\infty}\).
\end{defn}

\begin{anki}
TARGET DECK
Complex Qual::Complex Analysis
START
MathJaxCloze
Text: Let \(\Omega \subset \C\) be an open subset. Suppose that a function \(f\) has {{c1::poles at \(\left\{ z_i \right\}_{i=1}^{\infty}\)}}. Then \(f\) is **meromorphic** if {{c1::\(f\) is holomorphic on \(\Omega \setminus\left\{ z_i \right\}_{i=1}^{\infty}\)}}.
Tags: analysis complex_analysis singularities_residues defn
<!--ID: 1626125582286-->
END
\end{anki}


\begin{exmp}[Polynomials]
	Let \(p \in \C[z]\) be a polynomial. Then \(f(z) = \frac{1}{p(z)}\) is a meromorphic function with poles at the zeros of \(p\).
\end{exmp}

Earlier, we showed that a function holomorphic on \(\Omega \) is analytic on \(\Omega \), and hence is equal to a unique power series within \(\Omega \). Because poles are isolated, if \(f\) is meromorphic on a region \(\Omega \), then there exists a Laurent expansion at each pole \(\left\{ z_i \right\}_{i=1}^{\infty}\) which is valid on some open neighborhood of each pole. The Laurent expansion carries useful information about the pole, which we capture with the notion of \textit{residues}.

\begin{defn}[Residue]
	Let \(\Omega \subset \C\) be an open subset and let \(f\) be meromorphic on \(\Omega \) with a pole at \(z_0\) of order \(k\), with Laurent expansion
	\begin{align*}
		f(z) = \sum_{n=-k}^{\infty} a_n (z-z_0)^{n}.
	\end{align*}
	The coefficient \(a_{-1}\) is the \textbf{residue} of \(f\) at \(z_0\), which we will denote \(\textrm{Res}_{z_0}f := a_{-1}\). Furthermore, the partial sum
	\begin{align*}
		\sum_{n=-k}^{-1} a_n (z-z_0)^{n}
	\end{align*}
	is called the \textbf{principal part} of \(f\) at \(z_0\).
\end{defn}

We can use residues to calculate integrals rather easily.

\begin{anki}
START
MathJaxCloze
Text: Let \(\Omega \subset \C\) be an open subset and let \(f\) be meromorphic on \(\Omega \) with a pole at \(z_0\) of order \(k\), with Laurent expansion
\(\begin{align*}
  	f(z) = \sum_{n=-k}^{\infty} a_n (z-z_0)^{n}.
  \end{align*}\)
The {{c1::coefficient \(a_{-1}\)}} is the **residue** of \(f\) at \(z_0\), which we will denote {{c1::\(\textrm{Res}_{z_0}f := a_{-1}\)}}. Furthermore, the partial sum
{{c1::\(\begin{align*}
      	\sum_{n=-k}^{-1} a_n (z-z_0)^{n}
        \end{align*}\)}} 
is called the **principal part** of \(f\) at \(z_0\).
Tags: analysis complex_analysis singularities_residues defn
<!--ID: 1626125582303-->
END
\end{anki}


\begin{thm}
	Let \(z_0 \in \C\) be given, and suppose \(f\) is a complex-valued function meromorphic on \(D_r(z_0)\) for some \(r>0\) with exactly one pole at \(z_0\). Then
	\begin{align*}
		\int_{\partial D_r(z_0)}f \,d t = 2\pi i \textrm{Res}_{z_0}f. 
	\end{align*}
\end{thm}

We can extend this to a more general residue formula.

\begin{cor}[Residue Formula]
	Let \(\Omega \subset \C\) be an open set and let \(\gamma \) be a closed piecewise-smooth curve in \(\Omega \) homologous to a point. Suppose \(f\) is a meromorphic function on \(\Omega \) with poles \(\left\{ z_i \right\}_{i=1}^{n}\). Then
	\begin{align*}
		\int_{\gamma } f \,d t = 2\pi i \sum_{j=1}^{n} n(\gamma ,z_j) \textrm{Res}_{z_j}f. 
	\end{align*}
\end{cor}

\begin{anki}
START
MathJaxCloze
Text: Let \(\Omega \subset \C\) be an open set and let \(\gamma \) be a closed piecewise-smooth curve in \(\Omega \) homologous to a point. Suppose \(f\) is a meromorphic function on \(\Omega \) with poles \(\left\{ z_i \right\}_{i=1}^{n}\). Then
{{c1::\(\begin{align*}
        	\int_{\gamma } f \,d t = 2\pi i \sum_{j=1}^{n} n(\gamma ,z_j) \textrm{Res}_{z_j}f. 
        \end{align*}\)::residue formula}} 
Tags: analysis complex_analysis singularities_residues
<!--ID: 1626125582321-->
END
\end{anki}

\begin{exmp}
	\(\int_{-\infty}^{\infty} \frac{1}{1+x^2}\,d x = \pi \)
\end{exmp}

\begin{exmp}
	
\end{exmp}

\begin{lemma}[Tools to Help Calculate Residues]
	\begin{itemize}
		\item Let \(f\) be a meromorphic function with a simple pole at \(z_0\), and suppose \(g\) is a function holomorphic at \(z_0\). Then
			\begin{align*}
				\textrm{Res}_{z_0}(fg) = g(z_0) \textrm{Res}_{z_0}(f).
			\end{align*}
		\item Let \(f\) be a function with a simple zero at \(z_0\). Then \(\sfrac{1}{f}\) has a simple pole at \(z_0\) with residue \(\sfrac{1}{f'(z_0)}\).
		\item Let \(f\) be a function with a pole of order \(n\) at \(z_0\). Then
			\begin{align*}
				\textrm{Res}_{z_0}f = \lim_{z \to z_0} \frac{1}{(n-1)!} \left( \frac{d}{\,d z} \right)^{n-1}(z-z_0)^{n}f(z).
			\end{align*}
	\end{itemize}
\end{lemma}

\begin{anki}
START
MathJaxCloze
Text: 
* Let \(f\) be a meromorphic function with a simple pole at \(z_0\), and suppose \(g\) is a function holomorphic at \(z_0\). Then
 {{c1::\(\begin{align*}
         	\textrm{Res}_{z_0}(fg) = g(z_0) \textrm{Res}_{z_0}(f).
         \end{align*}\)::residue of product}} 
* Let \(f\) be a function with a simple zero at \(z_0\). Then {{c2::\(\sfrac{1}{f}\)}} has a simple pole at \(z_0\) with residue {{c2::\(\sfrac{1}{f'(z_0)}\)}} .
* Let \(f\) be a function with a pole of order \(n\) at \(z_0\). Then
{{c1::\(\begin{align*}
        	\textrm{Res}_{z_0}f = \lim_{z \to z_0} \frac{1}{(n-1)!} \left( \frac{d}{\,d z} \right)^{n-1}(z-z_0)^{n}f(z).
        \end{align*}\)::residue of point via higher derivatives}} 
Tags: analysis complex_analysis singularities_residues
<!--ID: 1626125582338-->
END
\end{anki}


%\begin{defn}[Bounding]
%	A cycle \(\gamma\) is said to bound the region \(M\) if and only if \(n(\gamma,a)\) is defined and equal to \(1\) for all points \(a \in M\) and either undefined or equal to zero for all points \(a\) not in \(M\).
%\end{defn}

\subsection{Argument Principle}
\label{sub:argument_principle}

There is a deep connection between residues, winding numbers, and the complex logarithm. It turns out that all of these notions capture in some form the change in argument of a meromorphic function-- the theorem that ties these ideas together is what we refer to as the \textit{argument principle}.\\

Recall that the goal of the complex logarithm is to provide an inverse function for \(e^{x}\). In other words, we define the logarithm so that
\begin{align*}
	\ln(\left| z \right| e^{i \theta }) = \ln_\R (\left| z \right| ) + i \theta .
\end{align*}
We refer to \(\theta \) as \(\textrm{Arg}(z)\) more generally. The imaginary term alone captures in its entirety the angle of the input \(z\)-- this observation leads us to utilize the logarithm as an intermediate tool to capture more generally the change in angle of a function.

\begin{general}[Logarithmic Derivative]
	Let \(f\) be a complex-valued meromorphic function. Consider the composition \(\ln(f)\). Observe that
	\begin{align*}
		\frac{\partial }{\partial z} \ln(f(z)) = \frac{f'(z)}{f(z)}.
	\end{align*}
	We refer to \(\frac{f'}{f}\) as the \textbf{logarithmic derivative} of \(f\). \\

	Notably, the logarithmic derivative carries an additive formula:
	\begin{align*}
		\frac{\partial }{\partial z} \ln\left( \prod_{n=\infty}^{N} f_n  \right) = \sum_{n=1}^{N} \frac{f'_n}{f_n}.
	\end{align*}
\end{general}

\begin{anki}
START
MathJaxCloze
Text: Let \(f\) be a complex-valued meromorphic function. Consider the composition \(\ln(f)\). Observe that
 {{c1::\(\begin{align*}
         	\frac{\partial }{\partial z} \ln(f(z)) = \frac{f'(z)}{f(z)}.
         \end{align*}\)}} 
We refer to {{c1::\(\frac{f'}{f}\)}} as the **logarithmic derivative** of \(f\). 

Notably, the logarithmic derivative carries an additive formula:
{{c1::\(\begin{align*}
        	\frac{\partial }{\partial z} \ln\left( \prod_{n=\infty}^{N} f_n  \right) = \sum_{n=1}^{N} \frac{f'_n}{f_n}.
        \end{align*}\)}}
Extra: What is the connection between the logarithm and residues? Suppose \(f\) is holomorphic with a zero of order \(n\) at \(z_0\). Then
\(\begin{align*}
  	f(z) = (z-z_0)^{n}g(z)
  \end{align*}\)
for \(g\) holomorphic and non-vanishing in a neighborhood of \(z_0\). The logarithmic derivative of \(f\) is hence
\(\begin{align*}
  	\frac{f'(z)}{f(z)} = \frac{n}{z-z_0}+ \frac{g'(z)}{g(z)}
  \end{align*}\)
by the additive property of the logarithmic derivative. But \(g\) and \(g'\) are holomorphic and non-vanishing-- and hence the logarithmic derivative has transformed \(f\) into a meromorphic function with a simple pole at \(z_0\) with residue \(\textrm{Res}_{z_0} = n\). Likewise, if \(f\) is meromorphic with a pole of order \(n\) at \(z_0\), then
\(\begin{align*}
  	f(z) = (z-z_0)^{-n}h(z)
  \end{align*}\)
for \(h\) holomorphic and non-vanishing. One can see the above follows but with \(-n\).
Tags: analysis complex_analysis singularities_residues defn
<!--ID: 1626125582356-->
END
\end{anki}


At this point, the connection between the logarithm and winding number is clear-- by performing a \(w\)-substitution with \(w = f(z)\), we see that that the logarithmic derivative is merely another way to write
\begin{align*}
	\frac{f'}{f} \,d z = \frac{dw}{w}
\end{align*}
and hence when integrated on a closed curve, captures the winding numbers of singularities of \(f\) inside the curve. Furthermore, we have the identity
\begin{align*}
	\int_\gamma \,d \ln(f(z)) = \int_\gamma \,d \left[ \ln \left| f(z) \right| + i \textrm{arg}(f(z)) \right] = \int_\gamma \,d \ln\left| f(z) \right| + \int_\gamma \,d \textrm{arg}(f(z)).
\end{align*}
Because the first integral always evaluates to zero (verify this), we see that the logarithmic derivative allows us to capture the change in argument of a function.\\

What is the connection between the logarithm and residues? Suppose \(f\) is holomorphic with a zero of order \(n\) at \(z_0\). Then
\begin{align*}
	f(z) = (z-z_0)^{n}g(z)
\end{align*}
for \(g\) holomorphic and non-vanishing in a neighborhood of \(z_0\). The logarithmic derivative of \(f\) is hence
\begin{align*}
	\frac{f'(z)}{f(z)} = \frac{n}{z-z_0}+ \frac{g'(z)}{g(z)}
\end{align*}
by the additive property of the logarithmic derivative. But \(g\) and \(g'\) are holomorphic and non-vanishing-- and hence the logarithmic derivative has transformed \(f\) into a meromorphic function with a simple pole at \(z_0\) with residue \(\textrm{Res}_{z_0} = n\). Likewise, if \(f\) is meromorphic with a pole of order \(n\) at \(z_0\), then
\begin{align*}
	f(z) = (z-z_0)^{-n}h(z)
\end{align*}
for \(h\) holomorphic and non-vanishing. One can see the above follows once more but with \(-n\).\\

In summary-- the logarithmic derivative transforms \textit{singularities of algebraic order \(n\)} into \textit{simple poles with residue \(n\)}. Thus, a contour integral of the logarithmic derivative will sum the algebraic orders of the singularities contained. This remarkable conclusion is the argument principle.

\begin{thm}[Argument Principle]
	If \(f\) is meromorphic in \(\Omega\subset \C \) with zeros \(\left\{ z_j \right\}\) and the poles \(\left\{ w_k \right\} \), then
	\begin{align*}
		\frac{1}{2\pi i} \int_{\gamma} \frac{f'}{f} \,d t = \sum_{j} n(\gamma,z_j) - \sum_{k} n(\gamma,w_k) 
	\end{align*}
	for every closed piecewise-smooth curve \(\gamma\) which is homologous to zero in \(\Omega \) and does not pass through any of the zeros or poles.
\end{thm}
In light of our earlier discussion, we refer to
\begin{align*}
	\frac{1}{2\pi i} \int_\gamma \frac{f'}{f}\,d t
\end{align*}
as the \textbf{winding number of \(f\) along \(\gamma \)}. We urge the reader to appreciate how remarkable it is that the change in angle of a meromorphic function can be captured so simply via its zeros and poles!

\begin{anki}
START
MathJaxCloze
Text: **Argument Principle**
If \(f\) is meromorphic in \(\Omega\subset \C \) with zeros \(\left\{ z_j \right\}\) and the poles \(\left\{ w_k \right\} \), then
{{c1::\(\begin{align*}
        	\frac{1}{2\pi i} \int_{\gamma} \frac{f'}{f} \,d t = \sum_{j} n(\gamma,z_j) - \sum_{k} n(\gamma,w_k) 
        \end{align*}\)}}
	for every closed piecewise-smooth curve \(\gamma\) which is homologous to zero in \(\Omega \) and does not pass through any of the zeros or poles.
Extra: We refer to
\(\begin{align*}
  	\frac{1}{2\pi i} \int_\gamma \frac{f'}{f}\,d t
  \end{align*}\)
as the \textbf{winding number of \(f\) along \(\gamma \)}.
Tags: analysis complex_analysis singularities_residues
<!--ID: 1626125582372-->
END
\end{anki}


\begin{cor}[Rouche's Theorem]
	Let \(\Omega \subset \C\) be an open bounded subset with \(\partial \Omega \) piecewise-smooth. Let \(f,g\) be holomorphic functions on \(\overline{\Omega }\) with \(\left| g(z) \right| < \left| f(z) \right| \) for all \(z \in \partial\Omega \). Then \(f\) and \(f+g\) have the same number of zeros in \(\Omega \).
\end{cor}
We can interpret \(g\) as a "holomorphic perturbation" of \(f\) in \(\Omega \)-- and hence the theorem really states that if the perturbation is bounded above by \(f\) along the boundary, then it cannot perturb any zero outside \(\Omega \). To see why this should be true, recall that the maximum principle tells us that \(g\) globally and locally achieves its maxima on the boundary. If \(g\) were to "push" a zero \(z_0\) of \(f\) outside \(\Omega \), then \(g\) would have to exceed \(f\) somewhere on the boundary for the maximum principle to hold. Formalizing this argument is difficult, however. The argument principle simplifies this process greatly.
\begin{proof}
	Note that the hypothesis implicitly requires that \(f\) and \(f+g\) are non-zero on \(\gamma \). Thus, we can rewrite
	\begin{align*}
		f+g &= f\left( 1+ \frac{g}{f} \right)
	\end{align*}
	Applying the logarithmic derivative gives us
	\begin{align*}
		\int_\gamma \frac{(f+g)'}{f+g}\,d t = \int_\gamma \frac{f'}{f} \,d t + \int_{\gamma }\frac{(1+\sfrac{g}{f})'}{1+\sfrac{g}{f}} \,d t.
	\end{align*}
	But we have that \(\left| \frac{g}{f} \right|<1\) on \(\gamma \), and so there are no zeros of \(\frac{g}{f}\) within \(\gamma \), and likewise no poles (because \(\frac{g}{f}\) is holomorphic). Hence, the argument principle tells us that the right-hand integral is zero, and hence
	\begin{align*}
		\int_\gamma \frac{(f+g)'}{f+g} \,d t = \int_\gamma \frac{f'}{f}\,d t
	\end{align*}
	which implies they must have the same zeros.
\end{proof}

\begin{anki}
START
MathJaxCloze
Text: **Rouche's Theorem**
Let \(\Omega \subset \C\) be an open bounded subset with \(\partial \Omega \) piecewise-smooth. Let \(f,g\) be holomorphic functions on \(\overline{\Omega }\) with \(\left| g(z) \right| < \left| f(z) \right| \) for all \(z \in \partial\Omega \). Then {{c1::\(f\) and \(f+g\) have the same number of zeros in \(\Omega \)}}.
Extra: We can interpret \(g\) as a "holomorphic perturbation" of \(f\) in \(\Omega \)-- and hence the theorem really states that if the perturbation is bounded above by \(f\) along the boundary, then it cannot perturb any zero outside \(\Omega \).
Tags: analysis complex_analysis singularities_residues
<!--ID: 1626125582389-->
END
\end{anki}

Now we will look at some applications of the two powerful theorems here to evaluating integrals.

%% Deal with later

% \subsection{Definite Integrals}
% \label{sec:definite_integrals}
% 
% First, note that all integrals of the form
% \begin{align*}
% 	\int_{0}^{2\pi} R(\cos(\theta),\sin(\theta)) \,d \theta 
% \end{align*}
% where the integrand is a rational function of the two trigonometric functions can be done via residues. Substituting \(z = e^{i\theta}\) yields
% \begin{align*}
% 	-i \int_{\left| z \right| =1} R \left[ \frac{1}{2}\left( z+ \frac{1}{z} \right) , \frac{1}{2i}\left( z-\frac{1}{z} \right)  \right] \frac{\,d z}{z}.
% \end{align*}
% 
% --
% 
% An integral of the form
% \begin{align*}
% 	\int_{-\infty}^{\infty} R(x) \,d x 
% \end{align*}
% converges if and only if the rational function \(R(x)\) the degree of the denomination is at least 2 degrees higher tahn taht of the numerator, and if there is no pole ON the real axis. We do this by integrating the complex function \(R(z)\) over a closed curve consisting of a line segment \((-p,p)\) and the semicircle from \(p,-p)\) in the upper half plane. Choosing \(p\) large enough  encloses all poles in the upper half plane, and so the integral is equal to \(2\pi i\) times the sum of the residues in the upper half plane. So,
% \begin{align*}
% 	\int_{-\infty}^{\infty} R(x) \,d x = 2\pi i \sum_{y>0} \textrm{Res}R(z) 
% \end{align*}
% 
% -
% 
% We can do this same method for integrals of the form
% \begin{align*}
% 	\int_{-\infty}^{\infty} R(x) e^{ix} \,d x 
% \end{align*}
% whose real and imaginary parts determine the integrals
% \begin{align*}
% 	\int_{-\infty}^{\infty} R(x) \cos(x) \,d x, \quad \int_{-\infty}^{\infty} R(x) \sin(x) \,d x  
% \end{align*}
% Because \(e^{-y}\) is bounded in the upper half plane, so the integral over the semicircle tends to zero (as long as \(R(z)\) has a zero of at least order two at infinity). Thus
% \begin{align*}
% 	\int_{-\infty}^{\infty} R(x) e^{ix}\,d x = 2\pi i \sum_{y>0} \textrm{Res}R(z) e^{ix} .
% \end{align*}
% This holds when \(R(z)\) only has order one zero at infinity, but not by the semicircle argument.\\
% 
% Note that we assumed that \(R(z)\) has no poles on the real axis; however, if it coincides with zeros of \(\sin(x)\), then it very well can be evaluated! It will work out to yield
% \begin{align*}
% 	\int_{-\infty}^{\infty} R(x)e^{ix}\,d x = 2\pi i \sum_{y>0} \textrm{Res}R(z) e^{iz} + \pi i \sum_{y=0} \textrm{Res}R(z)e^{iz} 
% \end{align*}
% Often, integrals containing powers of cosine and sine can be written as linear combinations via double angle identities, and hence
% \begin{align*}
% 	\int_{-\infty}^{\infty} R(x) e^{imx}\,d x = \frac{1}{m} \int_{-\infty}^{\infty} R\left( \frac{x}{m} \right) e^{ix} \,d x.  
% \end{align*}
% 
% -
% 
% Now consider
% \begin{align*}
% 	\int_{0}^{\infty} x^{\alpha}R(x) \,d x 
% \end{align*}
% with \(\alpha \in (0,1)\subset \R\). This only converges if \(R(z)\) has a zero of at least order two at \(\infty\) and at most a simple pole at the origin. Using the substitution of \(x = t^2\) and then omitting the negative imaginary axis, we can apply the residue theorem to yield
% \begin{align*}
% 	(1-e^{2\pi i \alpha}) \int_{0}^{\infty} z^{2\alpha+1}R(z^2) \,d z 
% \end{align*}
% Thus we determine the residues of the integrand in the upper hlaf plane. This is the same as the residues of \(z^{\alpha}R(z)\) in the whole plane.
% \printindex
\end{document}


% Section on partial fractions? Ahlfors 5.2

\chapter{Conformal Mappings}
\label{cha:conformal_mappings}

\section{Automorphisms in the Complex Plane}
\label{sec:automorphisms_in_the_complex_plane}

\documentclass{memoir}
\usepackage{notestemplate}

%\logo{~/School-Work/Auxiliary-Files/resources/png/logo.png}
%\institute{Rice University}
%\faculty{Faculty of Whatever Sciences}
%\department{Department of Mathematics}
%\title{Class Notes}
%\subtitle{Based on MATH xxx}
%\author{\textit{Author}\\Gabriel \textsc{Gress}}
%\supervisor{Linus \textsc{Torvalds}}
%\context{Well, I was bored...}
%\date{\today}

%\makeindex

\begin{document}

% \maketitle

% Notes taken on 07/09/21

Recall that a holomorphic isomorphism is a holomorphic map
\begin{align*}
	f:U\to V
\end{align*}
with a holomorphic inverse
\begin{align*}
	g:V\to U
\end{align*}
so that
\begin{align*}
	f\circ g &= \textrm{Id}_V\\
	g\circ f &= \textrm{Id}_U
\end{align*}
We say that \(f\) is an \textbf{automorphism} if \(U=V\), and refer to the set of automorphisms of \(U\) by \(\textrm{Aut}(U)\).

\subsection{Automorphisms of the Unit Disc}
\label{sub:automorphisms_of_the_unit_disc}

\begin{thm}[Schwarz' Lemma]
	Let \(f:D_1 \to D_1\) be a holomorphic automorphism of the unit disc with \(f(0) = 0\). Then
	\begin{align*}
		\left| f(z) \right| \leq \left| z \right| 
	\end{align*}
	for all \(z \in D_1\). If for some \(z_0\neq 0\) we have \(\left| f(z_0) \right| = \left| z_0 \right| \), then
	\begin{align*}
		f(z) = e^{i \theta } z
	\end{align*}
	for some \(\theta \in \R\).

	In particular, \(\left| f'(0) \right| \leq 1\). If equality holds, then \(f(z) = e^{i\theta } z\) for some \(\theta \in \R\).
\end{thm}
In other words, an automorphism of the unit disc that fixes the origin has an implicit "contraction" requisite. If it isn't a contraction, then it must be a rotation.

\begin{proof}
	
\end{proof}%Lang p.210, via maximum modulus principle

\begin{anki}
TARGET DECK
Complex Qual::Complex Analysis
START
MathJaxCloze
Text: **Schwarz' Lemma**
Let \(f:D_1 \to D_1\) be a holomorphic automorphism of the unit disc with \(f(0) = 0\). Then
 {{c1::\(\begin{align*}
        	\left| f(z) \right| \leq \left| z \right| 
        \end{align*}\)::modulus}}
for all \(z \in D_1\). If for some \(z_0\neq 0\) we have \(\left| f(z_0) \right| = \left| z_0 \right| \), then
 {{c1::\(\begin{align*}
        	f(z) = e^{i \theta } z
        \end{align*}\)::function is of form}}
for some \(\theta \in \R\).

In particular, {{c1::\(\left| f'(0) \right| \leq 1\)::modulus}}. If equality holds, then {{c1::\(f(z) = e^{i\theta } z\)::function is of form}} for some \(\theta \in \R\).
\end{thm}
Tags: analysis complex_analysis conformal_mappings
<!--ID: 1626293841377-->
END
\end{anki}


We can apply Schwarz Lemma to classify all the automorphisms of the unit disc. Because every automorphism must send exactly one point to zero, we can transform the general case to the origin.

\begin{thm}[Structure of Automorphisms of the Unit Disc]
\label{thm:classification_of_automorphisms_of_the_unit_disc}
	Let \(\sigma :D_1\to D_1\) be a holomorphic automorphism of the unit disc, and let \(z_0\in D_1\) be the point for which \(\sigma (z_0 )=0\). Then for some \(\theta \in \R\)
	\begin{align*}
		\sigma (z) = e^{i\theta } \frac{z-z_0}{1-\overline{z_0}z}
	\end{align*}
\end{thm}

Observe that when \(\theta =2\pi k\) for \(k \in \Z\), the automorphism of the disc satisfies \(\sigma (z_0) = 0\), \(\sigma (0) = z_0\), and \(\sigma \circ \sigma  = \textrm{Id}_{D_1}\). For this reason, we will denote by \(\sigma_{z_0}\) the \textbf{canonical automorphism of the unit disc centered at \(z_0\)} when \(\theta =2\pi k\) and \(\sigma (z_0)=0\).

\begin{anki}
START
MathJaxCloze
Text: \textbf{Structure of Automorphisms of the Unit Disc}
Let \(\sigma :D_1\to D_1\) be a holomorphic automorphism of the unit disc, and let \(z_0\in D_1\) be the point for which \(\sigma (z_0 )=0\). Then for some \(\theta \in \R\)
 {{c1::\(\begin{align*}
        	\sigma (z) = e^{i\theta } \frac{z-z_0}{1-\overline{z_0}z}
        \end{align*}\)}} 
Extra: Observe that when \(\theta =2\pi k\) for \(k \in \Z\), the automorphism of the disc satisfies \(\sigma (z_0) = 0\), \(\sigma (0) = z_0\), and \(\sigma \circ \sigma  = \textrm{Id}_{D_1}\). For this reason, we will denote by \(\sigma_{z_0}\) the **canonical automorphism of the unit disc centered at \(z_0\)** when \(\theta =2\pi k\) and \(\sigma (z_0)=0\).
Tags: analysis complex_analysis conformal_mappings
<!--ID: 1626293841400-->
END
\end{anki}

\subsection{Isomorphism from the Upper Half Plane}
\label{sub:isomorphism_from_the_upper_half_plane}

We began this chapter with automorphisms of the unit disc because it turns out that we can obtain an isomorphism from larger parts of the complex plane to the disc. Then we can use the automorphisms of the disc to characterize more general sections of the complex plane.\\

We will now show that the entire upper half complex plane is holomorphically isomorphic to the unit disc.

\begin{thm}[Isomorphism between \(\mathbb{H}\) and \(D_1\)]
	Let \(\mathbb{H}\subset \C\) be the upper half plane. The map
	\begin{align*}
		f&:\mathbb{H} \to D_1\\
		f&:z \mapsto \frac{z-i}{z+i}
	\end{align*}
	is a holomorphic isomorphism onto the unit disc.
\end{thm}
The inverse map for \(f\) is given by
\begin{align*}
	g&:D_1 \to \mathbb{H}\\
	g&:z\mapsto -i\frac{z+1}{z-1}
\end{align*}
We encourage the reader to verify that \(g\) is indeed an inverse for \(f\).

\begin{anki}
START
MathJaxCloze
Text: 
Let \(\mathbb{H}\subset \C\) be the upper half plane. The map
 {{c1::\(\begin{align*}
        	f&:\mathbb{H} \to D_1\\
        	f&:z \mapsto \frac{z-i}{z+i}
        \end{align*}\)}} 
is a holomorphic isomorphism onto the unit disc.

The inverse map for \(f\) is given by
 {{c1::\(\begin{align*}
        	g&:D_1 \to \mathbb{H}\\
        	g&:z\mapsto -i\frac{z+1}{z-1}
        \end{align*}\)}}
Extra: It follows that
\(\begin{align*}
  	\textrm{Aut}(\mathbb{H}) = f^{-1} \textrm{Aut}(D) f
  \end{align*}\)
Tags: analysis complex_analysis conformal_mappings
<!--ID: 1626293841426-->
END
\end{anki}


\begin{cor}[Automorphism Group of \(\mathbb{H}\)]
	\begin{align*}
		\textrm{Aut}(\mathbb{H}) = f^{-1} \textrm{Aut}(D) f
	\end{align*}
	where
	\begin{align*}
		f&:\mathbb{H} \to D_1\\
		f&:z \mapsto \frac{z-i}{z+i}
	\end{align*}
\end{cor}
Of course, it remains to actually calculate what automorphisms of the upper half plane look like.

\begin{thm}[Structure of Automorphisms of \(\mathbb{H}\)]
	Every automorphism of \(\mathbb{H}\) is a linear fractional translation of the form
	\begin{align*}
		A = \begin{pmatrix} a & b \\ c & d \end{pmatrix} \\
		S_A(z) =\frac{az+b}{c z + d}
	\end{align*}
	where \(a,b,c,d \in \R\) with \(\textrm{det}(A)\neq 0\).\\

	Two automorphisms \(S_A\) and \(S_{A'}\) are equal if and only if \(A' = \pm A\).
\end{thm}
Hence, the automorphism group \(\textrm{Aut}(\mathbb{H})\) is isomorphic to the projective special linear group-- that is,
 \begin{align*}
	 \textrm{Aut}(\mathbb{H}) \cong PSL_2(\R).
\end{align*}

\begin{anki}
START
MathJaxCloze
Text: **Structure of Automorphisms of \(\mathbb{H}\)**
Every automorphism of \(\mathbb{H}\) is a {{c1::linear fractional translation}} of the form
{{c1::\(\begin{align*}
        	A = \begin{pmatrix} a & b \\ c & d \end{pmatrix} \\
        	S_A(z) =\frac{az+b}{c z + d}
        \end{align*}\)}}
where {{c1::\(a,b,c,d \in \R\)}} with {{c1::\(\textrm{det}(A)\neq 0\)}}.\\

Two automorphisms \(S_A\) and \(S_{A'}\) are equal if and only if {{c2::\(A' = \pm A\)}} .
Extra: Hence, the automorphism group \(\textrm{Aut}(\mathbb{H})\) is isomorphic to the projective special linear group-- that is,
\( \begin{align*}
  	 \textrm{Aut}(\mathbb{H}) \cong PSL_2(\R).
  \end{align*}\)
Tags: analysis complex_analysis conformal_mappings
<!--ID: 1626293841448-->
END
\end{anki}


\subsection{Isomorphisms Between Boundaries of Closed Curves}
\label{sub:isomorphisms_between_boundaries_of_closed_curves}

We note that so far, many results have been shown by utilizing the behavior of the function on the boundary of an open set. This is no coincidence-- we formalize this notion via the following theorems.

\begin{thm}[Surjectivity via Boundary Mapping]
	Let \(\Omega \) be a bounded connected open set. Suppose \(f\) is a non-constant holomorphic function on \(\Omega \) that is continuous on \(\partial\Omega \), which maps \(\partial\Omega \) into \(\partial D_1\), that is,
	\begin{align*}
		\left| f(z) \right| =1
	\end{align*}
	for all \(z \in \partial\Omega \). Then
	\begin{align*}
		f(\Omega ) = D_1.
	\end{align*}
\end{thm}
If we further assume that \(f\) is injective, then it follows that \(f\) is an isomorphism.

\begin{anki}
START
MathJaxCloze
Text: Let \(\Omega \) be a bounded connected open set. Suppose \(f\) is a non-constant holomorphic function on \(\Omega \) that is continuous on \(\partial\Omega \), which maps \(\partial\Omega \) into \(\partial D_1\), that is,
\(\begin{align*}
  	\left| f(z) \right| =1
  \end{align*}\)
for all \(z \in \partial\Omega \). Then
 {{c1::\(\begin{align*}
        	f(\Omega ) = D_1.
        \end{align*}\)}}
Extra: If we further assume that \(f\) is injective, then it follows that \(f\) is an isomorphism.
Tags: analysis complex_analysis conformal_mappings
<!--ID: 1626293841465-->
END
\end{anki}


\begin{proof}%Lang VII 4.2, pg. 227
	
\end{proof}

We can further generalize this via interiors of curves, but first we prove a lemma that will be essential:

\begin{lemma}
	Let \(\gamma \) be a piecewise smooth closed curve in an open set \(\Omega \subset \C\). Suppose that \(\gamma \) has an interior denoted by \(\textrm{Int}(\gamma )\). Then
	\begin{align*}
		\textrm{Int}(\gamma ) \cup \gamma 
	\end{align*}
	is compact.
\end{lemma}
This follows from the compactness of \(\gamma \) combined with the continuity of the winding number within the interior.

\begin{anki}
START
MathJaxCloze
Text: Let \(\gamma \) be a piecewise smooth closed curve in an open set \(\Omega \subset \C\). Suppose that \(\gamma \) has an interior denoted by \(\textrm{Int}(\gamma )\). Then
\(\begin{align*}
  	\textrm{Int}(\gamma ) \cup \gamma 
  \end{align*}\)
is {{c1::compact::topological property}}.
Extra: This follows from the compactness of \(\gamma \) combined with the continuity of the winding number within the interior.
Tags: analysis complex_analysis complex_topology
<!--ID: 1626293841487-->
END
\end{anki}


\begin{thm}[Condition for Conformality on Open Connected Sets]
	Let \(f\) be a non-constant holomorphic function on some open connected set \(\Omega \subset \C\), and let \(\gamma \) be a piecewise-smooth closed curve in \(\Omega \) homologous to \(0\).\\

	If \(\gamma \) and \(f\circ \gamma \) have interiors so that
	\begin{align*}
		f\circ \gamma \cap f(\textrm{Int}\gamma ) = \emptyset
	\end{align*}
	then \(f\) is injective on \(\textrm{Int}(\gamma )\) and hence \(f:\textrm{Int}(\gamma )\to f(\textrm{Int}\gamma )\) is an isomorphism. If in addition \(\textrm{Int}(f\circ \gamma )\) is connected, then
	\begin{align*}
		f(\textrm{Int}\gamma ) = \textrm{Int}(f\circ \gamma ).
	\end{align*}
\end{thm}

\begin{proof}
	
\end{proof}

\begin{anki}
START
MathJaxCloze
Text: Let \(f\) be a non-constant holomorphic function on some open connected set \(\Omega \subset \C\), and let \(\gamma \) be a piecewise-smooth closed curve in \(\Omega \) homologous to \(0\).\\

If \(\gamma \) and \(f\circ \gamma \) have interiors so that
{{c1::\(\begin{align*}
        	f\circ \gamma \cap f(\textrm{Int}\gamma ) = \emptyset
        \end{align*}\)::necessary condition}} 
then \(f\) is injective on \(\textrm{Int}(\gamma )\) and hence \(f:\textrm{Int}(\gamma )\to f(\textrm{Int}\gamma )\) is an isomorphism. If in addition \(\textrm{Int}(f\circ \gamma )\) is connected, then
{{c1::\(\begin{align*}
        	f(\textrm{Int}\gamma ) = \textrm{Int}(f\circ \gamma ).
        \end{align*}\)::mapping of interior}}
Tags: analysis complex_analysis conformal_mappings
<!--ID: 1626293841503-->
END
\end{anki}


\begin{exmp}
	
\end{exmp}

\begin{exmp}%Unbounded example
	
\end{exmp}

% \printindex
\end{document}


\section{Riemann Mapping Theorem}
\label{sec:riemann_mapping_theorem}

\documentclass{memoir}
\usepackage{notestemplate}

%\logo{~/School-Work/Auxiliary-Files/resources/png/logo.png}
%\institute{Rice University}
%\faculty{Faculty of Whatever Sciences}
%\department{Department of Mathematics}
%\title{Class Notes}
%\subtitle{Based on MATH xxx}
%\author{\textit{Author}\\Gabriel \textsc{Gress}}
%\supervisor{Linus \textsc{Torvalds}}
%\context{Well, I was bored...}
%\date{\today}

%\makeindex

\begin{document}

% \maketitle

% Notes taken on 

We finally arrive at the cornerstone of conformal mappings.

\begin{thm}[Riemann Mapping Theorem]
	Let \(\Omega \subsetneq \C\) be a  simply connected region which is not the whole plane, and let a point \(z_0 \in \Omega \) be given. There exists a unique holomorphic isomorphism \(f:\Omega \to D_1\) satisfying
	\begin{align*}
		f(z_0) = 0\\
		f'(z_0)>0.
	\end{align*}
	We call holomorphic isomorphisms \(f:\Omega \to D_1\) satisfying \(f(z_0) = 0\) \textbf{Riemann mappings}, and the Riemann mapping satisfying \(f'(z_0)>0\) is referred to as the \textbf{canonical Riemann mapping}.
\end{thm}
An isomorphism to the unit disc with \(f(z_0) = 0\) is unique up to rotation by \(e^{i\theta }\), and hence the second condition fixes a representation.\\

Uniqueness can be easily shown from what we have proven earlier. However, we need to show the existence of injective holomorphic maps, the family's relative compactness, and finally that there is a maximum of \(f'(z_0)\), so that we obtain existence with the second condition.

\begin{anki}
TARGET DECK
Complex Qual::Complex Analysis
START
MathJaxCloze
Text: **Riemann Mapping Theorem**
Let \(\Omega \subsetneq \C\) be a  simply connected region which is not the whole plane, and let a point \(z_0 \in \Omega \) be given. There exists a unique {{c1::holomorphic isomorphism}} \(f:\Omega \to D_1\) satisfying
 {{c1::\(\begin{align*}
        	f(z_0) = 0\\
        	f'(z_0)>0
        \end{align*}\)}}
We call {{c1::holomorphic isomorphisms}} \(f:\Omega \to D_1\) satisfying {{c1::\(f(z_0) = 0\)}} **Riemann mappings**, and the Riemann mapping satisfying \(f'(z_0)>0\) is referred to as the **canonical Riemann mapping**.
Extra: An isomorphism to the unit disc with \(f(z_0) = 0\) is unique up to rotation by \(e^{i\theta }\), and hence the second condition fixes a representation.
Tags: analysis complex_analysis conformal_mappings
<!--ID: 1626294735091-->
END
\end{anki}


\subsection{Compact Sets in Function Spaces}
\label{sub:compact_sets_in_function_spaces}

Let \(\Omega \subset \C\) be open. We refer to the space of holomorphic functions on \(\Omega \) by \(\textrm{Hol}(\Omega )\).

\begin{defn}[Relatively Compact]
	A subset \(\Omega \subset \C\) is \textbf{relatively compact} if \(\overline{\Omega }\) is compact. In other words, \(\Omega \) is relatively compact if and only if every sequence in \(\Omega \) has a convergent subsequence.\\

	Similarly, let \(\Omega \subset \C\) be open and \(\mathcal{F}\subset \textrm{Hol}(\Omega )\) be a family of holomorphic functions on \(\Omega \). We say that \(\mathcal{F}\) is \textbf{relatively compact} if every sequence of functions \(\left\{ f_n \right\} \in \mathcal{F}\) contains a subsequence which converges uniformly on every \(K\subset \Omega \) compact.
\end{defn}
We note that \(\lim_{n \to \infty} \left\{ f_n \right\} \) need not be in \(\mathcal{F}\). Note that other texts often refer to relatively compact families as \textbf{normal families}-- we use the term relatively compact, as it better captures the properties without overloading an already bloated term.

\begin{anki}
START
MathJaxCloze
Text: A subset \(\Omega \subset \C\) is **relatively compact** if {{c1::\(\overline{\Omega }\) is compact}}. In other words, \(\Omega \) is relatively compact if and only if {{c1::every sequence in \(\Omega \) has a convergent subsequence}}.

Similarly, let \(\Omega \subset \C\) be open and \(\mathcal{F}\subset \textrm{Hol}(\Omega )\) be a family of holomorphic functions on \(\Omega \). We say that \(\mathcal{F}\) is **relatively compact** if every sequence of functions \(\left\{ f_n \right\} \in \mathcal{F}\) {{c2::contains a subsequence which converges uniformly on every \(K\subset \Omega \) compact}} .
Extra: \(\lim_{n \to \infty} \left\{ f_n \right\} \) need not be in \(\mathcal{F}\).

Another term for 'relatively compact families' is **normal families**.
Tags: analysis complex_analysis complex_topology defn
<!--ID: 1626294735110-->
END
\end{anki}


\begin{defn}[Uniformly Bounded on Compact Sets]
	Let \(\Omega \subset \C\) and consider a family of holomorphic functions \(\mathcal{F}\subset \textrm{Hol}(\Omega )\). We say \(\mathcal{F}\) is \textbf{uniformly bounded on compact subsets of \(\Omega \)} if for every \(K\subset \Omega \) compact, there exists a positive constant \(M_K\) such that
	\begin{align*}
		\left| f(z) \right| \leq M_K
	\end{align*}
	for all \(f \in \mathcal{F}\) and \(z \in K\).
\end{defn}

\begin{anki}
START
MathJaxCloze
Text: Let \(\Omega \subset \C\) and consider a family of holomorphic functions \(\mathcal{F}\subset \textrm{Hol}(\Omega )\). We say \(\mathcal{F}\) is **uniformly bounded on compact subsets of \(\Omega \)** if for every \(K\subset \Omega \) compact, there exists {{c1::a positive constant \(M_K\)}} such that
 {{c1::\(\begin{align*}
        	\left| f(z) \right| \leq M_K
        \end{align*}\)}}
for all \(f \in \mathcal{F}\) and \(z \in K\).
Tags: analysis complex_analysis complex_topology defn
<!--ID: 1626294735137-->
END
\end{anki}

\begin{defn}[Equicontinuous]
	Let \(\Omega \subset \C\) and consider \(K\subset \Omega \) compact. We say that \(\mathcal{F}\subset \textrm{Hol}(\Omega )\) is \textbf{equicontinuous on \(K\)} if for all \(\varepsilon>0\) there exists a \(\delta >0\) so that, for all \(z,z' \in K\) with \(\left| z-z' \right| <\delta \):
	\begin{align*}
		\left| f(z) - f(z')\right| < \varepsilon
	\end{align*}
	for all \(f \in \mathcal{F}\).
\end{defn}

\begin{anki}
START
MathJaxCloze
Text: Let \(\Omega \subset \C\) and consider \(K\subset \Omega \) compact. We say that \(\mathcal{F}\subset \textrm{Hol}(\Omega )\) is **equicontinuous on \(K\)** if for all {{c1::\(\varepsilon>0\)}} there exists {{c1::a \(\delta >0\)}} so that, for all {{c1::\(z,z' \in K\)}} with {{c1::\(\left| z-z' \right| <\delta \)}}:
{{c1::\(\begin{align*}
        	\left| f(z) - f(z')\right| < \varepsilon
        \end{align*}\)}} 
for all {{c1::\(f \in \mathcal{F}\)}}.
Tags: analysis complex_analysis complex_topology defn
<!--ID: 1626294735155-->
END
\end{anki}

Now we will tie these three definitions together quite nicely:

\begin{thm}
	Let \(\Omega \subset \C\) and consider a family of functions \(\mathcal{F}\subset \textrm{Hol}(\Omega )\). If \(\mathcal{F}\) is uniformly bounded on all \(K\subset \Omega \) compact, then
	\begin{itemize}
		\item \(\mathcal{F}\) is equicontinuous on all \(K\subset \Omega \) compact.
		\item \(\mathcal{F}\) is relatively compact.
	\end{itemize}
\end{thm}
Note that the first part follows from the holomorphicity of the functions in \(\mathcal{F}\), while the second part is merely topological, and follows from the fact that \(\mathcal{F}\) is uniformly bounded and equicontinuous. The second part is often referred to as the Arzela-Ascoli theorem.

\begin{proof}% By diagonalization in Lang p.309
	
\end{proof}

\begin{anki}
START
MathJaxCloze
Text: Let \(\Omega \subset \C\) and consider a family of functions \(\mathcal{F}\subset \textrm{Hol}(\Omega )\). If \(\mathcal{F}\) is uniformly bounded on all \(K\subset \Omega \) compact, then
	\begin{itemize}
		\item {{c1::\(\mathcal{F}\) is equicontinuous on all \(K\subset \Omega \) compact}}
		\item {{c2::\(\mathcal{F}\) is relatively compact}}
	\end{itemize}
Extra: Note that the first part follows from the holomorphicity of the functions in \(\mathcal{F}\), while the second part is merely topological, and follows from the fact that \(\mathcal{F}\) is uniformly bounded and equicontinuous. The second part is often referred to as the Arzela-Ascoli theorem.
Tags: analysis complex_analysis complex_topology
<!--ID: 1626294735173-->
END
\end{anki}


\begin{lemma}
	Let \(\Omega \subset \C\) be open. Then \(\Omega \) has an exhaustion by compact sets.\\

	That is, there exists a sequence \(\left\{ K_j \right\}_{j=1}^{\infty}\) of compact subsets \(K_j \subset \Omega \) with \(K_j \subset \textrm{Int}(K_{j+1})\) so that
	\begin{align*}
		\bigcup_{j=1}^{\infty}K_j = \Omega .
	\end{align*}
\end{lemma}

\begin{anki}
START
MathJaxCloze
Text: Let \(\Omega \subset \C\) be open. Then \(\Omega \) has {{c1::an exhaustion by compact sets}}.

That is, there exists a sequence \(\left\{ K_j \right\}_{j=1}^{\infty}\) of compact subsets \(K_j \subset \Omega \) with {{c1::\(K_j \subset \textrm{Int}(K_{j+1})\)}} so that
{{c1::\(\begin{align*}
		\bigcup_{j=1}^{\infty}K_j = \Omega .
	\end{align*}\)}}
Tags: analysis complex_analysis complex_topology
<!--ID: 1626294735193-->
END
\end{anki}


\subsection{Proof of the Riemann Mapping Theorem}
\label{sub:proof_of_the_riemann_mapping_theorem}

Let us briefly outline the path which we intend to prove the Riemann mapping theorem.\\

Consider a simply connected open set \(\Omega \subset \C\), and let \(z_0 \in \Omega \). Let \(\mathcal{F}\subset \textrm{Hol}(\Omega )\) be the family of injective holomorphic functions satisfying
\begin{align*}
	f:\Omega \to D_1\\
	f(z_0) = 0
\end{align*}
for all \(f \in \mathcal{F}\). We will show this family is non-empty, and then verify that it is uniformly bounded. Then it remains to show the existence of a \(f \in \mathcal{F}\) so that \(\left| f'(z_0) \right| \) is maximal, and verify that it is indeed the isomorphism. We already know that automorphisms of \(D_1\) fixing the origin are rotations, and hence we can rotate \(f\) so that \(f'(z_0)\) is real and positive, obtaining uniqueness.\\

Recall that holomorphic injective functions are isomorphic with their image, with non-zero derivative (this is the natural holomorphic isomorphism theorem \ref{sub:inverse_and_open_mapping_theorems}).

\begin{lemma}
\label{lemma:inj_or_const}
	Let \(\Omega \subset \C\) be connected and open, and let \(\left\{ f_n \right\} \subset \textrm{Hol}(\Omega )\) be a sequence of injective holomorphic maps of \(\Omega \) which converge uniformly on every \(K\subset \Omega \) compact. Then
	\begin{align*}
		\lim_{n \to \infty} \left\{ f_n \right\} = f
	\end{align*}
	is injective or constant.
\end{lemma}

\begin{anki}
START
MathJaxCloze
Text: Let \(\Omega \subset \C\) be connected and open, and let \(\left\{ f_n \right\} \subset \textrm{Hol}(\Omega )\) be a sequence of injective holomorphic maps of \(\Omega \) which converge uniformly on every \(K\subset \Omega \) compact. Then 
\(\begin{align*}
  	\lim_{n \to \infty} \left\{ f_n \right\} = f
  \end{align*}\)
is {{c1::injective or constant}}.
Tags: analysis complex_analysis conformal_mappings
<!--ID: 1626294735216-->
END
\end{anki}

We will provide one more lemma which will reduce the main case sufficiently for us to prove the desired result.

\begin{lemma}
	Let \(\Omega \subsetneq \C\) be an simply connected open subset. Then there exists a {{c1::holomorphic isomorphism}} \(f:\Omega \to U\) where \(U\subset D_1\) is an open subset of the disc.
\end{lemma}
\begin{proof}
	Let \(\alpha \in \C\) be a point with \(\alpha \not\in \Omega \). Because \(\Omega \) is simply connected, there is a branch of the complex logarithm so that
	\begin{align*}
		g(z) = \ln(z-\alpha )
	\end{align*}
	is holomorphic on \(\Omega \). We have then that \(g\) is injective:
	\begin{align*}
		g(z_1) = g(z_2) \implies e^{g(z_1)} = e^{g(z_2)} \implies z_1-\alpha  = z_2-\alpha \implies z_1=z_2.
	\end{align*}
	Because \(g\) is injective, then for any \(z_0 \in \Omega \)
	\begin{align*}
		g(z) \neq g(z_0) + 2\pi i
	\end{align*}
	for all \(z \in \Omega \). This holds because if equality were to hold, then exponentiation of the right-hand side would imply that \(z = z_0\) by the injectivity of \(g\), and hence \(g(z) = g(z_0)\), contradicting the equality.\\

	Furthermore, there exists a \(D_r(g(z_0)+2\pi i)\) for some \(r>0\) so that \(g(\Omega ) \cap D_r(g(z_0)+2\pi i) = \emptyset\). Once again we argue by contradiction-- if this intersection is non-empty for all \(r>0\), then there is a sequence \(\left\{ w_n \right\} \) so that \(\lim_{n \to \infty} g(w_n) = g(z_0)\), which yields a contradiction. Thus
	\begin{align*}
		\frac{1}{g(z) - g(z_0) - 2\pi i}
	\end{align*}
	is bounded on \(\Omega \) and is a holomorphic injection. We can then apply elementary transformations of translation and multiplication by a real number to obtain a function with \(f(z_0) = 0\) and \(\left| f(z) \right| <1\) for all \(z \in \Omega \), proving the lemma.
\end{proof}

\begin{anki}
START
MathJaxCloze
Text: **Lemma for Riemann Mapping Theorem**
Let \(\Omega \subsetneq \C\) be an simply connected open subset. Then there exists a {{c1::holomorphic isomorphism}} \(f:\Omega \to U\) where {{c1::\(U\subset D_1\) is an open subset of the disc}}.
Extra: 
Tags: 
<!--ID: 1626294735236-->
END
\end{anki}


Now we use this lemma to prove the Riemann mapping theorem, restated here for posterity:

\begin{thm}[Riemann Mapping Theorem]
	Let \(\Omega \subsetneq \C\) be a  simply connected region which is not the whole plane, and let a point \(z_0 \in \Omega \) be given. There exists a unique holomorphic isomorphism \(f:\Omega \to D_1\) satisfying
	\begin{align*}
		f(z_0) = 0\\
		f'(z_0)>0.
	\end{align*}
\end{thm}

\begin{proof}[Proof of Riemann Mapping Theorem]
	Observe that by the previous lemma, we can prove the Riemann mapping theorem under the assumption that \(\Omega \subset D_1\) containing the origin. Let \(\mathcal{F}\subset \textrm{Hol}(\Omega )\) be the family of all injective holomorphic maps \(f:U\to D_1\) with \(f(0) = 0\). Observe that \(\textrm{Id}_\Omega \in \mathcal{F}\) and hence the family is not empty. Applying Cauchy's integral formula:
	\begin{align*}
		f'(0) = \frac{1}{2\pi i}\int_{\partial D_1} \frac{f(z)}{z^2}\,d z
	\end{align*}
	we see that \(\left| f'(0) \right| \) is bounded above for all \(f \in \mathcal{F}\) by upper bounds \(\mathcal{M}\), as \(\mathcal{F}\) is uniformly bounded on \(\overline{D_\varepsilon}\) with \(\varepsilon>0\) small. Recall that \(\C\) has the least upper bound property, and hence there is a sequence \(\left\{ f_n \right\} \) so that \(\lim_{n \to \infty} \left| f'_n(0) \right| = M\) and \(\left| f(z) \right| \leq 1\), where \(M\leq M'\) for all upper bounds \(M' \in \mathcal{M}\) and \(M\) upper bounds \(\left| f'(0) \right| \) for all \(f \in \mathcal{F}\).\\

	By \ref{lemma:inj_or_const} it holds that \(f\) is injective. Furthermore, the maximum modulus principle already gives us
	\begin{align*}
		\left| f(z) \right| <1
	\end{align*}
	for all \(z \in \Omega \). Thus, we have proven that this \(f \in \mathcal{F}\) is maximal with regards to \(\left| f'(0) \right| \). Now we show that because \(f\) is maximal, \(f\) is additionally surjective onto the disk, proving the theorem.\\

	It turns out that surjectivity and maximality are equivalent conditions for mappings of this type. Suppose for the sake of contradiction that \(f\) is not onto. Then consider \(w \in D_1\setminus f(\Omega )\). Recall that by Theorem \ref{thm:classification_of_automorphisms_of_the_unit_disc} there is a canonical automorphism \(\sigma_w\) of \(D_1\) given by
	\begin{align*}
		\sigma_w(z) = \frac{z-w}{1-\overline{w}z}.
	\end{align*}
	We assert that there exists an injective holomorphic square root function \(g:(\sigma_w\circ f)(\Omega) \to D_1\) satisfying
	\begin{align*}
		(g(z))^2 = \sigma_w(z).
	\end{align*}
	This holds because \(\sigma_w\) has no zeros on \((\sigma_w \circ f)(\Omega) \). Now we apply a canonical automorphism once more via \(\sigma_{g(w)}\) and define a function:
	\begin{align*}
		h = \sigma_{g(w)}\circ g \circ \sigma_w \circ f.
	\end{align*}
	One can verify that \(h \in \mathcal{F}\), as injectivity and holomorphicity follows through composition, and \(h(0) = 0\). Now we show that if \(h \) exists, it satisfies \(\left| \varphi'(0) \right| > \left| f'(0) \right| \), yielding a contradiction. This follows from the fact that
	\begin{align*}
		f = \sigma^{-1}_w \circ g^{-1} \circ \sigma^{-1}_{g(w)} \circ h,
	\end{align*}
	as
	\begin{align*}
		\Phi := \sigma^{-1}_w \circ g^{-1} \circ \sigma^{-1}_{g(w)}
	\end{align*}
	is not injective (because \(g^{-1}\) is not injective), and so \(\left| \Phi'(0) \right| <1\) by Schwarz' lemma. Thus
	\begin{align*}
		f'(0) = \Phi'(0) h'(0) \implies \left| f'(0) \right| < \left| h'(0) \right| 
	\end{align*}
	contradicting the maximality of \(f\). Thus we have shown that there cannot exist \(w \in D_1\setminus f(\Omega )\). Therefore, \(f\) is surjective. By Schwarz' lemma, we can multiply \(f\) by some \(e^{i\theta }\) to ensure \(f'(0) > 0\), obtaining the desired unique isomorphism.
\end{proof}

%% NOTE-- we can replace simply connected with holomorphically simply connected; try to define and replace
%% Re: Not really. One can define holomorphically simply connected, but it turns out that the notion can be proven to be equivalent to simply connected. Not really worth mentioning (maybe to point out that such a concept is superfluous?)

\subsection{Explicit Calculations of Riemann mappings}
\label{sub:explicit_calculations_of_riemann_mappings}

While having the existence and uniqueness of the canonical Riemann mapping is nice, the proof of the Riemann mapping theorem doesn't offer much help in constructing the function. We will derive an explicit form of the Riemann mappings for polygons, and discuss how to obtain others.

%\begin{defn}[Polygons and Angles]
%	
%\end{defn}
%
%We can map each vertex of the polygon onto the boundary of \(D_1\) sequentially, so that the line segments between vertices map onto arcs between the vertices on \(\partial D_1\).

\begin{thm}[Schwarz-Christoffel Formula]
	For every polygon \(P\) with angles \(\alpha_k\pi \), there exists complex constants \(C_1,C_2 \in \C\) and sequence of points \(w_k \in \partial D_1\) so that the function \(z = F(w)\) given by
	\begin{align*}
		F(w) = C_1 \int_{0}^{w} \prod_{k=1}^{n} (w-w_k)^{-\beta_k} \,d w + C_2\\
		\beta_k = 1-\alpha_k
	\end{align*}
	is a holomorphic isomorphism \(F:D_1\to P\).
\end{thm}

%Way more\ldots read 6.2 in Ahlfors

% \printindex
\end{document}


\chapter{Harmonic Functions}
\label{cha:harmonic_functions}

In real analysis, harmonic functions form a relatively rigid class of functions, sharing many properties which are similar to holomorphic functions in many ways-- such as the mean value property and maximum modulus principle. It comes as no surprise that harmonic functions have an intimate connection to holomorphic functions. In this chapter, we will explore the connection between holomorphic and harmonic functions, and how it can be applied to learn more about the other. But-- we will also see where they differ, to better understand the kinds of functions which are harmonic but not holomorphic, and vice versa.

\section{Decomposability of Holomorphic Functions}
\label{sec:decomposability_of_holomorphic_functions}


\documentclass{memoir}
\usepackage{notestemplate}

%\logo{~/School-Work/Auxiliary-Files/resources/png/logo.png}
%\institute{Rice University}
%\faculty{Faculty of Whatever Sciences}
%\department{Department of Mathematics}
%\title{Class Notes}
%\subtitle{Based on MATH xxx}
%\author{\textit{Author}\\Gabriel \textsc{Gress}}
%\supervisor{Linus \textsc{Torvalds}}
%\context{Well, I was bored...}
%\date{\today}

%\makeindex

\begin{document}

% \maketitle

% Notes taken on 07/14/21

\begin{defn}[Harmonic Functions]
	Let \(u \in C^2(\Omega)\) for \(\Omega\subset \R^{n}\) be a \textit{real-valued} multivariate function with continuous second-derivative. We say that \(u\) is \textbf{harmonic} if it satisfies
	\begin{align*}
		\Delta u := \sum_{i=1}^{n} \frac{\partial^2 u}{\partial x_i^2} = 0.
	\end{align*}
	We call the operator \(\Delta \) the \textbf{Laplacian} and the equation above \textbf{Laplace's equation}.
\end{defn}
Now suppose \(f = (u,v)\) is holomorphic on \(\Omega \subset \C\). We have shown that \(u,v \in C^{\infty}\) and the Cauchy-Riemann equations tell us that
\begin{align*}
	\frac{\partial u}{\partial x} = \frac{\partial v}{\partial y} \\
	\frac{\partial u}{\partial y} = - \frac{\partial v}{\partial x} .
\end{align*}
Applying \(\frac{\partial }{\partial x} \) or \(\frac{\partial }{\partial y} \) to both sides brings us to our first connection between holomorphic and harmonic functions.

\begin{thm}
	Let \(f=(u,v)\) be a holomorphic function. Then \(u,v\) are harmonic.
\end{thm}

Thus, we have seen that holomorphicity already implicitly requires two harmonic functions. But the Cauchy-Riemann equations give us additional structure on \(u,v\):

\begin{defn}[Conjugate Harmonic Function]
	If two harmonic functions \(u,v\) satisfy the Cauchy-Riemann equations
\begin{align*}
	\frac{\partial u}{\partial x} &= \frac{\partial v}{\partial y} \\
	\frac{\partial u}{\partial y} &= - \frac{\partial v}{\partial x} 
\end{align*}
	then \(v\) is said to be the \textbf{conjugate harmonic function} of \(u\). 
\end{defn}
In fact, the Cauchy-Riemann equations provide us with the most structure we could expect-- a conjugate harmonic function \(v\) is uniquely determined (up to an additive constant). But exercise caution, as harmonic conjugacy is symmetric, not antisymmetric-- if \(v\) is the conjugate harmonic function of \(u\), then \(u\) is the conjugate harmonic function of \(-v\).\\

\begin{anki}
TARGET DECK
Complex Qual::Complex Analysis
START
MathJaxCloze
Text: If two harmonic functions \(u,v\) satisfy {{c1::the Cauchy-Riemann equations
\(\begin{align*}
        	\frac{\partial u}{\partial x} &= \frac{\partial v}{\partial y} \\
        	\frac{\partial u}{\partial y} &= - \frac{\partial v}{\partial x} 
        \end{align*}\)}} 
	then \(v\) is said to be the **conjugate harmonic function** of \(u\). 
Extra: In fact, the Cauchy-Riemann equations provide us with the most structure we could expect-- a conjugate harmonic function \(v\) is uniquely determined (up to an additive constant). But exercise caution, as harmonic conjugacy is symmetric, not antisymmetric-- if \(v\) is the conjugate harmonic function of \(u\), then \(u\) is the conjugate harmonic function of \(-v\).
Tags: analysis complex_analysis complex_analyticity defn 
<!--ID: 1624560720227-->
END
\end{anki}

Recall the complex differential forms:
\begin{align*}
	dz = (dx, dy)\\
	d \overline{z} = (dx,-dy)
\end{align*}
defined so that
\begin{align*}
	dx = (\sfrac{1}{2},0) (dz + d \overline{z})\\
	dy = (0, \sfrac{1}{2}) (dz - d \overline{z})
\end{align*}
is satisfied. We also defined the complex differential operator by:
\begin{align*}
	\frac{\partial }{\partial z} &= \frac{1}{2}\left( \frac{\partial }{\partial x} , \frac{\partial }{\partial y}  \right)\\
	\frac{\partial }{\partial \overline{z}} &= \frac{1}{2} \left( \frac{\partial }{\partial x} , - \frac{\partial }{\partial y}  \right) 
\end{align*}
defined so that
\begin{align*}
	df = \frac{\partial f}{\partial z} \,d z + \frac{\partial f}{\partial \overline{z}} \,d \overline{z}
\end{align*}
is satisfied.

\begin{prop}
	\begin{align*}
		\Delta = 4 \frac{\partial }{\partial z} \frac{\partial }{\partial \overline{z}} = 4 \frac{\partial }{\partial \overline{z}} \frac{\partial }{\partial z} .
	\end{align*}
\end{prop}
This follows directly from computations.\\

\begin{anki}
START
MathJaxCloze
Text: Consider the complex differential forms:
\(\begin{align*}
  	dz &= (dx,dy) & \quad dx &= (\sfrac{1}{2},0)(dz + d\overline{z}) \\
  	d \overline{z} &= (dx,-dy) &\quad \,d y &= (0,\sfrac{1}{2})(dz - d \overline{z})
  \end{align*}\)
  and the corresponding operators
  \(\begin{align*}
    	\frac{\partial }{\partial z} &= \frac{1}{2}\left( \frac{\partial }{\partial x} , \frac{\partial }{\partial y}  \right)\\
    	\frac{\partial }{\partial \overline{z}} &= \frac{1}{2} \left( \frac{\partial }{\partial x} , - \frac{\partial }{\partial y}  \right) 
    \end{align*}\). Then we can write the Laplacian operator by:
    {{c1::\(\begin{align*}
	    	\Delta = 4 \frac{\partial }{\partial z} \frac{\partial }{\partial \overline{z}} = 4 \frac{\partial }{\partial \overline{z}} \frac{\partial }{\partial z} .
	    \end{align*}\)}} 
Tags: analysis complex_analysis harmonic_functions
<!--ID: 1626993604616-->
END
\end{anki}

Recall from real analysis that a harmonic function is defined by its value on the boundary:
\begin{thm}
	Let \(U\subset \R^n\) be a bounded open set. Let \(u:\overline{U}\to \R^{n}\) be a continuous function harmonic on \(U\). Suppose there exists a second continuous function \(v:\overline{U}\to \R^{n}\) harmonic on \(U\), so that \(u = v\) on \(\partial U\). Then \(u=v\) on \(\overline{U}\).
\end{thm}
This theorem has two main weaknesses. The first is that the set is required to be bounded. The second is that the functions must be continuous on the boundary.\\

Complex analysis provides us with tools that can resolve these weaknesses. The Riemann mapping theorem allows us to obtain an isomorphism from most unbounded sets into a bounded region-- although we must be cautious that the boundary curves correspond to each other properly. We will need to build more theory to handle discontinuities on the boundary.\\

One can show that if \(u\) is \textit{bounded everywhere on \(\partial U\)} and continuous on \(\partial U\)-- except for a finite number of points, then the theorem holds as expected. We omit the proof here.\\

\begin{anki}
START
MathJaxCloze
Text: Let \(U\subset \R^n\) be a bounded open set. Let \(u:\overline{U}\to \R^{n}\) be a continuous function harmonic on \(U\). Suppose there exists a second continuous function \(v:\overline{U}\to \R^{n}\) harmonic on \(U\), so that \(u = v\) on \(\partial U\). Then \(u=v\) on \(\overline{U}\).
Extra: If \(u\) is \textit{bounded everywhere on \(\partial U\)} and continuous on \(\partial U\)-- except for a finite number of points, then the theorem holds as expected.
Tags: analysis complex_analysis harmonic_functions
<!--ID: 1626993604635-->
END
\end{anki}

Recall that we can compute the complex derivative of a function \(f=(u,v)\) directly by
\begin{align*}
	f'(z) = 2 \frac{\partial u}{\partial z} .
\end{align*}
We have a converse provided that the underlying space is simply connected.
 \begin{prop}
	 Let \(\Omega \subset \C\) be a simply connected set, and let \(u\) be a real-valued harmonic function on \(U\). Then there exists a holomorphic function \(f \in \textrm{Hol}(\Omega )\) so that \(u= \textrm{Re}(f)\). Furthermore, this holomorphic function is unique up to an imaginary constant.
\end{prop}
To see that simply connectedness is necessary, consider \(\ln(\left| z \right| )\) on \(D_1 \setminus\left\{ 0 \right\} \).
\begin{anki}
START
MathJaxCloze
Text: Let \(\Omega \subset \C\) be a simply connected set, and let \(u\) be a real-valued harmonic function on \(U\). Then there exists a holomorphic function \(f \in \textrm{Hol}(\Omega )\) so that {{c1::\(u= \textrm{Re}(f)\)}}. Furthermore, this holomorphic function is {{c1::unique up to an imaginary constant}}.
Extra: To see that simply connectedness is necessary, consider \(\ln(\left| z \right| )\) on \(D_1 \setminus\left\{ 0 \right\} \).
Tags: analysis complex_analysis harmonic_functions
<!--ID: 1626993604652-->
END
\end{anki}
Uniqueness will follow from a slightly more general lemma:
\begin{lemma}
	Let \(\Omega \subset \C\) be a connected open set, and \(f,g\) holomorphic functions on \(\Omega \). If \(\textrm{Re}(f) = \textrm{Re}(g)\), then
	\begin{align*}
		f = (g, K)
	\end{align*}
	for some constant \(K \in \R\).
\end{lemma}

\begin{anki}
START
MathJaxCloze
Text: Let \(\Omega \subset \C\) be a connected open set, and \(f,g\) holomorphic functions on \(\Omega \). If \(\textrm{Re}(f) = \textrm{Re}(g)\), then
\(\begin{align*}
  	f = (g, K)
  \end{align*}\)
for some constant \(K \in \R\).
Tags: analysis complex_analysis complex_analyticity
<!--ID: 1626993604668-->
END
\end{anki}

We will need some simple properties of composition, which we give now as an exercise.
\begin{hw}
	Let \(f:U\to V\) be a holomorphic function. Suppose \(g\) is harmonic on \(V\). Then \(g\circ f \) is harmonic on \(U\).\\

	Suppose instead that \(g\) is harmonic on the set
	\begin{align*}
		V' = \left\{ \overline{z} \mid z \in V \right\} .
	\end{align*}
	Then the function
	\begin{align*}
		g(\overline{f(z)})
	\end{align*}
	is harmonic on \(U\).
\end{hw}

Now we prove the mean value theorem for harmonic functions via Cauchy's theorem.
\begin{thm}[Mean Value Theorem]
	Let \(\Omega \subset \C\) be an open set and \(u\) a harmonic function on \(\Omega \). Let \(z_0 \in U\) be given and consider \(\overline{D_r(z_0)}\subset \Omega \) for some \(r>0\) sufficiently small. Then
	\begin{align*}
		u(z_0) = \frac{1}{2\pi } \int_{0}^{2\pi } u(z_0+re^{i\theta }) \,d \theta . 
	\end{align*}
\end{thm}
Naturally, the maximum principle for harmonic functions falls out from this.

\begin{anki}
START
MathJaxCloze
Text: **Mean Value Theorem for Harmonic Functions**
Let \(\Omega \subset \C\) be an open set and \(u\) a harmonic function on \(\Omega \). Let \(z_0 \in U\) be given and consider \(\overline{D_r(z_0)}\subset \Omega \) for some \(r>0\) sufficiently small. Then
 {{c1::\(\begin{align*}
         	u(z_0) = \frac{1}{2\pi } \int_{0}^{2\pi } u(z_0+re^{i\theta }) \,d \theta . 
         \end{align*}\)}} 
Tags: analysis complex_analysis harmonic_functions
<!--ID: 1626993604684-->
END
\end{anki}

\begin{thm}[Maximum Principle of Harmonic Functions]
	Let \(\Omega \subset \C\) be a connected open set with \(u\) harmonic on \(\Omega \).
\begin{itemize}
	\item If \(u\) has a maximum at a point \(z_0 \in \Omega \), then \(u\) is constant.
	\item If \(u\) is continuous on \(\overline{\Omega }\) and not constant on \(U\), then \(u\) obtains its maximum on \(\partial U\).
\end{itemize}
\end{thm}

We can apply the maximum modulus principle in a lot of non-intuitive circumstances. In fact, the maximum modulus is not arbitrary-- we can bound the maximum modulus on a circle provided we have information on the maximum modulus of \(f\) on concentric circles near it.

\begin{anki}
START
MathJaxCloze
Text: **Maximum Principle of Harmonic Functions**
Let \(\Omega \subset \C\) be a connected open set with \(u\) harmonic on \(\Omega \).

* If \(u\) has a maximum at a point \(z_0 \in \Omega \), then {{c1::\(u\) is constant}}.
* If \(u\) is continuous on \(\overline{\Omega }\) and not constant on \(U\), then {{c1::\(u\) obtains its maximum on \(\partial U\)}}.
Tags: analysis complex_analysis harmonic_functions
<!--ID: 1626993604707-->
END
\end{anki}

\begin{thm}[Hadamard's Three-Circle Theorem]
	Let \(0<R_1<R_2<\infty\) be given and consider the annulus given by \(\Omega = D_{R_2} \setminus \overline{D_{R_1}}\). Let
	\begin{align*}
		\prescript{}{r}\|f\|_\infty := \textrm{max}_{z \in \partial D_r} \left| f(z) \right| .
	\end{align*}
	If a function \(f\) is holomorphic on the open annulus and continuous on the closure of the annulus, then
	\begin{align*}
		\prescript{}{r}\|f\|_{\infty} &\leq \prescript{}{R_1}\|f\|_{\infty}^{\alpha } \prescript{}{R_2}\|f\|_{\infty}^{1-\alpha }\\
		\alpha &= \frac{\ln(\sfrac{R_2}{r})}{\ln(\sfrac{R_2}{R_1})}.
	\end{align*}
	In other words, the maximum modulus of \(f\) on a circle \(\partial D_r\) is bounded by a logarithmic interpolation between the maximum modulus of \(f\) on concentric circles surrounding \(\partial D_r\).
\end{thm}

\begin{anki}
START
MathJaxCloze
Text: **Hadamard's Three-Circle Theorem**
Let \(0<R_1<R_2<\infty\) be given and consider the annulus given by \(\Omega = D_{R_2} \setminus \overline{D_{R_1}}\). Let
\(\begin{align*}
  	\prescript{}{r}\|f\|_\infty := \textrm{max}_{z \in \partial D_r} \left| f(z) \right| .
  \end{align*}\)
If a function \(f\) is holomorphic on the open annulus and continuous on the closure of the annulus, then
{{c1::\(\begin{align*}
      	\prescript{}{r}\|f\|_{\infty} &\leq \prescript{}{R_1}\|f\|_{\infty}^{\alpha } \prescript{}{R_2}\|f\|_{\infty}^{1-\alpha }\\
      	\alpha &= \frac{\ln(\sfrac{R_2}{r})}{\ln(\sfrac{R_2}{R_1})}.
        \end{align*}\)}}
In other words, the maximum modulus of \(f\) on a circle \(\partial D_r\) is {{c1::bounded by a logarithmic interpolation}} between the {{c1::maximum modulus of \(f\) on concentric circles surrounding \(\partial D_r\)}}.
Tags: analysis complex_analysis harmonic_functions
<!--ID: 1626993604724-->
END
\end{anki}


\subsection{Harmonic Functions on Annuli}
\label{sub:harmonic_functions_on_annuli}

The annulus is an interesting case because it is not simply connected, and hence we cannot utilize our previous theorems to induce a holomorphic function. However, it has plenty of structure that allows us to induce the holomorphic function a different way.

\begin{thm}[Correspondence Between Harmonic and Holomorphic Functions (Annulus)]
	Let \(0\leq R_1<R_2\) be given and consider the annulus given by \(\Omega = D_{R_2}\setminus \overline{D_{R_1}}\). Suppose \(u\) is a harmonic function on \(\Omega \). Then there exists a constant \(a \in \R\) and a holomorphic function \(f \in \textrm{Hol}(\Omega )\) such that
	\begin{align*}
		\textrm{Re}(f) = u - a \ln(\left| z \right| ).
	\end{align*}
\end{thm}
Note that we allow \(R_1=0\) or \(R_2=\infty\) to include the special cases of the punctured plane \(\C\setminus\left\{ 0 \right\} \) and the punctured disk \(D_1\setminus\left\{ 0 \right\} \).

\begin{proof}
	
\end{proof}

\begin{anki}
START
MathJaxCloze
Text: **Correspondence between Harmonic and Holomorphic on the Annulus**
Let \(0\leq R_1<R_2\) be given and consider the annulus given by \(\Omega = D_{R_2}\setminus \overline{D_{R_1}}\). Suppose \(u\) is a harmonic function on \(\Omega \). Then there exists a constant \(a \in \R\) and a holomorphic function \(f \in \textrm{Hol}(\Omega )\) such that
{{c1::\(\begin{align*}
        	\textrm{Re}(f) = u - a \ln(\left| z \right| ).
        \end{align*}\)::real part of f}} 
Extra: Note that we allow \(R_1=0\) or \(R_2=\infty\) to include the special cases of the punctured plane \(\C\setminus\left\{ 0 \right\} \) and the punctured disk \(D_1\setminus\left\{ 0 \right\} \).
This also gives us a simple way to compute integrals of harmonic functions on the annulus:
\(\begin{align*}
  	\frac{1}{2\pi }\int_{0}^{2\pi } u(r,\theta )\,d \theta  = a \ln(\left| z \right| ) + b.
  \end{align*}\)
Tags: analysis complex_analysis harmonic_functions
<!--ID: 1626993604740-->
END
\end{anki}

This also gives us a simple way to compute integrals of harmonic functions on the annulus:
\begin{cor}
	Let \(0\leq R_1<R_2\) be given and consider the annulus given by \(\Omega  = D_{R_2}\setminus\left\{ D_{R_1} \right\} \). Let \(u\) be harmonic on \(\Omega \). Then there exists constants \(a,b\in \C\) so that
	\begin{align*}
		\frac{1}{2\pi }\int_{0}^{2\pi } u(r,\theta )\,d \theta  = a \ln(\left| z \right| ) + b.
	\end{align*}
\end{cor}

Nevertheless, it will continue to prove difficult to unify our study of harmonic and holomorphic functions until we can find conditions under which we can extend harmonic functions on a domain that is not simply connected. We will prove a result that we will use to extend to the general case.
\begin{lemma}[Extension of Bounded Harmonic Functions]
	Let \(\Omega = D_1\setminus\left\{ 0 \right\} \) be the punctured disc and suppose that \(u\) is a bounded harmonic function on \(\Omega \). Then \(u\) extends to a harmonic function on \(D_1\).
\end{lemma}

This gives us hope that harmonic functions can extend on removable singularities, but first we will need one more lemma.

\begin{lemma}[Analytic Continuation for Harmonic Functions]
	Let \(\Omega \subset \C\) be a connected open set. Suppose \(u\) is a harmonic function on \(\Omega \) and \(f\) a holomorphic function on \(\Omega \). If \(u = \textrm{Re}(f)\) on some \(D_r\subset \Omega \) for \(r>0\), then \(u = \textrm{Re}(f)\) everywhere in \(\Omega \).
\end{lemma}

Finally, we obtain the desired result:
\begin{thm}[Correspondence Between Harmonic and Holomorphic Functions (General)]
	Let \(\Omega\subset \C \) be a simply connected open set, and let \(\left\{ z_i \right\}_{i=1}^{n}\) be a sequence of distinct points in \(\Omega \). Consider \(\Omega^{*}= \Omega \setminus] \left\{ z_i \right\}_{i=1}^{n}\), and let \(u\) be a real harmonic function on \(\Omega ^{*}\). Then there exists constants \(\left\{ a_i \right\}_{i=1}^{n}\subset \C\) and a holomorphic function \(f \in \textrm{Hol}(\Omega^{*})\) such that, for all \(z \in \Omega^{*}\),
	\begin{align*}
		\textrm{Re}(f(z)) = u(z) - \sum_{i=1}^{n} a_i \ln \left| z-z_i \right| .
	\end{align*}
\end{thm}
In fact, we can actually consider deleted discs \(D_{r_i}(z_i)\) and obtain the same result, in which case our theorem on the annulus is a special case of the above theorem.

\begin{anki}
START
MathJaxCloze
Text: Let \(\Omega\subset \C \) be a simply connected open set, and let \(\left\{ z_i \right\}_{i=1}^{n}\) be a sequence of distinct points in \(\Omega \). Consider \(\Omega^{*}= \Omega \setminus] \left\{ z_i \right\}_{i=1}^{n}\), and let \(u\) be a real harmonic function on \(\Omega ^{*}\). Then there exists constants \(\left\{ a_i \right\}_{i=1}^{n}\subset \C\) and a holomorphic function \(f \in \textrm{Hol}(\Omega^{*})\) such that, for all \(z \in \Omega^{*}\),
\(\begin{align*}
  	\textrm{Re}(f(z)) = u(z) - \sum_{i=1}^{n} a_i \ln \left| z-z_i \right| .
  \end{align*}\)
Extra: We can actually consider deleted discs \(D_{r_i}(z_i)\) and obtain the same result, in which case our theorem on the annulus is a special case of the above theorem.
Tags: analysis complex_analysis harmonic_functions
<!--ID: 1626993604756-->
END
\end{anki}


\begin{proof}
	
\end{proof}

%\subsection{Poisson Formula}
%\label{sub:poisson_formula}

%\subsection{Construction of Harmonic Functions}
%\label{sub:construction_of_harmonic_functions}



% \printindex
\end{document}


\chapter{Analytic Continuation} % Cool, but not necessary for exam, so putting it on hold. May want to use it for reference later?
\label{cha:analytic_continuation}
%\input{AnalyticContinuation}

Under construction

\section{Continuation via Reflection}
\label{sec:continuation_via_reflection}

\documentclass{memoir}
\usepackage{notestemplate}

%\logo{~/School-Work/Auxiliary-Files/resources/png/logo.png}
%\institute{Rice University}
%\faculty{Faculty of Whatever Sciences}
%\department{Department of Mathematics}
%\title{Class Notes}
%\subtitle{Based on MATH xxx}
%\author{\textit{Author}\\Gabriel \textsc{Gress}}
%\supervisor{Linus \textsc{Torvalds}}
%\context{Well, I was bored...}
%\date{\today}

%\makeindex

\begin{document}

% \maketitle

% Notes taken on 

We have noted previously that a holomorphic function on a domain \(U\subset \Omega \subset \C\) may have a holomorphic extension onto \(\Omega \) which we refer to as its analytic continuation.
It is not always simple to construct such an extension-- however, if one considers the special case of a set symmetric over the real line, we can find conditions that give us these extensions.

\begin{defn}[Symmetric Set over \(\R\)]
	We say a set \(\Omega \) is \textbf{symmetric over the real axis} if \(z \in \Omega \iff \overline{z} \in \Omega \). If this holds, we denote by \(\Omega^{+}\subset \Omega \), \(\Omega^{-}\subset \Omega \), and \(\Omega^{\R}\) the subsets given by 
	\begin{align*}
		\Omega^{+} &:= \left\{z \in \Omega  \mid \textrm{Im}(z)>0 \right\} \\
		\Omega^{\R}&:= \left\{z \in \Omega  \mid \textrm{Im}(z)=0 \right\} \\
		\Omega^{-} &:= \left\{z \in \Omega  \mid \textrm{Im}(z)<0 \right\} 
	\end{align*}
	so that
	\begin{align*}
		\Omega = \Omega^{+} \sqcup \Omega^{\R} \sqcup \Omega^{-}.
	\end{align*}
\end{defn}

\begin{anki}
TARGET DECK
Complex Qual::Complex Analysis
START
MathJaxCloze
Text: We say a set \(\Omega \) is \textbf{symmetric over the real axis} if {{c1::\(z \in \Omega \iff \overline{z} \in \Omega \)}}. If this holds, we denote by \(\Omega^{+}\subset \Omega \), \(\Omega^{-}\subset \Omega \), and \(\Omega^{\R}\) the subsets given by 
{{c1::\(\begin{align*}
        	\Omega^{+} &:= \left\{z \in \Omega  \mid \textrm{Im}(z)>0 \right\} \\
        	\Omega^{\R}&:= \left\{z \in \Omega  \mid \textrm{Im}(z)=0 \right\} \\
        	\Omega^{-} &:= \left\{z \in \Omega  \mid \textrm{Im}(z)<0 \right\} 
        \end{align*}\)}}
	so that
\(\begin{align*}
	\Omega = \Omega^{+} \sqcup \Omega^{\R} \sqcup \Omega^{-}.
  \end{align*}\)
Tags: analysis complex_analysis analytic_continuation defn
<!--ID: 1626804188322-->
END
\end{anki}

Before we discuss holomorphic functions on symmetric sets, we address the more rigid class of harmonic functions.

\begin{thm}[Schwarz Reflection Principle (Harmonic Functions)]
	Let \(\Omega \subset \C\) be an open symmetric set, and let \(v\) be a continuous real-valued harmonic function on \(\Omega^{+}\cup \Omega^{\R}\) with \(v(\Omega^{\R})=0\).
	Then \(v\) extends to a harmonic function on \(\Omega \).
\end{thm}
We chose to refer to this harmonic function as \(v\) because one can see that we can treat \(v\) as the imaginary part of a holomorphic function, which will give us the Schwarz Reflection Principle for holomorphic functions.

\begin{anki}
START
MathJaxCloze
Text: **Schwarz Reflection Principle for Harmonic Functions**
Let \(\Omega \subset \C\) be an open symmetric set, and let \(v\) be a continuous real-valued harmonic function on {{c1::\(\Omega^{+}\cup \Omega^{\R}\)}} with \(v(\Omega^{\R})=0\). Then \(v\) extends to a harmonic function on \(\Omega \).
Extra: We chose to refer to this harmonic function as \(v\) because one can see that we can treat \(v\) as the imaginary part of a holomorphic function, which will give us the Schwarz Reflection Principle for holomorphic functions.
Tags: analysis complex_analysis analytic_continuation
<!--ID: 1626804188340-->
END
\end{anki}


\begin{thm}[Schwarz Reflection Principle (Holomorphic Functions)]
Let \(\Omega \subset \C\) be an open set symmetric on the real axis. Let \(f\) be a complex-valued function on \(\Omega \).
\begin{enumerate}[(i).]
	\item If \(f\) is holomorphic on \(\Omega^{+}\cup \Omega^{-}\) and continuous on \(\Omega^{\R}\), then \(f\) is holomorphic on \(\Omega \).
	\item It follows from (i). that if \(f\) is holomorphic on \(\Omega^{+}\) and continuous on \(\Omega^{\R}\) with \(f(\Omega^{\R}) \subset \R\) (that is, \(f\) is real-valued on \(\Omega^{\R}\)), then \(f\) has a unique analytic continuation \(F\) on \(\Omega \) given by
		\begin{align*}
			F(z) = \begin{cases}
				f(z) & z \in \Omega^{+}\cup \Omega^{\R}\\
				\overline{f(\overline{z})} & z \in \Omega^{-}
			\end{cases}.
		\end{align*}
\end{enumerate}
\end{thm}
There are actually two ways to prove this theorem-- we give both here as it is valuable to understand how to prove it both ways.

\begin{proof}[Schwarz Reflection Principle via Harmonic Functions]
	
\end{proof}

\begin{proof}[Schwarz Reflection Principle via Cauchy's Theorem]
	
\end{proof}

\begin{anki}
START
MathJaxCloze
Text: **Schwarz Reflection Principle for Holomorphic Functions**
Let \(\Omega \subset \C\) be an open set symmetric on the real axis. Let \(f\) be a complex-valued function on \(\Omega \).

* If \(f\) is holomorphic on {{c1::\(\Omega^{+}\cup \Omega^{-}\)}} and continuous on {{c1::\(\Omega^{\R}\)}}, then \(f\) is holomorphic on \(\Omega \).
* It follows from the above that if \(f\) is holomorphic on {{c1::\(\Omega^{+}\)}} and continuous on {{c1::\(\Omega^{\R}\)}} with {{c1::\(f(\Omega^{\R}) \subset \R\)}} (that is, \(f\) is {{c1::real-valued on \(\Omega^{\R}\)}}), then \(f\) has a unique analytic continuation \(F\) on \(\Omega \) given by
{{c1::\(\begin{align*}
        	F(z) = \begin{cases}
        		f(z) & z \in \Omega^{+}\cup \Omega^{\R}\\
        		\overline{f(\overline{z})} & z \in \Omega^{-}
        	\end{cases}.
        \end{align*}\)}}
Tags: analysis complex_analysis analytic_continuation
<!--ID: 1625527512412-->
END
\end{anki}

If our holomorphic function is an isomorphism, we can give an interesting statement that will prove to be quite powerful.

\begin{prop}[Schwarz Principle for Isomorphisms]
	Let \(\Omega \subset \C\) be given and let \(f\) be a holomorphic function on \(\Omega \) with \(f(\Omega^{\R}) \subset \R\).
	If \(f\) is an isomorphism on \(\Omega^{+}\cup \Omega^{-}\) with
	\begin{align*}
		f(\Omega^{+}) &\subset \mathbb{H}\\
		f(\Omega^{-}) &\subset \overline{\mathbb{H}}
	\end{align*}
	then \(f:\Omega \to f(\Omega )\) is an isomorphism.
\end{prop}
Of course, the Schwarz Reflection Principle applies on \(\C\) so the isomorphism allows us to apply the principle to our original set \(\Omega \).

\begin{proof}
	
\end{proof}

\begin{anki}
START
MathJaxCloze
Text: **Schwarz Principle for Isomorphisms** 
Let \(\Omega \subset \C\) be given and let \(f\) be a holomorphic function on \(\Omega \) with \(f(\Omega^{\R}) \subset \R\).
If \(f\) is an isomorphism on \(\Omega^{+}\cup \Omega^{-}\) with
{{c1::\(\begin{align*}
		f(\Omega^{+}) &\subset \mathbb{H}\\
		f(\Omega^{-}) &\subset \overline{\mathbb{H}}
	\end{align*}\)}}
	then \(f:\Omega \to f(\Omega )\) is {{c1::an isomorphism}}.
Extra: Of course, the Schwarz Reflection Principle applies on \(\C\) so the isomorphism allows us to apply the principle to our original set \(\Omega \).
Tags: analysis complex_analysis analytic_continuation
<!--ID: 1626993866883-->
END
\end{anki}


\subsection{Reflection Over Holomorphic Arcs}
\label{sub:reflection_over_Holomorphic_arcs}

The real axis gave us a nice condition to impose on our function in order to induce the analytic continuation, but it is not hard to imagine that the idea can be extended to a more general notion of symmetry.

\begin{defn}[Isomorphically Symmetric]
	We say an open set \(\Omega\subset \C \) is \textbf{isomorphically symmetric over \(\gamma \)} if there exists a disjoint partition of \(\Omega \) into two open sets and a curve:
	\begin{align*}
		\Omega = \Omega^{+} \sqcup \gamma \sqcup \Omega^{-}
	\end{align*}
	and a holomorphic isomorphism \(\psi:\Omega \to \Omega'\) into a set symmetric over the real axis, satisfying
	\begin{align*}
		\psi (\Omega^{+}) &= \Omega'^{+}\\
		\psi(\gamma ) &= \Omega'^{\R}\\
		\psi (\Omega^{-}) &= \Omega'^{-}.
	\end{align*}
\end{defn}

\begin{anki}
START
MathJaxCloze
Text: We say an open set \(\Omega\subset \C \) is **isomorphically symmetric over \(\gamma \)** if there exists a disjoint partition of \(\Omega \) into two open sets and a curve:
\(\begin{align*}
  	\Omega = \Omega^{+} \sqcup \gamma \sqcup \Omega^{-}
  \end{align*}\)
and a {{c1::holomorphic isomorphism}} \(\psi:\Omega \to \Omega'\) into a set \(\Omega '\) {{c1::symmetric over the real axis}}, satisfying
{{c1::\(\begin{align*}
        	\psi (\Omega^{+}) &= \Omega'^{+}\\
        	\psi(\gamma ) &= \Omega'^{\R}\\
        	\psi (\Omega^{-}) &= \Omega'^{-}.
        \end{align*}\)}} 
Tags: analysis complex_analysis analytic_continuation defn
<!--ID: 1626993866909-->
END
\end{anki}

The Schwarz Reflection Principle follows directly by composition.
\begin{thm}[Schwarz Reflection Principle Over Arcs]
	Let  \(\Omega \subset \C\) be an set isomorphically symmetric over \(\gamma \), and suppose there is a function \(f\) on \(\Omega \). Then
	\begin{enumerate}[(i).]
		\item If \(f\) is holomorphic on \(\Omega^{+}\) and \(\Omega^{-}\) and continuous on \(\gamma \), then \(f\) is holomorphic on \(\Omega \).
		\item If \(f\) is holomorphic on \(\Omega^{+}\) and continuous on \(\gamma \) with \(f(\gamma)\subset \R\), then \(f\) is holomorphic on \(\Omega \).
	\end{enumerate}
\end{thm}

This is very natural due to the Riemann Mapping Theorem, but \(\gamma \) can be very poorly behaved, and so it may be difficult for \(f\) to be continuous on \(\gamma \).
It will help us to define some additional structure on our curve \(\gamma \).

\begin{anki}
START
MathJaxCloze
Text: **Schwarz Reflection Principle Over Arcs**
Let  \(\Omega \subset \C\) be an set isomorphically symmetric over \(\gamma\subset \Omega \), and suppose there is a function \(f\) on \(\Omega \). Then

* If \(f\) is holomorphic on \(\Omega^{+}\) and \(\Omega^{-}\) and continuous on \(\gamma \), then {{c1::\(f\) is holomorphic on \(\Omega \)}}.
* If \(f\) is holomorphic on \(\Omega^{+}\) and continuous on \(\gamma \) with \(f(\gamma)\subset \R\), then \(f\) is holomorphic on \(\Omega \).
Extra: This is very natural due to the Riemann Mapping Theorem, but \(\gamma \) can be very poorly behaved, and so it may be difficult for \(f\) to be continuous on \(\gamma \).
Tags: analysis complex_analysis analytic_continuation
<!--ID: 1626993866925-->
END
\end{anki}


%% All in Lang IX sec 2 "Reflection Across Analytic Arcs"

% Defn for real analytic and proper analytic

% Open neighborhood around [a,b] in C for the curve that is analytic isomorphism

% Analytic continuation across gamma

% Harmonic version

% \printindex
\end{document}


\chapter{Entire and Meromorphic Functions}
\label{cha:entire_and_meromorphic_functions}

\section{Partial Fraction Decomposition}
\label{sec:partial_fraction_decomposition}

%\input{PartialFraction}
Under construction

\section{Infinite Products}
\label{sec:infinite_products}


\documentclass{memoir}
\usepackage{notestemplate}

%\logo{~/School-Work/Auxiliary-Files/resources/png/logo.png}
%\institute{Rice University}
%\faculty{Faculty of Whatever Sciences}
%\department{Department of Mathematics}
%\title{Class Notes}
%\subtitle{Based on MATH xxx}
%\author{\textit{Author}\\Gabriel \textsc{Gress}}
%\supervisor{Linus \textsc{Torvalds}}
%\context{Well, I was bored...}
%\date{\today}

%\makeindex

\begin{document}

% \maketitle

% Notes taken on 

\begin{defn}[Convergence of Infinite Products]
	Let \(\left\{ a_n \right\}_{n=1}^{\infty}\) be a sequence of non-zero complex numbers. We say that the **infinite produc**t
	\begin{align*}
		\prod_{n=1}^{\infty} a_n 
	\end{align*}
	\textbf{converges absolutely} if
	 \begin{align*}
		\lim_{n \to \infty} a_n = 1
	\end{align*}
	and if the corresponding series
	\begin{align*}
		\sum_{n=1}^{\infty} \ln(a_n)
	\end{align*}
	converges absolutely.
\end{defn}
The transformation between the products of \(a_n\) and the sum of \(\ln(a_n)\) is natural, as exponentiation gives equality for partial sums/products.
\begin{anki}
TARGET DECK
Complex Qual::Complex Analysis
START
MathJaxCloze
Text: Let \(\left\{ a_n \right\}_{n=1}^{\infty}\) be a sequence of non-zero complex numbers. We say that the **infinite product**
\(\begin{align*}
  	\prod_{n=1}^{\infty} a_n 
  \end{align*}\)
**converges absolutely** if
{{c1::\( \begin{align*}
      	\lim_{n \to \infty} a_n = 1
        \end{align*}\)::coefficients}}
and if the corresponding power series
{{c1::\(\begin{align*}
        	\sum_{n=1}^{\infty} \ln(a_n)
        \end{align*}\)}} 
converges absolutely.
Extra: The transformation between the products of \(a_n\) and the sum of \(\ln(a_n)\) is natural, as exponentiation gives equality for partial sums/products.
Tags: analysis complex_analysis entire_meromorphic
<!--ID: 1626995479520-->
END
\end{anki}
An astute reader will notice that we need to be careful about our determination of \(\ln(a_n)\).
For finitely many \(\ln(a_n)\), we can take any determination without concern-- but as \(a_n\) approaches 1, we will face some issues.
Fortunately, there exists an \(N\) so that for all \(n\geq N\), we can express \(a_n = 1-\alpha_n\) for some \(\left| \alpha_n \right| <1\), for which the logarithm will remain well-defined for the rest of the sequence.
Of course, we ought to verify this transformation won't affect convergence.

\begin{lemma}
	Let \(\left\{ a_n \right\}_{n=1}^{\infty}\) be a sequence of complex numbers with \(a_n\neq 1\) for all \(n\). Suppose that \(\left\{ a_n \right\} \) is absolutely convergent:
	\begin{align*}
		\sum_{n=1}^{\infty} \left| a_n \right| 
	\end{align*}
	Then
	\begin{align*}
		\prod_{n=1}^{\infty} (1-a_n) 
	\end{align*}
	converges absolutely.
\end{lemma}
This lemma makes our study of infinite products convenient, as it allows us to reduce infinite products to infinite sums, a case which we are already familiar with.\\

If we consider infinite products of functions, we see something rather interesting.
The lemma above already gives us conditions for uniform convergence of
\begin{align*}
	\prod_{n=1}^{\infty} (1-g_n(z)) .
\end{align*}
Furthermore, we can leverage our knowledge of the logarithmic derivative to obtain more information about uniformly convergent products of functions.
\begin{lemma}
	Let \(\Omega \subset \C\) be an open set and \(\left\{ f_n \right\}_{n=1}^{\infty}\) a sequence of holomorphic functions on \(\Omega \).
	Consider the corresponding sequence \(\left\{ g_n \right\}_{n=1}^{\infty}\) determined so that \(f_n(z) = 1 + g_n(z)\), and suppose that
	\begin{align*}
		\sum_{n=1}^{\infty} g_n(z)
	\end{align*}
	converges uniformly and absolutely on \(\Omega \).
	Let \(K\subset \Omega \) be a compact subset so that \(f_n^{-1}(0)\cap K = \emptyset\) for all \(n\).
	Then the infinite product of \(f_n\) converges to a holomorphic function on \(\Omega \):
	\begin{align*}
		\prod_{n=1}^{\infty} f_n = f 
	\end{align*}
	and we have absolute and uniform convergence on \(K\) for the following sum:
	\begin{align*}
		\frac{f'}{f} = \sum_{n=1}^{\infty} \frac{f_n'}{f_n}.
	\end{align*}
\end{lemma}

\begin{anki}
START
MathJaxCloze
Text: Let \(\Omega \subset \C\) be an open set and \(\left\{ f_n \right\}_{n=1}^{\infty}\) a sequence of holomorphic functions on \(\Omega \).
Consider the corresponding sequence \(\left\{ g_n \right\}_{n=1}^{\infty}\) determined so that \(f_n(z) = 1 + g_n(z)\), and suppose that
\(\begin{align*}
  	\sum_{n=1}^{\infty} g_n(z)
  \end{align*}\)
converges {{c1::uniformly and absolutely on \(\Omega \)}}.
Let \(K\subset \Omega \) be a compact subset so that {{c1::\(f_n^{-1}(0)\cap K = \emptyset\)}} for all \(n\).
Then the infinite product of \(f_n\) {{c1::converges to a holomorphic function}} on \(\Omega \):
{{c1::\(\begin{align*}
        	\prod_{n=1}^{\infty} f_n = f 
        \end{align*}\)}}
and we have absolute and uniform convergence on \(K\) for the following sum:
{{c1::\(\begin{align*}
        	\frac{f'}{f} = \sum_{n=1}^{\infty} \frac{f_n'}{f_n}.
        \end{align*}\)::logarithmic derivative}} 
Tags: analysis complex_analysis entire_meromorphic
<!--ID: 1626995479539-->
END
\end{anki}


\begin{hw}[Blaschke Products]
	Let \(\left\{ a_n \right\} \subset D_1 \) be a sequence in the unit disc such that \(a_n\neq 0\) for all \(n\) and
	\begin{align*}
		\sum_{n=1}^{\infty} (1-\left| a_n \right| )
	\end{align*}
	converges.
	Show that the \textbf{Blaschke product}
	\begin{align*}
		f(z) = \prod_{n=1}^{\infty} \frac{\left| a_n \right| }{a_n}\cdot \frac{a_n - z}{1 - \overline{a}_n z} 
	\end{align*}
	converges uniformly on \(\left| z \right| \leq r<1\) for some fixed \(r\) and defines a holomorphic function on \(D_1\) having only the zeros \(\left\{ a_n \right\} \).
	Furthermore, show that  \(\left| f(z) \right| \leq 1\).\\

	(Hint: prove that for \(0<\left| a \right| <1\) and for some fixed \(r\) and  \(\left| z \right| \leq r<1\), the inequality
	\begin{align*}
		\left| \frac{a + \left| a \right| z}{(1-\overline{a}z)a} \right| \leq \frac{1+r}{1-r}
	\end{align*}
	holds)
\end{hw}
One can use Blaschke products to construct some unusual functions.
For example, if we choose \(a_n = 1-\sfrac{1}{n^2}\), then our resulting function is holomorphic on the unit disc with a zero at 1.
Modifying this construction allows us to construct a bounded holomorphic function \(f\) on \(D_1\) for which each point of the unit circle is a singularity.
(Note that this is a useful example that demonstrates that non-isolated singularities need not conform to our standard types of singularities-- we refer to this form of non-isolated singularity as a \textbf{natural boundary})

\begin{anki}
START
MathJaxCloze
Text: Let \(\left\{ a_n \right\} \subset D_1 \) be a sequence in the unit disc such that \(a_n\neq 0\) for all \(n\) and
\(\begin{align*}
  	\sum_{n=1}^{\infty} (1-\left| a_n \right| )
  \end{align*}\)
converges.
Then the **Blaschke product**
\(\begin{align*}
  	f(z) = \prod_{n=1}^{\infty} \frac{\left| a_n \right| }{a_n}\cdot \frac{a_n - z}{1 - \overline{a}_n z} 
  \end{align*}\)
  {{c1::converges uniformly}} on \(\left| z \right| \leq r<1\) for some fixed \(r\) and defines a {{c1::holomorphic function on \(D_1\) having only the zeros \(\left\{ a_n \right\} \)}}.
Furthermore,  \(\left| f(z) \right| \leq 1\).
Extra: One can use Blaschke products to construct some unusual functions. For example, if we choose \(a_n = 1-\sfrac{1}{n^2}\), then our resulting function is holomorphic on the unit disc with a zero at 1. Modifying this construction allows us to construct a bounded holomorphic function \(f\) on \(D_1\) for which each point of the unit circle is a singularity.
Tags: analysis complex_analysis entire_meromorphic
<!--ID: 1626995479562-->
END
\end{anki}

\subsection{Weierstrass Products}
\label{sub:weierstrass_products}

Our goal will be to show that we can use infinite products to classify entire functions.
We will classify a restricted class of entire functions, then utilize this to extend to the general case.

\begin{thm}[Non-vanishing Entire Functions]
	Let \(f\) be a non-vanishing entire function. Then there exists a second entire function \(g\) so that
	\begin{align*}
		f(z) = e^{g(z)}.
	\end{align*}
	Furthermore, \(g\) is unique up to an additive constant. That is, if
	\begin{align*}
		f(z) = \lambda e^{h(z)}
	\end{align*}
	for some \(\lambda \in \C\setminus\left\{ 1 \right\} \), then
	\begin{align*}
		h(z) = g(z) + \ln(\lambda ).
	\end{align*}
\end{thm}
This follows from our logarithmic derivatives directly.
We leave the verification as an exercise to the reader.

\begin{anki}
START
MathJaxCloze
Text: Let \(f\) be a non-vanishing entire function. Then there exists a second entire function \(g\) so that
{{c1::\(\begin{align*}
        	f(z) = e^{g(z)}.
        \end{align*}\)}}
Furthermore, \(g\) is unique up to an additive constant. That is, if
{{c1::\(\begin{align*}
      	f(z) = \lambda e^{h(z)}
        \end{align*}\)}}
for some \(\lambda \in \C\setminus\left\{ 1 \right\} \) and distinct entire function \(h\), then
{{c1::\(\begin{align*}
      	h(z) = g(z) + \ln(\lambda ).
        \end{align*}\)}}
Extra: This follows from our logarithmic derivatives directly.
Suppose \(f,g\) are two entire functions with the same zeros of equal multiplicity. Then it follows that
\(\begin{align*}
  	f(z) = g(z)e^{h(z)}
  \end{align*}\)
for some entire function \(h(z)\) (uniquely determined up to an additive constant).
It also follows that for \(h\) entire,
\(\begin{align*}
  	g(z) = 0 \iff g(z)e^{h(z)} = 0
  \end{align*}\)
with the same multiplicities.
Tags: analysis complex_analysis entire_meromorphic
<!--ID: 1626995479580-->
END
\end{anki}

Suppose \(f,g\) are two entire functions with the same zeros of equal multiplicity. Then it follows that
\begin{align*}
	f(z) = g(z)e^{h(z)}
\end{align*}
for some entire function \(h(z)\) (uniquely determined up to an additive constant).
It also follows that for \(h\) entire,
\begin{align*}
	g(z) = 0 \iff g(z)e^{h(z)} = 0
\end{align*}
with the same multiplicities.
Hence, we can construct a canonical entire function for a set of zeros of fixed multiplicity-- and thus all entire functions with those zeros of fixed multiplicity can be expressed in terms of the canonical form.\\

We will now give the intuition for the canonical form.
First, we should order our zeros by increasing absolute value, so that our zeros \(\left\{ z_n \right\} \) satisfy
\begin{align*}
	\left| z_1 \right| \leq \left| z_2 \right| \leq \ldots
\end{align*}
We'd like to define our function by the infinite product
\begin{align*}
	\prod_{n=1}^{\infty} \left( 1 - \frac{z}{z_n} \right)  
\end{align*}
but this product may not converge.
To resolve this, we introduce a convergence factor which does not introduce any zeros-- an exponential.
Our exponent in this term ought to be a polynomial whose degree is dependent on the term of the sequence (to ensure independence between terms).
In other words, our convergence factor will be of the form
\begin{align*}
	e^{w_n + \frac{1}{2}w_n^2 + \ldots + \frac{1}{n-1}w_n^{n-1}}
\end{align*}
where \(w_n = \sfrac{z}{z_n}\). We combine the convergence term with our original terms and write
\begin{align*}
	E_n(w) = (1-w) e^{w_n + \frac{1}{2}w_n^2 + \ldots + \frac{1}{n-1}w_n^{n-1}}.
\end{align*}
The polynomial in the exponent is chosen because
\begin{align*}
	\ln(1-z) = \sum_{n=1}^{\infty} -\frac{z^{n}}{n}
\end{align*}
and so
\begin{align*}
	\ln \left( \prod_{n=1}^{\infty} E_n\left(\sfrac{z}{z_n}\right)  \right)\\
	= \sum_{n=1}^{\infty} \ln \left(E_n\left(\sfrac{z}{z_n}\right) \right)\\
	= \sum_{n=1}^{\infty} \left( \frac{z}{z_n} + \ldots + \frac{1}{n-1}\frac{z}{z_{n-1}} \right) + \ln \left( 1 - \frac{z}{z_n}\right) \\
	= \sum_{n=1}^{\infty} \sum_{k=n}^{\infty} - \frac{1}{k} \left( \frac{z}{z_n} \right)^{k} 
\end{align*}
Of course, this identity is desirable as we want to show our infinite product converges absolutely.\\

Now we formally justify the work we've shown thus far.
First, we verify that convergence will occur as we expect.

\begin{lemma}
	If \(\left| w \right| \leq \frac{1}{2}\) then
	\begin{align*}
		\frac{\left| \ln E_n(w) \right| }{\left| w \right|^{n}} \leq 2.
	\end{align*}
	Furthermore, let a sequence of complex numbers \(\left\{ z_n \right\} \) be given with \(\left| z_1 \right| \leq \left| z_2 \right| \leq \ldots\). There exists a corresponding increasing sequence of positive integers \(\left\{ k_n \right\} \) so that, for all positive real \(a>0\) 
	\begin{align*}
		\sum_{n=1}^{\infty} \left( \frac{a}{\left| z_n \right| } \right)^{k_n} 
	\end{align*}
	converges. In fact, for every \(a>0\) there is a corresponding integer \(N_a>0\) so that for all \(k_n \geq N_a\):
	\begin{align*}
		\left( \frac{a}{\left| z_n \right| } \right)^{k_n} \leq \frac{1}{2^{k_n}}.
	\end{align*}
\end{lemma}

Now that we have introduced this sequence \(\left\{ k_n \right\} \), we can finally connect the various ideas we've constructed together into a well-defined product with the properties desired.

\begin{thm}[Weierstrass Product Theorem]
	Let \(\left\{ z_n \right\}_{n=1}^{\infty} \subset \C\setminus\left\{ 0 \right\} \) be a sequence of complex numbers in the complex plane with
	\begin{align*}
		\left| z_1 \right| \leq \left| z_2 \right| \leq \ldots
	\end{align*}
	and let \(\left\{ k_n \right\} \subset \N\) be the corresponding smallest sequence of positive integers so that for all positive real \(a>0\)
	\begin{align*}
		\sum_{n=1}^{\infty} \left( \frac{a}{\left| z_n \right| } \right)^{k_n} 
	\end{align*}
	converges. Define
	\begin{align*}
		P_n(z) &= \sum_{k=1}^{k_n-1} \frac{z^{k}}{k}\\
		E_n( \sfrac{z}{z_n}) &= \left( 1- \frac{z}{z_n} \right) e^{P_n(z / z_n)} .
	\end{align*}
	Then
	\begin{align*}
		\prod_{n=1}^{\infty} E_n( \sfrac{z }{z_n})   
	\end{align*}
	converges uniformly and absolutely on every disc \(D_a\), and hence defines an entire function with zeros exclusively at \(\left\{ z_n \right\} \).\\

	If \(\sup_{n} k_n = k< \infty\), then we take the canonical sequence to be \(k_n = \sup_{n} k_n\). In this case, we refer to \(E_n(\sfrac{z}{z_n})\) as the \textbf{elementary form} and the product
	 \begin{align*}
		 z^{m} \prod_{n=1}^{\infty} E_n \left( \sfrac{z}{z_n} \right)
	\end{align*}
	as the \textbf{Weierstrass product} and take it to be the canonical form for a set of zeros \(\left\{ z_n \right\} \subset \C\setminus\left\{ 0 \right\} \).
\end{thm}

\begin{anki}
START
MathJaxCloze
Text: **Weierstrass Product Theorem**
Let \(\left\{ z_n \right\}_{n=1}^{\infty} \subset \C\setminus\left\{ 0 \right\} \) be a sequence of complex numbers in the complex plane with
\(\begin{align*}
  	\left| z_1 \right| \leq \left| z_2 \right| \leq \ldots
  \end{align*}\)
and let \(\left\{ k_n \right\} \subset \N\) be the corresponding smallest sequence of positive integers so that for all positive real \(a>0\)
\(\begin{align*}
  	\sum_{n=1}^{\infty} \left( \frac{a}{\left| z_n \right| } \right)^{k_n} 
  \end{align*}\)
converges. Define
\(\begin{align*}
  	P_n(z) &= \sum_{k=1}^{k_n-1} \frac{z^{k}}{k}\\
  	E_n( \sfrac{z}{z_n}) &= \left( 1- \frac{z}{z_n} \right) e^{P_n(z / z_n)} .
  \end{align*}\)
Then
 {{c1::\(\begin{align*}
        	\prod_{n=1}^{\infty} E_n( \sfrac{z }{z_n})   
        \end{align*}\)}} 
converges {{c1::uniformly}} and {{c1::absolutely}} on every disc \(D_a\), and hence defines an entire function with zeros {{c1::exclusively at \(\left\{ z_n \right\} \)}}.

If \(\sup_{n} k_n = k< \infty\), then we take the canonical sequence to be \(k_n = \sup_{n} k_n\). In this case, we refer to \(E_n(\sfrac{z}{z_n})\) as the **elementary form** and the product
{{c1::\( \begin{align*}
        	 z^{m} \prod_{n=1}^{\infty} E_n \left( \sfrac{z}{z_n} \right)
        \end{align*}\)}}
as the \textbf{Weierstrass product} and take it to be the canonical form for a set of zeros \(\left\{ z_n \right\} \subset \C\setminus\left\{ 0 \right\} \).
Tags: analysis complex_analysis entire_meromorphic defn
<!--ID: 1626995479602-->
END
\end{anki}


\begin{cor}[Hadamard's Theorem]
	Every entire function \(f\) with zeros exactly at \(\left\{ z_n \right\}_{n=1}^{\infty}\subset \C\setminus\left\{ 0 \right\}\) and possibly at zero with order \(m\) can be written uniquely in the form
	\begin{align*}
	  	f(z) = e^{g(z)} z^{m} \prod_{n=1}^{\infty} E_n( \sfrac{z}{z_n})
	  \end{align*}
	where \(g\) is a polynomial of fixed degree \(\leq \sup_{n} k_n\) uniquely determined up to an additive constant.
\end{cor}

\begin{anki}
START
MathJaxCloze
Text: **Hadamard's Theorem**
Every entire function \(f\) with zeros exactly at \(\left\{ z_n \right\}_{n=1}^{\infty}\subset \C\setminus\left\{ 0 \right\}\) and possibly at zero with order \(m\) can be written uniquely in the form
{{c1::\(\begin{align*}
      	f(z) = e^{g(z)} z^{m} \prod_{n=1}^{\infty} E_n( \sfrac{z}{z_n})
        \end{align*}\)}}
where \(g\) is a polynomial of fixed degree \(\leq \sup_{n} k_n\) uniquely determined up to an additive constant.
Tags: analysis complex_analysis entire_meromorphic
<!--ID: 1626995479618-->
END
\end{anki}


\begin{proof}[Proof of Weierstrass Product Theorem]
	
\end{proof}

Of course, we can immediately leverage our classification of entire functions to classify meromorphic functions on \(\C\).

\begin{cor}[Classification of Meromorphic Functions on \(\C\)]
	Every function \(F\) which is meromorphic in the whole plane can be expressed uniquely by:
	\begin{align*}
		F(z) = f(z)\frac{g(z)}{h(z)}
	\end{align*}
	where \(f(z)\) is a non-vanishing entire function, \(g(z)\) is the canonical Weierstrass product corresponding to the zeros of \(F\) and \(h(z)\) is the canonical Weierstrass product corresponding to the poles of \(F\).
\end{cor}
We can use either this form or equivalently the form
\begin{align*}
	F(z) = e^{f(z)} \frac{g(z)}{h(z)}
\end{align*}
as our canonical choice depending on the context.

\begin{anki}
START
MathJaxCloze
Text: Every function \(F\) which is meromorphic in the whole plane can be expressed uniquely by:
{{c1::\(\begin{align*}
        	F(z) = f(z)\frac{g(z)}{h(z)}
        \end{align*}\)}} 
where \(f(z)\) is a non-vanishing entire function, \(g(z)\) is the canonical Weierstrass product corresponding to the {{c1::zeros of \(F\)}} and \(h(z)\) is the canonical Weierstrass product corresponding to the {{c1::poles of \(F\)}}.
Extra: We can use either this form or equivalently the form
\(\begin{align*}
  	F(z) = e^{f(z)} \frac{g(z)}{h(z)}
  \end{align*}\)
as our canonical choice depending on the context.
Tags: analysis complex_analysis entire_meromorphic
<!--ID: 1626995479636-->
END
\end{anki}
While this form is natural, we will explore other constructions for meromorphic functions later that will prove more fruitful.\\

First, we give an example of a few Weierstrass products and investigate the structure of \(\left\{ k_n \right\} \).

\begin{exmp}
	\begin{align*}
		\sin(\pi z) = \pi z \prod_{n=1}^{\infty} \left( 1- \frac{z^2}{n^2} \right)  .
	\end{align*}
	so
	\begin{align*}
		\frac{\pi ^2}{\sin^2(\pi z)} = \sum_{n=-\infty}^{\infty} \frac{1}{(z-n)^{2}}
	\end{align*}
\end{exmp}


% \printindex
\end{document}


\section{Entire Functions of Finite Order}
\label{sub:entire_functions_of_finite_order}

\documentclass{memoir}
\usepackage{notestemplate}

%\logo{~/School-Work/Auxiliary-Files/resources/png/logo.png}
%\institute{Rice University}
%\faculty{Faculty of Whatever Sciences}
%\department{Department of Mathematics}
%\title{Class Notes}
%\subtitle{Based on MATH xxx}
%\author{\textit{Author}\\Gabriel \textsc{Gress}}
%\supervisor{Linus \textsc{Torvalds}}
%\context{Well, I was bored...}
%\date{\today}

%\makeindex

\begin{document}

% \maketitle

% Notes taken on 


In the statement of the Weierstrass Product Theorem, we implicitly assumed that
\begin{itemize}
	\item there exists a sequence \(\left\{ k_n \right\} \subset \N\) so that
		\begin{align*}
			\sum_{n=1}^{\infty} \left( \frac{a}{\left| z_n \right| } \right)^{k_n}
		\end{align*}
		converges
	\item within the set of sequences that give convergence, there exists a "minimal" sequence.
\end{itemize}
We proved the first point in the proof of the theorem-- one can see that choosing \(k_n = n\) will guarantee convergence in all cases. As for the question of minimality, we explore in this section what a minimal sequence might be like, and how it reflects the underlying structure of our entire function.

\begin{defn}[Order of Entire Function]
	Let \(f\) be an entire function. We say that \(f\) is of \textbf{order \(\leq \rho \)} for real \(\rho >0\) if for all \(\varepsilon>0\), there exists a constant \(C_\varepsilon\) and a corresponding \(R_{C_\varepsilon} >0\) such that, for all \(R>R_{C_\varepsilon}\),
	\begin{align*}
		\sup_{z \in D_R} \left| f(z) \right| \leq C_\varepsilon^{R \raisebox{+4pt}{\scalebox{0.65}{\(\,\rho +\varepsilon\)} } }.
	\end{align*}
	We say that \(f\) is of \textbf{strict order \(\leq \rho \)} if the inequality holds without the \(\varepsilon\):
	\begin{align*}
		\sup_{z \in D_R} \left| f(z) \right| \leq C_\varepsilon^{R \raisebox{+4pt}{\scalebox{0.65}{\(\,\rho\)} } }.
	\end{align*}
	Finally, the function \(f\) is of \textbf{order \(\rho \)} or \textbf{strict order \(\rho \)} if \(\rho \) is the greatest lower bound to make the inequality valid.
\end{defn}

\begin{anki}
TARGET DECK
Complex Qual::Complex Analysis
START
MathJaxCloze
Text: Let \(f\) be an entire function. We say that \(f\) is of **order \(\leq \rho \)** for real \(\rho >0\) if for all \(\varepsilon>0\), there exists a constant \(C_\varepsilon\) and a corresponding \(R_{C_\varepsilon} >0\) such that, for all \(R>R_{C_\varepsilon}\),
{{c1::\(\begin{align*}
      	\sup_{z \in D_R} \left| f(z) \right| \leq C_\varepsilon^{R^{\rho +\varepsilon}} .
        \end{align*}\)}}
We say that \(f\) is of **strict order \(\leq \rho \)** if the inequality holds without the \(\varepsilon\):
{{c1::\(\begin{align*}
        	\sup_{z \in D_R} \left| f(z) \right| \leq C_\varepsilon^{R^{\rho} }.
        \end{align*}\)}}
Finally, the function \(f\) is of **order \(\rho \)** or **strict order \(\rho \)** if \(\rho \) is the {{c1::greatest lower bound}} to make the inequality valid.
Tags: analysis complex_analysis entire_meromorphic
<!--ID: 1626995840083-->
END
\end{anki}


\begin{exmp}
	The function \(e^{z}\) is strict order 1 because
	\begin{align*}
		\left| e^{z} \right| = e^{x} \leq e^{\left| z \right| }.
	\end{align*}
\end{exmp}

For a more involved example, if we have a canonical Weierstrass product with \(\sup_{n} \left\{ k_n \right\} =k\), then for all \(\rho \) satisfying \(k-1<\rho <k\), the Weierstrass product is of order \(\leq \rho \). The following lemma allows us to obtain the converse.

\begin{lemma}
	Let \(f\) be an entire function of strict order \(\leq \rho \). Then for \(R\) sufficiently large,
	\begin{align*}
		\frac{1}{2\pi i}\int_{\partial D_R}\frac{f'}{f} \,d t \leq R^{\rho }.
	\end{align*}
	Of course because \(f\) is entire, the left-hand side represents the number of zeros within the disc of radius \(R\).
\end{lemma}
The following corollary immediately follows:
\begin{cor}
	Let \(f\) have strict order \(\leq \rho \), and let \(\left\{ z_n \right\} \subset \C\setminus\left\{ 0 \right\} \) be the zeros of \(f\) ordered by increasing modulus. Then for every \(\varepsilon>0\), the series
	\begin{align*}
		\sum_{n=1}^{\infty} \frac{1}{\left| z_n \right|^{\rho + \delta }}
	\end{align*}
	converges.
\end{cor}
This corollary then gives us the converse-- every entire function of strict order \(\leq \rho \) has a Weierstrass product form with \(\rho < \sup_{n} k_n < \rho +1\).
We summarize these results into the minimum modulus theorem.
\begin{thm}[Minimum Modulus Theorem]
	Let \(f\) be an entire function of order \(\leq \rho \), and let \(\left\{ z_n \right\} \subset \C\) be its sequence of zeros ordered by increasing modulus. Then for all \(s>\rho \), \(\varepsilon>0\), there is a corresponding \(R_0\) so that, for all \(R > R_0\) and \(z \in \C\setminus \overline{D_s}\):
	\begin{align*}
		\left| f(z) \right| \geq e^{-R\raisebox{+4pt}{\scalebox{0.65}{\(\,\rho +\varepsilon\)} }}
	\end{align*}
\end{thm}
This ensures the construction we gave in Hadamard's Theorem is valid and unique.

\begin{anki}
START
MathJaxCloze
Text: **Minimum Modulus Theorem**
Let \(f\) be an entire function of order \(\leq \rho \), and let \(\left\{ z_n \right\} \subset \C\) be its sequence of zeros ordered by increasing modulus. Then for all \(s>\rho \), \(\varepsilon>0\), there is a corresponding \(R_0\) so that, for all \(R > R_0\) and \(z \in \C\setminus \overline{D_s}\):
{{c1::\(\begin{align*}
        	\left| f(z) \right| \geq e^{-R ^{\rho +\varepsilon} }
        \end{align*}\)}}
Extra: This ensures the construction we gave in Hadamard's Theorem is valid and unique.
Tags: analysis complex_analysis entire_meromorphic
<!--ID: 1626995840102-->
END
\end{anki}


% \printindex
\end{document}


\section{Meromorphic Functions}
\label{sec:meromorphic_functions}


\documentclass{memoir}
\usepackage{notestemplate}

%\logo{~/School-Work/Auxiliary-Files/resources/png/logo.png}
%\institute{Rice University}
%\faculty{Faculty of Whatever Sciences}
%\department{Department of Mathematics}
%\title{Class Notes}
%\subtitle{Based on MATH xxx}
%\author{\textit{Author}\\Gabriel \textsc{Gress}}
%\supervisor{Linus \textsc{Torvalds}}
%\context{Well, I was bored...}
%\date{\today}

%\makeindex

\begin{document}

% \maketitle

% Notes taken on 

Recall that if a function \(f\) is meromorphic with a pole at \(z_0\) of order \(k\), then \(f\) has a Laurent expansion at \(z_0\) of the form
\begin{align*}
	f(z) = \sum_{n=-k}^{\infty} a_n (z-z_0)^{n}
\end{align*}
and we refer to the partial sum
\begin{align*}
	\sum_{n=-k}^{-1} a_n (z-z_0)^{n}
\end{align*}
as the principal part of \(f\) at \(z_0\).

\begin{thm}[Mittag-Leffler Theorem]
	Let \(\left\{ z_n \right\} \subset \C\) be a sequence of distinct complex numbers with \( \lim_{n \to \infty} \left| z_n \right|= \infty \). Let \(\left\{ p_n(z) \right\} \) be a sequence of polynomials with \(p_n(0)=0\). Then there exists a function \(f\) meromorphic on the complex plane with poles exclusively at \(\left\{ z_n \right\} \) whose principal part at each \(z_n\) is \(p_n(\frac{1}{z-z_n})\). In fact, every meromorphic function \(f\) with poles at \(\left\{ z_n \right\} \) is of the form
	\begin{align*}
		f(z) = \sum_{n=1}^{\infty} \left[ p_n\left( \frac{1}{z-z_n} \right) - q_n(z) \right] + \varphi (z)
	\end{align*}
	for some polynomials \(\left\{ q_n \right\} \) and an entire function \(\varphi \). The sum converges absolutely and uniformly on all \(K\subset \C\) compact with \(K \cap \left\{ z_n \right\} = \emptyset\).
\end{thm}

\begin{anki}
TARGET DECK
Complex Qual::Complex Analysis
START
MathJaxCloze
Text: **Mittag-Leffler Theorem**
Let \(\left\{ z_n \right\} \subset \C\) be a sequence of distinct complex numbers with \( \lim_{n \to \infty} \left| z_n \right|= \infty \). Let \(\left\{ p_n(z) \right\} \) be a sequence of polynomials with \(p_n(0)=0\). Then there exists a function \(f\) meromorphic on the complex plane with poles exclusively at \(\left\{ z_n \right\} \) whose principal part at each \(z_n\) is \(p_n(\frac{1}{z-z_n})\). In fact, every meromorphic function \(f\) with poles at \(\left\{ z_n \right\} \) is of the form
{{c1::\(\begin{align*}
        	f(z) = \sum_{n=1}^{\infty} \left[ p_n\left( \frac{1}{z-z_n} \right) - q_n(z) \right] + \varphi (z)
        \end{align*}\)}}
for some polynomials \(\left\{ q_n \right\} \) and an entire function \(\varphi \). The sum converges {{c1::absolutely}} and {{c1::uniformly}} on all \(K\subset \C\) compact with {{c1::\(K \cap \left\{ z_n \right\} = \emptyset\)}}.
Tags: analysis complex_analysis entire_meromorphic
<!--ID: 1626995970482-->
END
\end{anki}


\begin{proof}
\end{proof}

% \printindex
\end{document}


\chapter{Special Functions}
\label{cha:special_functions}

\section{Subharmonic Functions}
\label{sec:subharmonic_functions}

Under construction

\subsection{Dirichlet's Problem}
\label{sub:dirichlet_s_problem}

Under construction

\section{Elliptic Functions}
\label{sec:elliptic_functions}

Under construction

\subsection{Fourier Transform}
\label{sub:fourier_transform}
%\input{FourierTransform}

Under construction

\section{Riemann Zeta Function}
\label{sec:riemann_zeta_function}

Under construction

\subsection{Gamma Function}
\label{sub:gamma_function}

Under construction


\end{document}
