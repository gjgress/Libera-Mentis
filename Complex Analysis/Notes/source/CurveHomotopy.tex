\documentclass{memoir}
\usepackage{notestemplate}

%\logo{~/School-Work/Auxiliary-Files/resources/png/logo.png}
%\institute{Rice University}
%\faculty{Faculty of Whatever Sciences}
%\department{Department of Mathematics}
%\title{Class Notes}
%\subtitle{Based on MATH xxx}
%\author{\textit{Author}\\Gabriel \textsc{Gress}}
%\supervisor{Linus \textsc{Torvalds}}
%\context{Well, I was bored...}
%\date{\today}

%\makeindex

\begin{document}

% \maketitle

% Notes taken on 

\subsection{Free Homotopy of Curves}
\label{sub:free_homotopy_of_curves}

In order to state Cauchy's theorem in complete generality, we need to construct a notion of homotopy for curves.

\begin{defn}[Free Homotopy]
	A \textbf{free homotopy} of closed curves in \(\Omega \) is a continuous map \(\gamma(\tau,t)\) from \([0,1]\times [t_0,t_1]\) to \(\Omega \) such that
\begin{align*}
	\gamma(\tau,t_0) = \gamma(\tau,t_1)
\end{align*}
for every \(\tau\in [0,1]\). We can denote \(\gamma_\tau(t) := \gamma(\tau,t)\).\\

We say \(\gamma_0,\gamma_1\) are \textbf{homotopic} if there exists a free homotopy with \(\gamma(0,t) = \gamma_0\) and \(\gamma(1,t) = \gamma_1\).
\end{defn}
Of course, if \(\Omega \) is convex, then any two curves are automatically freely homotopic.


% \printindex
\end{document}
