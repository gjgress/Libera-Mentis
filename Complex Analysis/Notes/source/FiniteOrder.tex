\documentclass{memoir}
\usepackage{notestemplate}

%\logo{~/School-Work/Auxiliary-Files/resources/png/logo.png}
%\institute{Rice University}
%\faculty{Faculty of Whatever Sciences}
%\department{Department of Mathematics}
%\title{Class Notes}
%\subtitle{Based on MATH xxx}
%\author{\textit{Author}\\Gabriel \textsc{Gress}}
%\supervisor{Linus \textsc{Torvalds}}
%\context{Well, I was bored...}
%\date{\today}

%\makeindex

\begin{document}

% \maketitle

% Notes taken on 


In the statement of the Weierstrass Product Theorem, we implicitly assumed that
\begin{itemize}
	\item there exists a sequence \(\left\{ k_n \right\} \subset \N\) so that
		\begin{align*}
			\sum_{n=1}^{\infty} \left( \frac{a}{\left| z_n \right| } \right)^{k_n}
		\end{align*}
		converges
	\item within the set of sequences that give convergence, there exists a "minimal" sequence.
\end{itemize}
We proved the first point in the proof of the theorem-- one can see that choosing \(k_n = n\) will guarantee convergence in all cases. As for the question of minimality, we explore in this section what a minimal sequence might be like, and how it reflects the underlying structure of our entire function.

\begin{defn}[Order of Entire Function]
	Let \(f\) be an entire function. We say that \(f\) is of \textbf{order \(\leq \rho \)} for real \(\rho >0\) if for all \(\varepsilon>0\), there exists a constant \(C_\varepsilon\) and a corresponding \(R_{C_\varepsilon} >0\) such that, for all \(R>R_{C_\varepsilon}\),
	\begin{align*}
		\sup_{z \in D_R} \left| f(z) \right| \leq C_\varepsilon^{R \raisebox{+4pt}{\scalebox{0.65}{\(\,\rho +\varepsilon\)} } }.
	\end{align*}
	We say that \(f\) is of \textbf{strict order \(\leq \rho \)} if the inequality holds without the \(\varepsilon\):
	\begin{align*}
		\sup_{z \in D_R} \left| f(z) \right| \leq C_\varepsilon^{R \raisebox{+4pt}{\scalebox{0.65}{\(\,\rho\)} } }.
	\end{align*}
	Finally, the function \(f\) is of \textbf{order \(\rho \)} or \textbf{strict order \(\rho \)} if \(\rho \) is the greatest lower bound to make the inequality valid.
\end{defn}

\begin{anki}
TARGET DECK
Complex Qual::Complex Analysis
START
MathJaxCloze
Text: Let \(f\) be an entire function. We say that \(f\) is of **order \(\leq \rho \)** for real \(\rho >0\) if for all \(\varepsilon>0\), there exists a constant \(C_\varepsilon\) and a corresponding \(R_{C_\varepsilon} >0\) such that, for all \(R>R_{C_\varepsilon}\),
{{c1::\(\begin{align*}
      	\sup_{z \in D_R} \left| f(z) \right| \leq C_\varepsilon^{R^{\rho +\varepsilon}} .
        \end{align*}\)}}
We say that \(f\) is of **strict order \(\leq \rho \)** if the inequality holds without the \(\varepsilon\):
{{c1::\(\begin{align*}
        	\sup_{z \in D_R} \left| f(z) \right| \leq C_\varepsilon^{R^{\rho} }.
        \end{align*}\)}}
Finally, the function \(f\) is of **order \(\rho \)** or **strict order \(\rho \)** if \(\rho \) is the {{c1::greatest lower bound}} to make the inequality valid.
Tags: analysis complex_analysis entire_meromorphic
<!--ID: 1626995840083-->
END
\end{anki}


\begin{exmp}
	The function \(e^{z}\) is strict order 1 because
	\begin{align*}
		\left| e^{z} \right| = e^{x} \leq e^{\left| z \right| }.
	\end{align*}
\end{exmp}

For a more involved example, if we have a canonical Weierstrass product with \(\sup_{n} \left\{ k_n \right\} =k\), then for all \(\rho \) satisfying \(k-1<\rho <k\), the Weierstrass product is of order \(\leq \rho \). The following lemma allows us to obtain the converse.

\begin{lemma}
	Let \(f\) be an entire function of strict order \(\leq \rho \). Then for \(R\) sufficiently large,
	\begin{align*}
		\frac{1}{2\pi i}\int_{\partial D_R}\frac{f'}{f} \,d t \leq R^{\rho }.
	\end{align*}
	Of course because \(f\) is entire, the left-hand side represents the number of zeros within the disc of radius \(R\).
\end{lemma}
The following corollary immediately follows:
\begin{cor}
	Let \(f\) have strict order \(\leq \rho \), and let \(\left\{ z_n \right\} \subset \C\setminus\left\{ 0 \right\} \) be the zeros of \(f\) ordered by increasing modulus. Then for every \(\varepsilon>0\), the series
	\begin{align*}
		\sum_{n=1}^{\infty} \frac{1}{\left| z_n \right|^{\rho + \delta }}
	\end{align*}
	converges.
\end{cor}
This corollary then gives us the converse-- every entire function of strict order \(\leq \rho \) has a Weierstrass product form with \(\rho < \sup_{n} k_n < \rho +1\).
We summarize these results into the minimum modulus theorem.
\begin{thm}[Minimum Modulus Theorem]
	Let \(f\) be an entire function of order \(\leq \rho \), and let \(\left\{ z_n \right\} \subset \C\) be its sequence of zeros ordered by increasing modulus. Then for all \(s>\rho \), \(\varepsilon>0\), there is a corresponding \(R_0\) so that, for all \(R > R_0\) and \(z \in \C\setminus \overline{D_s}\):
	\begin{align*}
		\left| f(z) \right| \geq e^{-R\raisebox{+4pt}{\scalebox{0.65}{\(\,\rho +\varepsilon\)} }}
	\end{align*}
\end{thm}
This ensures the construction we gave in Hadamard's Theorem is valid and unique.

\begin{anki}
START
MathJaxCloze
Text: **Minimum Modulus Theorem**
Let \(f\) be an entire function of order \(\leq \rho \), and let \(\left\{ z_n \right\} \subset \C\) be its sequence of zeros ordered by increasing modulus. Then for all \(s>\rho \), \(\varepsilon>0\), there is a corresponding \(R_0\) so that, for all \(R > R_0\) and \(z \in \C\setminus \overline{D_s}\):
{{c1::\(\begin{align*}
        	\left| f(z) \right| \geq e^{-R ^{\rho +\varepsilon} }
        \end{align*}\)}}
Extra: This ensures the construction we gave in Hadamard's Theorem is valid and unique.
Tags: analysis complex_analysis entire_meromorphic
<!--ID: 1626995840102-->
END
\end{anki}


% \printindex
\end{document}
