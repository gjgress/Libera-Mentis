\documentclass{memoir}
\usepackage{notestemplate}

%\logo{~/School-Work/Auxiliary-Files/resources/png/logo.png}
%\institute{Rice University}
%\faculty{Faculty of Whatever Sciences}
%\department{Department of Mathematics}
%\title{Class Notes}
%\subtitle{Based on MATH xxx}
%\author{\textit{Author}\\Gabriel \textsc{Gress}}
%\supervisor{Linus \textsc{Torvalds}}
%\context{Well, I was bored...}
%\date{\today}

%\makeindex

\begin{document}

% \maketitle

% Notes taken on 

\begin{defn}[Noetherian and Artinian Rings]
	A ring is \textbf{Noetherian} if it satisfies the ascending chain condition; there is no infinite strictly increasing chain \(I_1<I_2<\ldots\) of ideals of \(R\).\\

	A ring is \textbf{Artinian} if it satisfies the descending chain condition; there is no infinitely strictly chain \(I_1 > I_2 > \ldots\) of ideals of \(R\).
\end{defn}
For the sake of this section, we will only need to deal with Noetherian rings. We will see both types however when working with modules.
\begin{prop}[Equivalent Definitions]
	A ring \(R\) is Noetherian if and only if every ideal of \(R\) is finitely generated.\\

	A ring \(R\) is Artinian if and only if it is Noetherian and every prime ideal is maximal.
\end{prop}
Principal ideal domains are Noetherian because every ideal in such a ring is generated by one element. A Noetherian ring always has a finite chain of increasing prime ideals-- in an Artinian ring, the chain terminates immediately.

\begin{cor}
	Let \(R\) be a Noetherian ring. Every proper ideal \(I\) of \(R\) is contained in a maximal ideal.
\end{cor}

\begin{thm}[Hilbert Basis Theorem]
	Let \(R\) be a Noetherian ring. The polynomial ring \(R[x]\) is Noetherian.
\end{thm}

\begin{prop}[Quotients of Noetherian]
	Let \(R\) be a Noetherian ring, and let \(I\) be an ideal of \(R\). Any ring that is isomorphic to the quotient ring \(\overline{R} = R / I\) is Noetherian.
\end{prop}
\begin{cor}
	Let \(P\) be a polynomial ring in a finite number of variables over the integers/field. Any ring \(R\) that is isomorphic to the quotient ring \(P / I\) is Noetherian.
\end{cor}
\begin{lemma}
	Let \(R\) be a ring, let \(I\) be an ideal of the polynomial ring \(R[x]\). The set \(A\) whose elements are the leading coefficients of the nonzero polynomials in \(I\), together with the zero element of \(R\), is an ideal of \(R\), the \textbf{ideal of leading coefficients}.
\end{lemma}
\begin{proof}
	
\end{proof}


% \printindex
\end{document}
