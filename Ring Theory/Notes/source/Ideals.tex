\documentclass{memoir}
\usepackage{notestemplate}

%\logo{~/School-Work/Auxiliary-Files/resources/png/logo.png}
%\institute{Rice University}
%\faculty{Faculty of Whatever Sciences}
%\department{Department of Mathematics}
%\title{Class Notes}
%\subtitle{Based on MATH xxx}
%\author{\textit{Author}\\Gabriel \textsc{Gress}}
%\supervisor{Linus \textsc{Torvalds}}
%\context{Well, I was bored...}
%\date{\today}

%\makeindex

\begin{document}

% \maketitle

% Notes taken on 

\section{Properties of Ideals}
\label{sec:properties_of_ideals}

\begin{defn}
	Let \(A\leq R\) be a subset of a ring \(R\).
	\begin{itemize}
		\item  The \textbf{ideal generated by \(A\)} is the  smallest ideal of \(R\) containing \(A\), written by \((A)\).
		\item 
			\begin{align*}
				RA = \left\{r_1a_1 + r_2a_2 + \ldots + r_na_n \mid r_i \in R,\, a_i \in A,\, n \in \Z_+ \right\} \\
				AR = \left\{a_1r_1 + a_2r_2 + \ldots + a_nr_n \mid a_i \in A,\,r_i \in R,\,  n \in \Z_+ \right\} \\
				RAR = \left\{r_1a_1r'_1 + r_2a_2r_2' + \ldots + r_na_nr_n' \mid r_i,r'_i \in R,\, a_i \in A,\, n \in \Z_+ \right\}
			\end{align*}
			are the \textbf{left, right, and two-sided ideal generated by \(A\)}.
		\item If \(\left| A \right| = n < \infty\), then \((A)\) is a \textbf{finitely generated ideal}.
		\item If \(\left| A \right| =1\) then \((A)\) is a \textbf{principal ideal}.
	\end{itemize}
\end{defn}
If \(A = \left\{ a_1,a_2,\ldots \right\} \) then we write \((A) = (a_1,a_2,\ldots)\) for simplicity. Notice that \(b\in R\) is in \((a)\) if and only if \(b = ra\) for some \(r \in R\), which is equivalent to \((b) \subset (a)\).

\begin{prop}
	Let \(I \triangleleft R\). Then \(I = R\) if and only if there exists a unit \(u \in I\).\\

	If \(R\) is commutative, then \(R\) is a field if and only if the only ideals of \(R\) are \(0\) and \(R\).
\end{prop}

\begin{cor}
	If \(R\) is a field and \(R'\) an arbitrary ring, then any nonzero ring homomorphism \(\varphi :R\to R'\) is injective.
\end{cor}

\begin{defn}[Maximal Ideal]
	A proper ideal \(M\) of a ring \(R\) is a \textbf{maximal ideal} of \(R\) if there does not exist another ideal of \(R\) that contains \(M\) besides \(R\) itself.
\end{defn}
Every proper ideal is containd in a maximal ideal.

\begin{anki}
TARGET DECK
Current Math::Abstract Algebra II

% Up to 5 consequences
START
Definition
Name: Maximal Ideal
Premise 1: \(M\) is a proper ideal of ring \(R\)
Consequence 1: \(M\) is maximal if there does not exist another ideal of \(R\) that contains \(M\) besides \(R\) itself
Tags: ring_ideals
<!--ID: 1611701297894-->
END
\end{anki}

\begin{thm}[Classification of Maximal Ideals]
	Let \(R\) be a commutative ring with identity and \(M\) an ideal in \(R\). Then \(M\) is a maximal ideal of \(R\) if and only if \(R / M\) is a field.
\end{thm}

\begin{anki}
% Up to 4 premises
% Up to 4 equivalences
START
Theorem
Name: Classification of Maximal Ideals
Premise 1: \(R\) commutative ring with identity
Premise 2: \(M\) ideal in \(R\)
Consequence 1: \(M\) maximal ideal \(\iff\) \(R/M\) is a field
Tags: ring_ideals
<!--ID: 1611701297914-->
END
\end{anki}

\begin{defn}[Prime Ideal]
	A proper ideal \(P\) in a commutative ring \(R\) is a \textbf{prime ideal} if whenever \(ab \in P\), then either \(a \in P\) or \(b \in P\).
\end{defn}
Note that it is possible to define prime ideals in a noncommutative setting.

\begin{anki}
% Up to 5 consequences
START
Definition
Name: Prime Ideal
Premise 1: Proper ideal \(P\) in commutative ring \(R\)
Consequence 1: \(P\) prime ideal iff \(ab \in P \implies a \in P\) or \(b \in P\)
Tags: rings_ideals
<!--ID: 1611701297931-->
END
\end{anki}

\begin{prop}
	Let \(R\) be a commutative ring with identity \(1_R\neq 0\). Then \(P\) is a prime ideal in \(R\) if and only if \(R / P\) is an integral domain.
\end{prop}

\begin{anki}
% Up to 4 premises
% Up to 4 equivalences
START
Theorem
Premise 1: \(R\) commutative ring w identity
Consequence 1: \(P\) prime ideal in \(R\) iff \(R / P\) integral domain
Tags: rings_ideals
<!--ID: 1611701297949-->
END
\end{anki}

As an example, note that every ideal in \(\Z\) is of the form \(n\Z\), and that \(\Z_n\) is an integral domain only when \(n\) is prime. This is why ideals of the form \(\Z_p\) are viewed as prime ideals.
\begin{cor}
	Every maximal ideal in a commutative ring with identity is also a prime ideal.
\end{cor}

\begin{anki}
START
MathJaxCloze
Text: Every maximal ideal in a commutative ring with identity is also a {{c1::prime ideal}}. 
Tags: rings_ideals
<!--ID: 1611701297965-->
END
\end{anki}
% \printindex
\end{document}
