\documentclass{memoir}
\usepackage{notestemplate}

% \begin{figure}[ht]
%     \centering
%     \incfig{riemmans-theorem}
%     \caption{Riemmans theorem}
%     \label{fig:riemmans-theorem}
% \end{figure}

\begin{document}

\chapter{Module Theory}
\label{cha:module_theory}

We will consider systems \(AX =B\) where \(A,B\) have elements in a ring \(R\), and looking for solutions \(X = (x_1,\ldots,x_n)^{t}\) with entries in \(R\).

\section{Modules}
\label{sec:modules}

The analogue for a ring \(R\) of a vector space over a field is called a module.

\begin{defn}[Module]
	Let \(R\) be a ring. An \textbf{\(R\)-module} \(V\) is an abelian group with a law of composition written +, and a scalar multiplication \(R \times V \to V\), written \(r,v \mapsto rv\), that satisfies:
	\begin{align*}
		1v = v\\
		(rs)v = r(sv) \\
		(r+s)v = rv + sv \\
		r(v+v') = rv + rv'
	\end{align*}
	for all \(r,s \in R\) and \(v,v' \in V\).
\end{defn}
This is precisely the axioms for a vector space; but the elements of a ring do not need to be invertible.

\begin{exmp}[Free Modules]
	The modules \(R^{n}\) of \(R\)-vectors are a first example, called \textbf{free modules}. Then
	\begin{align*}
		\begin{pmatrix} a_1 \\ \vdots \\ a_n \end{pmatrix} + \begin{pmatrix} b_1 \\ \vdots \\ b_n \end{pmatrix}  = \begin{pmatrix} a_1+b_1 \\ \vdots \\ a_n + b_n \end{pmatrix} \quad r \begin{pmatrix} a_1 \\ \vdots \\ a_n \end{pmatrix} = \begin{pmatrix} ra_1 \\ \vdots \\ ra_n \end{pmatrix} 
	\end{align*}
\end{exmp}

An abelian group \(V\) can be made into a module over the integers in exactly one way, with
\begin{align*}
	nv = v + \ldots_{n} + v\\
	(-n)v = -(nv)
\end{align*}
and hence \(V\) is a \(\Z\)-module; this is the only way to do so. It also holds that any \(\Z\)-module has the structure of an abelian group, and hence abelian groups and \(\Z\)-modules are equivalent concepts.

\begin{defn}[Submodule]
	A \textbf{submodule} \(W\) of an \(R\)-module \(V\) is a non-empty subset that is closed under addition and scalar multiplication. Using the laws of composition on \(V\), \(W\) is also a module.
\end{defn}

\begin{prop}
	The submodules of the \(R\)-module \(R\) are the ideals of \(R\).
\end{prop}
Of course, we can define a homomorphism of \(R\)-modules in the same way one would expect:
\begin{defn}[Homomorphism of \(R\)-modules]
	A \textbf{homomorphism} \(\varphi:V\to W\) of \(R\)-modules follows the laws of composition
	\begin{align*}
		\varphi(v+v') = \varphi(v) + \varphi(v')\\
		\varphi(rv) = r\varphi(v)
	\end{align*}
	If it is bijective, then it is an \textbf{isomorphism}. The \textbf{kernel} of a homomorphism \(\varphi:V\to W\) is the set of elements \(v \in V\) such that \(\varphi(v)= 0\) \textit{is a submodule of the domain \(V\)}, and the \textbf{image} of \(\varphi\) \textit{is a submodule of the range \(W\)}.
\end{defn}

Finally, we can extend the quotient construction to modules. Let \(W\) be a submodule of an \(R\)-module \(V\). The quotient module \(\overline{V}= V / W\) is the group of additive cosets \(\overline{v} = v+W\), made into an \(R\)-module by the rule
\begin{align*}
	r \overline{v} = \overline{rv}.
\end{align*}
\begin{thm}
	Let \(W\) be a submodule of an \(R\)-module \(V\).
	\begin{itemize}
		\item The set \(\overline{V}\) of additive cosets of \(W\) in \(V\) is an \(R\)-module, and the canonical map from \(V\to \overline{V}\) by \(v\mapsto v+W\) is a surjective homomorphism of \(R\)-modules whose kernel is \(W\).
		\item Let \(f:V\to V'\) be a homomorphism of \(R\)-modules whose kernel \(K\) contains \(W\). There is a unique homomorphism \(\overline{f}:\overline{V}\to V'\) such that \(f = \overline{f}\circ \pi\).
		\item Let \(f:V\to V'\) be a surjective homomorphism of \(R\)-modules whose kernel is equal to \(W\). Then \(\overline{f}\) as defined above is an isomorphism (First Isomorphism Theorem)
		\item Let \(f:V \to V'\) be a surjective homomorphism of \(R \)-modules with kernel \(W\). There is a bijective correspondence between submodules of \(V'\) and those of \(V\) that contain \(W\). Namely, if \(S'\) is a submodule of \(V'\), the corresponding submodule of \(V\) is \(S = f^{-1}(S')\) ; and if \(S\) is a submodule of \(V\) that contains \(W\), the corresponding submodule of \(V'\) is \(S' = f(S)\). If the two are corresponding modules, then \(V / S\) is isomorphic to \(V' / S'\)
	\end{itemize}
\end{thm}

\section{Free Modules}
\label{sec:free_modules}

\begin{defn}[R-Matrix]
	Let \(R\) be a ring. An \textbf{\(R\)-matrix} is a matrix whose entries are in \(R\). An \textbf{invertible \(R\)-matrix} is an \(R\)-matrix that has an inverse that is also an \(R\)-matrix. The \(n \times n\) invertible \(R\)-matrices form a group called the \textbf{general linear group over \(R\)}:
	\begin{align*}
		GL_n(R) = \left\{n\times n \text{ invertible \(R\)-matrices} \right\} .
	\end{align*}
	The \textbf{determinant} of an \(R\)-matrix \(A = (a_{ij})\) is defined in the usual way
	\begin{align*}
		\textrm{det}(A) = \sum_{p} \pm a_{1,p 1}\ldots a_{n, pn}.
	\end{align*}
	or the sum over all permutations of the indices and the sign being the sign of the permutation. Of course, all the usual properties of determinants hold for \(R\)-matrices.
\end{defn}

\begin{lemma}
	Le \(R\) be a non-zero ring. Then a square \(R\)-matrix \(A\) is invertible if and only if it has either a left inverse or a right inverse, and only if its determinant is a unit of the ring. Furthermore, an invertible \(R\)-matrix is square.
\end{lemma}

\begin{defn}[Span]
	An ordered set \((v_1,\ldots,v_k)\) of elements of a module \(V\) is said to \textbf{generate} \(V\) or \textbf{span} \(V\) if every element \(v\) is a linear combination
	\begin{align*}
		v = r_1v_1 + \ldots + r_kv_k
	\end{align*}
	with coefficients in \(R\). We call \(v_i\) \textbf{generators}. We say a module \(V\) is \textbf{finitely generated} if there exists a finite set of generators.\\

	We say a set of elements \((v_1,\ldots,v_n)\) of a module \(V\) is \textbf{independent} if, whenever a linear combination \(r_1v_1 + \ldots + r_nv_n\) with \(r_i \in R\) is zero, then all the coefficients \(r_i\) are zero. We say this is a basis if every \(v\) is a unique linear combination of those elements.
\end{defn}
The standard multiplication works the same way; i.e.
\begin{align*}
	BX = (v_1,\ldots,v_n) \begin{pmatrix} x_1 \\ \vdots \\ x_n \end{pmatrix} = v_1x_1 + \ldots + v_nx_n
\end{align*}
which defines a homorphism of modules that we may denote by
\begin{align*}
	R^{n} \to^{B} V.
\end{align*}
The homomorphism is surjective iff \(B\) generates \(V\), injective iff \(B\) is independent, and bijective iff \(B\) is a basis. In other words, \(V\) has a basis if and only if it is isomorphic to one of the free modules \(R^{k}\), and if so, it is called a \textbf{free module} too. A module is free if and only if it has a basis.\\

Most modules have no basis! A free \(Z\)-module is also called a \textbf{free abelian group}; lattices in \(\R^2\) are free abelian groups, while finite, non-zero abelian groups are not free.

\begin{prop}
	Let \(R\) be a non-zero ring. Then the matrix \(P\) of a change of basis in a free module is an invertible \(R\)-matrix. Furthermore, any two bases of the same free module over \(R\) have the same cardinality.
\end{prop}
We call the number of elements for a free module \(V\) the \textbf{rank} of \(V\). It is analogous to the dimension of a vector space. Likewise, every homomorphism \(f\) between two free modules is given by left multiplication by an \(R\)-matrix.

%---
%
%diagonalizaton of modules etc
%
%---

\section{Generators and Relations}
\label{sec:generators_and_relations}

\begin{defn}[Presentations]
	Let an \(m\times n\) \(R\)-matrix denoted by \(A\) be a homomorphism of \(R\)-modules
	\begin{align*}
		R^{n}\to^{A} R^{m}.
	\end{align*}
	We can denote its image by \(AR^{n}\). We say that the quotient module \(V = R^{m}/ AR^{n}\) is \textbf{presented} by the matrix \(A\). Any isomorphism \(\sigma:R^{m} / AR^{n} \to V\) is a \textbf{presentation} of a module \(V\), of which \(A\) is a \textbf{presentation matrix} for \(V\) if there is such an isomorphism.
\end{defn}

We use the canonical map \(\pi:R^{m}\to V = R^{m} / AR^{n}\) to interpret the quotient module as follows:
\begin{prop}
\(V\) is generated by a set of elements \(B = (v_1,\ldots,v_m)\), the images of the standard basis elements of \(R^{m}\). Furthermore, if \(Y\) is a column vector in \(R^{m}\), the element \(BY\) of \(V\) is zero if and only if \(Y\) is a linear combination of the columns of \(A\), with coefficients in \(R\), if and only if there exists a column vector \(X\) with entries in \(R\) such that \(Y=AX\).
\end{prop}
If a module \(V\) is generated by a set \(B = (v_1,\ldots,v_m)\), we call any element \(Y \in R^{m}\) such that \(BY = 0\) a \textbf{relation vector}, or simply a \textbf{relation} among the generators. A set \(S\) of relations is a \textbf{complete set} if every relation is a linear combination of \(S\) with coefficients in the ring.
\begin{prop}[Theoretical Method of Finding a Presentation]
	First, choose a set of generators \(B = (v_1,\ldots,v_m)\) for \(V\). These generators give a surjective homomorphism \(R^{m}\to V\) that sends a column vector \(Y\) to the linear combination \(BY = v_1y_1+\ldots+v_my_m\). Denote the kernel of the map by \(W\). It is the \textbf{module of relations}; its elements are the relation vectors.\\

	Repeat this procedure, choosing a set of generators \(C = (w_1,\ldots,w_m)\) for \(W\), and define a surjective map \(R^{n}\to W\) using them. Here the generators \(w_j\) are elements of \(R^{m}\), and thus column vectors. Assemble the coordinate vectors \(A_j\) of \(w_j\) into a matrix with \(A_i\) as column \(i\). Then multiplication by \(A\) defines
	\begin{align*}
		R^{n}\to^{A} R^{m}
	\end{align*}
	which sends \(e_j \mapsto A_j = w_j\), as it is the composition of \(R^{n}\to W\) with the inclusion \(W \subset R^{m}\). By construction \(W\) is its image and we denote it by \(AR^{n}\). Because the map \(R^{m}\to V\) is surjective, by the First Isomorphism Theorem, \(V\) is isomorphic to \(R^{m}/W = R^{m}/AR^{n}\). Hence \(V\) is presented by the matrix \(A\).\\

	In short the presentation matrix \(A\) for a module \(V\) is determined by the set of generators for \(V\), and the set of generators for the module of relations \(W\). Assuming the set of generators does not form a basis, the number of generators will be equal to the number of rows of \(A\).
\end{prop}
Note that this relies on the assumption that \(V\) has finite generators. We must also assume that \(W\) has a finite set of generators, which is slightly more problematic.

\begin{prop}[Rules for manipulating \(A\) without changing isomorphism class]
	Let \(A\) be an \(m \times n\) presentation matrix for a module \(V\). The following matrices \(A' \) present the same module \(V\):
	\begin{itemize}
		\item \(A' = Q^{-1}A\), \(Q \in GL_m(R)\)
		\item \(A' = AP\) with \(P \in GL_n(R)\)
		\item \(A'\) is obtained by deleting a column of zeroes
		\item if the \(j\)-th column of \(A\) is \(e_i\), then removing row \(i\) and column \(j\) preserves the presentation
	\end{itemize}
\end{prop}

\section{Noetherian Rings}
\label{sec:noetherian_rings}

\begin{prop}
	The following conditions on an \(R\)-module \(V\) are equivalent:
	\begin{itemize}
		\item Every submodule of \(V\) is finitely generated
		\item There is no infinite strictly increasing chain \(W_1 < W_2 < \ldots\) of submodules of \(V\).
	\end{itemize}
\end{prop}

\begin{defn}[Noetherian]
	A ring \(R\) is \textbf{noetherian} if every ideal of \(R\) is finitely generated.
\end{defn}
\begin{cor}
	A ring is noetherian if and only if it satisfies the ascending chain condition; there is no infinite strictly increasing chain \(I_1<I_2<\ldots\) of ideals of \(R\).
\end{cor}
Principal ideal domains are noetherian because every ideal in such a ring is generated by one element.

\begin{cor}
	Let \(R\) be a noetherian ring. Every proper ideal \(I\) of \(R\) is contained in a maximal ideal.
\end{cor}

\begin{thm}[Submodules of Noetherian]
	Let \(R\) be a noetherian ring. Every submodule of a finitely generated \(R\)-module \(V\) is finitely generated.
\end{thm}
\begin{lemma}
	Let \(\varphi:V\to V'\) be a homomorphism of \(R\)-modules.
	\begin{itemize}
		\item If \(V\) is finitely generated and \(\varphi\) is surjective, then \(V'\) is finitely generated.
		\item If the kernal and image of \(\varphi\) are finitely generated, then \(V\) is finitely generated.
		\item Let \(W\) be a submodule of an \(R\)-module \(V\). If both \(W\) and \(\overline{V} = V / W\) are finitely generated, then \(V\) is finitely generated. If \(V\) is finitely generated, so is \(\overline{V}\).
	\end{itemize}
\end{lemma}

\begin{thm}[Hilbert Basis Theorem]
	Let \(R\) be a noetherian ring. The polynomial ring \(R[x]\) is noetherian.
\end{thm}

\begin{prop}[Quotients of Noetherian]
	Let \(R\) be a noetherian ring, and let \(I\) be an ideal of \(R\). Any ring that is isomorphic to the quotient ring \(\overline{R} = R / I\) is noetherian.
\end{prop}
\begin{cor}
	Let \(P\) be a polynomial ring in a finite number of variables over the integers/field. Any ring \(R\) that is isomorphic to the quotient ring \(P / I\) is noetherian.
\end{cor}
\begin{lemma}
	Let \(R\) be a ring, let \(I\) be an ideal of the polynomial ring \(R[x]\). The set \(A\) whose elements are the leading coefficients of the nonzero polynomials in \(I\), together with the zero element of \(R\), is an ideal of \(R\), the \textbf{ideal of leading coefficients}.
\end{lemma}

\section{Structure of Abelian Groups}
\label{sec:structure_of_abelian_groups}

\begin{defn}[Direct Sum of Modules]
	Let \(W_1,\ldots,W_k\) be submodules of an \(R\)-module \(V\). Their sum is the submodule that they generate;
	\begin{align*}
		W_1 + \ldots + W_k = \left\{v \in V \mid v = w_1+\ldots+w_k \quad w_i \in W_i \right\} 
	\end{align*}
	If \(W_1 + \ldots + W_k = V\) AND they are independent (\(w_1+\ldots+w_k=0 \iff w_i = 0\) ) then \(V\) is the direct sum of the submodules.
\end{defn}
Similarly, \(V = W_1\bigoplus W_2 \iff W_1+W_2=V, \quad W_1 \cap W_2 = 0\).

\begin{thm}[Structure Theorem for Abelian Groups]
	A finitely generated abelian group V is a direct sum of cyclic subgroups \(C_{d_1},\ldots,C_{d_k}\) and a free abelian group \(L\):
	\begin{align*}
		V = C_{d_1}\bigoplus \ldots \bigoplus C_{d_k}\bigoplus L,
	\end{align*}
	where the order \(d_i\) of \(C_{d_i}\) is greater than one, and \(d_i \mid d_{i+1}\) for \(i<k\).
\end{thm}

\begin{thm}[Structure Theorem (Alternate Form)]
	Every finite abelian group is a direct sum of cyclic groups of prime power orders.
\end{thm}
\begin{thm}[Uniqueness for Structure Theorem]
	Suppose that a finite abelian group \(V\) is a direct sum of cyclic groups of prime power orders \(d_j = p_j^{r_k}\). The integers \(d_j\) are uniquely determined by the group \(V\).
\end{thm}

\section{Analogues for Polynomial Rings and Linear Operators}
\label{sec:analogues_for_polynomial_rings_and_linear_operators}

\begin{thm}
	Let \(R = F[t]\) be a polynomial ring in one variable over a field \(F\) and let \(A\) be an \(m\times n\) \(R\)-matrix. There are products \(Q,P\) of elementary \(R\)-matrices such that
	\begin{align*}
		A' = Q^{-1}AP
	\end{align*}
	is diagonal, each non-zero diagonal entry \(d_i\) of \(A'\) is a monic polynmial, and \(d_1\mid \ldots\mid d_k\).
\end{thm}
\begin{defn}[Cyclic Modules]
We define a \textbf{cyclic \(R\)-module \(C\)}, where \(R\) is any ring, to be a module that is generated by a single element \(v\).
\end{defn}
Then there is a surjective homomorphism \(\varphi:R\to C\) defined by \(r\mapsto rv\). The kernel of \(\varphi\) is a submodule of \(R\), an ideal \(I\). Therefore, \(C\) is isomorphic to the \(R\)-module \(R / I\). When \(R = F[t]\), the ideal \(I\) will be principal.

\begin{thm}[Structure Theorem for Modules over Polynomial Rings]
	Let \(R = F[t]\) be the ring of polynomials in one variable with coefficients in a field \(F\). Let \(V\) be a finitely generated module over \(R\). Then V is a direct sum of cyclic modules \(C_1,C_2,\ldots,C_k\) and a free module \(L\), where \(C_i\) is isomorphic to \(R / (d_i)\), the elements \(d_1,\ldots,d_k\) are monic polynomials of positive degree and satisfy both (but not simultaneously)
\begin{itemize}
	\item \(d_1\mid d_2\mid \ldots\mid d_k\) 
	\item Each \(d_i\) is a power of a monic irreducible polynomial
\end{itemize}
\end{thm}

\end{document}
