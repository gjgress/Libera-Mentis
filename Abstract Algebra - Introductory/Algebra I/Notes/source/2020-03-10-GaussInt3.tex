\documentclass{memoir}
\usepackage{notestemplate}

% \begin{figure}[ht]
%     \centering
%     \incfig{riemmans-theorem}
%     \caption{Riemmans theorem}
%     \label{fig:riemmans-theorem}
% \end{figure}

\begin{document}
% \section{}	
\begin{prop}[Disjoint Partitions of Fields]
	Let \(R\) be a field. Then we can partition \(R\) into disjoint sets by taking all sets of the form
	\begin{align*}
		\left\{ a, -a, a^{-1}, (-a)^{-1} \right\} 
	\end{align*}
	where \(a\) is non-zero, and taking the set \(\left\{ 0 \right\} \).
\end{prop}
\begin{thm}[Two Squares Theorem]
Consider the equation \(x^2+y^2 = n\), and let \(n = 2^{\alpha}p_1^{\beta_1}\ldots p_r^{\beta_r}q_1^{\gamma_1}\ldots q_s^{\gamma_s}\) be the Gaussian factorization of \(n\). Then, \(x^2+y^2 = n\) is solvable in \(\Z\) if and only if all \(\gamma_j\) are even. Furthermore, the number of solutions is
\begin{align*}
	4 \prod_{j=1}^{r} (\beta_j + 1) 
\end{align*}
\end{thm}
\begin{proof}
	First, we write \(n = x^2+y^2 = (x+yi)(x-yi)\). Using the Gaussian factorization, we rewrite
	\begin{align*}
		n = 2^{\alpha}p_1^{\beta_1}\cdot \ldots\cdot q_1^{\gamma_1}\cdot \ldots = (-i)^{\alpha}(1+i)^{2\alpha}\pi_1^{\beta_1}\overline{\pi_1}^{\beta_1}\cdot \ldots\cdot q_1^{\gamma_1}
	\end{align*}
	Now observe that
	\begin{align*}
		(x+yi)\mid n \implies x+yi = \varepsilon (1+i)^{\alpha'}\pi_1^{\beta_1'}\overline{\pi_1}^{\beta_1''}\cdot \ldots\cdot q_1^{\gamma_1'}\cdot \ldots \\
		\implies x-yi = \overline{\varepsilon}(1-i)^{\alpha'}\overline{\pi_1}^{\beta_1'}\pi_1^{\beta_1'}\cdot \ldots\cdot q_1^{\gamma_1}
	\end{align*}
	Then because
	\begin{align*}
		n = (x+yi)(x-yi) \implies\\
		2\alpha = \alpha' + \alpha' \iff\alpha' = \alpha\\
		\beta_1 = \beta_1' + \beta_1'' \iff \beta_1' = 0,1,\ldots,\beta_1; \beta_1'' = \beta_1-\beta_1' \\
		\gamma_1 = \gamma_1' + \gamma_1' \iff \gamma_1 \text{ even}, \gamma_1' = \frac{\gamma_1}{2}\\
		(-i)^{\alpha} = \varepsilon \overline{\varepsilon}(-i)^{\alpha} \iff 1 = \varepsilon \overline{\varepsilon} \text{ which always holds}
	\end{align*}
	Thus, the equation is always solvable if all the \(\gamma\) are even. Looking at the above, the number of solutions will be
	\begin{align*}
		1 \cdot (\beta_j+1) \cdot 1 \cdot 4 = 4 \prod_{j=1}^{r} (\beta_j + 1) 
	\end{align*}
\end{proof}

\subsection{Fermat's Last Theorem}
\label{subsec:fermat_s_last_theorem}
\begin{thm}[Fermat's Last Theorem]
	Let \(n\geq 3\). Does \(x^{n}+y^{n}=z^{n}\) have positive integer solutions?
\end{thm}
It is clear that if it is true for \(n  = 4\), \(n= p\) prime, then it holds, as of course it will hold for any multiples. We can rewrite this as
\begin{align*}
	x^{p} = z^{p}-y^{p}
\end{align*}
\(y\) is a parameter, so the roots are
\begin{align*}
	z^{p} = y^{p} \implies z = (y^{p})^{\frac{1}{p}} = z, z\rho, z\rho^2,\ldots,z\rho^{p-1} \text{ where } \rho = \cos \frac{\pi}{p} + i \sin^2 \frac{2\pi}{p}
\end{align*}
So we can rewrite this as
\begin{align*}
	x^{p} = z^{p}-y^{p} = (z-y)(z-\rho y)\ldots(z-\rho y^{p-1})
\end{align*}
Observe that each prime must be a \(p\)-th power as they cannot share factors. Let
\begin{align*}
	H_p = \left\{ a_0 + a_1\rho + \ldots + a_{p-2}\rho^{p-2} \right\}, a_j \in \Z \\
	\rho^{p-1}+\rho^{p-2} + \ldots + \rho + 1 = 0
\end{align*}
If the factors on the RHS are pairwise coprime, then \(z-y = \varepsilon_0 \theta_0^{p}\), \(z-y\rho = e_1 \theta_1^{p}\)
BUT the units are non-trivial for these types of coefficients, AND we don't have UFT.\\

Then Kummer did a different direction. Note that ~"if there is a gcd then it is UFT" (not exactly, but dw about it). Then consider for \(a,b \in \Z\)
\begin{align*}
	\textrm{gcd}(a,b) = d \implies d = au + bv
\end{align*}
Consider the set
\begin{align*}
	\left\{ak + bl \mid k,l \in \Z \right\} = \left\{dn \mid n \in \Z \right\} 
\end{align*}
Now consider
\begin{align*}
	\left\{ak + bl \mid k,l \in H_p \right\} 
\end{align*}
If \( \exists \textrm{gcd}(a,b) \implies\) this set is the set of multiples of gcd. So for "ideal numbers" UFT holds and the proof holds. So for "ideal numbers" UFT holds and the proof holds.
\end{document}
