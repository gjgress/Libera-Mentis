\documentclass{memoir}
\usepackage{notestemplate}

% \begin{figure}[ht]
%     \centering
%     \incfig{riemmans-theorem}
%     \caption{Riemmans theorem}
%     \label{fig:riemmans-theorem}
% \end{figure}

\begin{document}
\section{Structure of groups of special sizes}
\label{sec:structure_of_groups_of_special_sizes}
\begin{prop}
	\begin{itemize}
		\item If \(\left| G \right| = p\), then \(G \cong Z_p\)
		\item If \(\left| G \right| =p^2\), then \(G\) is Abelian and \(G \cong Z_{p^2} \text{ OR }Z_p \times Z_p\)
		\item Let \(p>2\), if \(\left| G \right| = 2p\), then \(G \cong Z_{2p} \text{ OR } D_p\).
	\end{itemize}
\end{prop}
\begin{defn}[Permutation Groups]
	Permutation groups are subgroups of \(S_n\).
\end{defn}
\begin{thm}[Cayley's Theorem]
	Every group of size \(n\) is isomorphic to a subgroup of \(S_n\).
\end{thm}
\begin{proof}
	Let \(S_n\) be all permutations of the group \(G = \left\{ e,g_2,\ldots,g_n \right\} \) and define \(\varphi:G\to S_n\) by \(g \mapsto {g_i}\choose{gg_i}\). In other words, we assign to \(g\in G\) the permutation of \(G\) onto itself; we multiply every \(g_i\) by the given \(g\) from the left. One can check that \(\varphi\) is an injective homomorphism, and so \(G \cong \textrm{Image}(\varphi)\leq S_n\).
\end{proof}
\begin{defn}[Simple]
	A group \(G\) is \textbf{simple} if the only normal subgroups are the trivial ones.
\end{defn}
Finite simple groups are important, and only recently have they all been characterized.
\end{document}
