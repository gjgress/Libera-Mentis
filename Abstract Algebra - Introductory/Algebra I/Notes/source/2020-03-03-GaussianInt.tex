\documentclass{memoir}
\usepackage{notestemplate}

% \begin{figure}[ht]
%     \centering
%     \incfig{riemmans-theorem}
%     \caption{Riemmans theorem}
%     \label{fig:riemmans-theorem}
% \end{figure}

\begin{document}

\section{Gaussian Integers and Applications}
\label{sec:gaussian_integers_and_applications}

\begin{defn}[Gaussian Integers]
	The ring \(G\) with elements \(a+bi \in \C\), with \(a,b \in \Z\), is called the ring of \textbf{Gaussian Integers}.
\end{defn}[Norm]
Let \(\alpha \in \C\). Then the norm of \(\alpha\) is defined by \(\alpha \overline{\alpha}\)
\begin{thm}
	\(x^2+y^2 = n\) solvable iff number of solutions
\end{thm}
\end{document}
