\documentclass{memoir}
\usepackage{notestemplate}

% \begin{figure}[ht]
%     \centering
%     \incfig{riemmans-theorem}
%     \caption{Riemmans theorem}
%     \label{fig:riemmans-theorem}
% \end{figure}

\begin{document}
\section{Direct Product of Groups}	
\begin{defn}[Direct Product]
	The \textbf{direct product} \(G_1\times G_2\) of groups \(G_1,G_2\) is the group of all ordered pairs \((g_1,g_2)\) where \(g_i \in G_i\), with the usual definition of multiplication:
	\begin{align*}
		(g_1,g_2)(h_1,h_2) = (g_1h_1,g_2h_2).
	\end{align*}
\end{defn}
	It is clearly a group, and the projection subsets \(G_1^{*}= \left\{(g_1,e_2) \mid g_1 \in G_1 \right\} \) and \(G_2^{*}=\left\{(e_1,g_2) \mid g_2 \in G_2 \right\} \) are isomorphic to their respective groups.\\

Furthermore, \(G_1^{*},G_2^{*}\) are normal subgroups  in the direct product, and every \(u \in G_1\times G_2\) can be decomposed as \(u = u_1u_2\), \(u_i \in G_i^{*}\). The converse holds as well; if \(N,M\) are normal subgroups of a group \(G\), and every \(g \in G\) can be written as \(g = nm\), \(n \in N, m \in M\), then \(G \cong N\times M\).\\

Of course, \((G_1\times G_2) / G_1^{*} \cong G_2\) and vice versa; we can even take the direct product of more than two groups.\\

The Fundamental Theorem of Finite Abelian Groups states that every finite Abelian group is the direct product of cyclic groups of prime power size, and the list of the direct factors is uniquely determined.
\end{document}
