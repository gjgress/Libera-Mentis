\documentclass{memoir}
\usepackage{notestemplate}

% \begin{figure}[ht]
%     \centering
%     \incfig{riemmans-theorem}
%     \caption{Riemmans theorem}
%     \label{fig:riemmans-theorem}
% \end{figure}

\begin{document}


\chapter{Divisibility, Congruences, Euler's Function}
All numbers are assumed to be \(\Z\)\\
\begin{align*}
	a\mid b := \quad \exists c \text{ s.t. } ac = b
\end{align*}
\begin{defn}
	We say that \(a \equiv b\) (mod \(m\)) (a is equivalent to b)  if \(m|a-b\) 
\end{defn}
\begin{defn}[Euler's Function]
	\(\varphi(n)\) is defined as the number of integers coprime to \(n\) in \(\left\{ 1,2,\ldots,n \right\} \). In other words, it is the magnitude of \(\left\{ c \mid 1\leq c\leq n, (c,n) = 1 \right\} \)
\end{defn}
Note that if \(p\) is prime, then \(\varphi(p) = p-1\).
\begin{thm}
	\(n = p_1^{k_1}\cdot\ldots\cdotp_r^{k_r}\), \(p_i \neq p_j\), \(p\) prime, \(k_i > 0\) \(\implies \varphi(n) = (p_1^{k_1-1}-p_1^{k_1-1})\cdot\ldots\cdot (p_r^{k_r}-p_r^{k_r-1})\)
\end{thm}
\begin{thm}[Euler-Fermat Theorem]
	If \(c,m\) are coprime, then \(c^{\varphi(m)} \equiv 1\) (mod \(m\)).
\end{thm}
\begin{thm}[Linear Congruence]
	A linear congruence \(ax \equiv b\) (mod \(m\)) is solvable if and only if \((a,m) \mid b\). Furthermore, the number of pairwise incongruent solutions is \((a,m)\).
\end{thm}
\begin{defn}[Linear Diophantine equation]
	A linear Diophantine equation in two variables is \(Ax+By=C\) where \(A,B,C\) are given integers, \(A,B\) not both zero, with integer solutions for  \(x,y \).
\end{defn}
A linear Diophantine equation is solvable if and only if \((A,B)\mid C\), in which case there are infinite solutions.\\
Note that a linear Diophantine equation can be transformed into a linear congruence \(Ax \equiv C \pmod {\left| B \right|}\) or \(By \equiv C \pmod {\left| A \right|} \)
\end{document}
