\documentclass{memoir}
\usepackage{notestemplate}

% \begin{figure}[ht]
%     \centering
%     \incfig{riemmans-theorem}
%     \caption{Riemmans theorem}
%     \label{fig:riemmans-theorem}
% \end{figure}

\begin{document}
\begin{defn}[Rings]
	A ring is a non-empty set with two operations, \(+\) and \(\cdot \). The \(+\) operation is commutative, associative, has an identity, and inverses for all elements. The \(\cdot \) operation is associative. Both operations have two distributive rules, namely, for all \(a,b,c\),
	\begin{align*}
		(a+b)c = ac + bc\\
		a(b+c) = ab+ac 
	\end{align*}
\end{defn}
If multiplication is commutative, then it is a commutative ring. If in addition, multiplication has an identity, and an inverse for all except the additive inverse, then it is a field.\\

Unless specified otherwise, we will assume our rings have a multiplicative identity.
\begin{defn}[Zero Divisors and Units]
	Let \(R\) be a ring. A nonzero element \(a \in R\) is called a \textbf{zero divisor} if there exists a nonzero element \(b \in R\) such that \(ab=0\) or \(ba=0\).\\

	An element \(u \in R\) is called a \textbf{unit} in \(R\) if there is a \(v \in R\) such that \(uv = vu = 1\)-- that is, \(u\) has a multiplicative inverse. The set of units in \(R\) is denoted \(R^{X}\).
\end{defn}
Every element besides \(0\) is a unit in a field. We can also see that units form a group under multiplication.\\

Notice that zero divisors and units are distinct concepts-- a zero divisor can never be a unit.

\begin{defn}[Integral Domain]
	A commutative ring is called an \textbf{integral domain} if it has no zero divisors.
\end{defn}

\begin{prop}
	Let \(a,b,c \in R\) be elements of a ring with \(a\) not a zero divisor. Then
	\begin{align*}
		ab = ac \implies a=0\text{ or } b = c
	\end{align*}
	In particular, if \(R\) is an integral domain this always holds.
\end{prop}

\begin{hw}
	Prove that any finite integral domain is a field.\\
	(Hint: consider \(x\mapsto ax\) for \(a\) nonzero)
\end{hw}

\end{document}
