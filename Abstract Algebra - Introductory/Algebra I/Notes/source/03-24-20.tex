\documentclass{memoir}
\usepackage{notestemplate}

% \begin{figure}[ht]
%     \centering
%     \incfig{riemmans-theorem}
%     \caption{Riemmans theorem}
%     \label{fig:riemmans-theorem}
% \end{figure}

\begin{document}
\chapter{Groups}
\label{cha:groups}
\begin{defn}[Group]
	A \textbf{group} is a set \(G\) that has an associative operation with an identity \(e\) and inverses for all elements. If the operation is commutative, then we say \(G\) is \textbf{abelian}.
\end{defn}
One particular important class of groups are cyclic groups. A \textbf{cyclic group} is a group consisting of powers of a single element \(g\), denoted by \( \langle g \rangle \).

\begin{defn}[Order]
	Let \(g \in G\). The \textbf{order of \(g\)}, denoted by \(o(g)\), is the smallest positive integer \(k\) satisfying \(g^{k}=e\). If there is no such \(k\), then we say that \(o(g) = \infty\).
\end{defn}
In the finite case, the powers of \(g\) are periodic, and so the order is the smallest period. \\

A finite group \(G\) is cyclic if and only if \(\left| G \right| = o(g)\) for some \(g \in G\). Lagrange's theorem implies that \(o(g) \mid \left| G \right| \) for \(\left| G \right| <\infty\).
	
\end{document}
