\documentclass{memoir}
\usepackage{notestemplate}

% \begin{figure}[ht]
%     \centering
%     \incfig{riemmans-theorem}
%     \caption{Riemmans theorem}
%     \label{fig:riemmans-theorem}
% \end{figure}

\begin{document}

\begin{defn}[Subgroup]
	A set \(H \subset G\) is a \textbf{subgroup} of \(G\) if it is a group under the operation of \(G\). We denote this by \(H \leq G\).
\end{defn}

\(H\leq G\) if and only if it contains the identity of \(G\) and is closed under the operation in \(G\), and for inverses. We can reduce these conditions to requiring that \(H\) satisfies the criterion
\begin{align*}
	xy^{-1} \in H \quad \forall x,y \in H
\end{align*}

\begin{prop}
	Let \(\mathcal{A}\) be a nonempty collection of subgroups of \(G\). Then their intersection is a subgroup:
	\begin{align*}
		K = \bigcap_{H \in \mathcal{A}} H \leq G.
	\end{align*}
\end{prop}
This leads us to the following definition.
\begin{defn}[Generated Subgroup]
	Let \(A\subset G\) be a subset of the group \(G\). We define
	\begin{align*}
		\langle A \rangle = \bigcap_{A\subset H\leq G} H
	\end{align*}
	to be the \textbf{subgroup of \(G\) generated by \(A\)}.\\

	Alternatively, we define \(\overline{A}\) to be the (finite product) closure of \(A\) under the group operation of \(G\). One should check that \(\langle A \rangle = \overline{A}\).
\end{defn}
In short, we can generate a subgroup on a set by looking for the smallest subgroup that contains the set. This is a unique minimal element. This will also be the exact same subgroup formed by taking the closure of \(A\) under the group operations of \(G\).

\begin{defn}[Coset]
	Let \(H\leq G\) and \(g \in G\). Then \(gH = \left\{gh \mid h \in H \right\} \) is a \textbf{left coset}, and \(Hg = \left\{hg \mid h \in H \right\} \) is a \textbf{right coset}.
\end{defn}

Two left cosets are either disjoint or equal, and every coset is of the same size.

\begin{prop}
	Let \(H\leq G\) be a subgroup of \(G\). Then the set of left cosets of \(H\) form a partition of \(G\). That is,
	\begin{align*}
		G = \bigcup_{g \in G} gH.
	\end{align*}
	Furthermore, for all \(g,g' \in G\), \(gH = g'H\) if and only if \(g'^{-1}g \in H\). In other words, \(g\) and \(g'\) are representatives of the same coset.
\end{prop}
The set of left cosets form a group by the operation
\begin{align*}
	gH \cdot g'H = (gg')H
\end{align*}
provided that \(ghg^{-1} \in H\) for all \(g \in G\) and \(h \in H\). We will soon give this property a name. But first, we will state and prove a fundamental theorem for groups.

\begin{thm}[Lagrange's Theorem]
	Let \(H\) be a subgroup of a finite group \(G\). Then \(\left| H \right| \mid \left| G \right| \), and te number of left cosets of \(H\) in \(G\) is \(\dfrac{\left| G \right| }{\left| H \right| }\).
\end{thm}
This gives a new perspective on subgroups-- we can view subgroups as a means of partitioning a group.

\begin{defn}[Index]
	If \(G\) is a group and \(H\leq G\), the number of left cosets of \(H\) in \(G\) is called the \textbf{index} of \(H \) in \(G\). We denote this by \(\left| G : H \right| \).
\end{defn}
Lagrange's Theorem gives us a lot of really nice results. For example, we can immediately see that \(\left| g \right| \mid \left| G \right| \) for \(g \in G\) in a finite group, and moreover \(g^{\left| G \right| }= 1\) for all \(g \in G\).\\

We will define a way to compose two subgroups of a group that can sometimes be convenient.

\begin{defn}
	Let \(H,K \leq G\) be subgroups of \(G\). We define
	\begin{align*}
		HK := \left\{hk \mid h \in H, k \in K \right\} .
	\end{align*}
\end{defn}

\begin{prop}
	If \(H,K\leq G\) are finite subgroups of \(G\), then
	\begin{align*}
		\left| HK \right| = \dfrac{\left| H \right| \left| K \right| }{\left| H \cap K \right| }.
	\end{align*}
\end{prop}
This equation becomes nicer when \(H,K\) are disjoint (with the exception of identity of course), as it implies that \(\left| HK \right|  = \left| H \right| \left| K \right| \). However, one needs to be careful-- this product isn't necessarily a subgroup, as it may not contain inverses.

\begin{prop}
	Let \(H,K \leq G\) be subgroups of \(G\). \(HK\) is a subgroup if and only if
	\begin{align*}
		HK = KH.
	\end{align*}
\end{prop}
If this holds, then inverses will be contained within \(HK\), and hence it will be closed under group operations.

\end{document}
