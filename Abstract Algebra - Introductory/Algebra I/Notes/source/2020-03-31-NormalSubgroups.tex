\documentclass{memoir}
\usepackage{notestemplate}

% \begin{figure}[ht]
%     \centering
%     \incfig{riemmans-theorem}
%     \caption{Riemmans theorem}
%     \label{fig:riemmans-theorem}
% \end{figure}

\begin{document}

\subsection{Normal Subgroups}
\label{sec:normal_subgroups}

Cosets are such a vital construction to visualize subgroups, as they will become key to understanding quotient groups. However, we notice one flaw that makes this characterization difficult-- left and right cosets of a subgroup are not necessarily equivalent. This property is important enough to earn itself a name.

\begin{defn}[Normal Subgroup]
	A subgroup \(N\) of \(G\) is \textbf{normal} if the left and right cosets are the same; i.e.
\begin{align*}
	N\leq G \text{ and }gN = Ng
\end{align*}
for every \(g \in G\). We denote this by \(N \triangleleft G\).
\end{defn}
We can also think of this as \(Ng \subset gN\) and \(gN \subset Ng\), or in other words, \(g^{-1}ng \in N\) and \(gng^{-1} \in N\).
\begin{align*}
	N \triangleleft G \iff N \leq G \text{ and } g^{-1}ng \in N \text{for every }g \in G, n \in N
\end{align*}
We call \(g^{-1}ng\) a \textbf{conjugate} of \(n\).\\

Observe that, because the trivial subgroups are trivially normal, then all subgroups \(H\) of index 2 are normal.

\begin{thm}
	Let \(N \leq G\) be a subgroup of a group \(G\). The following are equivalent:
	\begin{itemize}
		\item \(N \triangleleft G\) 
		\item \(gN = Ng\) for all \(g \in G\) 
		\item The left cosets of \(N\) in \(G\) form a group by the natural group operation
			\begin{align*}
				gN \cdot g'N = (gg')N
			\end{align*}
		\item \(gNg^{-1} \subset N\) for all \(g \in G\)
	\end{itemize}
\end{thm}

Every group also has a special normal subgroup called the \textbf{center}.

\begin{defn}[Center]
	The \textbf{center} of a group is the normal subgroup given by
\begin{align*}
	Z(G) = \left\{g \in G \mid xg = gx \text{ for every }x \in G \right\} .
\end{align*}
\end{defn}
Every subgroup of the center is normal.
\end{document}
