\documentclass{memoir}
\usepackage{notestemplate}

% \begin{figure}[ht]
%     \centering
%     \incfig{riemmans-theorem}
%     \caption{Riemmans theorem}
%     \label{fig:riemmans-theorem}
% \end{figure}

\begin{document}
\begin{defn}[Subgroup]
	A set \(H \subset G\) is a \textbf{subgroup} of \(G\) if it is a group under the operation of \(G\). We denote this by \(H \leq G\).
\end{defn}
\(H\leq G\) if and only if it contains the identity of \(G\) and is closed under the operation in \(G\), and for inverses.
\begin{defn}[Coset]
	Let \(H\leq G\) and \(g \in G\). Then \(gH = \left\{gh \mid h \in H \right\} \) is a \textbf{left coset}, and \(Hg = \left\{hg \mid h \in H \right\} \) is a \textbf{right coset}.
\end{defn}
Two left cosets are either disjoint or equal, and every coset is of the same size.
\begin{thm}[Lagrange's Theorem]
	Let \(H\) be a subgroup of \(G\). Then \(\left| H \right| \mid \left| G \right| \) for \(\left| G \right| <\infty\).
\end{thm}
Important subgroups and their properties:
\begin{defn}[Permutations]
	We define by \(S_n\) the set of all bijections of \(\left\{ 1,2,\ldots,n \right\} \) onto itself, with the operation being composition.
\end{defn}
\begin{cor}
With "old permutations" we say a permutation is even or odd based on the number of inversions. With this definition, we say that a permutation is even if and only if the permutation is the product of an even number of transpositions.
\end{cor}
Let \(A_n\) denote the set of even permutations. This is clearly a subgroup of \(S_n\). We would like to show that
\begin{align*}
	\left| S_n : A_n \right| =2.
\end{align*}
Observe that there is a bijection from the even permutations to the odd permutations by simply transposing the first two elements.\\

\textbf{The order of an element \(g \in S_n\) is the LCM of the lengths of disjoint cycles.}
\end{document}
