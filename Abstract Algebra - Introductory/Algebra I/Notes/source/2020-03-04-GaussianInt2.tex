\documentclass{memoir}
\usepackage{notestemplate}

% \begin{figure}[ht]
%     \centering
%     \incfig{riemmans-theorem}
%     \caption{Riemmans theorem}
%     \label{fig:riemmans-theorem}
% \end{figure}

\begin{document}
% \section{}	
\begin{thm}
	The Gaussian Integers form a Euclidean Domain.
\end{thm}
\begin{proof}
	We will show that \(f(\alpha) = N(\alpha)\) suffices. Observe that
	\begin{align*}
		\alpha = \beta \rho + \theta \iff\\
		\frac{\alpha}{\beta = \rho + \frac{\theta}{\beta}}\iff\\
		\frac{\alpha}{\beta}-\rho = \frac{\theta}{\beta} \iff\\
		\left| \frac{\alpha}{\beta} - \rho \right| < 1
	\end{align*}
	Of course, this is the distance between \(\frac{\alpha}{\beta}\) and \(\rho\). But there always exists a lattice point within distance \(1\) of any \(\C\), and so therefore the statement holds.
\end{proof}
Now we characterize all \(G\)-primes.
\begin{prop}
To every Gaussian prime \(\pi\), there exists exactly one positive prime number \(p\) satisfying \(\pi\mid p\). Furthermore, every positive prime \(p\) is either a Gaussian prime, or the product of two complex conjugate Gaussian primes with norm \(p\).
\end{prop}
\begin{thm}
	All Gaussian primes take on the form:
	\begin{itemize}
		\item \(\varepsilon(1+i)\) 
		\item \(\varepsilon q\), with \(q\) a positive prime of the form \(4k-1\) 
		\item \(\pi\) where \(N(\pi)\) is a positive prime of the form \(4k+1\)
	\end{itemize}
	where \(\varepsilon\) is a unit.
\end{thm}
\end{document}
