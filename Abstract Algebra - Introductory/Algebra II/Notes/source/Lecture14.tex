\documentclass{memoir}
\usepackage{notestemplate}

%\logo{~/School-Work/Auxiliary-Files/resources/png/logo.png}
%\institute{Rice University}
%\faculty{Faculty of Whatever Sciences}
%\department{Department of Mathematics}
%\title{Class Notes}
%\subtitle{Based on MATH xxx}
%\author{\textit{Author}\\Gabriel \textsc{Gress}}
%\supervisor{Linus \textsc{Torvalds}}
%\context{Well, I was bored...}
%\date{\today}

\begin{document}

% \maketitle

% Notes taken on ??

\chapter{Field Extensions}
\label{cha:field_extensions}

Recall the definition of a field.
\begin{defn}
	A \textbf{field} is a commutative ring \(F\) with multiplicative identity \(1_F\) in which every nonzero element has a multiplicative inverse.
\end{defn}
Furthermore, recall that the \textbf{characteristic} of a field \(F\), denoted \(\textrm{char}(F)\), is the smallest positive integer \(n\) such that
\begin{align*}
	1_F + 1_F + \ldots_n + 1_F = 0_F
\end{align*}
if such an \(n \in \N\) exists. Otherwise, we say that \(\textrm{char}(F)=0\).

\begin{prop}
	For a field \(F\), we have that \(\textrm{char}(F)=0\) or \(\textrm{char}(F) = p\) for a prime integer \(p\). If \(\textrm{char}(F) = p\), then \(p\cdot \alpha  = \alpha + \ldots_p + \alpha  = 0_F\) for all \(\alpha  \in F\).
\end{prop}
We often refer to fields with prime characteristics as \textbf{fields of positive characteristic}.\\

Some fields of characteristic zero include \(\Q\), \(\R\), and \(\C\). Any field of the form \(\Z / p\Z := \mathbb{F}_p\) is a field of characteristic \(p\).

\section{Subfields}
\label{sec:subfields}

\begin{defn}
	A \textbf{subfield} of a field \(F\) is a nonempty subset \(S\) containing \(1_F\) that is a subring under the addition and multiplication of \(F\), and so that \(S\) is closed under taking multiplicative inverse.\\

	The \textbf{prime subfield} of a field \(F\) is the subfield generated by the multiplicative identity \(1_F\) of \(F\), that is, it is the smallest subfield of \(F\) containing \(1_F\).
\end{defn}

\begin{prop}
	The prime subfield of a field \(F\) is either \(\Q\) if \(\textrm{char}(F) = 0\), or \(\mathbb{F}_p\) if \(\textrm{char}(F) = p\).
\end{prop}

\begin{defn}
	A \textbf{homomorphism \(\Phi:F_1\to F_2\) between fields \(F_1\) and \(F_2\)} is a unital ring homomorphism: \(\forall x,y \in F_1\)
	\begin{align*}
		\varphi (x+y) = \varphi (x) + \varphi (y)\\
		\varphi (xy) = \varphi (x) \varphi (y), \quad \varphi (1_{F_1}) = 1_{F_2}
	\end{align*}
\end{defn}

A lot of fields are better viewed via a ring homomorphism. We can quotient out a ring \(R\) by any maximal ideal \(I\) of \(R\) to get an object isomorphic to a field.
\begin{exmp}
	Consider the principal ideal domain \(\Q[x]\). For any irreducible polynomial \(p(x)\), we have that
	\begin{align*}
		\Q[x] / (p(x))
	\end{align*}
	is a field, where \((p(x))\) denotes the root of \(p(x)\). We can in fact see that this space is equivalent to \(\Q\) but including the roots of \(x^2-2\), namely \(\sqrt{2} \). One can construct a unital isomorphism so that
	\begin{align*}
		\Q[x] / (x^2-2) \cong \Q(\sqrt{2} )
	\end{align*}
\end{exmp}

\section{Extension of Fields}
\label{sec:extension_of_fields}

\begin{defn}
	If \(K\) is a field containing a subfield \(F\), then \(K\) is said to be an \textbf{extension of \(F\)}, denoted by \(K / F\).\\

	The field \(F\) is sometimes called the \textbf{base field} of the extension.
\end{defn}
Note that if \(K\) is an extension of a field \(F\), then \(K\) is a \(F\)-vector space via the typical \(F\) action.

\begin{defn}
	The \textbf{degree} or \textbf{index} of a field extension \(K / F\), denoted \([K:F]\), is defined to be \(\textrm{dim}_F K\), the dimension of \(K\) as an \(F\)-vector space.
\end{defn}
For example, \([\Q(\sqrt{2} ) : \Q] = 2\) and \([\C: \R] = 2\). One can see the latter example by observing that \(\C\cong \R[x] / (x^2+1)\).

\end{document}
