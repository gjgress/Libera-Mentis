\documentclass{memoir}
\usepackage{notestemplate}

%\logo{./resources/pdf/logo.pdf}
%\institute{Rice University}
%\faculty{Faculty of Whatever Sciences}
%\department{Department of Mathematics}
%\title{Class Notes}
%\subtitle{Based on MATH xxx}
%\author{\textit{Author}\\Gabriel \textsc{Gress}}
%\supervisor{Linus \textsc{Torvalds}}
%\context{Well, I was bored...}
%\date{\today}

\begin{document}

% \maketitle

% Notes taken on 01/27/21

\begin{defn}[Norm of Integral Domain]
	Let \(R\) be an integral domain. Any function \(N:R\to \Z^{+}\cup \left\{ 0 \right\} \) with \(N(0)\) is called a \textbf{norm} on the integral domain \(R\). If \(N(a) > 0\) for \(a\neq 0\) then \(N\) is a \textbf{positive norm}
\end{defn}
This is a pretty loose construction, and an integral domain can have many norms on it.\\

\begin{defn}[Euclidean Function]
	An integral domain \(R\) is a \textbf{Euclidean Domain} if there exists a norm called the \textbf{Euclidean function} \(N:R\setminus\left\{ 0 \right\} \to \N\) that satisfies \(\forall a,b \in R\setminus \left\{ 0 \right\} \):
	\begin{align*}
		N(ab) \geq \textrm{max}\left\{ N(a),N(b) \right\}\\
		\exists q,r \in R \text{ such that }a = qb + r \text{ and }\left[ r = 0 \text{ or }N(r)<N(b) \right] 
	\end{align*}
	We call the element \(q\) the \textbf{quotient} and the element \(r\) the  \textbf{remainder}.
\end{defn}
The existence of a Euclidean function is integral to constructing a Euclidean algorithm to perform division of elements \(a,b \in R\). We can perform successive divisions to get
\begin{align*}
	a = q_0b + r_0\\
	b = q_1r_0 + r_1\\
	r_0 = q_2r_1 + r_2\\
	\vdots\\
	r_{n-1}= q_nr_{n-1}+r_n\\
	r_{n-1} = q_{n+1}r_n
\end{align*}
where \(r_n\) is the last nonzero remainder. This \(r_n\) always exists as the norms form a decreasing sequence of nonnegative integers. However, these elements are not necessarily unique.

\begin{prop}
	Every ideal in a Euclidean Domain is principal. That is, if \(I\triangleleft R\) is a nontrivial ideal in \(R\), then \(I = (d)\) for some nonzero element of \(I\) with minimum norm.
\end{prop}
This also makes it convenient to show an integral domain is not a Euclidean Domain by simply finding a non-principal ideal. Moreover, it motivates the notion of greatest common divisors from \(\Z\) into comumutative rings.

\begin{defn}[Greatest Common Divisor]
	Let \(R\) be a commutative ring and let \(a,b \in R\) with \(b\neq 0\). Then \(a\) is said to be a \textbf{multiple} of \(b\) if there exists an element \(x \in R\) with \(a=bx\). In this case, we say \(b\) \textbf{divides} \(a\) (or is a \textbf{divisor} of \(a \)) written \(b\mid a\).\\

	A \textbf{greatest common divisor} of \(a,b\) is a nonzero element \(d\) such that
	\begin{align*}
		d \mid a, \; d \mid b \\
		d'\mid a \; d'\mid b \implies d'\mid d
	\end{align*}
	We will denote a greatest common divisor by \(\textrm{gcd}(a,b)\), or sometimes simply \((a,b)\) if it is clear from context.\\

	If \(gcd(a,b) = 1_R\), then we say that \(a\) and \(b\) are \textbf{relatively prime}.
\end{defn}
We can easily extend this to finite sequences of elements \((a_1,a_2,\ldots,a_n)\).\\

Recall that in a ring \(b\mid a \iff a \in (b) \iff (a) \subset (b)\). Hence, we can discuss greatest common factors in terms of ideals. That is, if \(I = (a,b)\) is the ideal of \(R\) generated by \(a,b\), then \(d = \textrm{gcd}(a,b)\) if \(I\subset (d)\) and if \(I\subset (d') \implies (d)\subset (d')\). Thus, it is the unique smallest principal ideal containing \(a\) and \(b\). However, it may not exist in all rings.

\begin{prop}[Sufficient Conditions for Existence]
	If \(a,b \in R\) are nonzero elements in a commutative ring such that \(I = (a,b) = (d)\), then \(d\) is the greatest common divisor of \(a,b\).
\end{prop}
Obviously this is a sufficient and not a necessary condition. But it also clarifies why \((a,b)\) is used both for ideals and greatest common divisors. Any integral domain that satisfies the above condition for all ideals of two elements is called a \textbf{Bezout Domain}.

\begin{prop}
	Let \(R\) be an integral domain. If two elements \(d,d' \in R\) generate the same principal ideal, i.e. \((d) = (d')\), then \(d' = ud\) for some unit \(u \in R\). In particular, this tells us that greatest common divisors are unique up to units.
\end{prop}

\begin{thm}
	Let \(R\) be a Euclidean Domain and let \(a,b \in R\) be nonzero. Let \(d = r_n\) be the last nonzero remainder in the Euclidean Algorithm for \(a,b\) described earlier. Then \(d = \textrm{gcd}(a,b)\) and \((d) = (a,b)\). That is, \(d\) can be written as an \textbf{\(R\)-linear combination} of \(a,b\):
	\begin{align*}
		d = ax+by
	\end{align*}
	for some \(x,y \in R\).
\end{thm}
Notice that \(x,y\) are not unique in this case. One can show that that if \(x_0,y_0\) are solutions to
\begin{align*}
	ax+by=N
\end{align*}
then any other solutions are of the form
\begin{align*}
	x = x_0 + m \frac{b}{(a,b)}\\
	y = y_0 - m \frac{a}{(a,b)}
\end{align*}
for \(m \in \Z\). This is really strong as it gives a complete solution of the first order Diophantine equation provided we have one solution. Our work here essentially tells us that \(ax+by = N\) is solvable in integers \(x,y\) if and only if \(\textrm{gcd}(a,b)\mid N\).

\begin{proof}
	
\end{proof}

Finally, we discuss a definition that is useful to determine whether an integral domain is a Euclidean Domain.
\begin{defn}[Universal Side Divisor]
	Let \(R\) be an integral domain, and define \(\tilde{R}= R^{\times }\cup \left\{ 0 \right\} \). We say an element \(u \in R-\tilde{R}\) is a \textbf{universal side divisor} if for every \(x \in R\) there is a \(z \in \tilde{R}\) such that
	\begin{align*}
		u\mid x-z
	\end{align*}
	That is, every \(x\) can be written
	\begin{align*}
		x = qu + z
	\end{align*}
	where \(z\) is either zero or a unit.
\end{defn}

\begin{prop}
	Let \(R\) be an integral domain that is not a field. If \(R\) is a Euclidean Domain, then there exist universal side divisors in \(R\).
\end{prop}
It is often simpler to show that an integral domain can't have universal side divisors by assuming one exists of minimal norm, finding candidates, then showing they fail to satisfy the necessary properties.

\begin{hw}
	Let \(F = \Q(\sqrt{D} )\) be a quadratic field with quadratic integer ring \(\mathcal{O}\) and field norm \(N\).
	\begin{itemize}
		\item Suppose \(D \in \left\{ -1,-2,-3,-7,-11 \right\} \). Prove that \(\mathcal{O}\) is a Euclidean Domain with respect to \(N\). %Hint: Modify the proof for Z[i] in the text. For D = -3,-7, -11, prove that every element of F differs from an element in O by an element whose norm is at most (1+|D|)^2 / (16|D|), which is less than 1 for these values. Plotting the points of O might be helpful
		\item Suppose that \(D \in \left\{ -43,-67,-163 \right\} \). Prove that \(\mathcal{O}\) is not a Euclidean Domain with respect to any norm. %Apply same proof as D = -19 from text
	\end{itemize}
	These numbers are specially chosen because they are the only negative values of \(D\) that makes every ideal in \(\mathcal{O}\) principal.
\end{hw}

\subsection{Principal Ideal Domains}
\label{sub:principal_ideal_domains}

\begin{defn}[Princiapl Ideal Domain]
	A \textbf{Principal Ideal Domain} is an integral domain in which every ideal is principal.
\end{defn}
We have already shown that every Euclidean Domain is a Principal Ideal Domain. The converse does not hold. The biggest difference from a practicality angle is that while PIDs have gcds, there is no algorithm to compute them.

\begin{prop}
	Let \(R\) be a Principal Ideal Domain and let \(a,b \in R\) be nonzero. Let \(d\) be a generator for the principal ideal generated by \(a,b\). Then \(d = \textrm{gcd}(a,b)\) and can be written as an \(R\)-linear combination
	\begin{align*}
		d = ax+by
	\end{align*}
	for \(x,y \in R\). Finally, \(d\) is unique up to multiplication by a unit.
\end{prop}
Recall that maximal ideals are always prime ideals but the converse is not true in general. Fortunately, PIDs have enough structure so this holds.

\begin{prop}
	Every nonzero prime ideal in a Principal Ideal Domain is a maximal ideal.
\end{prop}
\begin{proof}% Try yourself
	
\end{proof}

\begin{cor}
	If \(R\) is any commutative ring such that \(R[x]\) is a PID, then \(R\) is necessarily a field.
\end{cor}
\begin{proof}
	
\end{proof}

We construct some definitions that help us distinguish PIDs and EDs.

\begin{defn}[Dedeking-Hasse Norm]
	Define \(N\) to be a \textbf{Dedekind-Hasse norm} if \(N\) is a positive norm and for every \(a,b \in R\) nonzero either \(a \in (b)\) or there exists \(s,t \in R\) with \(0 < N(sa-tb) < N(b)\) (that is, a nonzero element in the ideal \((a,b)\) with norm smaller than \(b\)).
\end{defn}
This is a weakening of the Euclidean condition. \(R\) is an ED with respect to a positive norm \(N\) if it is always possble to satisfy the above condition with \(s=1\).

\begin{prop}
	The integral domain \(R\) is a PID if and only if \(R\) has a Dedeking-Hasse norm.
\end{prop}

\begin{exmp}
	%Show that in Z[1+sqrt(-19) / 2] N()=a^2+ab+5b%2 is a Dedekind-Hasse norm
\end{exmp}

\subsection{Unique Factorization Domain}
\label{sub:unique_factorization_domain}

Unique Factorization Domains capture the idea that some rings admit a proper factorization on elements.

\begin{defn}[Reducibility and Primes]
	Let \(R\) be an integral domain
	\begin{itemize}
		\item Suppose \(r \in R\) is nonzero and not a unit. Then \(r\) is called \textbf{irreducible in \(R\)} if for all \(a,b \in R\), \(r = ab\) implies that either \(a\) or \(b\) is a unit. Otherwise, we say \(r\) is \textbf{reducible}.
		\item Let \(p \in R\) be nonzero. We say it is \textbf{prime in \(R\)} if the ideal \((p)\) is a prime ideal. An equivalent statement is that \(p\) is not a unit and if \(p\mid ab\), then \(p\mid a\) or \(p\mid b\).
		\item If \(a=ub\) for \(a,b \in R\) and \(u \in R\) a unit, then we say \(a\) and \(b\) are \textbf{associates}.
	\end{itemize}
\end{defn}

\begin{prop}
	In an integral domain, a prime element is always irreducible.
\end{prop}
The converse does not hold in general. However, in a PID, the converse does hold.

\begin{proof}% Try yourself, also try showing that in a PID it does hold
	
\end{proof}
This is also a useful tool to show a ring is not a PID.

\begin{defn}[Proper Factorization]
	Let \(a \in R\) be a nonzero nonunit. A \textbf{proper factorization} of \(a\) is a finite product \(a=p_1p_2\ldots p_n\), where \(p_i\) are not units of \(R\). If this exists, we say \(\left\{ p_i \right\} \) are \textbf{proper factors} of \(a\).
\end{defn}
Of course, an irreducible element has no proper factorizations.

\begin{defn}[Unique Factorization Domain]
	An integral domain \(R\) is a \textbf{unique factorization domain} if every nonzero, non-unit element has a proper factorization
	\begin{align*}
		r = p_1p_2\ldots p_n
	\end{align*}
	where \(\left\{ p_i \right\} \) are irreducible elements and unique up to associates and reordering.
\end{defn}
It turns out that primes are equivalent to irreducibles in a UFD as well.

\begin{prop}
	In a Unique Factorization Domain, a nonzero element is a prime if and only if it is an irreducible.
\end{prop}
We will also see that UFDs admit a greatest common divisor via its factorization

\begin{prop}
	Let \(a,b \in R\) be nonzero elements of a UFD \(R\) and suppose
	\begin{align*}
		a = up_1^{e_1}p_2^{e_2}\ldots p_n^{e_n}\\
		b= v p_1^{f_1}p_2^{f_2}\ldots p_n^{f_n}
	\end{align*}
	are prime factorizations with \(u,v\) units, primes \(p_1,p_2,\ldots,p_n\) distinct, and exponents \(e_i,f_i \geq 0\). Then the element
	\begin{align*}
		d = p_1^{\textrm{min}(e_1,f_1)}p_2^{\textrm{min}(e_2,f_2)}\ldots p_n^{\textrm{min}(e_n,f_n)}
	\end{align*}
	is a greatest common divisor of \(a\) and \(b\).
\end{prop}

\begin{hw}
	Let \(R\) be a UFD.
	\begin{enumerate}[(a).]
		\item Let \(b\) and \(a_1,\ldots,a_s\) be nonzero elements of \(R\). For \(d \in R\), show that
			\begin{align*}
				bd = gcd(ba_1,\ldots,ba_s) \iff d = gcd(a_1,\ldots,a_s)
			\end{align*}
		\item Let \(f(x) \in R[x]\) where \(f(x) = bf_1(x)\) for \(f_1(x)\) \textbf{primitive} (i.e. \(gcd(\text{coefficients of }f_1(x)) = 1_R\) ). Show that
			\begin{align*}
				b = gcd(\left\{ \text{coefficients of }f(x) \right\} ).
			\end{align*}
	\end{enumerate}
\end{hw}

This leads us to the full description of the structure of these domains.

\begin{thm}
	Every Principal Ideal Domain is a Unique Factorization Domain. Hence, every Euclidean Domain is a Unique Factorization Domain.
\end{thm}
\begin{proof}
	
\end{proof}

This forms a strict classification hierarchy by
\begin{align*}
	\textbf{Euclidean Domains} \subset \textbf{Principal Ideal Domains} \subset \textbf{Unique Factorization Domains} \subset \textbf{Integral Domains} \subset \textbf{commutative rings}
\end{align*}

\begin{cor}
	The integers \(\Z\) are a UFD.
\end{cor}

\begin{cor}
	Let \(R\) be a PID. Then there exists a multiplicative Dedekind-Hausse norm on \(R\).
\end{cor}

\end{document}
