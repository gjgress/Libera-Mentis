\documentclass{memoir}
\usepackage{notestemplate}

%\logo{~/School-Work/Auxiliary-Files/resources/png/logo.png}
%\institute{Rice University}
%\faculty{Faculty of Whatever Sciences}
%\department{Department of Mathematics}
%\title{Class Notes}
%\subtitle{Based on MATH xxx}
%\author{\textit{Author}\\Gabriel \textsc{Gress}}
%\supervisor{Linus \textsc{Torvalds}}
%\context{Well, I was bored...}
%\date{\today}

\begin{document}

% \maketitle

% Notes taken on 05/10/21

Now we apply this theorem to finite fields. Consider \(\mathbb{F}_{p^{n}}\), the splitting field of \(x^{p^{n}}-x\). This is Galois over \(\mathbb{F}_p\). Thus we have \(\left| \textrm{Aut}(\mathbb{F}_{p^{n}} / \mathbb{F}_p )\right| = [ \mathbb{F}_{p^{n}}: \mathbb{F}_p ] = n \). This gives us \(\textrm{Gal}(\mathbb{F}_{p^{n}} / \mathbb{F}_p ) = \Z / n\Z\) and the Galois group consists solely of the Frobenius endomorphism.\\

One can see then that all subfields \(\mathbb{F}_p\subset E\subset \mathbb{F}_{p^{n}}\) have the form \(E \cong \mathbb{F}_{p^{d}}\) for some \(d\mid n\). Of course, this means that \(E / F\) is necessarily Galois as well!

\section{Applications of Galois Theory}
\label{sec:applications_of_galois_theory}

\begin{prop}
	The irreducible polynomial \(x^{4}+1 \in \Z[x]\) is reducible over \(\mathbb{F}_p\) for any prime \(p\).
\end{prop}
\begin{proof}
	One can check this directly for \(p=2\). If \(p>2\), then observe that \(p \cong 1,3,5\) or \(7 \mod 8\), and hence \(p^2 \cong 1 \mod 8\). Therefore we have that \(x^{8}-1 \mid x^{p^2-1}-1\) over \(\mathbb{F}_p\).\\

	Of course, \(x^{4}+1 \mid x^{8}-1\) and so any root of \(x^{4}+1\) is a root of \(x^{p^2}-x\) and hence are elements of the field \(\mathbb{F}_{p^2}\). Since \([\mathbb{F}_{p^2}:\mathbb{F}_p] = 2\), the degree of the extension is no more than 2. Of course, if \(x^{4}+1\) were irreducible over \(\mathbb{F}_p\), then it would necessarily be 4, and hence it must be reducible.
\end{proof}

\begin{prop}
	\begin{align*}
		x^{p^{n}}-x = \prod_{d\mid n} \left\{ \text{irreducible polynomial in \(\mathbb{F}_p[x]\) of degree \(d\)} \right\}  
	\end{align*}
\end{prop}
We can use this recursively as \(n\) increases.
\end{document}
