\documentclass{memoir}
\usepackage{notestemplate}

%\logo{~/School-Work/Auxiliary-Files/resources/png/logo.png}
%\institute{Rice University}
%\faculty{Faculty of Whatever Sciences}
%\department{Department of Mathematics}
%\title{Class Notes}
%\subtitle{Based on MATH xxx}
%\author{\textit{Author}\\Gabriel \textsc{Gress}}
%\supervisor{Linus \textsc{Torvalds}}
%\context{Well, I was bored...}
%\date{\today}

\begin{document}

% \maketitle

% Notes taken on ??

Of course, we need to be sure that our construction of characters is well-defined. It turns out that they capture the properties of our representation well and are unique.

\begin{prop}
	\begin{enumerate}
		\item The definition of \(X_V\) is independent of choice of basis of \(V\)
		\item If \(V \cong W\), then \(X_V = X_W\) 
		\item If \(g,h \in G\) are conjugate, then \(X_V(g) = X_V(h)\)
	\end{enumerate}
\end{prop}

\begin{defn}
	The \textbf{character table} of \(G\) is defined as
	\begin{align*}
		\begin{bmatrix} X_{V_1}(g_1) & X_{V_1}(g_2) & \ldots & X_{V_1}(g_k)\\
		\vdots & & & \vdots \\
	X_{V_\ell}(g_1) & X_{V_\ell}(g_2) & \ldots & X_{V_\ell}(g_k)\end{bmatrix} 
	\end{align*}
\end{defn}
The number of irreducible characters of \(G\) is the same as the number of conjugacy classes of elements of \(G\). Furthermore, the character table is a square matrix with entries in \(\C\) when the rows are indexed by irreducible representations of \(G\) and the columns are indexed by conjugacy classes representations of elements of \(G\).\\

In this case, \((X_{V_i}(g_j))\) is an invertible matrix.

% Examples

\begin{prop}
	Let \((V,\rho )\) be a representation of \(G\), and take \(g \in G\). Then
	\begin{enumerate}
		\item \(X_V(e) = \textrm{dim}(V)\) 
		\item \(X_V(g)\) is a sum of roots of unity
		\item \(X_{V\oplus W}(g) = X_{V}(g) + X_{W}(g)\)
		\item \(X_V(g^{-1}) = \overline{X_V(g)}\) 
		\item \(\overline{X_V}\) is a character of \(G\)
	\end{enumerate}
\end{prop}

Let \(X_1,\ldots,X_r\) be irreducible characters of a finite group \(G\). Define
\begin{align*}
	\langle X_i, X_j \rangle := \frac{1}{\left| G \right| } \sum_{g \in G} \overline{X_i(g)}X_j(g)
\end{align*}

\begin{thm}
	\begin{enumerate}
		\item \(\langle X_i, X_j \rangle = \delta _{ij}\) 
		\item 
			\begin{align*}
				\sum_{i=1}^{r} \overline{X_i(x)}X_i(y) = \begin{cases}
					\left| C_G(x) \right| & x,y \text{ conjugate in }G\\
					0 & \text{otherwise}
				\end{cases}
			\end{align*}
	\end{enumerate}
\end{thm}
			Here, \(C_G(x)\) is the centralizer of \(x \in G\), that is, \(C_G(x) = \textrm{Stab}_x(G) = \left\{g \in G \mid g x g^{-1} = x \right\} \).

% Example

\begin{thm}
	If \(V,W\) are irreducible representations of \(G\), then
	\begin{align*}
		\langle X_V,X_V \rangle = 1\\
		\langle X_V,X_W \rangle = 0 \text{ when }V\not\cong W
	\end{align*}
\end{thm}
This shows us that characters completely determine representations, and forthermore characters completely determine irreducibility.
\end{document}
