\documentclass{memoir}
\usepackage{notestemplate}

%\logo{./resources/pdf/logo.pdf}
%\institute{Rice University}
%\faculty{Faculty of Whatever Sciences}
%\department{Department of Mathematics}
%\title{Class Notes}
%\subtitle{Based on MATH xxx}
%\author{\textit{Author}\\Gabriel \textsc{Gress}}
%\supervisor{Linus \textsc{Torvalds}}
%\context{Well, I was bored...}
%\date{\today}

\begin{document}

% \maketitle

% Notes taken on 02/01/21

\section{Irreducibility Criteria}
\label{sec:irreducibility_criteria}

If \(R\) is an integral domain, then for \(f(x) \in R[x]\) monic, of degree \(>0\), is irreducible if and only if \(f(x)\) cannot be factored as a product of two polynomials of deg \(\geq 1\).
Fortunately, we have a few tools to get irreducibility of polynomials, such as Gauss' Lemma.\\

Another direction we can take is roots:
\begin{prop}
	Let \(f(x) \in F[x]\) for \(F \) a field. Then
	\begin{itemize}
		\item \(f(x)\) has a degree 1 factor if and only if \(f(x)\) has a root \(\alpha\) in \(F\), i.e. \(\exists \alpha \in F\) such that \(f(\alpha)=0\) 
		\item \(f(x)\) of degree 2 or 3 is reducible if and only if \(f(x)\) has a root in \(F\)
	\end{itemize}
\end{prop}
If we look at \(R = \Z\) and \(F = \Q\) specifically, we have more options.
\begin{prop}[Rational Root Test]
	Let \(f(x) = \sum_{i=1}^{n} a_i x^{i} \in \Z[x]\). 
\begin{itemize}
	\item If \(\frac{r}{s}\in \Q\) with \(gcd(r,s) = 1\) and \(\frac{r}{s}\) is a root of \(f(x)\), then \(r\mid a_0\) and \(s\mid a_n\).
	\item If \(f(x) \in Z[x]\) is monic and if \(f(\alpha)\neq 0\) for all \(\alpha \in \Z\) dividing \(\alpha_0\), then \(f(x)\) has no roots in \(\Q\).
\end{itemize}
\end{prop}
Unfortunately, while these theorems are powerful, they are relatively dependent on the polynomial being of low degree. Ideals can help us extend these ideas to higher degree polynomials.

\begin{prop}
	Let \(R\) be an integral domain, and let \(I \triangleleft R\) be a proper ideal of \(R\). Take \(f(x) \in R[x]\) a monic polynomial of degree \(\geq 1\). If the image of \(f(x)\) in \((R / I)[x]\) is irreducible, then \(f(x)\) is irreducible in \(R[x]\).
\end{prop}

While nice, unfortunately many irreducible polynomials are reducible when modulated by the ideal.

\begin{thm}[Eisenstein-Schonemann Criteria]
	Let \(P\) be a prime ideal of an integral domain \(R\), and take
	\begin{align*}
		f(x) = x^{n}+ a_{n-1}x^{n-1}+ \ldots + a_1x + a_0
	\end{align*}
	to be a monic polynomial in \(R[x]\) of degree \(\geq 1\). Suppose \(a_{n-1},\ldots,a_1,a_0 \in P\) and \(a_0 \not\in P^2\). Then \(f(x) \) is irreducible in \(R[x]\).
\end{thm}
A trick we can use to help apply this is that if \(f(x)\) doesn't satisfy the criteria, use \(f(x-c)\) and try again. If it is (ir)reducible for \(f(x-c)\), it is ir(reducible) for \(f(x)\).
\end{document}
