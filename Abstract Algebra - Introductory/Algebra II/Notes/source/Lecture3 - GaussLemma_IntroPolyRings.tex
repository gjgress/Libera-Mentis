\documentclass{memoir}
\usepackage{notestemplate}

%\logo{./resources/pdf/logo.pdf}
%\institute{Rice University}
%\faculty{Faculty of Whatever Sciences}
%\department{Department of Mathematics}
%\title{Class Notes}
%\subtitle{Based on MATH xxx}
%\author{\textit{Author}\\Gabriel \textsc{Gress}}
%\supervisor{Linus \textsc{Torvalds}}
%\context{Well, I was bored...}
%\date{\today}

\begin{document}

% \maketitle

% Notes taken on 01/29/21

For posterity we restate the definition of a polynomial ring.

\begin{defn}[Polynomial Ring]
	Let \(R\) be a commutative ring. We define a \textbf{polynomial} in \(x\) to be the formal sum
	\begin{align*}
		a_n x^{n} + a_{n-1}x^{n-1} + \ldots + a_1 x + a_0
	\end{align*}
	where \(n\geq 0\) and \(a_i \in R\). If \(a_n \neq 0\), then the polynomial is of \textbf{degree \(n\)}, and \(a_nx^{n}\) is te \textbf{leading term} (\(a_n\) is the \textbf{leading coefficient}). Furthermore, we say the polnyomial is \textbf{monic} if \(a_n=1\).\\

	The set of all such polynomials is called the \textbf{ring of polynomials in \(\R\)} and will be denoted \(R[x]\). We define addition and multiplication by the standard version from algebra:
	\begin{align*}
		(a_nx^{n} + \ldots + a_1x + a_0) + (b_nx^{n} + \ldots + b_1x + b_0) = (a_n+b_n)x^{n} + \ldots + (a_1+b_1)x + (a_0 + b_0)\\
		(a_0+a_1x + a_2x^2+\ldots) \times (b_0 + b_1x + b_2x^2 + \ldots) = a_0b_0 + (a_0b_1 + a_1b_0)x + (a_0b_2 + a_1b_1 + a_2b_0)x^2 + \ldots
	\end{align*}
	That is, the coefficient in the product of \(x^{k}\) is \(\sum_{i=0}^{k} a_i b_{k-i}\).
\end{defn}
Recall that \(\textrm{deg}(p(x)q(x)) = \textrm{deg}(p(x)) + \textrm{deg}(q(x))\) if \(p,q \neq 0\). Furthermore, the units of \(R[x]\) are the units of \(R\), and \(R[x]\) is an integral domain.

\begin{prop}
	Let \(I\triangleleft R\) be an ideal and let \(\left( I \right) = I[x]\) denote the ideal in \(R[x]\) generated by \(I\). Then
	\begin{align*}
		R[x] / (I) \cong (R / I)[x]
	\end{align*}
	and hence if \(I\) is a prime ideal of \(R\), \((I)\) is a prime ideal of \(R[x]\).
\end{prop}
This does not hold for maximal ideals, but \((I,x)\) is maximal in \(R[x]\) if \(I\) is maximal in \(R\).

\begin{defn}[Polynomial Rings over Multiple Variables]
	We inductively define the \textbf{polynomial ring in the variables} \(x_1,x_2,\ldots,x_n\) with coefficients in \(R\) to be
	\begin{align*}
		R[x_1,x_2,\ldots,x_n] := R[x_1,x_2,\ldots,x_n][x_n]
	\end{align*}
	Hence, we can view polynomial rings of multiple variables as polynomial rings on a single variable, with polynomials of \(n-1\) variables as coefficients.
\end{defn}
We say a polynomial is \textbf{homogeneous} if all its terms have the same degree. If \(f\) is a nonzero polynomial in \(n\) variables, the sum of all monomial terms in \(f\) of degree \(k\) is called the \textbf{homogeneous component of \(f\) of degree \(k\)}.

\section{Polynomial Rings over Fields}
\label{sec:polynomial_rings_over_fields}

Let \(R = F\) be a field. We can define a natural norm on \(F[x]\) by
\begin{align*}
	N(p(x)) = \textrm{deg}(p(x)).
\end{align*}
\begin{thm}
	Let \(F\) be a field. The polynomial ring \(F[x]\) is a Euclidean Domain. This implies that if \(a(x),b(x) \in F[x]\) with \(b(x)\) nonzero, then
	\begin{align*}
		a(x) = q(x)b(x) + r(x)
	\end{align*}
	where \(q(x),r(x) \in F[x]\) are unique polynomials, and \(r(x) = 0\) or \(\textrm{deg}(r(x)) < \textrm{deg}(b(x))\).
\end{thm}

\begin{proof}
	
\end{proof}
Of course, this tells us that \(F[x]\) is a PID and a UID.\\

In fact, the quotient and remainder in the division algorithm are \textit{independent of field extensions}. That is, if \(F\subset E\) is a field extension, then \(b(x)\mid a(x)\) in \(E[x]\) if and only if \(b(x)\mid a(x)\) in \(F[x]\), and \(\textrm{gcd}(a(x),b(x))\) is the same in both fields.

\begin{prop}
	The maximal ideals in \(F[x]\) are the ideals \((f(x))\) generated by irreducible polynomials \(f(x)\). In particular, \(F[x] / (f(x))\) is a field if and only if \(f(x)\) is irreducible.
\end{prop}

\begin{prop}
	Let \(g(x) \in F[x]\) be nonconstant and let
	\begin{align*}
		g(x) = f_1(x)^{n_1}f_2(x)^{n_2}\ldots f_k(x)^{n_k}
	\end{align*}
	be its factorization into irreducibles, where the \(f_i(x)\) are distinct. Then
	\begin{align*}
		F[x] / (g(x)) \cong F[x] / (f_1(x)^{n_1}) \times F[x] / (f_2(x)^{n_2}) \times  \ldots \times F[x] / (f_k(x)^{n_k}).
	\end{align*}
\end{prop}

Notice that if \(f(x)\) has roots \(\alpha_1,\alpha_2,\ldots,\alpha_k\) in \(F\), then \(f(x)\) has \((x-\alpha_1)\ldots(x-\alpha_k)\) as a factor. In other words, a polynomial of degree \(n\) over a field has at most \(n\) roots in \(F\).

\begin{prop}
	A finite subgroup of the multiplicative group of a field is cyclic. In particular, if \(F\) is a finite field, then \(F^{\times }\) is a cyclic group.
\end{prop}
\begin{proof}
	
\end{proof}

\begin{cor}
	Let \(p\) be a prime. Then \((\Z / p\Z)^{\times }\) of nonzero residue classes \(\pmod{p}\) is cyclic.
\end{cor}

\begin{cor}
	Let \(n\geq 2\) be an integer with factorization
	\begin{align*}
		n = p_1^{\alpha_1}p_2^{\alpha_2}\ldots p_r^{\alpha_r}
	\end{align*}
	with \(p_1,\ldots,p_r\) are distinct. Then
	\begin{align*}
		(\Z / n\Z)^{\times } \cong ( \Z/ p_1^{\alpha_1} \Z)^{\times } \times (\Z / p_2^{\alpha_2}\Z)^{\times } \times \ldots \times (\Z / p_r^{\alpha_r}\Z)^{\times },
	\end{align*}
	in particular \((\Z / 2^{\alpha }\Z)^{\times }\) is the direct product of a cyclic group of order 2 and a cyclic group of order \(2^{\alpha -2}\) for all \(\alpha \geq 2\).\\

	Finally, \((\Z / p^{\alpha }\Z)^{\times }\) is a cyclic group of order \(p^{\alpha-1}(p-1)\) for all odd primes \(p\).
\end{cor}
These describe the group theory structure of the automorphism group of the cyclic group \(\Z_n\), as \(\textrm{Aut}(\Z_n) \cong (\Z / n\Z)^{\times }\).

\begin{proof}
	
\end{proof}

\section{Polynomial Rings and UFDs}
\label{sec:polynomial_rings_and_ufds}


\begin{defn}[Primitive]
	A polynomial \(f(x) \in R[x]\) is \textbf{primitive} if \( \textrm{gcd}(\left\{ \text{coeff of }f(x) \right\} ) = 1_R\)
\end{defn}
Recall that since \(R\) is an integral domain, one can form its field of fractions by
\begin{align*}
	F:= \textrm{Frac}(R) = \left\{ \frac{r}{s} \mid r,s \in R, \; s \neq 0 \right\} 
\end{align*}
\begin{lemma}[Gauss' Lemma]
	Let \(R\) be a UFD with \(F = \textrm{Frac}(R)\).
	\begin{itemize}
		\item If \(f(x),g(x) \in R[x]\) are primitive, then so is \(f(x)\cdot g(x)\).
		\item Take \(f(x) \in R[x]\). Then \(f(x) = \varphi(x) \psi(x) \in F[x]\) with \( \textrm{deg}(\varphi) \textrm{deg}(\psi)\geq 1 \iff f(x) = \psi(x) \varphi(x)\) in \(R[x]\).
	\end{itemize}
\end{lemma}
The elements of the ring \(R\) become units in the UFD \(F[x]\).


\begin{cor}
	Let \(R\) be a UFD. The irreducible elements of \(R[x]\) are of two types:
	\begin{itemize}
		\item nonzero scalar polynomials that are irreducible as elements of \(R\) 
		\item primitive polynomials in \(R[x]\) that are irreducible in \(F[x]\)
	\end{itemize}
\end{cor}
Essentially, a polynomial \(p(x)\) with \(\textrm{deg}(p(x))\geq 1\) is irreducible in \(R[x]\) if and only if it is irreducible in \(F[x]\).

\begin{thm}
	\(R\) is a Unique Factorization Domain if and only if \(R[x]\) is a Unique Factorization Domain.
\end{thm}
\begin{proof}
	
\end{proof}
By induction, it holds that \(R[x_1,x_2,\ldots,x_n]\) is a UFD if and only if \(R\) is a UFD.

\end{document}
