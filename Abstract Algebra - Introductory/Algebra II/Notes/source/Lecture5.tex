\documentclass{memoir}
\usepackage{notestemplate}

%\logo{./resources/pdf/logo.pdf}
%\institute{Rice University}
%\faculty{Faculty of Whatever Sciences}
%\department{Department of Mathematics}
%\title{Class Notes}
%\subtitle{Based on MATH xxx}
%\author{\textit{Author}\\Gabriel \textsc{Gress}}
%\supervisor{Linus \textsc{Torvalds}}
%\context{Well, I was bored...}
%\date{\today}

\begin{document}

% \maketitle

% Notes taken on 02/05/21

\section{Modules}
\label{sec:modules}

Briefly, an \(R\)-module \(M\) is an abelian group that comes equipped with a binary operation
\begin{align*}
	* : R\times M \to M
\end{align*}
that is compatible with operations of both \(M\) and \(R\).

\begin{defn}
	Let \(R\) be a ring. A \textbf{left \(R\)-module} is a pair \((M,*:R\times M\to M := _RM)\) where \(M\) is an abelian group, and \(*\) is a binary operation so that
	\begin{align*}
		\forall r,s \in R, \; m,n \in M:\\
		r*(m+n) = (r*m) + (r*n)\\
		(r+s)*m = (r*m) + (s*m)\\
		(rs)*m = r*(s*m)
	\end{align*}
	If \(R\) is unital, then we also require
	\begin{align*}
		1_R * m = m.
	\end{align*}
	The map is called the (left) \textbf{\(R\)-action map}.
\end{defn}

\begin{exmp}
	\begin{enumerate}
		\item If \(R\) is a field \(F\), then the \(R\)-module is an \(F\)-vector space.
		\item Take \(M = \left\{(t_1,\ldots,t_n) \mid t_i \in R \right\}:= R^{n}\). Let
			\begin{align*}
				R\times M \to M\\
				(r,(t_1,\ldots,t_n)) \mapsto (rt_1,\ldots,rt_n).
			\end{align*}
			This yields a left \(R\)-action on \(M = R^{n}\). This left \(R\)-module \(_\R^{n}:= R^{n}\) is called the \textbf{free left \(R\)-module of rank \(n\)}.
	\end{enumerate}
\end{exmp}

\end{document}
