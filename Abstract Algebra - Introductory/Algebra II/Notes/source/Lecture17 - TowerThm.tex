\documentclass{memoir}
\usepackage{notestemplate}

%\logo{~/School-Work/Auxiliary-Files/resources/png/logo.png}
%\institute{Rice University}
%\faculty{Faculty of Whatever Sciences}
%\department{Department of Mathematics}
%\title{Class Notes}
%\subtitle{Based on MATH xxx}
%\author{\textit{Author}\\Gabriel \textsc{Gress}}
%\supervisor{Linus \textsc{Torvalds}}
%\context{Well, I was bored...}
%\date{\today}

\begin{document}

% \maketitle

% Notes taken on whenever

\section{Algebraic Extensions}
\label{sec:algebraic_extensions}

\begin{thm}[Tower Theorem]
	Let \(F \hookrightarrow E \hookrightarrow K\) be a composition of field extensions. Then \([K:F] = [K:E] [E:F]\).
\end{thm}
One can show this via vector space arguments (look at the bases of the spaces).
\begin{cor}
	If \(K / F\) is a finite extension, and \(E\) is a subfield of \(K\) containing \(F\), then \([E:F] \mid [K:F] \).
\end{cor}

\begin{exmp}
	Let 
\begin{align*}
	K &= \Q(\sqrt[6]{2} )\\
	E &= \Q(\sqrt{2} )\\
	F &= \Q
\end{align*} It follow directly from previous work that \([\Q(\sqrt[6]{2}) : \Q] = 6\) and \([\Q(\sqrt{2} ):\Q] = 2\). As for \(K / E\), the minimal polynomial is \(x^3-\sqrt{2} \), which gives \([\Q(\sqrt[6]{2} ): \Q(\sqrt{2} )] = 3\), which corresponds to what the tower theorem gives us.
\end{exmp}

\begin{defn}
	An extension \(K / F\) is called \textbf{finitely generated} if there exist elements \(\alpha_1, \alpha_2,\ldots,\alpha_n\) such that
	\begin{align*}
		K = F(\alpha_1, \alpha_2, \ldots, \alpha_n) \quad \text{for }n<\infty
	\end{align*}
	Such an extension can be obtained recursively via simple extensions.
\end{defn}
	We have that \(F(\alpha ,\beta ) = (F(\alpha ))(\beta )\), hence the definition above is consistent.
\begin{exmp}
	\begin{itemize}
		\item \(\Q(\sqrt[6]{2} ,\sqrt{2} ) = \left( \Q(\sqrt[6]{2} ) \right) (\sqrt{2} ) = \Q(\sqrt[6]{2} )\) because \(\sqrt{2} = (\sqrt[6]{2} )^3\).
		\item One can check that \(\Q(\sqrt{2} ,\sqrt{3} )\) is a proper field extension for both \(\Q(\sqrt{2} )\) and \(\Q(\sqrt{3} )\).
	\end{itemize}
\end{exmp}
\begin{thm}
	\(K / F\) is finite if and only if \(K\) is generated by a finite number of algebraic elements over \(F\).
\end{thm}
We denote by \(\overline{\Q}\) the subfield of \(\C\) generated by all algebraic elements of \(\C\) over \(\Q\). \(\overline{\Q}\) is an infinite algebraic extension of \(\Q\), and referred to as the \textbf{field of algebraic numbers}. \\

\begin{thm}
	If \(E / F\) and \(K / E\) are algebraic, then \(K / F\) is algebraic.
\end{thm}
\end{document}
