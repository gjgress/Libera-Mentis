\documentclass{memoir}
\usepackage{notestemplate}

%\logo{~/School-Work/Auxiliary-Files/resources/png/logo.png}
%\institute{Rice University}
%\faculty{Faculty of Whatever Sciences}
%\department{Department of Mathematics}
%\title{Class Notes}
%\subtitle{Based on MATH xxx}
%\author{\textit{Author}\\Gabriel \textsc{Gress}}
%\supervisor{Linus \textsc{Torvalds}}
%\context{Well, I was bored...}
%\date{\today}

\begin{document}

% \maketitle

% Notes taken on 05/10/21

\begin{thm}
	Let \(K / \mathbb{F}_p\) be a field extension of the prime subfield \(\mathbb{F}_p\).
	\begin{itemize}
		\item If \(K\) is finite, then \(\left| K \right| = p^{n}\) for some positive integer \(n\).
		\item \(\left| K \right| = p^{n}\) if and only if \(K\) is the splitting field of \(x^{p^{n}}-x\) over \(\mathbb{F}_p\).
	\end{itemize}
	By the uniqueness of splitting fields, we can simply denote \(K\) by \(\mathbb{F}_{p^{n}}\).
\end{thm}
This theorem gives us a complete characterization of finite fields. The first part is proven in Dummitt-Foote 13.2 \#1.

\begin{cor}
	For all prime \(p\), for all \(n \in \Z_+\), there exists a field of cardinality \(p^{n}\). Furthermore, any two finite fields of the same cardinality are isomorphic.
\end{cor}

\section{Simple Extensions}
\label{sec:simple_extensions}

\begin{thm}
	If \(\left| F \right| < \infty\), and \(K / F\) is a finite extension of \(F\), then \(K = F(\alpha )\) for some \(\alpha  \in K\).	
\end{thm}

This holds because \(K^{\times }\) is a cyclic group, and so there must exist \(\alpha \) so \(\langle \alpha  \rangle = K^{\times }\), and hence \(K = F(\alpha )\).

\begin{thm}
	If \(F\) is an infinite field, and \(K / F\) is a finite separable extension, then \(K = F(\alpha )\) for some \(\alpha  \in K\).
\end{thm}
Every field extension can be written by appending a sequence of elements, and we can reduce the elements to one by the combination \(\alpha  = \beta + \gamma  \delta \), where \((\beta ,\gamma )\) is the two additional elements, and \(\delta \neq \frac{\beta_i - \beta }{\gamma  - \gamma_j}\). Often we can simply choose \(\delta =1\) if we are lucky.

\end{document}
