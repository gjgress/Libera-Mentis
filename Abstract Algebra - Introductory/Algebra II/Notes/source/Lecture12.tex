\documentclass{memoir}
\usepackage{notestemplate}

%\logo{~/School-Work/Auxiliary-Files/resources/png/logo.png}
%\institute{Rice University}
%\faculty{Faculty of Whatever Sciences}
%\department{Department of Mathematics}
%\title{Class Notes}
%\subtitle{Based on MATH xxx}
%\author{\textit{Author}\\Gabriel \textsc{Gress}}
%\supervisor{Linus \textsc{Torvalds}}
%\context{Well, I was bored...}
%\date{\today}

\begin{document}

% \maketitle

% Notes taken on 02/26/21

\section{G-homomorphisms}
\label{sec:g_homomorphisms}

Let \(K\) be a field and \(G\) be a group. We are going to look at a structure of interest: we define \(K\)-linear representations of \(G\) as a \(K\)-vector space \(V\) equipped with a group homomorphism
\begin{align*}
	\rho:G\to GL(V)\\
	g\mapsto [\rho g:V\to V]
\end{align*}

\begin{defn}[\(G\)-homomorphism]
	Let \((V',\rho')\) and \((V'',\rho'')\) be representations of \(G\) over \(K\). A \textbf{\(G\)-homomorphism from \((V',\rho')\) to \(V'',\rho'')\)} is a \(K\) linear map \(\varphi:V'\to V''\) which intertwines with the action of \(G\) :
	\begin{align*}
		\varphi(\rho'g(v')) = \rho''g(\varphi(v')) \quad \forall g \in G, v' \in V'
	\end{align*}
	We denote the collection of \(G\)-homomorphisms from \((V', \rho')\) to \((V'', \rho'')\) by \( \textrm{Hom}_G(V',V'')\), and \(\textrm{End}_G(V') := \textrm{Hom}_G(V',V')\). Finally, a \textbf{\(G\)-isomorphism} is an invertible \(G\)-homomorphism.
\end{defn}

This is really just a change in basis.

\begin{prop}
	If \(\varphi \in \textrm{Hom}_G(V,W)\), then \(\varphi^{-1} \in \textrm{Hom}_G(W,V)\).
\end{prop}
\begin{prop} %Exercise 1 on HW 5
	Take \(\varphi \in \textrm{Hom}_G(V,W)\). Then
	\begin{enumerate}[(a).]
		\item \(\textrm{Ker}\varphi \) is a subrepresentation of \(V\), and
		\item \(\textrm{Im}\varphi\) is a subrepresentation of \(W\).
	\end{enumerate}
\end{prop}
For the rest of the section, we take \(K = \C\).

\begin{lemma}[Schur's Lemma]
	Let \((V,\rho)\) be an irreducible representation of \(G\). If \(\varphi \in \textrm{End}_G(V)\), then \(\varphi\) is a scalar multiple of \(\textrm{Id}V\) :
	\begin{align*}
		\exists \lambda \in \C \; s.t. \; \varphi(v) = \lambda v \quad \forall v \in V
	\end{align*}
\end{lemma}
This result has many applications.
\begin{thm}
	All nonzero complex irreducible representations of an abelian group \(G\) have degree 1.
\end{thm}
Using these tools, we now can complete a few problems.

\begin{hw}
	Given a finite abelian group \(G\), describe its irreducible representations, up to equivalence. Illustrate this for the Klein-four group \(G = C_2\times C_2\).
\end{hw}
Moreover, one can apply Schur's lemma to complete the following problem:
\begin{hw}
	Let \(V\) and \(W\) be irreducible representations of \(G\), and take \(\varphi \in \textrm{Hom}_G(V,W)\). Show that
	\begin{enumerate}[(a).]
		\item If \(V \not\cong W\), then \(\varphi\) is the zero map.
		\item If \(V \cong W\) and \(\varphi\neq 0\), then \(\varphi\) is a \(G\)-isomorphism.
	\end{enumerate}
\end{hw} % Hint: hw5 exercise 1


\section{Character Theory}
\label{sec:character_theory}

Character theory will serve as a very convenient bookkeeping tool for representations of \(G\) when \(G\) is finite. We still keep \(K = \C\).

\begin{defn}
	Let \((V,\rho)\) be a \(\C\)-representation of \(G\) of finite degree \(n\). Choose any basis of \(V\) and express \(\rho_g\) as a matrix in \(GL_n(\C)\), for all \(g \in G\). The \textbf{character of \((V,\rho)\)}, denoted \(X_V\) is the function
	\begin{align*}
		X_V:G\to \C\\
		g\mapsto \textrm{Tr}(\rho g)
	\end{align*}
	We say that \(X_V\) is \textbf{irreducible} if \((V,\rho)\) is irreducible.
\end{defn}
It turns out that characters detect irreducibility. Let \(X_V,\psi_W\) be given. We define a scalar by
\begin{align*}
	\langle X_V,\psi_W \rangle := \frac{1}{\left| G \right| }\sum_{g \in G} \overline{X_V(g)}\psi_W(g).
\end{align*}
\begin{prop}
	Let \(V\) be a representation of \(G\). Then
	\begin{align*}
		V \text{ is irreducible }\iff\langle X_V,X_V \rangle =1.
	\end{align*}
\end{prop}

\end{document}
