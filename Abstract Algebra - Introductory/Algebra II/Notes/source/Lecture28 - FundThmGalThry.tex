\documentclass{memoir}
\usepackage{notestemplate}

%\logo{~/School-Work/Auxiliary-Files/resources/png/logo.png}
%\institute{Rice University}
%\faculty{Faculty of Whatever Sciences}
%\department{Department of Mathematics}
%\title{Class Notes}
%\subtitle{Based on MATH xxx}
%\author{\textit{Author}\\Gabriel \textsc{Gress}}
%\supervisor{Linus \textsc{Torvalds}}
%\context{Well, I was bored...}
%\date{\today}

\begin{document}

% \maketitle

% Notes taken on 05/10/21

We can summarize our characterization thus far by a set of equivalences. The following are equivalent:
\begin{itemize}
	\item A finite field extension \(K / F\) is Galois
	\item \(\left| \textrm{Aut}(K / F) \right| = [K:F]\) 
	\item \(K / F\) is the splitting field of a separable polynomial over \(F\) 
	\item \(K / F\) is normal and separable
	\item \(F = K^{\textrm{Aut}(K / F)}\)
\end{itemize}

\section{Fundamental Theorem of Galois Theory}
\label{sec:fundamental_theorem_of_galois_theory}

\begin{thm}[Fundamental Theorem of Galois Theory]
	Let \(K / F\) be Galois and set \(G:= \textrm{Gal}(K / F)\). Then there exists a bijection between the subfields \(E\subset K\) with \(F\subset E\) and the subgroups \(H \leq G\) given by
\begin{align*}
	E \mapsto \textrm{Aut}(K / E)\\
	H \mapsto K^{H}
\end{align*}
and these maps are inverses of each other. Furthermore, this bijection has some additonal properties:
\begin{itemize}
	\item If \(E_1 \leftrightarrow H_1\) and \(E_2 \leftrightarrow H_2\), then \(E_1 \subset E_2 \iff H_2 \leq H_1\).
	\item If \(E \leftrightarrow H\), then \([K:E] = \left| H \right| \) and \([E:F] = [G:H]\).
	\item \(K / E\) is always Galois for \(F \subset E \subset  K\).
	\item \(E / F\) is Galois if and only if \(H \triangleleft G\). In this case, \(\textrm{Gal}(E / F) \cong G / H\).
	\item If \(E_1 \leftrightarrow H_1\) and \(E_2 \leftrightarrow H_2\), then \(E_1 \cap E_2 \leftrightarrow \langle H_1,H_2 \rangle \) and \(E_1E_2 \leftrightarrow H_1 \cap H_2\).
\end{itemize}
\end{thm}
Remember that \(H \triangleleft G\) is equivalent to \(\textrm{Aut}(K / E) \triangleleft \textrm{Aut}(K / F)\). Also recaall that \(\langle H_1,H_2 \rangle \) is the smallest subgroup of \(G\) that contains \(H_1,H_2\), and \(E_1E_2\) is the smallest subfield of \(K\) containing \(E_1,E_2\). They are not necessarily equivalent!

% Examples here

% Proof here

\vspace{5mm}

\end{document}
