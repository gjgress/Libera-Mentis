\documentclass{memoir}
\usepackage{notestemplate}

%\logo{~/School-Work/Auxiliary-Files/resources/png/logo.png}
%\institute{Rice University}
%\faculty{Faculty of Whatever Sciences}
%\department{Department of Mathematics}
%\title{Class Notes}
%\subtitle{Based on MATH xxx}
%\author{\textit{Author}\\Gabriel \textsc{Gress}}
%\supervisor{Linus \textsc{Torvalds}}
%\context{Well, I was bored...}
%\date{\today}

\begin{document}

% \maketitle

% Notes taken on 03/17/21

\begin{prop}
	Take \(f(x) \in F[x]\) of degree \(n\). Then for \(K :=\) splitting field of \(f(x)\), we get that \([K:F] \leq n!\).
\end{prop}
Now we discuss the uniqueness of splitting fields.
\begin{thm}
	Let \(\varphi :F\to F'\) be an isomorphism of fields. Let
	\begin{align*}
		f(x) = a_nx^{n}+ \ldots + a_1x + a_0 \in F[x]\\
		f'(x) = \varphi (a_n)x^{n}+ \ldots + \varphi (a_1) x + \varphi (a_0) \in F'[x].
	\end{align*}
	Let \(E\) be the splitting field of \(f(x)\) over \(F\) and \(E'\) be the splitting field of \(f'(x)\) over \(F'\). Then the isomorphism \(\varphi \) extens to an isomorphism \(\sigma :E\to E'\), so that \(\sigma \mid_F = \varphi \).
\end{thm}
This can be proven by induction on the degree of \(f(x)\).
\begin{cor}
	Any two splitting fields for a polynomial \(f(x) \in F[x]\) over a field \(F\) are isomorphic.
\end{cor}

Thus we can safely refer to -the- splitting field of a polynomial over a field.

\begin{defn}
	If \(K\) is an algebraic extension of \(F\), which is the splitting field over \(F\) for a collection of polynomials \(\left\{ f_i(x) \right\} \in F[x]\), then \(K\) is called a \textbf{normal} extension of \(F\).
\end{defn}
In other words, a normal extension is simply an algebraic extension that is also a splitting field.
\begin{hw}
	Determine the splitting field of \(x^{6}-4\) over \(\Q\) and its degree over \(\Q\).
\end{hw}
\end{document}
