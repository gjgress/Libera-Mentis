\documentclass{memoir}
\usepackage{notestemplate}

%\logo{~/School-Work/Auxiliary-Files/resources/png/logo.png}
%\institute{Rice University}
%\faculty{Faculty of Whatever Sciences}
%\department{Department of Mathematics}
%\title{Class Notes}
%\subtitle{Based on MATH xxx}
%\author{\textit{Author}\\Gabriel \textsc{Gress}}
%\supervisor{Linus \textsc{Torvalds}}
%\context{Well, I was bored...}
%\date{\today}

\begin{document}

% \maketitle

% Notes taken on 05/10/21

\begin{exmp}
	Let \(K = \Q(\sqrt{2} ,\sqrt{3} )\). Then one can see that \(\Q(\sqrt{2} )\), \(\Q(\sqrt{3} )\), and \(\Q(\sqrt{6} )\) are all subfields for which \(K\) is a Galois extension. Furthermore, these fields are all Galois extensions of \(\Q\).
\end{exmp}

\begin{exmp}
	Consider the quotient field \(\mathbb{F}_2(t)\) of \(\mathbb{F}_2[t]\) and consider \(f(x) = x^2-t \in \mathbb{F}_2(t)[x]\). One can show that \(f(x)\) is irreducible but not separable over \(\mathbb{F}_2(t)\), and hence if \(\theta \) is a root of \(f(x)\), \(\mathbb{F}_2(t)(\theta )\) is NOT a Galois extension of \(\mathbb{F}_2(t)\).
\end{exmp}

\begin{exmp}
	Let \(K\) be the splitting field of \(x^3-2\), i.e. \(K = \Q(\sqrt[3]{2} ,\omega )\). \(K\) is Galois over \(\Q\), but \(\Q(\sqrt[3]{2} )\) is NOT Galois over \(\Q\).\\

	In fact, \(\textrm{Gal}(K / \Q) \) is a nonabelian group of order 6, and thus is isomorphic to \(S_3\).
\end{exmp}
\end{document}
