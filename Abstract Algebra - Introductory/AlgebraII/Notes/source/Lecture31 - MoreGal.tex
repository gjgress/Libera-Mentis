\documentclass{memoir}
\usepackage{notestemplate}

%\logo{~/School-Work/Auxiliary-Files/resources/png/logo.png}
%\institute{Rice University}
%\faculty{Faculty of Whatever Sciences}
%\department{Department of Mathematics}
%\title{Class Notes}
%\subtitle{Based on MATH xxx}
%\author{\textit{Author}\\Gabriel \textsc{Gress}}
%\supervisor{Linus \textsc{Torvalds}}
%\context{Well, I was bored...}
%\date{\today}

\begin{document}

% \maketitle

% Notes taken on 05/10/21

\begin{defn}[Radical Extension]
	A field \(K\) is said to be a \textbf{radical extension} of a field \(F\) if there is a chain of fields
	\begin{align*}
		F = F_0 \subset F_1 \subset F_2 \subset \ldots\subset F_n = K
	\end{align*}
	such that, for each \(i=1,\ldots,n\), \(F_i = F_{i-1}(\alpha_i)\) and some power of \(\alpha_i\) is in \(F_{i-1}\).
\end{defn}
Let \(f \in F[x]\). The equation \(f(x) = 0_F\) is \textbf{solvable by radicals} if there exists a radical extension of \(F\) that contains a splitting field of \(f(x)\). This is equivalent to the notion of there existing a "formula" for the solutions.

\section{Solvable Groups}
\label{sec:solvable_groups}

\begin{defn}[Solvable]
	A group \(G\) is said to be \textbf{solvable} if it has a chain of subgroups
	\begin{align*}
		\langle e \rangle = G_n \triangleleft \ldots \triangleleft G_1 \triangleleft G_0 = G
	\end{align*}
	such that each quotient group \(G_{i-1} / G_i\) is abelian.
\end{defn}
Notice that all abelian groups are solvable.
\begin{prop}
	For \(n\geq 5\) the group \(S_n\) is not solvable.
\end{prop}
\begin{thm}
	Every homomorphic image of a solvable group \(G\) is solvable.
\end{thm}

Our goal is to prove the Galois Criterion. That is, let \(f \in F[x]\). \(f(x) = 0_F\) is solvable by radicals if and only if the Galois group of \(f(x)\) is a solvable group.

\begin{lemma}
	Let \(F\) be a field and \(\eta \) a primitive \(n\)-th root of unity in \(F\). Then \(F\) contains a primitive \(d\)-th root of unity for every positive \(d\mid n\).
\end{lemma}

This combined with the next two theorems will allow us to prove the Galois Criterion.

\begin{thm}
	Let \(F\) be a field of characteristic zero and \(\eta \) a primitive \(n\)-th root of unity in some field extension of \(F\). Then \(K = F(\eta )\) is a normal extension of \(F\) and \(\textrm{Gal}_F(K)\) is abelian.
\end{thm}

\begin{thm}
	Let \(F\) be a field of characteristic zero that contains a primitive \(n\)-th root of unity. If \(\alpha \) is a root of \(x^{n}-c \in F[x]\) in some extension field of \(F\), then \(K = F(\alpha )\) is a normal extension of \(F\) and \(\textrm{Gal}_F(K)\) is abelian.
\end{thm}

\begin{lemma}
	Let \(F,E,K\) be fields of characteristic zero with
	\begin{align*}
		F\subset E\subset K = E(\alpha )\quad \alpha^k \in E
	\end{align*}
	If \(K\) is finite-dimensional over \(F\) and \(E\) is normal over \(F\), then there exists a field extension \(L\) of \(K\) which is a radical extension of \(E\) and a normal extension of \(F\).
\end{lemma}

\begin{thm}[Galois Criterion]
Let \(f \in F[x]\). \(f(x) = 0_F\) is solvable by radicals if and only if the Galois group of \(f(x)\) is a solvable group.

\end{thm}

We can use this to show that there is no formula for the solutions of all fifth-degree polynomials, as there are fifth-degree polynomials whose Galois group is \(S_5\).

\begin{thm}
	Let \(F\) be a field of characteristic zero and \(f(x) \in F[x]\). If \(f(x) = 0_F\) is solvable by radicals, then there is a normal radical field extension of \(F\) that contains the splitting field of \(f(x)\).
\end{thm}

\begin{thm}
	Let \(K\) be a normal radical field extension of \(F\) and \(E\) an intermediate field, all of characteristic zero. If \(E\) is normal over \(F\), then \(\textrm{Gal}_F(E)\) is a solvable group.
\end{thm}

\end{document}
