\documentclass{memoir}
\usepackage{notestemplate}

%\logo{~/School-Work/Auxiliary-Files/resources/png/logo.png}
%\institute{Rice University}
%\faculty{Faculty of Whatever Sciences}
%\department{Department of Mathematics}
%\title{Class Notes}
%\subtitle{Based on MATH xxx}
%\author{\textit{Author}\\Gabriel \textsc{Gress}}
%\supervisor{Linus \textsc{Torvalds}}
%\context{Well, I was bored...}
%\date{\today}

\begin{document}

% \maketitle

% Notes taken on 02/10/21

Let \(R\) be a ring with \(1_R\).
\begin{prop}
	Let \(M\) be an \(R\)-module, with \(N_1,\ldots,N_t\) as \(R\)-submodules. Then
	\begin{enumerate}
		\item The sum of \(\left\{ N_i \right\}_{i=1}\) is
			\begin{align*}
				N_1+\ldots+N_t = \left\{n_1+\ldots+n_t \mid n_i \in N_i, \; i=1,\ldots,t \right\} 
			\end{align*}
			and is an \(R\)-module.
		\item The direct product of \(\left\{ N_i \right\} \) is
			\begin{align*}
				N_1\times \ldots\times N_t = \left\{(n_1,\ldots,n_t) \mid n_i \in N_i, \; i=1,\ldots,t \right\} 
			\end{align*}
			and is an \(R\)-module.
	\end{enumerate}
\end{prop}
One can also define the direct product of \(R\)-modules (i.e. not just submodules).

\begin{prop}
	Let \(N_1,\ldots,N_t\) be \(R\)-submodules of an \(R\)-module \(M\). Then the following are equivalent:
	\begin{enumerate}
		\item
			\begin{align*}
				\varphi:N_1\times \ldots\times N_t \to N_1+\ldots+N_t\\
				(n_1,\ldots,n_t) \mapsto n_1+\ldots+n_t
			\end{align*}
			is an \(R\)-module isomorphism
		\item \(N_j \cap (N_1+\ldots+N_{j-1}+N_{j+1} + \ldots N_t) = 0\) for all \(j=1,\ldots,t\)
		\item Every \(x \in N_1+\ldots+N_t\) can be written as \(n_1+\ldots+n_t\) uniquely for some \(n_i \in N_i\) for all \(i=1,\ldots,t\)
	\end{enumerate}
\end{prop}
If the proposition holds, then
\begin{align*}
	N_1\times \ldots\times N_t \cong N_1+\ldots+N_t
\end{align*}
and we refer to the structure as the direct sum of \(R\)-modules.

\section{Generation}
\label{sec:generation}

\begin{defn}
	Take an \(R\)-module \(M\) and a subset \(X\) of \(M\).
	\begin{enumerate}
		\item The \textbf{\(R\)-submodule generated by \(X\)} is
			\begin{align*}
				RX = \left\{r_1x_1+\ldots+r_mx_m \mid r_i \in R, \; x_i \in X, \; m \in \Z>0 \right\} .
			\end{align*}
			The set \(X\) is called the \textbf{generating set} of \(RX\).
		\item A \(R\)-submodule \(N\) of \(M\) is \textbf{finitely generated} if \(N = RX\) for \(\left| X \right| <\infty\) and \textbf{cyclic} if \(N=RX\) for \(\left| X \right| =1\).
	\end{enumerate}
\end{defn}

\section{Free Modules}
\label{sec:free_modules}

\begin{defn}
	We call \(X = \left\{x_1,\ldots,x_n \right\} \) \textbf{\(R\)-linearly independent} if
	\begin{align*}
		r_1x_1+\ldots+r_nx_n = 0 \implies r_i = 0 \; \forall i = 1,\ldots,n
	\end{align*}
\end{defn}

\begin{defn}
	We say that an \(R\)-module \(M\) is \textbf{free on the subset \(X\)} of \(M\) if
	\begin{align*}
		M = RX\\
		X \text{ is \(R\)-linearly independent}
	\end{align*}
In this case, we call \(X\) the \textbf{basis} of \(M\), and sometimes denote \(M\) by \(F(X)\).
\end{defn}
This illustrates a key difference between vector spaces and modules-- vector spaces are always free, while modules need not be.

\end{document}
