\documentclass{memoir}
\usepackage{notestemplate}

%\logo{~/School-Work/Auxiliary-Files/resources/png/logo.png}
%\institute{Rice University}
%\faculty{Faculty of Whatever Sciences}
%\department{Department of Mathematics}
%\title{Class Notes}
%\subtitle{Based on MATH xxx}
%\author{\textit{Author}\\Gabriel \textsc{Gress}}
%\supervisor{Linus \textsc{Torvalds}}
%\context{Well, I was bored...}
%\date{\today}

\begin{document}

% \maketitle

% Notes taken on whenever

\section{Composite Field Extensions}
\label{sec:composite_field_extensions}

\begin{defn}[Composite Field]
	Let \(K_1\) and \(K_2\) be two subfields of a field \(K\). Then the \textbf{composite field of \(K_1\) and \(K_2\)}, denoted by \(K_1K_2\) is the smallest subfield of \(K\) containing both \(K_1\) and \(K_2\).
\end{defn}
The composite of any collection of subfields \(\left\{ K_i \right\} \) is defined similarly.\\\

\begin{prop}
	Let \(K_1\) and \(K_2\) be two finite extensions of \(F\) contained in \(K\). Then
	\begin{align*}
		[K_1K_2:F] \leq [K_1:F][K_2:F]
	\end{align*}
	with equality if and only if an \(F\)-vector space basis for \(K_1\) is linearly independent over \(K_2\) (or vice versa).
\end{prop}
If the \(F\)-vector space basis of \(K_1\) is \(\alpha_1,\ldots,\alpha_n\) and the \(F\)-vector space basis of \(K_2\) is \(\beta_1,\ldots,\beta_m\), then \(\left\{ \alpha_i \beta_j \right\}_{i,j = 1}^{n,m}\) is a \(F\)-vector span of \(K_1K_2\).
\begin{cor}
	If, furthermore, \([K_1:F] = n\) and \([K_2:F] = m\) with \(\textrm{gcd}(n,m) = 1\), then \([K_1K_2:F] = [K_1:F][K_2:F] = nm\).
\end{cor}

\begin{exmp}
	\begin{itemize}
		\item Consider \(K = \Q(\sqrt{2} )\Q(\sqrt[3]{2}) \). We have
			\begin{align*}
				\Q &\hookrightarrow^{2} \Q(\sqrt{2} ) \hookrightarrow^{3} \Q(\sqrt{2})\Q(\sqrt[3]{2}) = \Q(\sqrt[6]{2})\\
				\Q &\hookrightarrow^{3} \Q(\sqrt[3]{2}) \hookrightarrow^{2} \Q(\sqrt[6]{2})\\
				\Q &\hookrightarrow^{6} \Q(\sqrt[6]{2})
			\end{align*}
			where \(\hookrightarrow^{k}\) represents a degree \(k\) extension. % Fill in rest of field extensions here, and perhaps turn into diagrams
	\end{itemize}
\end{exmp}

\section{Splitting Fields}
\label{sec:splitting_fields}

Recall that for any field \(F\) and any polynomial \(f(x) \in F[x]\), there exists a field extension \(K\) over \(F\) that contains a root, say \(\alpha  \in K\), of \(f(x)\). In this case, \(f(x) = (x-\alpha )g(x)\) in \(K[x]\) as \(K[x]\) is a Euclidean domain.\\

Now we want a field extension \(K / F\) so that \(f(x) \in F[x]\) splits completely into linear factors in \(K[x]\).

\begin{defn}
	A field extension \(K\) of \(F\) is called a \textbf{splitting field for \(f(x) \in F[x]\)} if \(f(x) = \prod_{i} (x - \alpha_i) \) in \(K[x]\) and \(f(x)\) does NOT factor completely in \(K'[x]\) for any proper subfield \(K'\) of \(K\).
\end{defn}
\(f(x) \in K[x]\) splits completely if and only if \(K\) contains all roots of \(f(x)\).

\begin{exmp}
	\begin{itemize}
		\item The splitting field of \(x^2-2\) over \(\Q\) is \(\Q(\sqrt{2} )\) 
		\item The splitting field of \((x^2-2)(x^2-3)\) over \(\Q\) is \(\Q(\sqrt{2} ,\sqrt{3} )\)
		\item The splitting field of \(x^3-2\) over \(\Q\) is NOT \(\Q(\sqrt[3]{2})\). The roots \(\sqrt[3]{2}\omega \) and \(\sqrt[3]{2}\omega^2 \) are in fact imaginary and hence are not in \(\Q(\sqrt[3]{2})\) (note that \(\omega \) represents the principal root of unity).
	\end{itemize}
\end{exmp}

\begin{thm}
	Splitting fields always exist. For any field \(F\), if \(f(x) \in F[x]\), then there exists a field extension \(K\) of \(F\) that is a splitting field for \(f(x)\).
\end{thm}
\end{document}
