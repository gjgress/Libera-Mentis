\documentclass{memoir}
\usepackage{notestemplate}

%\logo{./resources/pdf/logo.pdf}
%\institute{Rice University}
%\faculty{Faculty of Whatever Sciences}
%\department{Department of Mathematics}
%\title{Class Notes}
%\subtitle{Based on MATH xxx}
%\author{\textit{Author}\\Gabriel \textsc{Gress}}
%\supervisor{Linus \textsc{Torvalds}}
%\context{Well, I was bored...}
%\date{\today}

\begin{document}

% \maketitle

% Notes taken on 01/29/21

\begin{thm}
	If \(R\) is a UFD, then so is \(R[x]\).
\end{thm}

\begin{defn}
	A polynomial \(f(x) \in R[x]\) is \textbf{primitive} if \( \textrm{gcd}(\left\{ \text{coeff of }f(x) \right\} ) = 1_R\)
\end{defn}
Recall that since \(R\) is an integral domain, one can form its field of fractions by
\begin{align*}
	F:= \textrm{Frac}(R) = \left\{ \frac{r}{s} \mid r,s \in R, \; s \neq 0 \right\} 
\end{align*}
\begin{lemma}[Gauss' Lemma]
	Let \(R\) be a UFD with \(F = \textrm{Frac}(R)\).
	\begin{itemize}
		\item If \(f(x),g(x) \in R[x]\) are primitive, then so is \(f(x)\cdot g(x)\).
		\item Take \(f(x) \in R[x]\). Then \(f(x) = \varphi(x) \psi(x) \in F[x]\) with \( \textrm{deg}(\varphi) \textrm{deg}(\psi)\geq 1 \iff f(x) = \psi(x) \varphi(x)\) in \(R[x]\).
	\end{itemize}
\end{lemma}

\begin{cor}
	Let \(R\) be a UFD. The irreducible elements of \(R[x]\) are of two types:
	\begin{itemize}
		\item nonzero scalar polynomials that are irreducible as elements of \(R\) 
		\item primitive polynomials in \(R[x]\) that are irreducible in \(F[x]\)
	\end{itemize}
\end{cor}

\begin{thm}
	If \(R\) is a UFD, then \(R[x]\) is a UFD.
\end{thm}

\end{document}
