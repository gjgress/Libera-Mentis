\documentclass[num=5,duedate=03-10-21,course=Algebra\ II,proflastname=Walton]{hwtemplate}

%%% Options for hwtemplate.cls:
%
%% Required:
%
% num - Assignment number
% course - Name or course ID
% proflastname - Last name of professor
% duedate - date that homework is due, in mm-dd-yy
%
%% Optional:
%
% type - type of document (default: Homework)
% studentid - student id used for emails etc. (default: gjg3)
% name - your full name (default: Gabriel Gress)
% emaildomain - domain of email (default: rice.edu)
%
%%%

\begin{document}

\lstset{language=Matlab,%
	%basicstyle=\color{red},
	breaklines=true,%
	morekeywords={matlab2tikz},
	keywordstyle=\color{blue},%
	morekeywords=[2]{1}, keywordstyle=[2]{\color{black}},
	identifierstyle=\color{black},%
	stringstyle=\color{mylilas},
	commentstyle=\color{mygreen},%
	showstringspaces=false,%without this there will be a symbol in the places where there is a space
	numbers=left,%
	numberstyle={\tiny \color{black}},% size of the numbers
	numbersep=9pt, % this defines how far the numbers are from the text
	emph=[1]{for,end,break},emphstyle=[1]\color{red} %some words to emphasise
	%emph=[2]{word1,word2}, emphstyle=[2]{style},
}

% \lstinputlisting{foo.m}

\maketitle
Collaborated with the Yellow group.
\pagebreak
\problem[1]
\begin{claim}
Take a \(G\)-homomorphism \(\varphi:V\to W\). Show that
\begin{enumerate}[(a).]
	\item \(\textrm{Ker}(\varphi)\) is a subrepresentation of \(V\), and
	\item \(\textrm{Im}(\varphi)\) is a subrepresentation of \(W\).
\end{enumerate}
\end{claim}

\begin{proof}
	\begin{enumerate}[(a).]
		\item First, observe that \(\textrm{Ker}(\varphi )\) is a vector subspace of \(V\). To verify that it is a subrepresentation, we apply the \(g\)-action on an arbitrary element. Let \(v \in \textrm{Ker}(\varphi )\):
			\begin{align*}
				\varphi (\rho_g(v)) = \rho_g(\varphi (v)) = \rho_g(0) = 0.
			\end{align*}
			And hence it is closed under the \(g\)-action.
		\item Liekwise, we know that \(\textrm{Im}(\varphi )\) is a vector subpace of \(V\). Now we verify it is a subrepresentation:
			\begin{align*}
				\rho_g(w) = \rho _g(\varphi (v)) = \varphi (\rho_g (v))
			\end{align*}
			and hence \(\varphi_g(v)\) is in the image of \(\varphi \) as desired.
	\end{enumerate}
\end{proof}

\problem[2]
\begin{claim} %Hint for (5), use Exercise B of HW 4
	Let \((V,\rho )\) be a representation of \(G\), and take \(g \in G\). Let \(X_V\) represent the character table of \(G\). Prove:
	\begin{enumerate}
		\item \(X_V(e) = \textrm{dim}V\)
		\item \(X_{V\oplus W}(g) = X_V(g) + X_W(g)\)
		\item \(X_V(g^{-1}) = \overline{X_V(g)}\)
		\item \(\overline{X_V}\) is a character of \(G\)
	\end{enumerate}
\end{claim}
\begin{proof}
	\begin{enumerate}
		\item Because \(\rho \) is a group homomorphism, \(\rho (e) = I\)
	\end{enumerate}
\end{proof}

\problem[3] % Dummit-Foote Section 13.1 Exercise #4
\begin{claim}
	Prove directly that the map \(a+b\sqrt{2} \mapsto a-b \sqrt{2} \) is an isomorphism of \(\Q(\sqrt{2} )\) with itself.
\end{claim}
\begin{proof}
	
\end{proof}

\separator

\problem[4]
\begin{claim}
	Given a finite abelian group \(G\), describe its irreducible complex representations, up to equivalence. Illustrate this for the Klein-four group \(G = C_2 \times C_2\).
\end{claim}
\begin{proof}
	
\end{proof}

\problem[5]
\begin{claim}
	Let \(V\) and \(W\) be irreducible complex representations of a group \(G\), and take \(\varphi \in \textrm{Hom}_G(V,W)\). Show that
	\begin{enumerate}[(a).]
		\item If \(V \not\cong W\), then \(\varphi\) is the zero map.
		\item If \(V \cong W\) and \(\varphi \neq 0\), then \(\varphi \) is a \(G\)-isomorphism.
	\end{enumerate}
\end{claim}
\begin{proof}
	
\end{proof}

\problem[6]
\begin{claim}
	There is a group \(G\) of order \(8\) that has five conjugacy classes of elements; suppose that the representations are \(g_1,\ldots,g_5\). Given irreducible characters \(\chi_1,\ldots,\chi _4\), complete the character table below for the row corresponding to the final irreducible character \(\chi_5\). Make sure to verify that \(\chi_5\) is indeed irreducible and pairwise inequivalent to \(\chi _1,\ldots,\chi _4\) by using its values \(\chi _5(g_j)\) for \(j = 1,\ldots,5\).
		\begin{table}[ht]
			\centering
			\(\begin{array}{c | c c c c c}
			 g_i & g_1 & g_2 & g_3 & g_4 & g_5\\
			 \left| C_G(g_i) \right| & 8 & 8 & 4 & 4 & 4\\
			 \hline
			 \chi_1 & 1 & 1 & 1 & 1 & 1\\
			 \chi_2 & 1 & 1 & 1 & -1 & -1 \\
			 \chi _3 & 1 & 1 & -1 & 1 & -1\\
			 \chi_4 & 1 & 1 & -1 & -1 & 1\\
			 \chi_5 & ? & ? & ? & ? & ?
		 \end{array}\)
		\end{table}
\end{claim}
\begin{proof}
	
\end{proof}
\problem[7]
\begin{claim}
	Let \(\chi_1,\ldots,\chi_r\) be the irreducible characters of a finite group \(G\). Show that
	\begin{align*}
		Z(G) = \left\{g \in G \mid \sum_{i=1}^{r} \overline{\chi_i(g)}\chi_i(g) = \left| G \right|  \right\} .
	\end{align*}
	Here, \(Z(G)\) is the center of \(G\), which consist of elements \(g \in G\) so that \(gh = hg\) for each \(h \in G\).
\end{claim}
\begin{proof}
	
\end{proof}
\end{document}
