\documentclass[num=10,duedate=04-14-21,course=Algebra\ II,proflastname=Walton]{hwtemplate}

%%% Options for hwtemplate.cls:
%
%% Required:
%
% num - Assignment number
% course - Name or course ID
% proflastname - Last name of professor
% duedate - date that homework is due, in mm-dd-yy
%
%% Optional:
%
% type - type of document (default: Homework)
% studentid - student id used for emails etc. (default: gjg3)
% name - your full name (default: Gabriel Gress)
% emaildomain - domain of email (default: rice.edu)
%
%%%

\begin{document}

\lstset{language=Matlab,%
	%basicstyle=\color{red},
	breaklines=true,%
	morekeywords={matlab2tikz},
	keywordstyle=\color{blue},%
	morekeywords=[2]{1}, keywordstyle=[2]{\color{black}},
	identifierstyle=\color{black},%
	stringstyle=\color{mylilas},
	commentstyle=\color{mygreen},%
	showstringspaces=false,%without this there will be a symbol in the places where there is a space
	numbers=left,%
	numberstyle={\tiny \color{black}},% size of the numbers
	numbersep=9pt, % this defines how far the numbers are from the text
	emph=[1]{for,end,break},emphstyle=[1]\color{red}, %some words to emphasise
	%emph=[2]{word1,word2}, emphstyle=[2]{style},
}

% \lstinputlisting{foo.m}

\maketitle
\pagebreak
\problem[1]
\begin{claim}
	Let \(\tau \) be the map \(\tau :\C\to \C\) defined by \(\tau (a+bi) = a-bi\). Prove that \(\tau \) is an automorphism of \(\C\), and then use this to determine the fixed field of \(\tau \) on \(\C\).
\end{claim}

\begin{proof}
Observe that
\begin{align*}
	\tau((a+bi)+(c+di)) = (a+c) - (b+d)i = \tau(a+bi) + \tau (c+di)\\
	\tau((a+bi)(c+di)) = \tau ((ac-bd) + (ad+cd)i) = (ac-bd) - (ad+cd)i = (a-bi)(c-di) = \tau (a+bi)\tau (c+di)
\end{align*}
and hence \(\tau \) is an automorphism of \(\C\). Now we will show that the fixed field of \(\tau \) is \(\R\). Observe that
\begin{align*}
	\tau (a+bi) = a-bi = a+bi \iff b=0
\end{align*}
which is exactly the space of \(\R\).
\end{proof}

\problem[2]
\begin{claim}
	\begin{enumerate}[(a).]
		\item Determine the automorphisms of the extension \(\Q(\sqrt[4]{2}) / \Q(\sqrt{2} ) \) explicitly.
		\item Show that \(\textrm{Aut}(\Q(\sqrt[4]{2}) / \Q) \) is not the trivial group.
	\end{enumerate}
\end{claim}
\begin{proof}
	\begin{enumerate}[(a).]
		\item We are looking for the automorphisms of \(\Q(\sqrt[4]{2} \) that fix \(\Q(\sqrt{2} )\). Observe that the minimal polynomial for \(\Q(\sqrt[4]{2}) \) is \(x^{4}-2\), which has roots in \(\Q(\sqrt[4]{2}) \) given by \(\left\{ \sqrt[4]{2}, - \sqrt[4]{2}   \right\} \). Because the other two roots are complex and hence not in the space, we only have the two automorphisms: the identity automorphism, and the automorphism given by \(\begin{cases}
				\sqrt[4]{2} \mapsto -\sqrt[4]{2} \\
				-\sqrt[4]{2} \mapsto \sqrt[4]{2}  
		\end{cases}\). Explicitly:
		\begin{align*}
			\textrm{Aut}(\Q(\sqrt[4]{2}) / \Q(\sqrt{2} )) = \left\{ \textrm{Id}_{\Q(\sqrt[4]{2}) }; \sigma: \sqrt[4]{2}\mapsto -\sqrt[4]{2}   \right\} 
		\end{align*}
	\item Observe that \(\sqrt{2} \) is not a root of our minimal polynomial as given above. As a result, the automorphisms of \(\Q\sqrt[4]{2} / \Q(\sqrt{2} ) \) are exactly the same as the automorphisms of \(\Q(\sqrt[4]{2})/ \Q\). Hence we see that the automorphism group is the finite group of order two as given above (and hence not the trivial group).
	\end{enumerate}
\end{proof}

\problem[3]
\begin{claim}
	Let \(\zeta_n\) be a primitive \(n\)th root of unity in \(\C\). Show that \(\textrm{Aut}(\Q(\zeta_n)/\Q)\) is an abelian group.
\end{claim}
\begin{proof}
	By construction any automorphism on the space fixes \(\Q\). As a result, for any automorphism defined on \(\varphi (\zeta_n)\) must satisfy \(\varphi (\zeta_n)^{n} = \varphi (\zeta_n ^{n}) = \varphi (1) = 1\). In other words, we can see that the automorphisms must permute the roots of unity. In other words, we can represent automorphisms of \(\Q(\zeta_n) / \Q\) by
	\begin{align*}
		\varphi_i (\zeta_n) = \zeta^{i}_n.
	\end{align*}
	Now we simply check that
	\begin{align*}
		\varphi_i\varphi_j(\zeta_n) = \varphi_i(\zeta^{j}_n) = \zeta^{ji}_n = \zeta^{ij}_n = \varphi_j (\zeta^{i}_n) = \varphi_j \varphi_i (\zeta _n)
	\end{align*}
	And hence we have that they must commute for all automorphisms on \(\Q(\zeta_n) / \Q\).
\end{proof}
\separator

\problem[4]
\begin{claim}
	\begin{enumerate}[(a).]
	\item Show that if the field \(K\) is generated over \(F\) by the elements \(\alpha _1,\ldots,\alpha_n\), then an automorphism \(\sigma \) of \(K\) fixing \(F\) is uniquely determined by \(\sigma (\alpha _1),\ldots,\sigma (\alpha _n)\).
	\item Let \(G\leq \textrm{Gal}(K / F)\) be a subgroup of the Galois group of the extension \(K / F\) and suppose \(\sigma_1,\ldots,\sigma_k\) are generators for \(G\). Show that the subfield \(E / K\) is fixed by \(G\) if and only if it is fixed by the generators \(\sigma_1,\ldots,\sigma_k\).
	\end{enumerate}
\end{claim}
\begin{proof}
	\begin{enumerate}[(a).]
		\item Let \(K / F\) be a field extension generated by \(\alpha_1,\ldots,\alpha_n\), let \(\sigma \) be an automorphism of \(K\) that fixes \(F\), and let \(x \in K / F\) be given by \(x = \lambda_1 \alpha_1 + \ldots + \lambda_n \alpha_n\). Then
			\begin{align*}
				\sigma(x) = \sigma (\sum_{i=1}^{n} \lambda_i\alpha_i) = \sum_{i=1}^{n} \lambda_i \sigma (\alpha_i)
			\end{align*}
			and hence if \(F\) is to be fixed, then \(\left\{ \lambda_i \right\} \) is uniquely determined for a given automorphism.
		\item Suppose \(E / F\) is fixed by \(G\leq \textrm{Gal}(K / F)\) generated by \(\sigma_1,\ldots,\sigma_k\). It immediately follows that if it is fixed by \(G\) it must be fixed by the generators themselves. The reverse direction does not follow as directly: if \(E / F\) is fixed by \(\sigma_1,\ldots,\sigma_k\) individually, then consider \(\sigma  \in G\). Observe that \(\sigma \in \textrm{Aut}(K / F)\), which implies that \(\sigma \) fixes \(K\). Because \(E / F \subset K / F\), it must hold that \(\sigma \) fixes \(E\), and hence we are done.
	\end{enumerate}
\end{proof}

\problem[5]
\begin{claim}
	Let \(F\subset E\subset K\) be a composition of field extensions so that \(E\) is normal over \(F\). Prove that if \(\sigma  \in \textrm{Aut}(K / F)\), then \(\sigma (E) = E\).
\end{claim}
\begin{proof}
	Let \(\sigma \in \textrm{Aut}(K / F)\) be given. \(K / F\) is hence Galois, and so by a proposition from class, we know that \(K\) is the splitting field of a separable polynoimal \(p \in F[x]\). Let \(\alpha_1,\ldots,\alpha _n\) be roots of \(p\), all distinct. Because \(\alpha_i\) generate \(K / F\), there exists a subset of \(\left\{ \alpha_i \right\} \) that generates \(E\subset F\) which we will denote by \(\alpha'_1,\ldots,\alpha'_m\) with \(m\leq n\). Because \(E / F\) is normal we know that \(\alpha'_i\) must be the roots of a separable polynomial \(p' \in F[x]\) (as the \(\alpha'_i\) are distinct). This gives us that \(E / F\) is Galois, and hence \(\sigma (E) = E\) as desired.
\end{proof}

\problem[6]
\begin{claim}
	\begin{enumerate}[(a).]
	\item Prove that any \(\sigma \in \textrm{Aut}(\R / \Q)\) takes squares to squares and positive reals to positive reals. Use this to conclude that \(a<b\) implies \(\sigma a < \sigma b\) for \(a,b \in \R\).
	\item Prove that \(-\frac{1}{m}<a-b<\frac{1}{m}\) implies \(-\frac{1}{m}<\sigma a-\sigma b<\frac{1}{m}\) for every positivfe integer \(m\). Conclude that \(\sigma \) is the continuous map on \(\R\).
	\item Prove that any continous map on \(\R\) which is the identity of \(\Q\) is the identity map, and hence \(\textrm{Aut}(\R/\Q)=1\).
	\end{enumerate}
\end{claim}
\begin{proof}
	\begin{enumerate}[(a).]
		\item Let \(\sigma \in \textrm{Aut}(\R / \Q)\) be given, and let \(x \in \R\) be an element which can be expressed as \(x=y^2\) for some \(y \in \R\). Of course it then holds that \(\sigma (x) = \sigma (y^2) = \sigma(y)^2\). Now let \(x\) be an arbitrary positive real number. By the property of \(\R\),  \(\sqrt{x} \) is well defined and hence \(x = \sqrt{x} ^2\). Hence
			\begin{align*}
				\sigma (x) = \sigma (\sqrt{x} )^2
			\end{align*}
			which must be greater than zero, as \(\sqrt{x } \neq 0\). Now let \(a<b\) be given. Observe that \(b-a\) is a positive real number, and hence
			\begin{align*}
				b-a>0 \implies \sigma (b-a)>0 \implies \sigma (b) - \sigma (a) > 0 \implies \sigma (a) < \sigma (b)
			\end{align*}
			as desired.
		\item Now suppose that for some positive integer \(m\) we have
			\begin{align*}
				-\frac{1}{m}< a-b < \frac{1}{m}.
			\end{align*}
			Then
			\begin{align*}
				-1 < m(a-b)<1 \implies \sigma (-1) < \sigma (m(a-b)) < \sigma (1) \implies -1 < \sigma (m(a-b))<1\\
				\implies -\frac{1}{m} < \sigma (a)-\sigma (b) < \frac{1}{m}.
			\end{align*}
			Now we argue that this gives continuity. For every \(\varepsilon>0\) there exists a positive integer \(m\) such that \(\frac{1}{m}< \varepsilon\) by the Archimedean property. Hence if \(\left| a-b \right| < \frac{1}{m}\), then \(\left| \sigma (a)-\sigma (b) \right| < \frac{1}{m} < \varepsilon\), and hence we can choose our delta to be \(\frac{1}{m}\), giving us continuity as defined on \(\R\).
		\item This directly follows from the fact that continuous functions that agree on dense sets of a space must agree on the whole space. As a result, if a continuous map is identity on \(\Q\), it must be identity on \(\R\), as \(\Q\) is dense in \(\R\).
	\end{enumerate}
\end{proof}
\end{document}
