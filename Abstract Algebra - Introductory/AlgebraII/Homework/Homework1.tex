\documentclass[num=1,duedate=02-03-21,course=Algebra\ II,proflastname=Walton]{hwtemplate}

%%% Options for hwtemplate.cls:
%
%% Required:
%
% num - Assignment number
% course - Name or course ID
% proflastname - Last name of professor
% duedate - date that homework is due, in mm-dd-yy
%
%% Optional:
%
% type - type of document (default: Homework)
% studentid - student id used for emails etc. (default: gjg3)
% name - your full name (default: Gabriel Gress)
% emaildomain - domain of email (default: rice.edu)
%
%%%

\usepackage{float}

\begin{document}

\lstset{language=Matlab,%
	%basicstyle=\color{red},
	breaklines=true,%
	morekeywords={matlab2tikz},
	keywordstyle=\color{blue},%
	morekeywords=[2]{1}, keywordstyle=[2]{\color{black}},
	identifierstyle=\color{black},%
	stringstyle=\color{mylilas},
	commentstyle=\color{mygreen},%
	showstringspaces=false,%without this there will be a symbol in the places where there is a space
	numbers=left,%
	numberstyle={\tiny \color{black}},% size of the numbers
	numbersep=9pt, % this defines how far the numbers are from the text
	emph=[1]{for,end,break},emphstyle=[1]\color{red}, %some words to emphasise
	%emph=[2]{word1,word2}, emphstyle=[2]{style},
}

% \lstinputlisting{foo.m}

\maketitle
\pagebreak
\problem[1]

\begin{figure}[H]
    \centering
     \def\svgwidth{1\linewidth}
     \input{./figures/worksheet-1.pdf_tex}
    \caption{Worksheet 1}
    \label{fig:worksheet-1}
\end{figure}

\problem[2]
\begin{claim}
	Let \(R\) be a UFD.
	\begin{enumerate}[(a).]
		\item Let \(b\) and \(a_1,\ldots,a_s\) be nonzero elements of \(R\). For \(d \in R\), show that
			\begin{align*}
				bd = gcd(ba_1,\ldots,ba_s) \iff d = gcd(a_1,\ldots,a_s)
			\end{align*}
		\item Let \(f(x) \in R[x]\) where \(f(x) = bf_1(x)\) for \(f_1(x)\) \textbf{primitive} (i.e. \(gcd(\text{coefficients of }f_1(x)) = 1_R\) ). Show that
			\begin{align*}
				b = gcd(\left\{ \text{coefficients of }f(x) \right\} ).
			\end{align*}
	\end{enumerate}

\end{claim}
\begin{proof}
	\begin{enumerate}[(a).]
		\item First, assume that \(bd = gcd(ba_1,\ldots,ba_s)\). Now consider an arbitrary \(d'\) that satisfies
			\begin{align*}
				d' \mid a_1,\ldots,a_s.
			\end{align*}
			We want to show that \(d' \mid d\). Observe that if \(d'\mid a_1,\ldots,a_s\), then \(d' \mid ba_1,\ldots,ba_s\), as the prime factorization of \(ba_i\) trivially contains the prime factorization of \(a_i\). This then implies that \(d' \mid bd\) by hypothesis. But we know that \(bd' \mid ba_i\) and hence \(bd' \mid bd\). This only holds if \(d'\mid d\), as desired.\\

			For the reverse direction, assume that \(d = gcd(a_1,\ldots,a_s)\). First, observe that \(bd \mid ba_1,\ldots,ba_s\). Thus it only remains to show that if \(d'\mid ba_1,\ldots,ba_s\), then \(d'\mid bd\). Assume for a contradiction that \(d'\) cannot be factored as \(bd''\). Then there exists a factor of \(b\) such that \(b_id' \mid ba_1,\ldots,ba_s\). Of course, if \(b_id'\) divides a term, then \(d'\) divides a term. In other words, we can simply look at the case where \(bd''\) exists, as if it doesn't, then we can multiply it by prime factors until it does, and division will still hold. Now we consider \(bd''\). Of course, if \(bd''\mid ba_i\), then \(d''\mid a_i\). Then we know that \(d''\mid d\) by hypothesis, in which case \(bd''\mid bd\), as desired.
\end{enumerate}
\end{proof}

\problem[3]
\begin{claim}
	\begin{enumerate}[(a).]
		\item Let \(R\) be a commutative ring with identity \(1\). Show that the polynomial rings \(R[x_1, \ldots,x_{n-1},x_n]\) and \(R[x_1,\ldots,x_{n-1}][x_n]\) can be identified.
		\item Show by induction that if \(K\) is a field, then, for all \(n\), \(K[x_1,\ldots,x_{n-1},x_n]\) is a unique factorization domain.
	\end{enumerate}
\end{claim}
\begin{proof}
Let $C_{\alpha_1...\alpha_n}$ be the coefficient of $x^{\alpha_1}...x^{\alpha_n}$ in $R[x_1, ... x_n]$. Let $\varphi:R[x_1, ... x_n] \rightarrow R[x_1, x_{n-1}][x_n]$ be given by
\begin{align}
\varphi \left( \sum_{(\alpha_1, ... \alpha_n) \in \mathbb{N}}C_{\alpha_1...\alpha_n}x_1^{\alpha_1}...x_n^{\alpha_n}\right) = \sum_{j\in \mathbb{N}}\left( \sum_{\alpha_1, ... \alpha_{n-1}}C_{\alpha_1...\alpha_n}x_1^{\alpha_1}...x_{n-1}^{\alpha_{n-1}}  \right)x_n^{\alpha_j}
\end{align}

One can observe that $\varphi$ is the identity map on elements with no \(x_n\) term, and clearly one-to-one for elements with \(x_n\), and hence an isomorphism. The proof of part (b) is by induction. The base case $n=1$ is given by Theorem 6.6.7. Then suppose the result holds for a nonnegative integer $n$. Then by induction hypothesis,
$$F[x_1, ... x_n]$$
is a UFD. By part (a), we have
\begin{align}
    F[x_1 ... x_{n+1}] \cong F[x_1, ... x_n][x_{n+1}]
\end{align}
which is a UFD by Theorem 6.6.7.
\end{proof}
\separator

\problem[4]
% Goodman Exercise 6.6.3
\begin{claim}
	Use Gauss' lemma to show that if a polynomial \(a_nx^{n}+ \ldots + a_1x + a_0 \in Z[x]\) has a rational root \(r / s\), where \(r\) and \(s\) are relatively prime, then \(s\mid a_n\) and \(r\mid a_0\). In particular, if the polynomial is monic, then its only rational roots are integers.
\end{claim}
\begin{proof}
	
		Because \(\Z\) is a UFD, we can apply Gauss' Lemma. Then if a polynomial has a rational root \(r / s\), we know by a proposition in class that it has a degree 1 factor in \(F[x]\). Hence it is reducible in \(F[x]\), which by Gauss' Lemma implies that it is reducible in \(R[x]\). Of course, in order for \(r / s\) to give a degree 1 factor in both domains, it must hold that the factor is in \(R\) and hence in \(Z\), and so in order for this to hold we can assure that \(s\mid a_0\) and \(r\mid a_0\), so that the rational roots are integers.

\end{proof}

\problem[5]
% Goodman Exercise 6.6.5
\begin{claim}
	Complete the details of this alternative proof of Gauss' Lemma:\\

	Let \(R\) be a UFD. For any irreducible \(p \in R\), consider the quotient map \(\pi_p:R\to R / pR\), and extend this to a homomorphism \(\pi_p:R[x]\to (R / pR)[x]\), defined by \(\pi_p(\sum a_ix^{i}) = \sum_i \pi_p(a_i)x^{i}\), using Corollary 6.2.9.
	\begin{enumerate}[(a).]
		\item Show that a polynomial \(h(x)\) is in the kernel of \(\pi_p\) if and only if \(p\) is a common divisor of the coefficients of \(h(x)\).
		\item Show that \(f(x) \in R[x]\) is primitive if and only if for all irreducible \(p, \pi_p(f(x))\neq 0\).
		\item Show that \((R / pR)[x]\) is integral domain for all irreducible \(p\).
		\item Conclude that if \(f(x)\) and \(g(x)\) are primitive in \(R[x]\), then \(f(x)g(x)\) is primitive as well.
	\end{enumerate}
\end{claim}
\begin{proof}
	\begin{enumerate}[(a).]
		\item First, assume that \(h(x)\) is in the kernel of \(\pi_p\), that is, \(\pi_p(h(x)) = 0\). Note that in order for this to be zero, each \(\pi_p(a_i)\) must be identically zero, as you cannot simplify \(x^{i}+x^{j}\) for \(i\neq j\). This only occurs if \(a_i\) is zero in \(R / pR\), which implies it is divisible by \(p\) because all ideals are prime ideals in \(R\). If instead every \(a_i\) is divisible by \(p\), then \(a_i\) must map to \(0\) in \(R/pR\) and so \(\pi_p\) maps \(h(x)\) to a sum with all coefficients zero, and hence \(\pi_p(h(x)) = 0\).
		\item Assume that there is an irreducible \(p\) such that \(\pi_p(f(x))=0\). Then by the first part, \(p\) divides all coefficients, and hence can factor from \(f(x)\), so it wouldn't be primitive. That shows that if \(f(x)\) is primitive, there cannot exist such a \(p\). Now assume that for all irreducible \(p\), \(\pi_p(f(x)) \neq 0\). Because \(R\) is a UFD, there exists a unique prime factorization for all \(a_i\). Because of the hypothesis, there cannot be a prime shared by all elements, otherwise it would be zero in the homomorphism. Then the set of coefficients share no prime factors, and hence the greatest common divisor must be \(1\), giving us that \(f(x)\) is primitive.
		\item Consider
			\begin{align*}
				\left( \sum \pi_p (a_i) x^{i} \right)\left( \pi_p(a'_i)x'^{i} \right) = \sum \sum \pi_p(a_i)\pi_p(a'_j) x^{i}x'^{j}.
			\end{align*}
			Observe that in order for this to be zero, we require \(a_ia'_j + a_ja'_i = 0\) for all \(i,j\). If \(p\) is irreducible, then we know that this implies that each term in the sum must have a zero. We will show that both terms must in fact be zero. Assume that \(a_i\) is zero, so that \(a_ia'_j\) is zero. If the first polynomial is non-zero, then there exists an \(a_k\) that is non-zero, and hence \(a_ka'_j\) would be positive. But then we wouldn't have all sums zero, so there is a contradiction. Thus, if any \(a_i\) or \(a'_j\) is nonzero, then the other must be zero for all terms. This implies that one or both of the polynomials must be identically zero, and hence there are no zero divisors.
		\item As expanded above, our coefficients for \(f(x)g(x)\) can be expressed by \(a_ia'_j + a_ja'_i\). This sum has a prime factorization-- assume for a contradiction that there is a \(p_i\) that factors out from all the sums. Then \(p_i\) must also divide each \(a_ia'_j\) AND \(a_ja'_i\). If it doesn't divide either of \(a_i\), \(a_j\), then it must divide every \(a'_i\), and so \(g(x)\) couldn't have been primitive. Otherwise, it will divide either \(a_i\) or \(a_j\) for each \(i,j\), and eventually will divide each element, giving that \(f(x)\) couldn't have been primitive. Ergo, the product cannot have a gcd that is non-unital, and hence it is primitive, as desired.
	\end{enumerate}
\end{proof}

\end{document}
