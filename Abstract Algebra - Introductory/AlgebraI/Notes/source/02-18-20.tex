\documentclass{memoir}
\usepackage{notestemplate}

% \begin{figure}[ht]
%     \centering
%     \incfig{riemmans-theorem}
%     \caption{Riemmans theorem}
%     \label{fig:riemmans-theorem}
% \end{figure}

\begin{document}
\section{Factor Ring}
\begin{defn}[Residue Class]
The equivalence class of the integer \(a\) with the congruence relation, denoted by \(\overline{a}_n\), is the set
\begin{align*}
	\left\{ \ldots,a-2n,a-n,a,a+n,a+2n,\ldots \right\} 
\end{align*}
In other words, the set of integers congruent to \(a \pmod n\) is the \textbf{residue class} of the integer \(a\) modulo \(n\). 
\end{defn}
We can combine residue classes by taking representative elements and working with them.

\begin{defn}[Coset]
	Let \(I \triangleleft R	\). A \textbf{coset} denoted \(r + I\) is the set
	\begin{align*}
	\left\{r+i \mid i \in I \right\} .
	\end{align*}
\end{defn}
Two cosets are either equal or disjoint. We define addition and multiplication of cosets by
\begin{align*}
	(r+I) + (s+I) = (r+s) + I \\
	(r+I)(s+I) = rs + I
\end{align*}
\begin{defn}[Factor Ring]
	Let \(I \triangleleft R\). We notate by \(R / I\) the \textbf{factor ring}, which is the ring with all of the cosets of \(I\) as elements, using the coset addition and multiplication defined above.
\end{defn}
\end{document}
